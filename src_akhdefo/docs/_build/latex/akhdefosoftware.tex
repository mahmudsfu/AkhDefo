%% Generated by Sphinx.
\def\sphinxdocclass{report}
\documentclass[letterpaper,10pt,english]{sphinxmanual}
\ifdefined\pdfpxdimen
   \let\sphinxpxdimen\pdfpxdimen\else\newdimen\sphinxpxdimen
\fi \sphinxpxdimen=.75bp\relax
\ifdefined\pdfimageresolution
    \pdfimageresolution= \numexpr \dimexpr1in\relax/\sphinxpxdimen\relax
\fi
%% let collapsible pdf bookmarks panel have high depth per default
\PassOptionsToPackage{bookmarksdepth=5}{hyperref}

\PassOptionsToPackage{booktabs}{sphinx}
\PassOptionsToPackage{colorrows}{sphinx}

\PassOptionsToPackage{warn}{textcomp}
\usepackage[utf8]{inputenc}
\ifdefined\DeclareUnicodeCharacter
% support both utf8 and utf8x syntaxes
  \ifdefined\DeclareUnicodeCharacterAsOptional
    \def\sphinxDUC#1{\DeclareUnicodeCharacter{"#1}}
  \else
    \let\sphinxDUC\DeclareUnicodeCharacter
  \fi
  \sphinxDUC{00A0}{\nobreakspace}
  \sphinxDUC{2500}{\sphinxunichar{2500}}
  \sphinxDUC{2502}{\sphinxunichar{2502}}
  \sphinxDUC{2514}{\sphinxunichar{2514}}
  \sphinxDUC{251C}{\sphinxunichar{251C}}
  \sphinxDUC{2572}{\textbackslash}
\fi
\usepackage{cmap}
\usepackage[T1]{fontenc}
\usepackage{amsmath,amssymb,amstext}
\usepackage{babel}



\usepackage{tgtermes}
\usepackage{tgheros}
\renewcommand{\ttdefault}{txtt}



\usepackage[Bjarne]{fncychap}
\usepackage{sphinx}

\fvset{fontsize=auto}
\usepackage{geometry}

\usepackage{nbsphinx}

% Include hyperref last.
\usepackage{hyperref}
% Fix anchor placement for figures with captions.
\usepackage{hypcap}% it must be loaded after hyperref.
% Set up styles of URL: it should be placed after hyperref.
\urlstyle{same}

\addto\captionsenglish{\renewcommand{\contentsname}{Home:}}

\usepackage{sphinxmessages}
\setcounter{tocdepth}{1}



\title{AkhDefo Software}
\date{Jan 09, 2024}
\release{2.2.61}
\author{Mahmud Mustafa Muhammad}
\newcommand{\sphinxlogo}{\vbox{}}
\renewcommand{\releasename}{Release}
\makeindex
\begin{document}

\ifdefined\shorthandoff
  \ifnum\catcode`\=\string=\active\shorthandoff{=}\fi
  \ifnum\catcode`\"=\active\shorthandoff{"}\fi
\fi

\pagestyle{empty}
\sphinxmaketitle
\pagestyle{plain}
\sphinxtableofcontents
\pagestyle{normal}
\phantomsection\label{\detokenize{index::doc}}


\noindent{\hspace*{\fill}\sphinxincludegraphics[width=200\sphinxpxdimen]{{akhdefo_logo}.svg}}

\sphinxstepscope


\chapter{Akhdefo}
\label{\detokenize{README:akhdefo}}\label{\detokenize{README::doc}}



\section{Computer Vision for Slope Stability: Land Deformation Monitoring}
\label{\detokenize{README:computer-vision-for-slope-stability-land-deformation-monitoring}}

\section{Background of Akh\sphinxhyphen{}Defo}
\label{\detokenize{README:background-of-akh-defo}}
\sphinxAtStartPar
\sphinxstylestrong{Akh\sphinxhyphen{}Defo} is derived from two distinct words:
\begin{enumerate}
\sphinxsetlistlabels{\arabic}{enumi}{enumii}{}{.}%
\item {} 
\sphinxAtStartPar
‘Akh’ in the Kurdish language, representing land, earth, or soil (originating from the Kurdish Badini dialect).

\item {} 
\sphinxAtStartPar
‘Defo’, a shorthand for the English term ‘deformation’.

\end{enumerate}


\section{Recommended Citation}
\label{\detokenize{README:recommended-citation}}
\sphinxAtStartPar
Muhammad M, Williams\sphinxhyphen{}Jones G, Stead D, Tortini R, Falorni G, and Donati D (2022) Applications of Image\sphinxhyphen{}Based Computer Vision for Remote Surveillance of Slope Instability. \sphinxstyleemphasis{Front. Earth Sci.} 10:909078. doi: \sphinxhref{https://doi.org/10.3389/feart.2022.909078}{10.3389/feart.2022.909078}


\section{Updates}
\label{\detokenize{README:updates}}\begin{itemize}
\item {} 
\sphinxAtStartPar
\sphinxstylestrong{Deprecated:} Akhdefo version one. \sphinxstyleemphasis{Current recommendation:} Use Akhdefo version 2.

\item {} 
\sphinxAtStartPar
\sphinxstylestrong{New Feature:} Cloud\sphinxhyphen{}based real\sphinxhyphen{}time processing capabilities.

\item {} 
\sphinxAtStartPar
\sphinxstylestrong{Expansion:} Over 20 modules for end\sphinxhyphen{}to\sphinxhyphen{}end Python\sphinxhyphen{}based GIS and Image Processing, and Customized Figure generation.

\item {} 
\sphinxAtStartPar
\sphinxstylestrong{Integration:} Access, filter, and download capabilities for Planet Labs data using the Planet Lab API.

\item {} 
\sphinxAtStartPar
\sphinxstylestrong{Enhancement:} Orthorectification feature for satellite images.

\end{itemize}


\section{Installation of Akhdefo Software}
\label{\detokenize{README:installation-of-akhdefo-software}}
\sphinxAtStartPar
Follow these steps to install the Akhdefo software:
\begin{enumerate}
\sphinxsetlistlabels{\arabic}{enumi}{enumii}{}{.}%
\item {} 
\sphinxAtStartPar
Create a new Python Anaconda environment using the command:

\begin{sphinxVerbatim}[commandchars=\\\{\}]
\PYG{n}{conda} \PYG{n}{create} \PYG{o}{\PYGZhy{}}\PYG{o}{\PYGZhy{}}\PYG{n}{name} \PYG{n}{akhdefo\PYGZus{}env}
\end{sphinxVerbatim}

\item {} 
\sphinxAtStartPar
Create  Anaconda environment and install the following libraries with Anaconda

\end{enumerate}

\begin{sphinxVerbatim}[commandchars=\\\{\}]
\PYG{n+nt}{dependencies}\PYG{p}{:}
\PYG{+w}{  }\PYG{p+pIndicator}{\PYGZhy{}}\PYG{+w}{ }\PYG{l+lScalar+lScalarPlain}{python=3.8}\PYG{+w}{  }\PYG{c+c1}{\PYGZsh{} Assuming Python 3.8, can be changed as needed}
\PYG{+w}{  }\PYG{p+pIndicator}{\PYGZhy{}}\PYG{+w}{ }\PYG{l+lScalar+lScalarPlain}{cmocean}
\PYG{+w}{  }\PYG{p+pIndicator}{\PYGZhy{}}\PYG{+w}{ }\PYG{l+lScalar+lScalarPlain}{pip}
\PYG{+w}{  }\PYG{p+pIndicator}{\PYGZhy{}}\PYG{+w}{ }\PYG{l+lScalar+lScalarPlain}{opencv}
\PYG{+w}{  }\PYG{p+pIndicator}{\PYGZhy{}}\PYG{+w}{ }\PYG{l+lScalar+lScalarPlain}{earthpy}
\PYG{+w}{  }\PYG{p+pIndicator}{\PYGZhy{}}\PYG{+w}{ }\PYG{l+lScalar+lScalarPlain}{flask}
\PYG{+w}{  }\PYG{p+pIndicator}{\PYGZhy{}}\PYG{+w}{ }\PYG{l+lScalar+lScalarPlain}{geopandas}
\PYG{+w}{  }\PYG{p+pIndicator}{\PYGZhy{}}\PYG{+w}{ }\PYG{l+lScalar+lScalarPlain}{glob2}
\PYG{+w}{  }\PYG{p+pIndicator}{\PYGZhy{}}\PYG{+w}{ }\PYG{l+lScalar+lScalarPlain}{gstools}
\PYG{+w}{  }\PYG{p+pIndicator}{\PYGZhy{}}\PYG{+w}{ }\PYG{l+lScalar+lScalarPlain}{hyp3\PYGZus{}sdk}
\PYG{+w}{  }\PYG{p+pIndicator}{\PYGZhy{}}\PYG{+w}{ }\PYG{l+lScalar+lScalarPlain}{ipywidgets}
\PYG{+w}{  }\PYG{p+pIndicator}{\PYGZhy{}}\PYG{+w}{ }\PYG{l+lScalar+lScalarPlain}{json5}
\PYG{+w}{  }\PYG{p+pIndicator}{\PYGZhy{}}\PYG{+w}{ }\PYG{l+lScalar+lScalarPlain}{matplotlib}
\PYG{+w}{  }\PYG{p+pIndicator}{\PYGZhy{}}\PYG{+w}{ }\PYG{l+lScalar+lScalarPlain}{numpy}
\PYG{+w}{  }\PYG{p+pIndicator}{\PYGZhy{}}\PYG{+w}{ }\PYG{l+lScalar+lScalarPlain}{gdal}
\PYG{+w}{  }\PYG{p+pIndicator}{\PYGZhy{}}\PYG{+w}{ }\PYG{l+lScalar+lScalarPlain}{pandas}
\PYG{+w}{  }\PYG{p+pIndicator}{\PYGZhy{}}\PYG{+w}{ }\PYG{l+lScalar+lScalarPlain}{recommonmark}
\PYG{+w}{  }\PYG{p+pIndicator}{\PYGZhy{}}\PYG{+w}{ }\PYG{l+lScalar+lScalarPlain}{sphinx}
\PYG{+w}{  }\PYG{p+pIndicator}{\PYGZhy{}}\PYG{+w}{ }\PYG{l+lScalar+lScalarPlain}{nbsphinx}
\PYG{+w}{  }\PYG{p+pIndicator}{\PYGZhy{}}\PYG{+w}{ }\PYG{l+lScalar+lScalarPlain}{sphinx\PYGZhy{}book\PYGZhy{}theme}
\PYG{+w}{  }\PYG{p+pIndicator}{\PYGZhy{}}\PYG{+w}{ }\PYG{l+lScalar+lScalarPlain}{myst\PYGZhy{}parser}
\PYG{+w}{  }\PYG{p+pIndicator}{\PYGZhy{}}\PYG{+w}{ }\PYG{l+lScalar+lScalarPlain}{plotly}
\PYG{+w}{  }\PYG{p+pIndicator}{\PYGZhy{}}\PYG{+w}{ }\PYG{l+lScalar+lScalarPlain}{pykrige}
\PYG{+w}{  }\PYG{p+pIndicator}{\PYGZhy{}}\PYG{+w}{ }\PYG{l+lScalar+lScalarPlain}{rasterio}
\PYG{+w}{  }\PYG{p+pIndicator}{\PYGZhy{}}\PYG{+w}{ }\PYG{l+lScalar+lScalarPlain}{requests}
\PYG{+w}{  }\PYG{p+pIndicator}{\PYGZhy{}}\PYG{+w}{ }\PYG{l+lScalar+lScalarPlain}{rioxarray}
\PYG{+w}{  }\PYG{p+pIndicator}{\PYGZhy{}}\PYG{+w}{ }\PYG{l+lScalar+lScalarPlain}{scipy}
\PYG{+w}{  }\PYG{p+pIndicator}{\PYGZhy{}}\PYG{+w}{ }\PYG{l+lScalar+lScalarPlain}{seaborn}
\PYG{+w}{  }\PYG{p+pIndicator}{\PYGZhy{}}\PYG{+w}{ }\PYG{l+lScalar+lScalarPlain}{shapely}
\PYG{+w}{  }\PYG{p+pIndicator}{\PYGZhy{}}\PYG{+w}{ }\PYG{l+lScalar+lScalarPlain}{scikit\PYGZhy{}image}\PYG{+w}{  }\PYG{c+c1}{\PYGZsh{} skimage}
\PYG{+w}{  }\PYG{p+pIndicator}{\PYGZhy{}}\PYG{+w}{ }\PYG{l+lScalar+lScalarPlain}{scikit\PYGZhy{}learn}\PYG{+w}{  }\PYG{c+c1}{\PYGZsh{} sklearn}
\PYG{+w}{  }\PYG{p+pIndicator}{\PYGZhy{}}\PYG{+w}{ }\PYG{l+lScalar+lScalarPlain}{statsmodels}
\PYG{+w}{  }\PYG{p+pIndicator}{\PYGZhy{}}\PYG{+w}{ }\PYG{l+lScalar+lScalarPlain}{tensorflow}
\PYG{+w}{  }\PYG{p+pIndicator}{\PYGZhy{}}\PYG{+w}{ }\PYG{l+lScalar+lScalarPlain}{tqdm}
\PYG{+w}{  }\PYG{p+pIndicator}{\PYGZhy{}}\PYG{+w}{ }\PYG{l+lScalar+lScalarPlain}{xmltodict}
\end{sphinxVerbatim}
\begin{enumerate}
\sphinxsetlistlabels{\arabic}{enumi}{enumii}{}{.}%
\setcounter{enumi}{2}
\item {} 
\sphinxAtStartPar
Download the Python package requirement file: \DUrole{xref,download,myst}{pip\_req.txt}.

\item {} 
\sphinxAtStartPar
Install required Python packages with the command:

\begin{sphinxVerbatim}[commandchars=\\\{\}]
\PYG{n}{pip} \PYG{n}{install} \PYG{o}{\PYGZhy{}}\PYG{n}{r} \PYG{n}{pip\PYGZus{}req}\PYG{o}{.}\PYG{n}{txt}
\end{sphinxVerbatim}

\item {} 
\sphinxAtStartPar
Install Akhdefo using the following command:

\begin{sphinxVerbatim}[commandchars=\\\{\}]
\PYG{n}{pip} \PYG{n}{install} \PYG{n}{akhdefo}\PYG{o}{\PYGZhy{}}\PYG{n}{functions}
\end{sphinxVerbatim}

\end{enumerate}

\sphinxstepscope


\chapter{Stawamus Chief Rockfall Monitoring Example}
\label{\detokenize{Stawamus_Chief:stawamus-chief-rockfall-monitoring-example}}\label{\detokenize{Stawamus_Chief::doc}}

\section{Example September 20th, 2021 RockFall Event}
\label{\detokenize{Stawamus_Chief:example-september-20th-2021-rockfall-event}}

\subsection{Introduction}
\label{\detokenize{Stawamus_Chief:introduction}}
\sphinxAtStartPar
This document presents a demonstration project utilizing the \sphinxstyleemphasis{Akhdefo} software for real\sphinxhyphen{}time monitoring of rockfalls. The project focuses on the application of optical flow processing using live camera feeds.


\subsection{Real\sphinxhyphen{}Time Monitoring and Analysis}
\label{\detokenize{Stawamus_Chief:real-time-monitoring-and-analysis}}
\sphinxAtStartPar
The core of this project lies in its capability to analyze and visualize displacement\sphinxhyphen{}time curves before and during rockfall events. Such data is invaluable in training neural network models aimed at enhancing early warning systems for landslides and rockfall hazards.


\subsection{Live Feed Integration}
\label{\detokenize{Stawamus_Chief:live-feed-integration}}
\sphinxAtStartPar
The following embedded frame provides direct access to our live monitoring interface:




\subsection{Visual Documentation}
\label{\detokenize{Stawamus_Chief:visual-documentation}}
\sphinxAtStartPar
Below is an image capturing a significant moment during the monitoring process:

\sphinxAtStartPar
\sphinxincludegraphics[width=1000\sphinxpxdimen,height=882\sphinxpxdimen]{{shot_squamish}.png}


\chapter{Documentation!}
\label{\detokenize{index:documentation}}
\sphinxstepscope


\section{Documentation}
\label{\detokenize{modules:documentation}}\label{\detokenize{modules::doc}}
\sphinxstepscope


\subsection{akhdefo\_functions package}
\label{\detokenize{akhdefo_functions:akhdefo-functions-package}}\label{\detokenize{akhdefo_functions::doc}}

\subsubsection{Submodules}
\label{\detokenize{akhdefo_functions:submodules}}

\subsubsection{akhdefo\_functions.AkhdefoPlot module}
\label{\detokenize{akhdefo_functions:module-akhdefo_functions.AkhdefoPlot}}\label{\detokenize{akhdefo_functions:akhdefo-functions-akhdefoplot-module}}\index{module@\spxentry{module}!akhdefo\_functions.AkhdefoPlot@\spxentry{akhdefo\_functions.AkhdefoPlot}}\index{akhdefo\_functions.AkhdefoPlot@\spxentry{akhdefo\_functions.AkhdefoPlot}!module@\spxentry{module}}\index{MeanProducts\_plot\_ts() (in module akhdefo\_functions.AkhdefoPlot)@\spxentry{MeanProducts\_plot\_ts()}\spxextra{in module akhdefo\_functions.AkhdefoPlot}}

\begin{fulllineitems}
\phantomsection\label{\detokenize{akhdefo_functions:akhdefo_functions.AkhdefoPlot.MeanProducts_plot_ts}}
\pysigstartsignatures
\pysiglinewithargsret{\sphinxcode{\sphinxupquote{akhdefo\_functions.AkhdefoPlot.}}\sphinxbfcode{\sphinxupquote{MeanProducts\_plot\_ts}}}{\sphinxparam{\DUrole{n}{path\_to\_shapefile}\DUrole{o}{=}\DUrole{default_value}{\textquotesingle{}\textquotesingle{}}}\sphinxparamcomma \sphinxparam{\DUrole{n}{dem\_path}\DUrole{o}{=}\DUrole{default_value}{\textquotesingle{}\textquotesingle{}}}\sphinxparamcomma \sphinxparam{\DUrole{n}{out\_folder}\DUrole{o}{=}\DUrole{default_value}{\textquotesingle{}Figs\_analysis\textquotesingle{}}}\sphinxparamcomma \sphinxparam{\DUrole{n}{color\_field}\DUrole{o}{=}\DUrole{default_value}{\textquotesingle{}\textquotesingle{}}}\sphinxparamcomma \sphinxparam{\DUrole{n}{Set\_fig\_MinMax}\DUrole{o}{=}\DUrole{default_value}{False}}\sphinxparamcomma \sphinxparam{\DUrole{n}{MinMaxRange}\DUrole{o}{=}\DUrole{default_value}{{[}\sphinxhyphen{}100, 100{]}}}\sphinxparamcomma \sphinxparam{\DUrole{n}{opacity}\DUrole{o}{=}\DUrole{default_value}{0.5}}\sphinxparamcomma \sphinxparam{\DUrole{n}{cmap}\DUrole{o}{=}\DUrole{default_value}{\textquotesingle{}jet\textquotesingle{}}}\sphinxparamcomma \sphinxparam{\DUrole{n}{point\_size}\DUrole{o}{=}\DUrole{default_value}{1}}\sphinxparamcomma \sphinxparam{\DUrole{n}{cbar\_label}\DUrole{o}{=}\DUrole{default_value}{\textquotesingle{}mm/year\textquotesingle{}}}\sphinxparamcomma \sphinxparam{\DUrole{n}{batch\_plot}\DUrole{o}{=}\DUrole{default_value}{False}}\sphinxparamcomma \sphinxparam{\DUrole{n}{dates\_list}\DUrole{o}{=}\DUrole{default_value}{\textquotesingle{}\textquotesingle{}}}}{}
\pysigstopsignatures
\sphinxAtStartPar
This program used to plot shapefile data


\paragraph{Parameters}
\label{\detokenize{akhdefo_functions:parameters}}
\sphinxAtStartPar
path\_to\_shapefile : str

\sphinxAtStartPar
dem\_path: str

\sphinxAtStartPar
out\_folder: str
\begin{description}
\sphinxlineitem{color\_field: str}
\sphinxAtStartPar
geopandas column name

\end{description}

\sphinxAtStartPar
Set\_fig\_MinMax: bool

\sphinxAtStartPar
MinMaxRange: list

\sphinxAtStartPar
opacity: float

\sphinxAtStartPar
cmap: str

\sphinxAtStartPar
point\_size: str
\begin{description}
\sphinxlineitem{cbar\_label: str}
\sphinxAtStartPar
“mm/year” or “degrees”, etc.. based on unit of the data column name in the color\_field

\end{description}


\paragraph{Returns}
\label{\detokenize{akhdefo_functions:returns}}
\sphinxAtStartPar
Figure

\end{fulllineitems}

\index{akhdefo\_ts\_plot() (in module akhdefo\_functions.AkhdefoPlot)@\spxentry{akhdefo\_ts\_plot()}\spxextra{in module akhdefo\_functions.AkhdefoPlot}}

\begin{fulllineitems}
\phantomsection\label{\detokenize{akhdefo_functions:akhdefo_functions.AkhdefoPlot.akhdefo_ts_plot}}
\pysigstartsignatures
\pysiglinewithargsret{\sphinxcode{\sphinxupquote{akhdefo\_functions.AkhdefoPlot.}}\sphinxbfcode{\sphinxupquote{akhdefo\_ts\_plot}}}{\sphinxparam{\DUrole{n}{path\_to\_shapefile}\DUrole{o}{=}\DUrole{default_value}{\textquotesingle{}\textquotesingle{}}}\sphinxparamcomma \sphinxparam{\DUrole{n}{dem\_path}\DUrole{o}{=}\DUrole{default_value}{\textquotesingle{}\textquotesingle{}}}\sphinxparamcomma \sphinxparam{\DUrole{n}{point\_size}\DUrole{o}{=}\DUrole{default_value}{1.0}}\sphinxparamcomma \sphinxparam{\DUrole{n}{opacity}\DUrole{o}{=}\DUrole{default_value}{0.75}}\sphinxparamcomma \sphinxparam{\DUrole{n}{cmap}\DUrole{o}{=}\DUrole{default_value}{\textquotesingle{}turbo\textquotesingle{}}}\sphinxparamcomma \sphinxparam{\DUrole{n}{Set\_fig\_MinMax}\DUrole{o}{=}\DUrole{default_value}{True}}\sphinxparamcomma \sphinxparam{\DUrole{n}{MinMaxRange}\DUrole{o}{=}\DUrole{default_value}{{[}\sphinxhyphen{}50, 50{]}}}\sphinxparamcomma \sphinxparam{\DUrole{n}{color\_field}\DUrole{o}{=}\DUrole{default_value}{\textquotesingle{}VEL\textquotesingle{}}}\sphinxparamcomma \sphinxparam{\DUrole{n}{user\_data\_points}\DUrole{o}{=}\DUrole{default_value}{\textquotesingle{}\textquotesingle{}}}\sphinxparamcomma \sphinxparam{\DUrole{n}{path\_saveData\_points}\DUrole{o}{=}\DUrole{default_value}{\textquotesingle{}\textquotesingle{}}}\sphinxparamcomma \sphinxparam{\DUrole{n}{save\_plot}\DUrole{o}{=}\DUrole{default_value}{False}}\sphinxparamcomma \sphinxparam{\DUrole{n}{Fig\_outputDir}\DUrole{o}{=}\DUrole{default_value}{\textquotesingle{}\textquotesingle{}}}\sphinxparamcomma \sphinxparam{\DUrole{n}{VEL\_Scale}\DUrole{o}{=}\DUrole{default_value}{\textquotesingle{}year\textquotesingle{}}}\sphinxparamcomma \sphinxparam{\DUrole{n}{filename\_dates}\DUrole{o}{=}\DUrole{default_value}{\textquotesingle{}\textquotesingle{}}}}{}
\pysigstopsignatures
\sphinxAtStartPar
This program used for analysis time\sphinxhyphen{}series velocity profiles


\paragraph{Parameters}
\label{\detokenize{akhdefo_functions:id1}}\begin{description}
\sphinxlineitem{user\_data\_points: str}
\sphinxAtStartPar
provide path to csv. file contains x and y coordinate for points of interest
you can generate this file by providing path to path\_saveData\_points (POI.csv).
This is useful to save mouse click positions to repeat the plots for different datasets for example if you plot several TS profiles for
EW velocity product, you can recreate TS for the same exact position by saving POI.csv with path\_saveData\_points and then use that as input for the another
plot such as NS velocity product via setting user\_datapoints=”POI.csv”

\sphinxlineitem{path\_to\_shapefile: str }
\sphinxAtStartPar
type path to timeseries shapefile in stack\_data/TS folder

\sphinxlineitem{dem\_path: str}
\sphinxAtStartPar
path to dem raster in geotif fromat

\sphinxlineitem{point\_size: float}
\sphinxAtStartPar
size of the sactter plot points

\sphinxlineitem{opacity: float }
\sphinxAtStartPar
transparency of the scater overlay

\sphinxlineitem{cmap: str}
\sphinxAtStartPar
Matplotlib colormap options example “RdYlBu\_r, jet, turbo, hsv, etc…”

\sphinxlineitem{Set\_fig\_MinMax: bool}
\sphinxAtStartPar
True or False

\sphinxlineitem{MinMaxRange: list}
\sphinxAtStartPar
{[}\sphinxhyphen{}50,50{]}  Normalize plot colormap range if Set\_fig\_MinMax=True

\sphinxlineitem{color\_field: str }
\sphinxAtStartPar
‘VEL’ ,”VEL\_2D”, ‘VEL\_N’, ‘VEL\_E’, ‘VELDir\_MEA’

\sphinxlineitem{path\_saveData\_points: str}
\sphinxAtStartPar
optional, provide directory path if you want to save profile data.
the data will be saved under POI.csv file

\sphinxlineitem{save\_plot: bool}
\sphinxAtStartPar
True or False

\sphinxlineitem{Fig\_outputDir: str}
\sphinxAtStartPar
if save\_plot=True then
you save your profile plots in interactive html file and jpg image

\sphinxlineitem{VEL\_Scale: str}
\sphinxAtStartPar
‘year’ or ‘month’ projects the velocity into provided time\sphinxhyphen{}scale

\sphinxlineitem{filename\_dates: str}
\sphinxAtStartPar
provide path to Names.txt file, this file generated at stack\_prep step

\end{description}


\paragraph{Returns}
\label{\detokenize{akhdefo_functions:id2}}
\sphinxAtStartPar
Interactive Figures

\end{fulllineitems}

\index{akhdefo\_viewer() (in module akhdefo\_functions.AkhdefoPlot)@\spxentry{akhdefo\_viewer()}\spxextra{in module akhdefo\_functions.AkhdefoPlot}}

\begin{fulllineitems}
\phantomsection\label{\detokenize{akhdefo_functions:akhdefo_functions.AkhdefoPlot.akhdefo_viewer}}
\pysigstartsignatures
\pysiglinewithargsret{\sphinxcode{\sphinxupquote{akhdefo\_functions.AkhdefoPlot.}}\sphinxbfcode{\sphinxupquote{akhdefo\_viewer}}}{\sphinxparam{\DUrole{n}{path\_to\_dem\_file}}\sphinxparamcomma \sphinxparam{\DUrole{n}{raster\_file}}\sphinxparamcomma \sphinxparam{\DUrole{n}{output\_folder}}\sphinxparamcomma \sphinxparam{\DUrole{n}{title}\DUrole{o}{=}\DUrole{default_value}{\textquotesingle{}\textquotesingle{}}}\sphinxparamcomma \sphinxparam{\DUrole{n}{pixel\_resolution\_meters}\DUrole{o}{=}\DUrole{default_value}{3.125}}\sphinxparamcomma \sphinxparam{\DUrole{n}{output\_file\_name}\DUrole{o}{=}\DUrole{default_value}{\textquotesingle{}\textquotesingle{}}}\sphinxparamcomma \sphinxparam{\DUrole{n}{alpha}\DUrole{o}{=}\DUrole{default_value}{0.5}}\sphinxparamcomma \sphinxparam{\DUrole{n}{unit\_conversion}\DUrole{o}{=}\DUrole{default_value}{None}}\sphinxparamcomma \sphinxparam{\DUrole{n}{no\_data\_mask}\DUrole{o}{=}\DUrole{default_value}{False}}\sphinxparamcomma \sphinxparam{\DUrole{n}{colormap}\DUrole{o}{=}\DUrole{default_value}{\textquotesingle{}jet\textquotesingle{}}}\sphinxparamcomma \sphinxparam{\DUrole{n}{min\_value}\DUrole{o}{=}\DUrole{default_value}{None}}\sphinxparamcomma \sphinxparam{\DUrole{n}{max\_value}\DUrole{o}{=}\DUrole{default_value}{None}}\sphinxparamcomma \sphinxparam{\DUrole{n}{normalize}\DUrole{o}{=}\DUrole{default_value}{False}}\sphinxparamcomma \sphinxparam{\DUrole{n}{colorbar\_label}\DUrole{o}{=}\DUrole{default_value}{None}}\sphinxparamcomma \sphinxparam{\DUrole{n}{show\_figure}\DUrole{o}{=}\DUrole{default_value}{True}}\sphinxparamcomma \sphinxparam{\DUrole{n}{aspect\_raster}\DUrole{o}{=}\DUrole{default_value}{None}}\sphinxparamcomma \sphinxparam{\DUrole{n}{cmap\_aspect}\DUrole{o}{=}\DUrole{default_value}{None}}\sphinxparamcomma \sphinxparam{\DUrole{n}{step}\DUrole{o}{=}\DUrole{default_value}{10}}}{}
\pysigstopsignatures
\sphinxAtStartPar
Overlays a raster file on a DEM hillshade and saves the plot as a PNG image.

\sphinxAtStartPar
Parameters:
path\_to\_dem\_file (str): Path to the DEM file.
raster\_file (str): Path to the raster file.
output\_folder (str): Path to the folder where the output image will be saved.
title (str, optional): Title of the plot. Defaults to the raster file’s basename.
pixel\_resolution\_meters (float, optional): Pixel resolution of the raster in meters. Default is None get resolution from raster input.
output\_file\_name (str, optional): Name of the output PNG image. Defaults to the raster file’s basename.
alpha (float, optional): Alpha value for the raster overlay. Default is 0.5.
unit\_conversion (str, optional): Unit conversion factor for the raster values. For example, ‘100cm’ for meters to centimeters conversion.
no\_data\_mask (bool, optional): If True, masks pixels with a value of 0 in the raster. Default is False.
colormap (str, optional): Colormap to use for the raster. Default is ‘jet’.
min\_value (float, optional): Minimum value for normalization. Uses raster’s minimum if None.
max\_value (float, optional): Maximum value for normalization. Uses raster’s maximum if None.
normalize (bool, optional): If True, normalizes the raster values. Default is False.
colorbar\_label (str, optional): Label for the colorbar. 
show\_figure (bool, optional): Whether to display the figure. Default is True.
aspect\_raster (str, optional): whetehr to plot displacement vector. Dedulat is None 
cmap\_aspect (str, optional): colormap to sue for the vector arrows
step (int, optional): density of the aspect vector arraows. Defulat is 10 pixel unit draw 1 arrow

\sphinxAtStartPar
Returns:
None

\end{fulllineitems}

\index{plot\_stackNetwork() (in module akhdefo\_functions.AkhdefoPlot)@\spxentry{plot\_stackNetwork()}\spxextra{in module akhdefo\_functions.AkhdefoPlot}}

\begin{fulllineitems}
\phantomsection\label{\detokenize{akhdefo_functions:akhdefo_functions.AkhdefoPlot.plot_stackNetwork}}
\pysigstartsignatures
\pysiglinewithargsret{\sphinxcode{\sphinxupquote{akhdefo\_functions.AkhdefoPlot.}}\sphinxbfcode{\sphinxupquote{plot\_stackNetwork}}}{\sphinxparam{\DUrole{n}{src\_folder}\DUrole{o}{=}\DUrole{default_value}{\textquotesingle{}\textquotesingle{}}}\sphinxparamcomma \sphinxparam{\DUrole{n}{output\_folder}\DUrole{o}{=}\DUrole{default_value}{\textquotesingle{}\textquotesingle{}}}\sphinxparamcomma \sphinxparam{\DUrole{n}{cmap}\DUrole{o}{=}\DUrole{default_value}{\textquotesingle{}tab20\textquotesingle{}}}\sphinxparamcomma \sphinxparam{\DUrole{n}{date\_plot\_interval}\DUrole{o}{=}\DUrole{default_value}{(5, 30)}}\sphinxparamcomma \sphinxparam{\DUrole{n}{marker\_size}\DUrole{o}{=}\DUrole{default_value}{15}}}{}
\pysigstopsignatures
\sphinxAtStartPar
This Program plots temporal network of triplets to be stacked for calculating 
Annual Mean Velocity from stacked optical images.


\paragraph{Parameters}
\label{\detokenize{akhdefo_functions:id3}}\begin{description}
\sphinxlineitem{src\_folder: str}
\sphinxAtStartPar
path to georeferenced\_folder

\sphinxlineitem{output\_folder: str}
\sphinxAtStartPar
path to output folder to save output Figure plot

\sphinxlineitem{cmap: str}
\sphinxAtStartPar
colormap for the plot default is “tab20”

\sphinxlineitem{date\_plot\_interval: list}
\sphinxAtStartPar
minumum and maximum plot x axis interval dates for the plot

\sphinxlineitem{marker\_size: float}
\sphinxAtStartPar
size of plotted points default is 15

\end{description}


\paragraph{Returns}
\label{\detokenize{akhdefo_functions:id4}}
\sphinxAtStartPar
Figure

\end{fulllineitems}



\subsubsection{akhdefo\_functions.Akhdefo\_Coreg module}
\label{\detokenize{akhdefo_functions:module-akhdefo_functions.Akhdefo_Coreg}}\label{\detokenize{akhdefo_functions:akhdefo-functions-akhdefo-coreg-module}}\index{module@\spxentry{module}!akhdefo\_functions.Akhdefo\_Coreg@\spxentry{akhdefo\_functions.Akhdefo\_Coreg}}\index{akhdefo\_functions.Akhdefo\_Coreg@\spxentry{akhdefo\_functions.Akhdefo\_Coreg}!module@\spxentry{module}}\index{Coregistration() (in module akhdefo\_functions.Akhdefo\_Coreg)@\spxentry{Coregistration()}\spxextra{in module akhdefo\_functions.Akhdefo\_Coreg}}

\begin{fulllineitems}
\phantomsection\label{\detokenize{akhdefo_functions:akhdefo_functions.Akhdefo_Coreg.Coregistration}}
\pysigstartsignatures
\pysiglinewithargsret{\sphinxcode{\sphinxupquote{akhdefo\_functions.Akhdefo\_Coreg.}}\sphinxbfcode{\sphinxupquote{Coregistration}}}{\sphinxparam{\DUrole{n}{input\_Folder}\DUrole{o}{=}\DUrole{default_value}{\textquotesingle{}\textquotesingle{}}}\sphinxparamcomma \sphinxparam{\DUrole{n}{output\_folder}\DUrole{o}{=}\DUrole{default_value}{\textquotesingle{}\textquotesingle{}}}\sphinxparamcomma \sphinxparam{\DUrole{n}{grid\_res}\DUrole{o}{=}\DUrole{default_value}{20}}\sphinxparamcomma \sphinxparam{\DUrole{n}{min\_reliability}\DUrole{o}{=}\DUrole{default_value}{60}}\sphinxparamcomma \sphinxparam{\DUrole{n}{window\_size}\DUrole{o}{=}\DUrole{default_value}{(64, 64)}}\sphinxparamcomma \sphinxparam{\DUrole{n}{path\_figures}\DUrole{o}{=}\DUrole{default_value}{\textquotesingle{}\textquotesingle{}}}\sphinxparamcomma \sphinxparam{\DUrole{n}{showFig}\DUrole{o}{=}\DUrole{default_value}{False}}\sphinxparamcomma \sphinxparam{\DUrole{n}{no\_data}\DUrole{o}{=}\DUrole{default_value}{{[}0, 0{]}}}\sphinxparamcomma \sphinxparam{\DUrole{n}{single\_ref\_path}\DUrole{o}{=}\DUrole{default_value}{\textquotesingle{}\textquotesingle{}}}\sphinxparamcomma \sphinxparam{\DUrole{n}{step\_size}\DUrole{o}{=}\DUrole{default_value}{3}}}{}
\pysigstopsignatures
\sphinxAtStartPar
Coregister multiple rasters using both the structural similarity index and the feature matching technique.
This function is based on the AROSICS Python library.


\paragraph{Parameters}
\label{\detokenize{akhdefo_functions:id5}}\begin{description}
\sphinxlineitem{input\_Folder}{[}str{]}
\sphinxAtStartPar
Path to input raster folders.

\sphinxlineitem{output\_folder}{[}str{]}
\sphinxAtStartPar
Directory to store coregistered and georeferenced raster in GeoTIFF format.

\sphinxlineitem{grid\_res}{[}int{]}
\sphinxAtStartPar
Grid resolution for coregistration.

\sphinxlineitem{min\_reliability}{[}int{]}
\sphinxAtStartPar
Structural similarity index threshold to differentiate deformation from raster shift (min=20, max=100).

\sphinxlineitem{window\_size}{[}tuple{]}
\sphinxAtStartPar
Window size for pixel search in coregistration.

\sphinxlineitem{path\_figures}{[}str{]}
\sphinxAtStartPar
Directory to store figures depicting displaced pixels in raster coordinate system units.

\sphinxlineitem{showFig}{[}bool{]}
\sphinxAtStartPar
If True, display results. If False, do not display results.

\sphinxlineitem{no\_data}{[}list{]}
\sphinxAtStartPar
No data values to be ignored for both reference and target image.

\sphinxlineitem{single\_ref\_path}{[}str{]}
\sphinxAtStartPar
Provide path to raster if interested in coregistering all rasters to a single reference. 
If left empty, the function will use subsequent rasters as reference.

\sphinxlineitem{step\_size}{[}int, default=3{]}
\sphinxAtStartPar
Determines how many images each image serves as a reference for. A value of 3 means every image 
acts as a reference for the next two images, and a value of 2 means every image acts as a reference for the next one.

\end{description}


\paragraph{Returns}
\label{\detokenize{akhdefo_functions:id6}}\begin{description}
\sphinxlineitem{None}
\sphinxAtStartPar
The function saves the coregistered rasters and corresponding figures to specified directories.

\end{description}

\end{fulllineitems}

\index{raster\_alignment() (in module akhdefo\_functions.Akhdefo\_Coreg)@\spxentry{raster\_alignment()}\spxextra{in module akhdefo\_functions.Akhdefo\_Coreg}}

\begin{fulllineitems}
\phantomsection\label{\detokenize{akhdefo_functions:akhdefo_functions.Akhdefo_Coreg.raster_alignment}}
\pysigstartsignatures
\pysiglinewithargsret{\sphinxcode{\sphinxupquote{akhdefo\_functions.Akhdefo\_Coreg.}}\sphinxbfcode{\sphinxupquote{raster\_alignment}}}{\sphinxparam{\DUrole{n}{src\_directory}}\sphinxparamcomma \sphinxparam{\DUrole{n}{ref\_filename}}\sphinxparamcomma \sphinxparam{\DUrole{n}{delete\_files}\DUrole{o}{=}\DUrole{default_value}{False}}}{}
\pysigstopsignatures
\sphinxAtStartPar
Aligns raster images in a given source directory to a reference image.

\sphinxAtStartPar
Args:
src\_directory (str): Path to the source directory containing images to align.
ref\_filename (str): Path to the reference image.
delete\_files (bool): If True, deletes the temporary directory created for alignment process. Defaults to False.

\sphinxAtStartPar
Returns:
str: Path to the directory containing all registered images.

\end{fulllineitems}



\subsubsection{akhdefo\_functions.Akhdefo\_GOI module}
\label{\detokenize{akhdefo_functions:module-akhdefo_functions.Akhdefo_GOI}}\label{\detokenize{akhdefo_functions:akhdefo-functions-akhdefo-goi-module}}\index{module@\spxentry{module}!akhdefo\_functions.Akhdefo\_GOI@\spxentry{akhdefo\_functions.Akhdefo\_GOI}}\index{akhdefo\_functions.Akhdefo\_GOI@\spxentry{akhdefo\_functions.Akhdefo\_GOI}!module@\spxentry{module}}\index{Optical\_flow\_akhdefo() (in module akhdefo\_functions.Akhdefo\_GOI)@\spxentry{Optical\_flow\_akhdefo()}\spxextra{in module akhdefo\_functions.Akhdefo\_GOI}}

\begin{fulllineitems}
\phantomsection\label{\detokenize{akhdefo_functions:akhdefo_functions.Akhdefo_GOI.Optical_flow_akhdefo}}
\pysigstartsignatures
\pysiglinewithargsret{\sphinxcode{\sphinxupquote{akhdefo\_functions.Akhdefo\_GOI.}}\sphinxbfcode{\sphinxupquote{Optical\_flow\_akhdefo}}}{\sphinxparam{\DUrole{n}{input\_dir}\DUrole{o}{=}\DUrole{default_value}{\textquotesingle{}\textquotesingle{}}}\sphinxparamcomma \sphinxparam{\DUrole{n}{output\_dir}\DUrole{o}{=}\DUrole{default_value}{\textquotesingle{}\textquotesingle{}}}\sphinxparamcomma \sphinxparam{\DUrole{n}{AOI}\DUrole{o}{=}\DUrole{default_value}{None}}\sphinxparamcomma \sphinxparam{\DUrole{n}{zscore\_threshold}\DUrole{o}{=}\DUrole{default_value}{2}}\sphinxparamcomma \sphinxparam{\DUrole{n}{ssim\_thresh}\DUrole{o}{=}\DUrole{default_value}{0.75}}\sphinxparamcomma \sphinxparam{\DUrole{n}{image\_resolution}\DUrole{o}{=}\DUrole{default_value}{\textquotesingle{}3125mm\textquotesingle{}}}\sphinxparamcomma \sphinxparam{\DUrole{n}{interpolate}\DUrole{o}{=}\DUrole{default_value}{None}}\sphinxparamcomma \sphinxparam{\DUrole{n}{show\_figure}\DUrole{o}{=}\DUrole{default_value}{False}}\sphinxparamcomma \sphinxparam{\DUrole{n}{point\_size}\DUrole{o}{=}\DUrole{default_value}{2}}\sphinxparamcomma \sphinxparam{\DUrole{n}{dem\_path}\DUrole{o}{=}\DUrole{default_value}{\textquotesingle{}\textquotesingle{}}}\sphinxparamcomma \sphinxparam{\DUrole{n}{smoothing\_kernel\_size}\DUrole{o}{=}\DUrole{default_value}{11}}\sphinxparamcomma \sphinxparam{\DUrole{n}{Vegetation\_mask}\DUrole{o}{=}\DUrole{default_value}{None}}\sphinxparamcomma \sphinxparam{\DUrole{n}{VEL\_scale}\DUrole{o}{=}\DUrole{default_value}{\textquotesingle{}year\textquotesingle{}}}\sphinxparamcomma \sphinxparam{\DUrole{n}{VEL\_Mode}\DUrole{o}{=}\DUrole{default_value}{\textquotesingle{}linear\textquotesingle{}}}\sphinxparamcomma \sphinxparam{\DUrole{n}{good\_match\_option}\DUrole{o}{=}\DUrole{default_value}{0.75}}\sphinxparamcomma \sphinxparam{\DUrole{n}{hillshade\_option}\DUrole{o}{=}\DUrole{default_value}{True}}\sphinxparamcomma \sphinxparam{\DUrole{n}{shapefile\_output}\DUrole{o}{=}\DUrole{default_value}{False}}\sphinxparamcomma \sphinxparam{\DUrole{n}{max\_triplet\_interval}\DUrole{o}{=}\DUrole{default_value}{24}}\sphinxparamcomma \sphinxparam{\DUrole{n}{pixel\_size}\DUrole{o}{=}\DUrole{default_value}{20}}\sphinxparamcomma \sphinxparam{\DUrole{n}{num\_chunks}\DUrole{o}{=}\DUrole{default_value}{10}}\sphinxparamcomma \sphinxparam{\DUrole{n}{overlap\_percentage}\DUrole{o}{=}\DUrole{default_value}{0}}\sphinxparamcomma \sphinxparam{\DUrole{n}{pyr\_scale}\DUrole{o}{=}\DUrole{default_value}{0.5}}\sphinxparamcomma \sphinxparam{\DUrole{n}{levels}\DUrole{o}{=}\DUrole{default_value}{15}}\sphinxparamcomma \sphinxparam{\DUrole{n}{winsize}\DUrole{o}{=}\DUrole{default_value}{32}}\sphinxparamcomma \sphinxparam{\DUrole{n}{iterations}\DUrole{o}{=}\DUrole{default_value}{7}}\sphinxparamcomma \sphinxparam{\DUrole{n}{poly\_n}\DUrole{o}{=}\DUrole{default_value}{7}}\sphinxparamcomma \sphinxparam{\DUrole{n}{poly\_sigma}\DUrole{o}{=}\DUrole{default_value}{1.5}}\sphinxparamcomma \sphinxparam{\DUrole{n}{flags}\DUrole{o}{=}\DUrole{default_value}{1}}\sphinxparamcomma \sphinxparam{\DUrole{n}{master\_reference}\DUrole{o}{=}\DUrole{default_value}{\textquotesingle{}single\textquotesingle{}}}\sphinxparamcomma \sphinxparam{\DUrole{n}{selection\_Mode}\DUrole{o}{=}\DUrole{default_value}{\textquotesingle{}triplet\textquotesingle{}}}\sphinxparamcomma \sphinxparam{\DUrole{n}{start\_date}\DUrole{o}{=}\DUrole{default_value}{None}}\sphinxparamcomma \sphinxparam{\DUrole{n}{end\_date}\DUrole{o}{=}\DUrole{default_value}{None}}\sphinxparamcomma \sphinxparam{\DUrole{n}{krig\_method}\DUrole{o}{=}\DUrole{default_value}{\textquotesingle{}ordinary\textquotesingle{}}}\sphinxparamcomma \sphinxparam{\DUrole{n}{spatial\_ref}\DUrole{o}{=}\DUrole{default_value}{False}}}{}
\pysigstopsignatures
\sphinxAtStartPar
Performs feature matching and velocity/displacement calculations across a series of images.


\paragraph{Parameters}
\label{\detokenize{akhdefo_functions:id7}}\begin{description}
\sphinxlineitem{input\_dir: str}
\sphinxAtStartPar
Path to the directory where the input images are stored.

\sphinxlineitem{output\_dir}{[}str{]}
\sphinxAtStartPar
Path to the directory where the output files will be saved.

\sphinxlineitem{AOI}{[}str{]}
\sphinxAtStartPar
The shapefile that represents the Area of Interest (AOI).

\sphinxlineitem{zscore\_threshold}{[}float{]}
\sphinxAtStartPar
The threshold value used to filter matches based on their Z\sphinxhyphen{}score.

\sphinxlineitem{image\_resolution}{[}str{]}
\sphinxAtStartPar
The resolution of the images specified per pixel. This can be expressed in various units 
like ‘3125mm’, ‘3.125m’ or ‘3.125meter’.

\sphinxlineitem{VEL\_scale: (str, optional)}
\sphinxAtStartPar
options year, month, None , default year

\sphinxlineitem{VEL\_Mode: str}
\sphinxAtStartPar
Options linear or mean , default linear

\sphinxlineitem{good\_match\_option: float}
\sphinxAtStartPar
ratio test as per Lowe’s paper default 0.75

\sphinxlineitem{shapefile\_output: bool}
\sphinxAtStartPar
True to export timeseries as deformation products as shapefile, default False

\sphinxlineitem{max\_triplet\_interval: int }
\sphinxAtStartPar
Maximum interval days between images allowed to form triplets

\sphinxlineitem{master\_reference: str }
\sphinxAtStartPar
single, multiple, None

\sphinxlineitem{Vegetation\_mask: (str, optional)}
\sphinxAtStartPar
Path to a raster file that represents a vegetation mask. Pixels in the input image
that correspond to non\sphinxhyphen{}vegetation in the mask will be set to one.

\sphinxlineitem{pyr\_scale: float}
\sphinxAtStartPar
parameter, specifying the image scale (\textless{}1) to build pyramids for each image; pyr\_scale=0.5 means a classical pyramid, where each next layer is twice smaller than the previous one.

\sphinxlineitem{levels: int}
\sphinxAtStartPar
number of pyramid layers including the initial image; levels=1 means that no extra layers are created and only the original images are used.

\sphinxlineitem{winsize: int}
\sphinxAtStartPar
averaging window size; larger values increase the algorithm robustness to image noise and give more chances for fast motion detection, but yield more blurred motion field.

\sphinxlineitem{iterations: int}
\sphinxAtStartPar
number of iterations the algorithm does at each pyramid level.

\sphinxlineitem{poly\_n: int}
\sphinxAtStartPar
size of the pixel neighborhood used to find polynomial expansion in each pixel; 
larger values mean that the image will be approximated with smoother surfaces, 
yielding more robust algorithm and more blurred motion field, typically poly\_n =5 or 7.

\sphinxlineitem{poly\_sigma: float}
\sphinxAtStartPar
standard deviation of the Gaussian that is used to smooth derivatives used as a basis for the polynomial expansion; 
for poly\_n=5, you can set poly\_sigma=1.1, for poly\_n=7, a good value would be poly\_sigma=1.5.

\sphinxlineitem{flags: 0 or 1}
\sphinxAtStartPar
operation flags that can be a combination of the following:
0 OPTFLOW\_USE\_INITIAL\_FLOW uses the input flow as an initial flow approximation.
1 OPTFLOW\_FARNEBACK\_GAUSSIAN uses the Gaussian winsize×winsize filter instead of a box filter of the same size for optical flow estimation; 
usually, this option gives z more accurate flow than with a box filter, at the cost of lower speed; 
normally, winsize for a Gaussian window should be set to a larger value to achieve the same level of robustness.

\sphinxlineitem{start\_date: str (example 20210203)}
\sphinxAtStartPar
The start date of the image series.

\sphinxlineitem{end\_date: str (example 20210503)}
\sphinxAtStartPar
The end date of the image series.

\sphinxlineitem{krig\_method: str ‘ordinary’ , ‘simple’ , ‘universal’}
\sphinxAtStartPar
selection of kriging interpolation method, the workflow is based on gstools library. default is ‘ordinary’

\end{description}


\paragraph{Returns}
\label{\detokenize{akhdefo_functions:id8}}\begin{description}
\sphinxlineitem{image1}{[}numpy.ndarray{]}
\sphinxAtStartPar
The first image in the series.

\sphinxlineitem{image3}{[}numpy.ndarray{]}
\sphinxAtStartPar
The third image in the series.

\sphinxlineitem{mean\_vel\_list}{[}list{]}
\sphinxAtStartPar
A list of mean velocity arrays, each array corresponding to a pair of images.

\sphinxlineitem{mean\_flowx\_list}{[}list{]}
\sphinxAtStartPar
A list of mean x\sphinxhyphen{}flow arrays, each array corresponding to a pair of images.

\sphinxlineitem{mean\_flowy\_list}{[}list{]}
\sphinxAtStartPar
A list of mean y\sphinxhyphen{}flow arrays, each array corresponding to a pair of images.

\sphinxlineitem{points1\_i}{[}numpy.ndarray{]}
\sphinxAtStartPar
Array of keypoints for the first image in the last pair.

\sphinxlineitem{points2}{[}numpy.ndarray{]}
\sphinxAtStartPar
Array of keypoints for the second image in the last pair.

\end{description}

\end{fulllineitems}

\index{calculate\_slopes\_std\_and\_residuals() (in module akhdefo\_functions.Akhdefo\_GOI)@\spxentry{calculate\_slopes\_std\_and\_residuals()}\spxextra{in module akhdefo\_functions.Akhdefo\_GOI}}

\begin{fulllineitems}
\phantomsection\label{\detokenize{akhdefo_functions:akhdefo_functions.Akhdefo_GOI.calculate_slopes_std_and_residuals}}
\pysigstartsignatures
\pysiglinewithargsret{\sphinxcode{\sphinxupquote{akhdefo\_functions.Akhdefo\_GOI.}}\sphinxbfcode{\sphinxupquote{calculate\_slopes\_std\_and\_residuals}}}{\sphinxparam{\DUrole{n}{geodataframe}}}{}
\pysigstopsignatures
\end{fulllineitems}

\index{correct\_date\_format() (in module akhdefo\_functions.Akhdefo\_GOI)@\spxentry{correct\_date\_format()}\spxextra{in module akhdefo\_functions.Akhdefo\_GOI}}

\begin{fulllineitems}
\phantomsection\label{\detokenize{akhdefo_functions:akhdefo_functions.Akhdefo_GOI.correct_date_format}}
\pysigstartsignatures
\pysiglinewithargsret{\sphinxcode{\sphinxupquote{akhdefo\_functions.Akhdefo\_GOI.}}\sphinxbfcode{\sphinxupquote{correct\_date\_format}}}{\sphinxparam{\DUrole{n}{col}}}{}
\pysigstopsignatures
\end{fulllineitems}

\index{find\_best\_match() (in module akhdefo\_functions.Akhdefo\_GOI)@\spxentry{find\_best\_match()}\spxextra{in module akhdefo\_functions.Akhdefo\_GOI}}

\begin{fulllineitems}
\phantomsection\label{\detokenize{akhdefo_functions:akhdefo_functions.Akhdefo_GOI.find_best_match}}
\pysigstartsignatures
\pysiglinewithargsret{\sphinxcode{\sphinxupquote{akhdefo\_functions.Akhdefo\_GOI.}}\sphinxbfcode{\sphinxupquote{find\_best\_match}}}{\sphinxparam{\DUrole{n}{gdf}}}{}
\pysigstopsignatures
\end{fulllineitems}

\index{interpolate\_xyz() (in module akhdefo\_functions.Akhdefo\_GOI)@\spxentry{interpolate\_xyz()}\spxextra{in module akhdefo\_functions.Akhdefo\_GOI}}

\begin{fulllineitems}
\phantomsection\label{\detokenize{akhdefo_functions:akhdefo_functions.Akhdefo_GOI.interpolate_xyz}}
\pysigstartsignatures
\pysiglinewithargsret{\sphinxcode{\sphinxupquote{akhdefo\_functions.Akhdefo\_GOI.}}\sphinxbfcode{\sphinxupquote{interpolate\_xyz}}}{\sphinxparam{\DUrole{n}{x}}\sphinxparamcomma \sphinxparam{\DUrole{n}{y}}\sphinxparamcomma \sphinxparam{\DUrole{n}{z}}\sphinxparamcomma \sphinxparam{\DUrole{n}{filename}}\sphinxparamcomma \sphinxparam{\DUrole{n}{reference\_raster}}\sphinxparamcomma \sphinxparam{\DUrole{n}{shapefile}\DUrole{o}{=}\DUrole{default_value}{None}}\sphinxparamcomma \sphinxparam{\DUrole{n}{interpolate}\DUrole{o}{=}\DUrole{default_value}{None}}\sphinxparamcomma \sphinxparam{\DUrole{n}{smoothing\_kernel\_size}\DUrole{o}{=}\DUrole{default_value}{None}}\sphinxparamcomma \sphinxparam{\DUrole{n}{mask}\DUrole{o}{=}\DUrole{default_value}{None}}}{}
\pysigstopsignatures
\sphinxAtStartPar
Save XYZ data as a GeoTIFF file using a reference raster for geospatial context.

\sphinxAtStartPar
This function takes X, Y, Z coordinate data and generates a GeoTIFF file. The geospatial context
is derived from a reference raster file. The function supports optional interpolation of Z values,
spatial smoothing, and masking based on a shapefile or a predefined  mask.
\begin{description}
\sphinxlineitem{Parameters:}
\sphinxAtStartPar
x (array\_like): Array of X coordinates.
y (array\_like): Array of Y coordinates.
z (array\_like): Array of Z values corresponding to X and Y coordinates.
filename (str): The base name of the output GeoTIFF file (without file extension).
reference\_raster (str): Path to the reference raster file used for spatial context (CRS, bounds, etc.).
shapefile (str, optional): Path to a shapefile for masking the output raster. Defaults to None.
interpolate (str, optional): Interpolation method to be used (e.g., ‘linear’, ‘nearest’). If None, a nearest\sphinxhyphen{}neighbor approach is applied. Defaults to None.
smoothing\_kernel\_size (int or float, optional): The size of the Gaussian kernel used for smoothing the Z values. Defaults to None.
mask (array\_like, optional): A boolean mask array to mask vegetation areas. Defaults to None.

\sphinxlineitem{Returns:}
\sphinxAtStartPar
numpy.ndarray: The array of interpolated/smoothed Z values, which is also saved as a GeoTIFF file.

\sphinxlineitem{Raises:}
\sphinxAtStartPar
Exception: If an error occurs during the process.

\sphinxAtStartPar
Notes:
\sphinxhyphen{} The function uses \sphinxtitleref{rasterio} for raster operations and \sphinxtitleref{numpy} and \sphinxtitleref{scipy} for data processing.
\sphinxhyphen{} The output GeoTIFF file will have the same spatial extent, resolution, and coordinate reference system (CRS) as the reference raster.
\sphinxhyphen{} If ‘interpolate’ is not None, Z values are interpolated over the grid defined by the reference raster. Out\sphinxhyphen{}of\sphinxhyphen{}range interpolated values are replaced with the mean of valid data points.
\sphinxhyphen{} If ‘smoothing\_kernel\_size’ is provided, a Gaussian smoothing is applied to the Z values.
\sphinxhyphen{} Masking with either a shapefile or a vegetation mask will set the corresponding areas to NaN.
\sphinxhyphen{} The output file is named using the ‘filename’ parameter with ‘.tif’ extension.

\end{description}

\end{fulllineitems}

\index{linear\_VEL() (in module akhdefo\_functions.Akhdefo\_GOI)@\spxentry{linear\_VEL()}\spxextra{in module akhdefo\_functions.Akhdefo\_GOI}}

\begin{fulllineitems}
\phantomsection\label{\detokenize{akhdefo_functions:akhdefo_functions.Akhdefo_GOI.linear_VEL}}
\pysigstartsignatures
\pysiglinewithargsret{\sphinxcode{\sphinxupquote{akhdefo\_functions.Akhdefo\_GOI.}}\sphinxbfcode{\sphinxupquote{linear\_VEL}}}{\sphinxparam{\DUrole{n}{df}}\sphinxparamcomma \sphinxparam{\DUrole{n}{dnames}}}{}
\pysigstopsignatures
\end{fulllineitems}

\index{mask\_all\_rasters\_in\_directory() (in module akhdefo\_functions.Akhdefo\_GOI)@\spxentry{mask\_all\_rasters\_in\_directory()}\spxextra{in module akhdefo\_functions.Akhdefo\_GOI}}

\begin{fulllineitems}
\phantomsection\label{\detokenize{akhdefo_functions:akhdefo_functions.Akhdefo_GOI.mask_all_rasters_in_directory}}
\pysigstartsignatures
\pysiglinewithargsret{\sphinxcode{\sphinxupquote{akhdefo\_functions.Akhdefo\_GOI.}}\sphinxbfcode{\sphinxupquote{mask\_all\_rasters\_in\_directory}}}{\sphinxparam{\DUrole{n}{directory}}\sphinxparamcomma \sphinxparam{\DUrole{n}{mask\_raster\_path}}}{}
\pysigstopsignatures
\sphinxAtStartPar
Masks all georeferenced raster files in a specified directory using a binary raster mask template.

\sphinxAtStartPar
Parameters:
\sphinxhyphen{} directory (str): Path to the directory containing the georeferenced raster files.
\sphinxhyphen{} mask\_raster\_path (str): Path to the binary raster mask template.

\sphinxAtStartPar
Returns:
None. Each raster file in the specified directory will be replaced by its corresponding masked raster.

\end{fulllineitems}

\index{mask\_raster() (in module akhdefo\_functions.Akhdefo\_GOI)@\spxentry{mask\_raster()}\spxextra{in module akhdefo\_functions.Akhdefo\_GOI}}

\begin{fulllineitems}
\phantomsection\label{\detokenize{akhdefo_functions:akhdefo_functions.Akhdefo_GOI.mask_raster}}
\pysigstartsignatures
\pysiglinewithargsret{\sphinxcode{\sphinxupquote{akhdefo\_functions.Akhdefo\_GOI.}}\sphinxbfcode{\sphinxupquote{mask\_raster}}}{\sphinxparam{\DUrole{n}{dem\_array}\DUrole{o}{=}\DUrole{default_value}{None}}\sphinxparamcomma \sphinxparam{\DUrole{n}{mask\_path}\DUrole{o}{=}\DUrole{default_value}{None}}\sphinxparamcomma \sphinxparam{\DUrole{n}{no\_data\_value}\DUrole{o}{=}\DUrole{default_value}{nan}}\sphinxparamcomma \sphinxparam{\DUrole{n}{scatter\_x}\DUrole{o}{=}\DUrole{default_value}{None}}\sphinxparamcomma \sphinxparam{\DUrole{n}{scatter\_y}\DUrole{o}{=}\DUrole{default_value}{None}}}{}
\pysigstopsignatures\begin{description}
\sphinxlineitem{Parameters:}\begin{itemize}
\item {} 
\sphinxAtStartPar
Mask a given raster (DEM) array using a binary mask and optionally filter scatter plot data based on the same mask.

\item {} 
\sphinxAtStartPar
dem\_array (np.ndarray, optional): The 2D or 3D input raster array to be masked. If 3D, the last dimension is assumed to be the channel dimension (e.g., RGB).

\item {} 
\sphinxAtStartPar
mask\_path (str): The path to the raster file containing the binary mask. Values of 1 in the mask represent areas to keep, and values of 0 represent areas to mask out.

\item {} 
\sphinxAtStartPar
no\_data\_value (scalar, optional): The value to replace the masked regions with in the \sphinxtitleref{dem\_array}. Defaults to np.nan.

\item {} 
\sphinxAtStartPar
scatter\_x (np.ndarray, optional): The x\sphinxhyphen{}coordinates of scatter plot data to be filtered based on the mask. If provided, \sphinxtitleref{scatter\_y} must also be provided.

\item {} 
\sphinxAtStartPar
scatter\_y (np.ndarray, optional): The y\sphinxhyphen{}coordinates of scatter plot data to be filtered based on the mask. If provided, \sphinxtitleref{scatter\_x} must also be provided.

\end{itemize}

\sphinxlineitem{Returns:}\begin{itemize}
\item {} 
\sphinxAtStartPar
np.ndarray: The masked raster array. This array will be of the same shape and data type as the input \sphinxtitleref{dem\_array}.

\item {} 
\sphinxAtStartPar
(If scatter\_x and scatter\_y are provided)

\item {} 
\sphinxAtStartPar
np.ndarray: The x\sphinxhyphen{}coordinates of the scatter plot data after filtering with the mask.

\item {} 
\sphinxAtStartPar
np.ndarray: The y\sphinxhyphen{}coordinates of the scatter plot data after filtering with the mask.

\end{itemize}

\sphinxlineitem{Note:}
\sphinxAtStartPar
If the \sphinxtitleref{dem\_array} data type is integer and the \sphinxtitleref{no\_data\_value} is np.nan, the function will 
replace NaN values with a default “no data” integer value (\sphinxhyphen{}9999) before casting back to the 
original data type.

\end{description}

\end{fulllineitems}

\index{mask\_raster\_with\_template() (in module akhdefo\_functions.Akhdefo\_GOI)@\spxentry{mask\_raster\_with\_template()}\spxextra{in module akhdefo\_functions.Akhdefo\_GOI}}

\begin{fulllineitems}
\phantomsection\label{\detokenize{akhdefo_functions:akhdefo_functions.Akhdefo_GOI.mask_raster_with_template}}
\pysigstartsignatures
\pysiglinewithargsret{\sphinxcode{\sphinxupquote{akhdefo\_functions.Akhdefo\_GOI.}}\sphinxbfcode{\sphinxupquote{mask\_raster\_with\_template}}}{\sphinxparam{\DUrole{n}{input\_raster\_path}}\sphinxparamcomma \sphinxparam{\DUrole{n}{mask\_raster\_path}}\sphinxparamcomma \sphinxparam{\DUrole{n}{noData\_value}\DUrole{o}{=}\DUrole{default_value}{nan}}}{}
\pysigstopsignatures
\sphinxAtStartPar
Masks a georeferenced raster file using a binary raster mask template.

\sphinxAtStartPar
Parameters:
\sphinxhyphen{} input\_raster\_path (str): Path to the input georeferenced raster file.
\sphinxhyphen{} mask\_raster\_path (str): Path to the binary raster mask template.

\sphinxAtStartPar
Returns:
None. The input raster file will be replaced by the masked raster.

\end{fulllineitems}

\index{merge\_geodataframes() (in module akhdefo\_functions.Akhdefo\_GOI)@\spxentry{merge\_geodataframes()}\spxextra{in module akhdefo\_functions.Akhdefo\_GOI}}

\begin{fulllineitems}
\phantomsection\label{\detokenize{akhdefo_functions:akhdefo_functions.Akhdefo_GOI.merge_geodataframes}}
\pysigstartsignatures
\pysiglinewithargsret{\sphinxcode{\sphinxupquote{akhdefo\_functions.Akhdefo\_GOI.}}\sphinxbfcode{\sphinxupquote{merge\_geodataframes}}}{\sphinxparam{\DUrole{n}{gdfs}}}{}
\pysigstopsignatures
\end{fulllineitems}

\index{merge\_geodataframes\_generator() (in module akhdefo\_functions.Akhdefo\_GOI)@\spxentry{merge\_geodataframes\_generator()}\spxextra{in module akhdefo\_functions.Akhdefo\_GOI}}

\begin{fulllineitems}
\phantomsection\label{\detokenize{akhdefo_functions:akhdefo_functions.Akhdefo_GOI.merge_geodataframes_generator}}
\pysigstartsignatures
\pysiglinewithargsret{\sphinxcode{\sphinxupquote{akhdefo\_functions.Akhdefo\_GOI.}}\sphinxbfcode{\sphinxupquote{merge\_geodataframes\_generator}}}{\sphinxparam{\DUrole{n}{gdfs}}}{}
\pysigstopsignatures
\end{fulllineitems}

\index{process\_shapefile\_with\_rasters() (in module akhdefo\_functions.Akhdefo\_GOI)@\spxentry{process\_shapefile\_with\_rasters()}\spxextra{in module akhdefo\_functions.Akhdefo\_GOI}}

\begin{fulllineitems}
\phantomsection\label{\detokenize{akhdefo_functions:akhdefo_functions.Akhdefo_GOI.process_shapefile_with_rasters}}
\pysigstartsignatures
\pysiglinewithargsret{\sphinxcode{\sphinxupquote{akhdefo\_functions.Akhdefo\_GOI.}}\sphinxbfcode{\sphinxupquote{process\_shapefile\_with\_rasters}}}{\sphinxparam{\DUrole{n}{shapefile\_path}}\sphinxparamcomma \sphinxparam{\DUrole{n}{rasterfile\_paths}}}{}
\pysigstopsignatures
\end{fulllineitems}

\index{update\_nodata\_values() (in module akhdefo\_functions.Akhdefo\_GOI)@\spxentry{update\_nodata\_values()}\spxextra{in module akhdefo\_functions.Akhdefo\_GOI}}

\begin{fulllineitems}
\phantomsection\label{\detokenize{akhdefo_functions:akhdefo_functions.Akhdefo_GOI.update_nodata_values}}
\pysigstartsignatures
\pysiglinewithargsret{\sphinxcode{\sphinxupquote{akhdefo\_functions.Akhdefo\_GOI.}}\sphinxbfcode{\sphinxupquote{update\_nodata\_values}}}{\sphinxparam{\DUrole{n}{shapefile\_path}\DUrole{o}{=}\DUrole{default_value}{\textquotesingle{}\textquotesingle{}}}\sphinxparamcomma \sphinxparam{\DUrole{n}{rasterfile\_paths}\DUrole{o}{=}\DUrole{default_value}{\textquotesingle{}\textquotesingle{}}}\sphinxparamcomma \sphinxparam{\DUrole{n}{interpolate}\DUrole{o}{=}\DUrole{default_value}{True}}\sphinxparamcomma \sphinxparam{\DUrole{n}{VEL\_Mode}\DUrole{o}{=}\DUrole{default_value}{None}}\sphinxparamcomma \sphinxparam{\DUrole{n}{VEL\_scale}\DUrole{o}{=}\DUrole{default_value}{None}}\sphinxparamcomma \sphinxparam{\DUrole{n}{master\_reference}\DUrole{o}{=}\DUrole{default_value}{True}}\sphinxparamcomma \sphinxparam{\DUrole{n}{Total\_days}\DUrole{o}{=}\DUrole{default_value}{None}}\sphinxparamcomma \sphinxparam{\DUrole{n}{spatial\_ref}\DUrole{o}{=}\DUrole{default_value}{False}}}{}
\pysigstopsignatures
\end{fulllineitems}



\subsubsection{akhdefo\_functions.Akhdefo\_TS module}
\label{\detokenize{akhdefo_functions:module-akhdefo_functions.Akhdefo_TS}}\label{\detokenize{akhdefo_functions:akhdefo-functions-akhdefo-ts-module}}\index{module@\spxentry{module}!akhdefo\_functions.Akhdefo\_TS@\spxentry{akhdefo\_functions.Akhdefo\_TS}}\index{akhdefo\_functions.Akhdefo\_TS@\spxentry{akhdefo\_functions.Akhdefo\_TS}!module@\spxentry{module}}\index{Time\_Series() (in module akhdefo\_functions.Akhdefo\_TS)@\spxentry{Time\_Series()}\spxextra{in module akhdefo\_functions.Akhdefo\_TS}}

\begin{fulllineitems}
\phantomsection\label{\detokenize{akhdefo_functions:akhdefo_functions.Akhdefo_TS.Time_Series}}
\pysigstartsignatures
\pysiglinewithargsret{\sphinxcode{\sphinxupquote{akhdefo\_functions.Akhdefo\_TS.}}\sphinxbfcode{\sphinxupquote{Time\_Series}}}{\sphinxparam{\DUrole{n}{stacked\_raster\_EW}\DUrole{o}{=}\DUrole{default_value}{\textquotesingle{}\textquotesingle{}}}\sphinxparamcomma \sphinxparam{\DUrole{n}{stacked\_raster\_NS}\DUrole{o}{=}\DUrole{default_value}{\textquotesingle{}\textquotesingle{}}}\sphinxparamcomma \sphinxparam{\DUrole{n}{velocity\_points}\DUrole{o}{=}\DUrole{default_value}{\textquotesingle{}\textquotesingle{}}}\sphinxparamcomma \sphinxparam{\DUrole{n}{dates\_name}\DUrole{o}{=}\DUrole{default_value}{\textquotesingle{}\textquotesingle{}}}\sphinxparamcomma \sphinxparam{\DUrole{n}{output\_folder}\DUrole{o}{=}\DUrole{default_value}{\textquotesingle{}\textquotesingle{}}}\sphinxparamcomma \sphinxparam{\DUrole{n}{outputFilename}\DUrole{o}{=}\DUrole{default_value}{\textquotesingle{}\textquotesingle{}}}\sphinxparamcomma \sphinxparam{\DUrole{n}{std}\DUrole{o}{=}\DUrole{default_value}{1}}\sphinxparamcomma \sphinxparam{\DUrole{n}{VEL\_Scale}\DUrole{o}{=}\DUrole{default_value}{\textquotesingle{}year\textquotesingle{}}}\sphinxparamcomma \sphinxparam{\DUrole{n}{velocity\_mode}\DUrole{o}{=}\DUrole{default_value}{\textquotesingle{}mean\textquotesingle{}}}\sphinxparamcomma \sphinxparam{\DUrole{n}{master\_reference}\DUrole{o}{=}\DUrole{default_value}{False}}}{}
\pysigstopsignatures
\sphinxAtStartPar
This program uses candiate velocity points from stackprep function and performs linear interpolation in time\sphinxhyphen{}domain to calibrate
stacked velocity. Additionally produces corrected timeseries velocity(daily) in a shapefile.


\paragraph{Parameters}
\label{\detokenize{akhdefo_functions:id9}}
\sphinxAtStartPar
b   
stacked\_raster\_EW: str

\sphinxAtStartPar
stacked\_raster\_NS: str
\begin{description}
\sphinxlineitem{velocity\_points: str }
\sphinxAtStartPar
Velcity Candidate points

\sphinxlineitem{dates\_name: str}
\sphinxAtStartPar
text file include name of each date in format YYYYMMDD

\end{description}

\sphinxAtStartPar
output\_folder: str

\sphinxAtStartPar
outputFilename: str
\begin{description}
\sphinxlineitem{VEL\_Scale: str}
\sphinxAtStartPar
‘year’ , “month” or empty  to calculate velocity within provided dataset date range

\sphinxlineitem{velocity\_mode: str}
\sphinxAtStartPar
“mean” or “linear”

\sphinxlineitem{master\_reference: bool}
\sphinxAtStartPar
True if calculate TS to a single reference date, False if calculate TS to subsequent Reference dates

\end{description}


\paragraph{Returns}
\label{\detokenize{akhdefo_functions:id10}}
\sphinxAtStartPar
Time\sphinxhyphen{}series shape file of velocity and direction EW, NS, and 2D(resultant Velocity and direction)

\end{fulllineitems}

\index{akhdefo\_dashApp() (in module akhdefo\_functions.Akhdefo\_TS)@\spxentry{akhdefo\_dashApp()}\spxextra{in module akhdefo\_functions.Akhdefo\_TS}}

\begin{fulllineitems}
\phantomsection\label{\detokenize{akhdefo_functions:akhdefo_functions.Akhdefo_TS.akhdefo_dashApp}}
\pysigstartsignatures
\pysiglinewithargsret{\sphinxcode{\sphinxupquote{akhdefo\_functions.Akhdefo\_TS.}}\sphinxbfcode{\sphinxupquote{akhdefo\_dashApp}}}{\sphinxparam{\DUrole{n}{Path\_to\_Shapefile}\DUrole{p}{:}\DUrole{w}{ }\DUrole{n}{str}\DUrole{w}{ }\DUrole{o}{=}\DUrole{w}{ }\DUrole{default_value}{\textquotesingle{}\textquotesingle{}}}\sphinxparamcomma \sphinxparam{\DUrole{n}{port}\DUrole{p}{:}\DUrole{w}{ }\DUrole{n}{int}\DUrole{w}{ }\DUrole{o}{=}\DUrole{w}{ }\DUrole{default_value}{8051}}\sphinxparamcomma \sphinxparam{\DUrole{n}{Column\_Name}\DUrole{p}{:}\DUrole{w}{ }\DUrole{n}{str\DUrole{w}{ }\DUrole{p}{|}\DUrole{w}{ }None}\DUrole{w}{ }\DUrole{o}{=}\DUrole{w}{ }\DUrole{default_value}{None}}\sphinxparamcomma \sphinxparam{\DUrole{n}{BaseMap}\DUrole{p}{:}\DUrole{w}{ }\DUrole{n}{bool}\DUrole{w}{ }\DUrole{o}{=}\DUrole{w}{ }\DUrole{default_value}{False}}\sphinxparamcomma \sphinxparam{\DUrole{n}{basemap\_type}\DUrole{p}{:}\DUrole{w}{ }\DUrole{n}{str\DUrole{w}{ }\DUrole{p}{|}\DUrole{w}{ }None}\DUrole{w}{ }\DUrole{o}{=}\DUrole{w}{ }\DUrole{default_value}{None}}}{}
\pysigstopsignatures
\sphinxAtStartPar
Initializes and runs a Dash web application for geospatial data analysis based on a provided shapefile.
This application provides a user interface to visualize geospatial data on maps and time\sphinxhyphen{}series plots.
It allows users to select a date range, a velocity range, and different plot types and options, such as
whether to include a trendline or not. It also has options for customizing the plot appearance by choosing 
from various color scales and editing the axis labels and title.
\begin{description}
\sphinxlineitem{Parameters:}\begin{itemize}
\item {} \begin{description}
\sphinxlineitem{Path\_to\_Shapefile (str): The file path to the shapefile that contains the geospatial data to be analyzed.}
\sphinxAtStartPar
If left empty, the application will not be able to start correctly.

\end{description}

\item {} 
\sphinxAtStartPar
port (int): The port number on which the Dash application will run. Defaults to 8051.

\end{itemize}

\sphinxlineitem{Dependencies:}\begin{itemize}
\item {} 
\sphinxAtStartPar
dash: The core library for running the Dash app.

\item {} 
\sphinxAtStartPar
dash\_bootstrap\_components: Provides Bootstrap components for a nicer layout and responsiveness.

\item {} 
\sphinxAtStartPar
geopandas: For reading and handling geospatial data in the form of shapefiles.

\item {} 
\sphinxAtStartPar
plotly and plotly.express: For creating interactive plots.

\item {} 
\sphinxAtStartPar
pandas: For data manipulation and analysis.

\item {} 
\sphinxAtStartPar
matplotlib and cmocean: For additional colormap options.

\item {} 
\sphinxAtStartPar
numpy: For numerical operations, especially related to creating custom trendlines.

\end{itemize}

\sphinxlineitem{The application consists of two main interactive elements:}\begin{itemize}
\item {} 
\sphinxAtStartPar
A map plot showing the spatial distribution of velocity data (\sphinxtitleref{VEL}) where the user can select points.

\item {} 
\sphinxAtStartPar
A time\sphinxhyphen{}series plot that displays the average of selected data points over time, with the option to add a trendline.
\begin{quote}

\sphinxAtStartPar
Additionally, the application includes a modal for editing plot titles and axis labels, inputs for
minimum and maximum velocity values, date pickers to filter data based on time, dropdowns for choosing color scales 
and plot types, and radio items for trendline options.
\end{quote}

\end{itemize}

\sphinxlineitem{Callbacks:}\begin{itemize}
\item {} \begin{description}
\sphinxlineitem{The application has callbacks to handle user interactions, updating plots based on user inputs, }
\sphinxAtStartPar
and toggling the modal for editing plot attributes.

\end{description}

\end{itemize}

\sphinxlineitem{Returns:}\begin{itemize}
\item {} 
\sphinxAtStartPar
The Dash app object configured to be run with \sphinxtitleref{app.run\_server(port=port)}.

\end{itemize}

\sphinxlineitem{Raises:}\begin{itemize}
\item {} 
\sphinxAtStartPar
IOError: If the shapefile path is incorrect or the file cannot be read.

\item {} 
\sphinxAtStartPar
Exception: If there are any other issues reading the geospatial data or initializing the app.

\end{itemize}

\sphinxlineitem{Usage:}
\sphinxAtStartPar
To use this function, ensure that you have a valid shapefile and simply call \sphinxtitleref{akhdefo\_dashApp(path\_to\_shapefile)}.
Access the app in a web browser by navigating to \sphinxtitleref{http://127.0.0.1:\textless{}port\textgreater{}} where \sphinxtitleref{port} is the port number passed to the function.

\end{description}

\end{fulllineitems}



\subsubsection{akhdefo\_functions.Akhdefo\_Tools module}
\label{\detokenize{akhdefo_functions:module-akhdefo_functions.Akhdefo_Tools}}\label{\detokenize{akhdefo_functions:akhdefo-functions-akhdefo-tools-module}}\index{module@\spxentry{module}!akhdefo\_functions.Akhdefo\_Tools@\spxentry{akhdefo\_functions.Akhdefo\_Tools}}\index{akhdefo\_functions.Akhdefo\_Tools@\spxentry{akhdefo\_functions.Akhdefo\_Tools}!module@\spxentry{module}}\index{LeNet (class in akhdefo\_functions.Akhdefo\_Tools)@\spxentry{LeNet}\spxextra{class in akhdefo\_functions.Akhdefo\_Tools}}

\begin{fulllineitems}
\phantomsection\label{\detokenize{akhdefo_functions:akhdefo_functions.Akhdefo_Tools.LeNet}}
\pysigstartsignatures
\pysigline{\sphinxbfcode{\sphinxupquote{class\DUrole{w}{ }}}\sphinxcode{\sphinxupquote{akhdefo\_functions.Akhdefo\_Tools.}}\sphinxbfcode{\sphinxupquote{LeNet}}}
\pysigstopsignatures
\sphinxAtStartPar
Bases: \sphinxcode{\sphinxupquote{object}}
\index{build() (akhdefo\_functions.Akhdefo\_Tools.LeNet static method)@\spxentry{build()}\spxextra{akhdefo\_functions.Akhdefo\_Tools.LeNet static method}}

\begin{fulllineitems}
\phantomsection\label{\detokenize{akhdefo_functions:akhdefo_functions.Akhdefo_Tools.LeNet.build}}
\pysigstartsignatures
\pysiglinewithargsret{\sphinxbfcode{\sphinxupquote{static\DUrole{w}{ }}}\sphinxbfcode{\sphinxupquote{build}}}{\sphinxparam{\DUrole{n}{width}}\sphinxparamcomma \sphinxparam{\DUrole{n}{height}}\sphinxparamcomma \sphinxparam{\DUrole{n}{depth}}\sphinxparamcomma \sphinxparam{\DUrole{n}{classes}}}{}
\pysigstopsignatures
\end{fulllineitems}


\end{fulllineitems}

\index{Lenet\_Model\_training() (in module akhdefo\_functions.Akhdefo\_Tools)@\spxentry{Lenet\_Model\_training()}\spxextra{in module akhdefo\_functions.Akhdefo\_Tools}}

\begin{fulllineitems}
\phantomsection\label{\detokenize{akhdefo_functions:akhdefo_functions.Akhdefo_Tools.Lenet_Model_training}}
\pysigstartsignatures
\pysiglinewithargsret{\sphinxcode{\sphinxupquote{akhdefo\_functions.Akhdefo\_Tools.}}\sphinxbfcode{\sphinxupquote{Lenet\_Model\_training}}}{\sphinxparam{\DUrole{n}{dataset}\DUrole{o}{=}\DUrole{default_value}{\textquotesingle{}DataForTraining\textquotesingle{}}}\sphinxparamcomma \sphinxparam{\DUrole{n}{model\_out}\DUrole{o}{=}\DUrole{default_value}{\textquotesingle{}foggy\_not\_foggy.model\textquotesingle{}}}\sphinxparamcomma \sphinxparam{\DUrole{n}{plot}\DUrole{o}{=}\DUrole{default_value}{\textquotesingle{}Model\_stat\_plot.png\textquotesingle{}}}\sphinxparamcomma \sphinxparam{\DUrole{n}{EPOCHS}\DUrole{o}{=}\DUrole{default_value}{100}}\sphinxparamcomma \sphinxparam{\DUrole{n}{INIT\_LR}\DUrole{o}{=}\DUrole{default_value}{0.001}}\sphinxparamcomma \sphinxparam{\DUrole{n}{BS}\DUrole{o}{=}\DUrole{default_value}{32}}}{}
\pysigstopsignatures
\sphinxAtStartPar
This function, Lenet\_Model\_train(), is designed to train a convolutional neural network (CNN) using the LeNet architecture. The network is trained on a dataset of images to classify whether they are “foggy” or “not foggy”.


\paragraph{Parameters:}
\label{\detokenize{akhdefo_functions:id11}}\begin{description}
\sphinxlineitem{dataset: str}
\sphinxAtStartPar
(default=”DataForTraining”) Path to the directory containing the image data for training. The images are expected to be in separate directories named after their corresponding class (“foggy” or “not foggy”).

\sphinxlineitem{model\_out: str}
\sphinxAtStartPar
(default=”foggy\_not\_foggy.model”) The name or path for the output file where the trained model will be saved in the h5 format.

\sphinxlineitem{plot: str}
\sphinxAtStartPar
(default=”Model\_stat\_plot.png”) The name or path for the output image file where a plot of the training loss and accuracy will be saved.

\sphinxlineitem{EPOCHS: int}
\sphinxAtStartPar
(default=100)The number of epochs to use for training.

\sphinxlineitem{INIT\_LR: float}
\sphinxAtStartPar
(default=1e\sphinxhyphen{}3)The initial learning rate for the Adam optimizer.

\sphinxlineitem{BS: int}
\sphinxAtStartPar
(default=32)The batch size for training.

\end{description}


\paragraph{Returns:}
\label{\detokenize{akhdefo_functions:id12}}\begin{itemize}
\item {} 
\sphinxAtStartPar
Trains a LeNet model on the given dataset.

\item {} 
\sphinxAtStartPar
Saves the trained model to disk in the h5 format.

\item {} 
\sphinxAtStartPar
Plots the training and validation loss and accuracy as a function of epoch number, and saves the plot to disk. The plot also includes the model summary.

\item {} 
\sphinxAtStartPar
Note: The function uses data augmentation techniques during training, including random rotations, width and height shifts, shearing, zooming, and horizontal flipping.

\item {} 
\sphinxAtStartPar
This function uses the TensorFlow, Keras, OpenCV, and matplotlib libraries.

\end{itemize}

\end{fulllineitems}

\index{akhdefo\_fitPlane() (in module akhdefo\_functions.Akhdefo\_Tools)@\spxentry{akhdefo\_fitPlane()}\spxextra{in module akhdefo\_functions.Akhdefo\_Tools}}

\begin{fulllineitems}
\phantomsection\label{\detokenize{akhdefo_functions:akhdefo_functions.Akhdefo_Tools.akhdefo_fitPlane}}
\pysigstartsignatures
\pysiglinewithargsret{\sphinxcode{\sphinxupquote{akhdefo\_functions.Akhdefo\_Tools.}}\sphinxbfcode{\sphinxupquote{akhdefo\_fitPlane}}}{\sphinxparam{\DUrole{n}{dem\_data}\DUrole{o}{=}\DUrole{default_value}{\textquotesingle{}\textquotesingle{}}}\sphinxparamcomma \sphinxparam{\DUrole{n}{line\_shapefile}\DUrole{o}{=}\DUrole{default_value}{None}}\sphinxparamcomma \sphinxparam{\DUrole{n}{out\_planeFolder}\DUrole{o}{=}\DUrole{default_value}{\textquotesingle{}Planes\_out\textquotesingle{}}}}{}
\pysigstopsignatures
\sphinxAtStartPar
Fit planes to points in a Digital Elevation Model (DEM) and visualize the results.
\begin{description}
\sphinxlineitem{Parameters:}\begin{itemize}
\item {} 
\sphinxAtStartPar
dem\_data (str): Path to the DEM data file (GeoTIFF format).

\item {} 
\sphinxAtStartPar
line\_shapefile (str): Path to the shapefile containing line features representing planes.

\item {} 
\sphinxAtStartPar
out\_planeFolder (str): Path to the folder where the output plane data will be saved.

\end{itemize}

\sphinxlineitem{How It Works:}
\sphinxAtStartPar
This function reads a DEM and shapefile, allows the user to interactively select points from the DEM,
fits planes to the selected points, and visualizes the results in 2D and 3D plots. It also provides options
to save the fitted planes as XYZ and DXF files. additionally plots poles to planes on polar grid and rose diagram for strike/trends of planes

\sphinxlineitem{Note:}\begin{itemize}
\item {} 
\sphinxAtStartPar
The function utilizes various libraries such as numpy, matplotlib, tkinter, osgeo (GDAL), and geopandas.

\item {} 
\sphinxAtStartPar
Ensure that the required libraries and dependencies are installed to use this function effectively.

\end{itemize}

\sphinxlineitem{Example Usage:}
\sphinxAtStartPar
akhdefo\_fitPlane(dem\_data=’path/to/dem.tif’,line\_shapefile=’path/to/lines.shp’,out\_planeFolder=’output/folder’)

\end{description}

\end{fulllineitems}

\index{assign\_fake\_projection() (in module akhdefo\_functions.Akhdefo\_Tools)@\spxentry{assign\_fake\_projection()}\spxextra{in module akhdefo\_functions.Akhdefo\_Tools}}

\begin{fulllineitems}
\phantomsection\label{\detokenize{akhdefo_functions:akhdefo_functions.Akhdefo_Tools.assign_fake_projection}}
\pysigstartsignatures
\pysiglinewithargsret{\sphinxcode{\sphinxupquote{akhdefo\_functions.Akhdefo\_Tools.}}\sphinxbfcode{\sphinxupquote{assign\_fake\_projection}}}{\sphinxparam{\DUrole{n}{input\_dir}}\sphinxparamcomma \sphinxparam{\DUrole{n}{output\_dir}}}{}
\pysigstopsignatures

\paragraph{Note}
\label{\detokenize{akhdefo_functions:note}}
\sphinxAtStartPar
This program assigns fake latlon geographic coordinates to ground\sphinxhyphen{}based images 
so that it can be ingest using gdal and rasterio geospatial libraries for further processing
\begin{description}
\sphinxlineitem{input\_dir: str}
\sphinxAtStartPar
path to image directories without projection info

\sphinxlineitem{output\_dir: str}
\sphinxAtStartPar
output path image directory for images included projection info

\end{description}

\end{fulllineitems}

\index{calculate\_and\_save\_aspect\_raster() (in module akhdefo\_functions.Akhdefo\_Tools)@\spxentry{calculate\_and\_save\_aspect\_raster()}\spxextra{in module akhdefo\_functions.Akhdefo\_Tools}}

\begin{fulllineitems}
\phantomsection\label{\detokenize{akhdefo_functions:akhdefo_functions.Akhdefo_Tools.calculate_and_save_aspect_raster}}
\pysigstartsignatures
\pysiglinewithargsret{\sphinxcode{\sphinxupquote{akhdefo\_functions.Akhdefo\_Tools.}}\sphinxbfcode{\sphinxupquote{calculate\_and\_save\_aspect\_raster}}}{\sphinxparam{\DUrole{n}{ew\_raster\_path}\DUrole{p}{:}\DUrole{w}{ }\DUrole{n}{str}\DUrole{w}{ }\DUrole{o}{=}\DUrole{w}{ }\DUrole{default_value}{\textquotesingle{}\textquotesingle{}}}\sphinxparamcomma \sphinxparam{\DUrole{n}{ns\_raster\_path}\DUrole{p}{:}\DUrole{w}{ }\DUrole{n}{str}\DUrole{w}{ }\DUrole{o}{=}\DUrole{w}{ }\DUrole{default_value}{\textquotesingle{}\textquotesingle{}}}\sphinxparamcomma \sphinxparam{\DUrole{n}{output\_raster\_path}\DUrole{p}{:}\DUrole{w}{ }\DUrole{n}{str}\DUrole{w}{ }\DUrole{o}{=}\DUrole{w}{ }\DUrole{default_value}{\textquotesingle{}\textquotesingle{}}}}{}
\pysigstopsignatures
\sphinxAtStartPar
Calculate the aspect raster from east\sphinxhyphen{}west and north\sphinxhyphen{}south displacement rasters and save it.

\sphinxAtStartPar
This function reads two raster files representing east\sphinxhyphen{}west (EW) and north\sphinxhyphen{}south (NS) 
displacements, calculates the aspect of each pixel, and saves the result as a new raster file.
Aspect is calculated in degrees from north (0 degrees), east (90 degrees), south (180 degrees),
and west (270 degrees).

\sphinxAtStartPar
Parameters:
\sphinxhyphen{} ew\_raster\_path (str): File path for the east\sphinxhyphen{}west displacement raster.
\sphinxhyphen{} ns\_raster\_path (str): File path for the north\sphinxhyphen{}south displacement raster.
\sphinxhyphen{} output\_raster\_path (str): File path for the output aspect raster.

\sphinxAtStartPar
Returns:
None. The result is saved as a new raster file at the specified output path.

\end{fulllineitems}

\index{calculate\_height\_change() (in module akhdefo\_functions.Akhdefo\_Tools)@\spxentry{calculate\_height\_change()}\spxextra{in module akhdefo\_functions.Akhdefo\_Tools}}

\begin{fulllineitems}
\phantomsection\label{\detokenize{akhdefo_functions:akhdefo_functions.Akhdefo_Tools.calculate_height_change}}
\pysigstartsignatures
\pysiglinewithargsret{\sphinxcode{\sphinxupquote{akhdefo\_functions.Akhdefo\_Tools.}}\sphinxbfcode{\sphinxupquote{calculate\_height\_change}}}{\sphinxparam{\DUrole{n}{slope}}\sphinxparamcomma \sphinxparam{\DUrole{n}{distance}}\sphinxparamcomma \sphinxparam{\DUrole{n}{dem}}}{}
\pysigstopsignatures
\sphinxAtStartPar
Calculate the height change using the slope and distance for each pixel.

\sphinxAtStartPar
Parameters:
slope (numpy.ndarray): Array of slope values in degrees.
distance (numpy.ndarray): Array of distance values.

\sphinxAtStartPar
Returns:
numpy.ndarray: Array of height changes.

\end{fulllineitems}

\index{calculate\_slope() (in module akhdefo\_functions.Akhdefo\_Tools)@\spxentry{calculate\_slope()}\spxextra{in module akhdefo\_functions.Akhdefo\_Tools}}

\begin{fulllineitems}
\phantomsection\label{\detokenize{akhdefo_functions:akhdefo_functions.Akhdefo_Tools.calculate_slope}}
\pysigstartsignatures
\pysiglinewithargsret{\sphinxcode{\sphinxupquote{akhdefo\_functions.Akhdefo\_Tools.}}\sphinxbfcode{\sphinxupquote{calculate\_slope}}}{\sphinxparam{\DUrole{n}{dem}}\sphinxparamcomma \sphinxparam{\DUrole{n}{aspect}}\sphinxparamcomma \sphinxparam{\DUrole{n}{dx}}\sphinxparamcomma \sphinxparam{\DUrole{n}{dy}}}{}
\pysigstopsignatures
\sphinxAtStartPar
Calculate the slope at each pixel using the aspect to determine direction.

\sphinxAtStartPar
Parameters:
dem (numpy.ndarray): Digital Elevation Model (DEM) array.
aspect (numpy.ndarray): Aspect array.
dx (float):  x Spatial resolution of the raster (distance between pixels).
dy (float):  y Spatial resolution of the raster (distance between pixels).

\sphinxAtStartPar
Returns:
numpy.ndarray: Array of slope values in degrees.

\end{fulllineitems}

\index{calculate\_volume() (in module akhdefo\_functions.Akhdefo\_Tools)@\spxentry{calculate\_volume()}\spxextra{in module akhdefo\_functions.Akhdefo\_Tools}}

\begin{fulllineitems}
\phantomsection\label{\detokenize{akhdefo_functions:akhdefo_functions.Akhdefo_Tools.calculate_volume}}
\pysigstartsignatures
\pysiglinewithargsret{\sphinxcode{\sphinxupquote{akhdefo\_functions.Akhdefo\_Tools.}}\sphinxbfcode{\sphinxupquote{calculate\_volume}}}{\sphinxparam{\DUrole{n}{elevation\_map}}\sphinxparamcomma \sphinxparam{\DUrole{n}{slope\_map}}\sphinxparamcomma \sphinxparam{\DUrole{n}{cell\_size}}\sphinxparamcomma \sphinxparam{\DUrole{n}{output\_file}}\sphinxparamcomma \sphinxparam{\DUrole{n}{plot\_map}\DUrole{o}{=}\DUrole{default_value}{False}}\sphinxparamcomma \sphinxparam{\DUrole{n}{plot\_file}\DUrole{o}{=}\DUrole{default_value}{None}}}{}
\pysigstopsignatures
\sphinxAtStartPar
Calculate the volume based on an elevation map and a slope map,
and save the volume map as a GeoTIFF file. Optionally, plot the volume map as a figure and save it.
\begin{description}
\sphinxlineitem{Args:}
\sphinxAtStartPar
elevation\_map (str): File path to the elevation map raster.
slope\_map (ndarray): 2D array representing the slope values.
cell\_size (float): Size of each cell in the map (e.g., length of one side of a square cell).
output\_file (str): Output file path for saving the volume map as a GeoTIFF.
plot\_map (bool, optional): Whether to plot the volume map as a figure. Default is False.
plot\_file (str, optional): Output file path for saving the volume map plot. Required if plot\_map is True.

\sphinxlineitem{Returns:}
\sphinxAtStartPar
ndarray: The calculated volume map.

\end{description}

\end{fulllineitems}

\index{calculate\_volume\_change() (in module akhdefo\_functions.Akhdefo\_Tools)@\spxentry{calculate\_volume\_change()}\spxextra{in module akhdefo\_functions.Akhdefo\_Tools}}

\begin{fulllineitems}
\phantomsection\label{\detokenize{akhdefo_functions:akhdefo_functions.Akhdefo_Tools.calculate_volume_change}}
\pysigstartsignatures
\pysiglinewithargsret{\sphinxcode{\sphinxupquote{akhdefo\_functions.Akhdefo\_Tools.}}\sphinxbfcode{\sphinxupquote{calculate\_volume\_change}}}{\sphinxparam{\DUrole{n}{height\_change}}\sphinxparamcomma \sphinxparam{\DUrole{n}{pixel\_area}}}{}
\pysigstopsignatures
\sphinxAtStartPar
Calculate the volume change for each pixel.

\sphinxAtStartPar
Parameters:
height\_change (numpy.ndarray): Array of height changes.
pixel\_area (float): Area of a single pixel.

\sphinxAtStartPar
Returns:
numpy.ndarray: Array of volume changes.

\end{fulllineitems}

\index{classification() (in module akhdefo\_functions.Akhdefo\_Tools)@\spxentry{classification()}\spxextra{in module akhdefo\_functions.Akhdefo\_Tools}}

\begin{fulllineitems}
\phantomsection\label{\detokenize{akhdefo_functions:akhdefo_functions.Akhdefo_Tools.classification}}
\pysigstartsignatures
\pysiglinewithargsret{\sphinxcode{\sphinxupquote{akhdefo\_functions.Akhdefo\_Tools.}}\sphinxbfcode{\sphinxupquote{classification}}}{\sphinxparam{\DUrole{n}{input\_dir}\DUrole{o}{=}\DUrole{default_value}{\textquotesingle{}dataset\_imagery\textquotesingle{}}}\sphinxparamcomma \sphinxparam{\DUrole{n}{trained\_model}\DUrole{o}{=}\DUrole{default_value}{\textquotesingle{}foggy\_not\_foggy.model\textquotesingle{}}}}{}
\pysigstopsignatures
\sphinxAtStartPar
Classifies images in the specified directory using a trained model.


\paragraph{Inputs:}
\label{\detokenize{akhdefo_functions:inputs}}\begin{itemize}
\item {} 
\sphinxAtStartPar
input\_dir (str, optional): Path to the directory containing the input images. Defaults to “dataset\_imagery”.

\item {} 
\sphinxAtStartPar
trained\_model (str, optional): Path to the trained model file. Defaults to “foggy\_not\_foggy.model”.

\end{itemize}


\paragraph{Returns:}
\label{\detokenize{akhdefo_functions:id13}}\begin{itemize}
\item {} 
\sphinxAtStartPar
The function assumes that the input directory contains image files in JPG format.

\item {} 
\sphinxAtStartPar
The function uses a trained convolutional neural network model to classify the images.

\item {} 
\sphinxAtStartPar
It saves the classified images into separate directories based on their classification.

\end{itemize}

\end{fulllineitems}

\index{displacement\_to\_volume() (in module akhdefo\_functions.Akhdefo\_Tools)@\spxentry{displacement\_to\_volume()}\spxextra{in module akhdefo\_functions.Akhdefo\_Tools}}

\begin{fulllineitems}
\phantomsection\label{\detokenize{akhdefo_functions:akhdefo_functions.Akhdefo_Tools.displacement_to_volume}}
\pysigstartsignatures
\pysiglinewithargsret{\sphinxcode{\sphinxupquote{akhdefo\_functions.Akhdefo\_Tools.}}\sphinxbfcode{\sphinxupquote{displacement\_to\_volume}}}{\sphinxparam{\DUrole{n}{dem\_path}\DUrole{o}{=}\DUrole{default_value}{\textquotesingle{}\textquotesingle{}}}\sphinxparamcomma \sphinxparam{\DUrole{n}{aspect\_path}\DUrole{o}{=}\DUrole{default_value}{\textquotesingle{}\textquotesingle{}}}\sphinxparamcomma \sphinxparam{\DUrole{n}{displacement\_path}\DUrole{o}{=}\DUrole{default_value}{\textquotesingle{}\textquotesingle{}}}\sphinxparamcomma \sphinxparam{\DUrole{n}{slope\_output\_path}\DUrole{o}{=}\DUrole{default_value}{\textquotesingle{}\textquotesingle{}}}\sphinxparamcomma \sphinxparam{\DUrole{n}{height\_output\_path}\DUrole{o}{=}\DUrole{default_value}{\textquotesingle{}\textquotesingle{}}}\sphinxparamcomma \sphinxparam{\DUrole{n}{volume\_output\_path}\DUrole{o}{=}\DUrole{default_value}{\textquotesingle{}\textquotesingle{}}}\sphinxparamcomma \sphinxparam{\DUrole{n}{dx}\DUrole{o}{=}\DUrole{default_value}{None}}\sphinxparamcomma \sphinxparam{\DUrole{n}{dy}\DUrole{o}{=}\DUrole{default_value}{None}}\sphinxparamcomma \sphinxparam{\DUrole{n}{pixel\_area}\DUrole{o}{=}\DUrole{default_value}{None}}}{}
\pysigstopsignatures
\sphinxAtStartPar
Process the DEM, aspect, and displacement rasters to calculate and export the slope, height change, and volume change.

\sphinxAtStartPar
Parameters:
dem\_path (str): Path to the DEM raster file.
aspect\_path (str): Path to the aspect raster file.
displacement\_path (str): Path to the displacement raster file.
slope\_output\_path (str): Path for the slope output GeoTIFF file.
height\_output\_path (str): Path for the height change output GeoTIFF file.
volume\_output\_path (str): Path for the volume change output GeoTIFF file.
dx (float):  x Spatial resolution of the raster (distance between pixels).
dy (float):  y Spatial resolution of the raster (distance between pixels).
pixel\_area (float): Area of a single pixel.

\sphinxAtStartPar
Returns:
None: Outputs GeoTIFF files at the specified output paths.

\end{fulllineitems}

\index{flip\_geotiff\_180() (in module akhdefo\_functions.Akhdefo\_Tools)@\spxentry{flip\_geotiff\_180()}\spxextra{in module akhdefo\_functions.Akhdefo\_Tools}}

\begin{fulllineitems}
\phantomsection\label{\detokenize{akhdefo_functions:akhdefo_functions.Akhdefo_Tools.flip_geotiff_180}}
\pysigstartsignatures
\pysiglinewithargsret{\sphinxcode{\sphinxupquote{akhdefo\_functions.Akhdefo\_Tools.}}\sphinxbfcode{\sphinxupquote{flip\_geotiff\_180}}}{\sphinxparam{\DUrole{n}{directory}}}{}
\pysigstopsignatures
\end{fulllineitems}

\index{move\_files() (in module akhdefo\_functions.Akhdefo\_Tools)@\spxentry{move\_files()}\spxextra{in module akhdefo\_functions.Akhdefo\_Tools}}

\begin{fulllineitems}
\phantomsection\label{\detokenize{akhdefo_functions:akhdefo_functions.Akhdefo_Tools.move_files}}
\pysigstartsignatures
\pysiglinewithargsret{\sphinxcode{\sphinxupquote{akhdefo\_functions.Akhdefo\_Tools.}}\sphinxbfcode{\sphinxupquote{move\_files}}}{\sphinxparam{\DUrole{n}{base\_directory}}}{}
\pysigstopsignatures
\sphinxAtStartPar
This function reorganizes files in the specified directory. 
It searches for timestamps in filenames, creates subdirectories based on the hour part of the timestamp,
and moves files to the appropriate subdirectories. The files are renamed based on the year, month, and day of the timestamp.
\begin{description}
\sphinxlineitem{Args:}
\sphinxAtStartPar
base\_directory (str): Path of the directory containing the files to be reorganized.

\end{description}

\end{fulllineitems}

\index{move\_files\_with\_string() (in module akhdefo\_functions.Akhdefo\_Tools)@\spxentry{move\_files\_with\_string()}\spxextra{in module akhdefo\_functions.Akhdefo\_Tools}}

\begin{fulllineitems}
\phantomsection\label{\detokenize{akhdefo_functions:akhdefo_functions.Akhdefo_Tools.move_files_with_string}}
\pysigstartsignatures
\pysiglinewithargsret{\sphinxcode{\sphinxupquote{akhdefo\_functions.Akhdefo\_Tools.}}\sphinxbfcode{\sphinxupquote{move\_files\_with\_string}}}{\sphinxparam{\DUrole{n}{source\_dir}\DUrole{p}{:}\DUrole{w}{ }\DUrole{n}{str}\DUrole{w}{ }\DUrole{o}{=}\DUrole{w}{ }\DUrole{default_value}{\textquotesingle{}\textquotesingle{}}}\sphinxparamcomma \sphinxparam{\DUrole{n}{dest\_dir}\DUrole{p}{:}\DUrole{w}{ }\DUrole{n}{str}\DUrole{w}{ }\DUrole{o}{=}\DUrole{w}{ }\DUrole{default_value}{\textquotesingle{}\textquotesingle{}}}\sphinxparamcomma \sphinxparam{\DUrole{n}{search\_string}\DUrole{p}{:}\DUrole{w}{ }\DUrole{n}{str}\DUrole{w}{ }\DUrole{o}{=}\DUrole{w}{ }\DUrole{default_value}{\textquotesingle{}.tif\textquotesingle{}}}}{}
\pysigstopsignatures
\sphinxAtStartPar
Move files from a source directory to a destination directory based on a search string present in their paths.

\sphinxAtStartPar
Parameters:
\sphinxhyphen{} source\_dir (str): The directory from which files are to be moved.
\sphinxhyphen{} dest\_dir (str): The destination directory where files will be moved.
\sphinxhyphen{} search\_string (str): The string to search for in the file paths.

\sphinxAtStartPar
This function traverses the source directory, including its subdirectories. 
Files whose paths contain the search string are moved to the destination directory. 
If a file with the same name exists in the destination, it’s renamed to avoid overwriting.

\sphinxAtStartPar
Errors during file movement (e.g., permission issues, non\sphinxhyphen{}existent directories) are logged but do not stop the process.

\end{fulllineitems}

\index{resample\_raster() (in module akhdefo\_functions.Akhdefo\_Tools)@\spxentry{resample\_raster()}\spxextra{in module akhdefo\_functions.Akhdefo\_Tools}}

\begin{fulllineitems}
\phantomsection\label{\detokenize{akhdefo_functions:akhdefo_functions.Akhdefo_Tools.resample_raster}}
\pysigstartsignatures
\pysiglinewithargsret{\sphinxcode{\sphinxupquote{akhdefo\_functions.Akhdefo\_Tools.}}\sphinxbfcode{\sphinxupquote{resample\_raster}}}{\sphinxparam{\DUrole{n}{src\_array}}\sphinxparamcomma \sphinxparam{\DUrole{n}{src\_transform}}\sphinxparamcomma \sphinxparam{\DUrole{n}{src\_crs}}\sphinxparamcomma \sphinxparam{\DUrole{n}{dst\_transform}}\sphinxparamcomma \sphinxparam{\DUrole{n}{dst\_crs}}\sphinxparamcomma \sphinxparam{\DUrole{n}{dst\_shape}}\sphinxparamcomma \sphinxparam{\DUrole{n}{resampling\_method}\DUrole{o}{=}\DUrole{default_value}{Resampling.nearest}}}{}
\pysigstopsignatures
\sphinxAtStartPar
Resample the source raster array to match the destination raster’s resolution and extent.

\sphinxAtStartPar
Parameters:
\sphinxhyphen{} src\_array: 2D numpy array of the source raster
\sphinxhyphen{} src\_transform: affine.Affine transform of the source raster
\sphinxhyphen{} src\_crs: CRS of the source raster
\sphinxhyphen{} dst\_transform: affine.Affine transform of the destination raster
\sphinxhyphen{} dst\_crs: CRS of the destination raster
\sphinxhyphen{} dst\_shape: Shape of the destination raster (height, width)
\sphinxhyphen{} resampling\_method: rasterio.enums.Resampling method to use for resampling

\sphinxAtStartPar
Returns:
\sphinxhyphen{} resampled\_array: 2D numpy array of the resampled source raster

\end{fulllineitems}

\index{scatter\_area\_mask() (in module akhdefo\_functions.Akhdefo\_Tools)@\spxentry{scatter\_area\_mask()}\spxextra{in module akhdefo\_functions.Akhdefo\_Tools}}

\begin{fulllineitems}
\phantomsection\label{\detokenize{akhdefo_functions:akhdefo_functions.Akhdefo_Tools.scatter_area_mask}}
\pysigstartsignatures
\pysiglinewithargsret{\sphinxcode{\sphinxupquote{akhdefo\_functions.Akhdefo\_Tools.}}\sphinxbfcode{\sphinxupquote{scatter\_area\_mask}}}{\sphinxparam{\DUrole{n}{input\_folder}}\sphinxparamcomma \sphinxparam{\DUrole{n}{output\_folder}}\sphinxparamcomma \sphinxparam{\DUrole{n}{plot\_folder}}\sphinxparamcomma \sphinxparam{\DUrole{n}{shapefile\_path}}\sphinxparamcomma \sphinxparam{\DUrole{n}{scatter\_Area\_threshold}\DUrole{o}{=}\DUrole{default_value}{1.1}}\sphinxparamcomma \sphinxparam{\DUrole{n}{vegetation\_mask\_path}\DUrole{o}{=}\DUrole{default_value}{None}}}{}
\pysigstopsignatures
\sphinxAtStartPar
Create an accumulated scatter area mask from a set of raster images based on a given threshold. the input dataset is taken from ASF RTC processing.
The scattering area for each pixel in the RTC image in square meters. The values are calculated based on the effectively illuminated gamma\sphinxhyphen{}0 terrain surface using a digital elevation model, 
the local incidence angle map, and the layover\sphinxhyphen{}shadow map. see detailes at the following website \sphinxurl{https://hyp3-docs.asf.alaska.edu/guides/rtc\_product\_guide/\#scattering-area-map}

\sphinxAtStartPar
The function processes each raster image in the input folder, crops it based on the provided AOI
from the shapefile, normalizes the cropped raster, and then converts the normalized image to a binary
mask based on the scatter\_percentageArea\_threshold. The binary masks from each raster are then accumulated
to generate the final scatter area mask.


\paragraph{Parameters:}
\label{\detokenize{akhdefo_functions:id14}}\begin{description}
\sphinxlineitem{input\_folder}{[}str{]}
\sphinxAtStartPar
Path to the folder containing raster files to be processed.

\sphinxlineitem{output\_folder}{[}str{]}
\sphinxAtStartPar
Directory where the final accumulated mask raster file will be saved.

\sphinxlineitem{plot\_folder}{[}str{]}
\sphinxAtStartPar
Directory where the visual representation (plot) of the accumulated mask will be saved.

\sphinxlineitem{shapefile\_path}{[}str{]}
\sphinxAtStartPar
Path to the shapefile containing the Area of Interest (AOI) for cropping the raster images.

\sphinxlineitem{scatter\_Area\_threshold}{[}float, optional (default=10) unit is perentage  square {]}
\sphinxAtStartPar
Threshold for determining the binary mask from the normalized raster image. Pixels with values 
less than this threshold are set to 0 and those above are set to 1.

\end{description}


\paragraph{Returns:}
\label{\detokenize{akhdefo_functions:id15}}
\sphinxAtStartPar
Shadow Mask for SAR image for sites less likey to have quality measurment points.
The results are saved as files in the specified output and plot directories.


\paragraph{Notes:}
\label{\detokenize{akhdefo_functions:notes}}\begin{itemize}
\item {} 
\sphinxAtStartPar
Assumes that there is only one geometry in the provided shapefile.

\item {} 
\sphinxAtStartPar
The accumulated mask is a result of multiplying binary masks from each raster. Therefore, a pixel in 
the accumulated mask will have a value of 1 only if all rasters have a value of 1 at that pixel location.

\end{itemize}

\end{fulllineitems}

\index{utm\_to\_latlon() (in module akhdefo\_functions.Akhdefo\_Tools)@\spxentry{utm\_to\_latlon()}\spxextra{in module akhdefo\_functions.Akhdefo\_Tools}}

\begin{fulllineitems}
\phantomsection\label{\detokenize{akhdefo_functions:akhdefo_functions.Akhdefo_Tools.utm_to_latlon}}
\pysigstartsignatures
\pysiglinewithargsret{\sphinxcode{\sphinxupquote{akhdefo\_functions.Akhdefo\_Tools.}}\sphinxbfcode{\sphinxupquote{utm\_to\_latlon}}}{\sphinxparam{\DUrole{n}{easting}}\sphinxparamcomma \sphinxparam{\DUrole{n}{northing}}\sphinxparamcomma \sphinxparam{\DUrole{n}{zone\_number}}\sphinxparamcomma \sphinxparam{\DUrole{n}{zone\_letter}}}{}
\pysigstopsignatures
\sphinxAtStartPar
This program converts geographic projection of shapefiles from UTM to LATLONG


\paragraph{Parameters}
\label{\detokenize{akhdefo_functions:id16}}
\sphinxAtStartPar
easting: Geopandas column with Easting

\sphinxAtStartPar
northing: Geopandas column with Northing

\sphinxAtStartPar
zone\_number: int

\sphinxAtStartPar
zone\_letter: “N” or “S”


\paragraph{Returns}
\label{\detokenize{akhdefo_functions:id17}}
\sphinxAtStartPar
{[}lon , lat {]}: List

\end{fulllineitems}



\subsubsection{akhdefo\_functions.Akhdefo\_utils module}
\label{\detokenize{akhdefo_functions:module-akhdefo_functions.Akhdefo_utils}}\label{\detokenize{akhdefo_functions:akhdefo-functions-akhdefo-utils-module}}\index{module@\spxentry{module}!akhdefo\_functions.Akhdefo\_utils@\spxentry{akhdefo\_functions.Akhdefo\_utils}}\index{akhdefo\_functions.Akhdefo\_utils@\spxentry{akhdefo\_functions.Akhdefo\_utils}!module@\spxentry{module}}\index{Akhdefo\_inversion() (in module akhdefo\_functions.Akhdefo\_utils)@\spxentry{Akhdefo\_inversion()}\spxextra{in module akhdefo\_functions.Akhdefo\_utils}}

\begin{fulllineitems}
\phantomsection\label{\detokenize{akhdefo_functions:akhdefo_functions.Akhdefo_utils.Akhdefo_inversion}}
\pysigstartsignatures
\pysiglinewithargsret{\sphinxcode{\sphinxupquote{akhdefo\_functions.Akhdefo\_utils.}}\sphinxbfcode{\sphinxupquote{Akhdefo\_inversion}}}{\sphinxparam{\DUrole{n}{horizontal\_InSAR}\DUrole{o}{=}\DUrole{default_value}{\textquotesingle{}\textquotesingle{}}}\sphinxparamcomma \sphinxparam{\DUrole{n}{Vertical\_InSAR}\DUrole{o}{=}\DUrole{default_value}{\textquotesingle{}\textquotesingle{}}}\sphinxparamcomma \sphinxparam{\DUrole{n}{EW\_Akhdefo}\DUrole{o}{=}\DUrole{default_value}{\textquotesingle{}\textquotesingle{}}}\sphinxparamcomma \sphinxparam{\DUrole{n}{NS\_Akhdefo}\DUrole{o}{=}\DUrole{default_value}{\textquotesingle{}\textquotesingle{}}}\sphinxparamcomma \sphinxparam{\DUrole{n}{output\_folder}\DUrole{o}{=}\DUrole{default_value}{\textquotesingle{}\textquotesingle{}}}\sphinxparamcomma \sphinxparam{\DUrole{n}{dem\_path}\DUrole{o}{=}\DUrole{default_value}{None}}}{}
\pysigstopsignatures
\sphinxAtStartPar
This program calculates 3D displacement velocity (East\sphinxhyphen{}West,North\sphinxhyphen{}South and vertical) using combined optical and InSAR products


\paragraph{Parameters}
\label{\detokenize{akhdefo_functions:id18}}\begin{description}
\sphinxlineitem{horizontal\_InSAR: str}
\sphinxAtStartPar
path to East Velocity InSAR product in geotif format

\sphinxlineitem{Vertical\_InSAR: str}
\sphinxAtStartPar
path to Vertical Velocity InSAR product in geotif format

\sphinxlineitem{EW\_Akhdefo: str }
\sphinxAtStartPar
path to east\sphinxhyphen{}west velocity  akhdefo(optical) product in geotif format

\sphinxlineitem{NS\_Akhdefo: str}
\sphinxAtStartPar
path to north\sphinxhyphen{}south velocity  akhdefo(optical) product in geotif format

\sphinxlineitem{dem\_path: str}
\sphinxAtStartPar
path to DEM raster in geotif format

\sphinxlineitem{output\_folder}{[}str{]}
\sphinxAtStartPar
path to save raster products

\end{description}


\paragraph{Returns}
\label{\detokenize{akhdefo_functions:id19}}\begin{description}
\sphinxlineitem{Three geotif rasters}
\sphinxAtStartPar
3D\sphinxhyphen{}Velocity (D3D in mm/year) raster
Plunge raster in degrees
Trend raster in degress

\end{description}

\end{fulllineitems}

\index{Akhdefo\_resample() (in module akhdefo\_functions.Akhdefo\_utils)@\spxentry{Akhdefo\_resample()}\spxextra{in module akhdefo\_functions.Akhdefo\_utils}}

\begin{fulllineitems}
\phantomsection\label{\detokenize{akhdefo_functions:akhdefo_functions.Akhdefo_utils.Akhdefo_resample}}
\pysigstartsignatures
\pysiglinewithargsret{\sphinxcode{\sphinxupquote{akhdefo\_functions.Akhdefo\_utils.}}\sphinxbfcode{\sphinxupquote{Akhdefo\_resample}}}{\sphinxparam{\DUrole{n}{input\_raster}\DUrole{o}{=}\DUrole{default_value}{\textquotesingle{}\textquotesingle{}}}\sphinxparamcomma \sphinxparam{\DUrole{n}{output\_raster}\DUrole{o}{=}\DUrole{default_value}{\textquotesingle{}\textquotesingle{}}}\sphinxparamcomma \sphinxparam{\DUrole{n}{xres}\DUrole{o}{=}\DUrole{default_value}{3.125}}\sphinxparamcomma \sphinxparam{\DUrole{n}{yres}\DUrole{o}{=}\DUrole{default_value}{3.125}}\sphinxparamcomma \sphinxparam{\DUrole{n}{SavFig}\DUrole{o}{=}\DUrole{default_value}{False}}\sphinxparamcomma \sphinxparam{\DUrole{n}{convert\_units}\DUrole{o}{=}\DUrole{default_value}{None}}}{}
\pysigstopsignatures
\sphinxAtStartPar
This program performs raster resampling for  rasters


\paragraph{Parameters}
\label{\detokenize{akhdefo_functions:id20}}\begin{description}
\sphinxlineitem{input\_raster: str}
\sphinxAtStartPar
path to input raster

\sphinxlineitem{output\_raster: str}
\sphinxAtStartPar
path to output raster

\sphinxlineitem{xres: float}
\sphinxAtStartPar
horizontal resolution

\sphinxlineitem{yres: float }
\sphinxAtStartPar
vertical resolution

\sphinxlineitem{SavFig: bool}
\sphinxAtStartPar
True to save output plot False to ignore exporting plot

\sphinxlineitem{convert\_units: float }
\sphinxAtStartPar
if not None converts raster value units from meter to mm: depends on your raster unit adjust this value

\end{description}

\end{fulllineitems}

\index{Auto\_Variogram() (in module akhdefo\_functions.Akhdefo\_utils)@\spxentry{Auto\_Variogram()}\spxextra{in module akhdefo\_functions.Akhdefo\_utils}}

\begin{fulllineitems}
\phantomsection\label{\detokenize{akhdefo_functions:akhdefo_functions.Akhdefo_utils.Auto_Variogram}}
\pysigstartsignatures
\pysiglinewithargsret{\sphinxcode{\sphinxupquote{akhdefo\_functions.Akhdefo\_utils.}}\sphinxbfcode{\sphinxupquote{Auto\_Variogram}}}{\sphinxparam{\DUrole{n}{data=\textquotesingle{}\textquotesingle{}}}\sphinxparamcomma \sphinxparam{\DUrole{n}{column\_attribute=\textquotesingle{}\textquotesingle{}}}\sphinxparamcomma \sphinxparam{\DUrole{n}{latlon=False}}\sphinxparamcomma \sphinxparam{\DUrole{n}{aoi\_shapefile=\textquotesingle{}\textquotesingle{}}}\sphinxparamcomma \sphinxparam{\DUrole{n}{pixel\_size=20}}\sphinxparamcomma \sphinxparam{\DUrole{n}{num\_chunks=10}}\sphinxparamcomma \sphinxparam{\DUrole{n}{overlap\_percentage=0}}\sphinxparamcomma \sphinxparam{\DUrole{n}{out\_fileName=\textquotesingle{}interpolated\_kriging\textquotesingle{}}}\sphinxparamcomma \sphinxparam{\DUrole{n}{plot\_folder=\textquotesingle{}kriging\_plots\textquotesingle{}}}\sphinxparamcomma \sphinxparam{\DUrole{n}{geo\_folder=\textquotesingle{}geo\_rasterFolder\textquotesingle{}}}\sphinxparamcomma \sphinxparam{\DUrole{n}{smoothing\_kernel=2}}\sphinxparamcomma \sphinxparam{\DUrole{n}{mask: {[}\textless{}class \textquotesingle{}numpy.ndarray\textquotesingle{}\textgreater{}{]} = None}}\sphinxparamcomma \sphinxparam{\DUrole{n}{UTM\_Zone=None}}\sphinxparamcomma \sphinxparam{\DUrole{n}{krig\_method=\textquotesingle{}ordinary\textquotesingle{}}}}{}
\pysigstopsignatures
\sphinxAtStartPar
This function performs automatic selection of the optimal variogram model for spatial data interpolation. 
It also supports clipping of the interpolation results to a specified Area of Interest (AOI). The function 
accepts both GeoDataFrame objects and file paths (specifically, shapefile paths) as input data sources.


\paragraph{Parameters}
\label{\detokenize{akhdefo_functions:id21}}\begin{description}
\sphinxlineitem{data}{[}str or gpd.GeoDataFrame, optional{]}
\sphinxAtStartPar
The path to a shapefile containing point data, or a GeoDataFrame. For shapefiles, they must include 
‘x’, ‘y’ coordinates (or ‘lat’, ‘lon’ if latlon is set to True). Defaults to an empty string.

\sphinxlineitem{column\_attribute}{[}str, optional{]}
\sphinxAtStartPar
The name of the attribute within the shapefile or GeoDataFrame to be interpolated.

\sphinxlineitem{latlon}{[}bool, optional{]}
\sphinxAtStartPar
Indicates whether the input data uses latitude and longitude (True) or Cartesian coordinates (False). 
Defaults to False.

\sphinxlineitem{aoi\_shapefile}{[}str, optional{]}
\sphinxAtStartPar
The path to a shapefile that defines the Area of Interest (AOI) for clipping the interpolation results. 
Defaults to an empty string.

\sphinxlineitem{pixel\_size}{[}int, optional{]}
\sphinxAtStartPar
The resolution size for the interpolated grid. Defaults to 20.

\sphinxlineitem{num\_chunks}{[}int, optional{]}
\sphinxAtStartPar
The number of chunks for processing to mitigate memory issues. Adjust as needed. Defaults to 10.

\sphinxlineitem{overlap\_percentage}{[}float, optional{]}
\sphinxAtStartPar
The percentage of overlap between chunks, ranging from 0 to 1.0. Defaults to 0.

\sphinxlineitem{out\_fileName}{[}str, optional{]}
\sphinxAtStartPar
The name for the output GeoTIFF file storing the interpolation results. Defaults to ‘interpolated\_kriging’.

\sphinxlineitem{plot\_folder}{[}str, optional{]}
\sphinxAtStartPar
Directory path to save plots. Defaults to ‘kriging\_plots’.

\sphinxlineitem{geo\_folder}{[}str, optional{]}
\sphinxAtStartPar
Directory path for saving geospatial raster files. Defaults to ‘geo\_rasterFolder’.

\sphinxlineitem{smoothing\_kernel}{[}int, optional{]}
\sphinxAtStartPar
The size of the smoothing kernel to be used. Defaults to 2.

\sphinxlineitem{mask}{[}{[}np.ndarray{]}, optional{]}
\sphinxAtStartPar
A numpy array to be used as a mask for the interpolation.

\sphinxlineitem{UTM\_Zone}{[}str, optional{]}
\sphinxAtStartPar
The UTM Zone designation (‘N’ for Northern Hemisphere or ‘S’ for Southern Hemisphere).

\sphinxlineitem{krig\_method}{[}str, optional{]}
\sphinxAtStartPar
The method of kriging to be used, either ‘ordinary’ or ‘universal’. Defaults to ‘ordinary’.

\end{description}


\paragraph{Returns}
\label{\detokenize{akhdefo_functions:id22}}\begin{description}
\sphinxlineitem{numpy.ndarray}
\sphinxAtStartPar
A 2D grid of interpolated values, clipped to the specified AOI if an AOI shapefile is provided.

\end{description}


\paragraph{Raises}
\label{\detokenize{akhdefo_functions:raises}}\begin{description}
\sphinxlineitem{ValueError}
\sphinxAtStartPar
If the input data is not a valid shapefile path or GeoDataFrame.
If essential columns (x, y or lat, lon) are missing in the input data.

\end{description}


\paragraph{Notes}
\label{\detokenize{akhdefo_functions:id23}}
\sphinxAtStartPar
The function generates two types of plots:
1. Fitted variogram models against the experimental variogram.
2. The interpolation result using the selected variogram model.


\paragraph{Dependencies}
\label{\detokenize{akhdefo_functions:dependencies}}
\sphinxAtStartPar
Requires geopandas, gstools, pykrige, matplotlib, and rasterio libraries.

\end{fulllineitems}

\index{akhdefo\_download\_planet() (in module akhdefo\_functions.Akhdefo\_utils)@\spxentry{akhdefo\_download\_planet()}\spxextra{in module akhdefo\_functions.Akhdefo\_utils}}

\begin{fulllineitems}
\phantomsection\label{\detokenize{akhdefo_functions:akhdefo_functions.Akhdefo_utils.akhdefo_download_planet}}
\pysigstartsignatures
\pysiglinewithargsret{\sphinxbfcode{\sphinxupquote{\DUrole{k}{async}\DUrole{w}{ }}}\sphinxcode{\sphinxupquote{akhdefo\_functions.Akhdefo\_utils.}}\sphinxbfcode{\sphinxupquote{akhdefo\_download\_planet}}}{\sphinxparam{\DUrole{n}{planet\_api\_key}\DUrole{o}{=}\DUrole{default_value}{\textquotesingle{}\textquotesingle{}}}\sphinxparamcomma \sphinxparam{\DUrole{n}{AOI}\DUrole{o}{=}\DUrole{default_value}{\textquotesingle{}plinth.json\textquotesingle{}}}\sphinxparamcomma \sphinxparam{\DUrole{n}{start\_date}\DUrole{o}{=}\DUrole{default_value}{\textquotesingle{}May 1, 2016\textquotesingle{}}}\sphinxparamcomma \sphinxparam{\DUrole{n}{end\_date}\DUrole{o}{=}\DUrole{default_value}{\textquotesingle{}\textquotesingle{}}}\sphinxparamcomma \sphinxparam{\DUrole{n}{limit}\DUrole{o}{=}\DUrole{default_value}{5}}\sphinxparamcomma \sphinxparam{\DUrole{n}{item\_type}\DUrole{o}{=}\DUrole{default_value}{\textquotesingle{}PSOrthoTile\textquotesingle{}}}\sphinxparamcomma \sphinxparam{\DUrole{n}{product\_bundle}\DUrole{o}{=}\DUrole{default_value}{\textquotesingle{}analytic\_sr\_udm2\textquotesingle{}}}\sphinxparamcomma \sphinxparam{\DUrole{n}{clear\_percent}\DUrole{o}{=}\DUrole{default_value}{90}}\sphinxparamcomma \sphinxparam{\DUrole{n}{cloud\_filter}\DUrole{o}{=}\DUrole{default_value}{0.1}}\sphinxparamcomma \sphinxparam{\DUrole{n}{output\_folder}\DUrole{o}{=}\DUrole{default_value}{\textquotesingle{}raw\_data\textquotesingle{}}}\sphinxparamcomma \sphinxparam{\DUrole{n}{clip\_flag}\DUrole{o}{=}\DUrole{default_value}{True}}\sphinxparamcomma \sphinxparam{\DUrole{n}{download\_data}\DUrole{o}{=}\DUrole{default_value}{False}}}{}
\pysigstopsignatures

\paragraph{Parameters}
\label{\detokenize{akhdefo_functions:id24}}

\paragraph{Note: To use this function need to call await as below:}
\label{\detokenize{akhdefo_functions:note-to-use-this-function-need-to-call-await-as-below}}

\paragraph{await akhdefo\_download\_planet()}
\label{\detokenize{akhdefo_functions:await-akhdefo-download-planet}}\begin{description}
\sphinxlineitem{planet\_api\_key: str = “  “}
\sphinxAtStartPar
input planet labs api

\sphinxlineitem{AOI: str = “plinth.json”}
\sphinxAtStartPar
input area of interest in json file format

\sphinxlineitem{start\_date: str = “May 1, 2016”}
\sphinxAtStartPar
input start date as the following format “Month Day, Year”

\sphinxlineitem{end\_date: str = “May 31, 2017”}
\sphinxAtStartPar
input end date as the following format “Month day, Year”

\sphinxlineitem{limit: int = 5}
\sphinxAtStartPar
input Maxumum number of images want to download; type None if you want to download images daily into the future but need to set the end\_date empty as follow end\_date=””

\sphinxlineitem{item\_type: str = “PSOrthoTile”}
\sphinxAtStartPar
input item type to downoload please refere to planet labs website for further detalis: 
PSScene:    PlanetScope 3, 4, and 8 band scenes captured by the Dove satellite constellation
REOrthoTile:        RapidEye OrthoTiles captured by the RapidEye satellite constellation
REScene:    Unorthorectified strips captured by the RapidEye satellite constellation
SkySatScene:        SkySat Scenes captured by the SkySat satellite constellation
SkySatCollect:      Orthorectified scene composite of a SkySat collection
SkySatVideo:        Full motion videos collected by a single camera from any of the active SkySats
Landsat8L1G Landsat8 Scenes: provided by USGS Landsat8 satellite
Sentinel2L1C:       Copernicus Sentinel\sphinxhyphen{}2 Scenes provided by ESA Sentinel\sphinxhyphen{}2 satellite

\sphinxlineitem{product\_bundle: str = “analytic\_sr\_udm2”}
\sphinxAtStartPar
please refer to planetlabs website for further details and different options of product\_bundle: default is analytic\_sr\_udm2
(analytic,analytic\_udm2, analytic\_3b\_udm2, analytic\_5b , analytic\_5b\_udm2 , analytic\_8b\_udm2, visual, uncalibrated\_dn, 
uncalibrated\_dn\_udm2, basic\_analytic, basic\_analytic\_udm2, basic\_analytic\_8b\_udm2, basic\_uncalibrated\_dn,
basic\_uncalibrated\_dn\_udm2, analytic\_sr, analytic\_sr\_udm2, analytic\_8b\_sr\_udm2, basic\_analytic\_nitf, 
basic\_panchromatic, basic\_panchromatic\_dn, panchromatic, panchromatic\_dn, panchromatic\_dn\_udm2,
pansharpened, pansharpened\_udm2 , basic\_l1a\_dn)

\sphinxlineitem{clear\_percent: int = 90}
\sphinxAtStartPar
Quality of the scene, if you need to download best images keep this value min 90. although, it will end up with less image acquistion

\sphinxlineitem{cloud\_filter: float = 0.1}
\sphinxAtStartPar
cloud percentage

\sphinxlineitem{output\_folder: str = “raw\_data”}
\sphinxAtStartPar
output directory to save the orders

\sphinxlineitem{clip\_flag: bool}
\sphinxAtStartPar
True to clip the downloads to Area of interest json file format provided above

\sphinxlineitem{download\_data: bool }
\sphinxAtStartPar
True to download the data or False to preview the data

\end{description}

\end{fulllineitems}

\index{akhdefo\_orthorectify() (in module akhdefo\_functions.Akhdefo\_utils)@\spxentry{akhdefo\_orthorectify()}\spxextra{in module akhdefo\_functions.Akhdefo\_utils}}

\begin{fulllineitems}
\phantomsection\label{\detokenize{akhdefo_functions:akhdefo_functions.Akhdefo_utils.akhdefo_orthorectify}}
\pysigstartsignatures
\pysiglinewithargsret{\sphinxcode{\sphinxupquote{akhdefo\_functions.Akhdefo\_utils.}}\sphinxbfcode{\sphinxupquote{akhdefo\_orthorectify}}}{\sphinxparam{\DUrole{n}{input\_Dir}\DUrole{p}{:}\DUrole{w}{ }\DUrole{n}{str}}\sphinxparamcomma \sphinxparam{\DUrole{n}{dem\_path}\DUrole{p}{:}\DUrole{w}{ }\DUrole{n}{str}}\sphinxparamcomma \sphinxparam{\DUrole{n}{output\_path}\DUrole{p}{:}\DUrole{w}{ }\DUrole{n}{str}}\sphinxparamcomma \sphinxparam{\DUrole{n}{ortho\_usingRpc}\DUrole{p}{:}\DUrole{w}{ }\DUrole{n}{bool}}\sphinxparamcomma \sphinxparam{\DUrole{n}{file\_ex}\DUrole{o}{=}\DUrole{default_value}{\textquotesingle{}.tif\textquotesingle{}}}}{}
\pysigstopsignatures

\paragraph{Parameteres}
\label{\detokenize{akhdefo_functions:parameteres}}\begin{description}
\sphinxlineitem{input\_Dir: str}
\sphinxAtStartPar
input unortho image directory path

\sphinxlineitem{dem\_path: str }
\sphinxAtStartPar
input path to DEM file

\sphinxlineitem{output\_path: str}
\sphinxAtStartPar
input path to output directory

\sphinxlineitem{ortho\_usingRpc: bool}
\sphinxAtStartPar
Use of RPC file for raw none\sphinxhyphen{}georectified satallite images

\end{description}

\end{fulllineitems}

\index{calculate\_new\_geotransform() (in module akhdefo\_functions.Akhdefo\_utils)@\spxentry{calculate\_new\_geotransform()}\spxextra{in module akhdefo\_functions.Akhdefo\_utils}}

\begin{fulllineitems}
\phantomsection\label{\detokenize{akhdefo_functions:akhdefo_functions.Akhdefo_utils.calculate_new_geotransform}}
\pysigstartsignatures
\pysiglinewithargsret{\sphinxcode{\sphinxupquote{akhdefo\_functions.Akhdefo\_utils.}}\sphinxbfcode{\sphinxupquote{calculate\_new\_geotransform}}}{\sphinxparam{\DUrole{n}{overlap\_box}}\sphinxparamcomma \sphinxparam{\DUrole{n}{src\_transform}}}{}
\pysigstopsignatures
\sphinxAtStartPar
Calculate a new geotransform matrix based on the overlap of image boundaries.

\sphinxAtStartPar
Parameters:
\sphinxhyphen{} overlap\_box (shapely.geometry.Polygon): The overlapping bounding box of all valid images.
\sphinxhyphen{} src\_transform (tuple): The source geotransform matrix to be modified.

\sphinxAtStartPar
Returns:
\sphinxhyphen{} tuple: A new geotransform matrix corresponding to the overlap\_box.

\end{fulllineitems}

\index{check\_shapefile\_columns() (in module akhdefo\_functions.Akhdefo\_utils)@\spxentry{check\_shapefile\_columns()}\spxextra{in module akhdefo\_functions.Akhdefo\_utils}}

\begin{fulllineitems}
\phantomsection\label{\detokenize{akhdefo_functions:akhdefo_functions.Akhdefo_utils.check_shapefile_columns}}
\pysigstartsignatures
\pysiglinewithargsret{\sphinxcode{\sphinxupquote{akhdefo\_functions.Akhdefo\_utils.}}\sphinxbfcode{\sphinxupquote{check\_shapefile\_columns}}}{\sphinxparam{\DUrole{n}{geodata}}\sphinxparamcomma \sphinxparam{\DUrole{n}{latlon}}}{}
\pysigstopsignatures
\end{fulllineitems}

\index{create\_vegetation\_mask() (in module akhdefo\_functions.Akhdefo\_utils)@\spxentry{create\_vegetation\_mask()}\spxextra{in module akhdefo\_functions.Akhdefo\_utils}}

\begin{fulllineitems}
\phantomsection\label{\detokenize{akhdefo_functions:akhdefo_functions.Akhdefo_utils.create_vegetation_mask}}
\pysigstartsignatures
\pysiglinewithargsret{\sphinxcode{\sphinxupquote{akhdefo\_functions.Akhdefo\_utils.}}\sphinxbfcode{\sphinxupquote{create\_vegetation\_mask}}}{\sphinxparam{\DUrole{n}{red\_band\_path}}\sphinxparamcomma \sphinxparam{\DUrole{n}{nir\_band\_path}}\sphinxparamcomma \sphinxparam{\DUrole{n}{output\_path}}\sphinxparamcomma \sphinxparam{\DUrole{n}{shapefile\_path}}\sphinxparamcomma \sphinxparam{\DUrole{n}{threshold}\DUrole{o}{=}\DUrole{default_value}{0.3}}\sphinxparamcomma \sphinxparam{\DUrole{n}{save\_plot}\DUrole{o}{=}\DUrole{default_value}{False}}\sphinxparamcomma \sphinxparam{\DUrole{n}{plot\_path}\DUrole{o}{=}\DUrole{default_value}{\textquotesingle{}plot.png\textquotesingle{}}}}{}
\pysigstopsignatures
\sphinxAtStartPar
Create a binary vegetation mask based on the NDVI (Normalized Difference Vegetation Index) calculation.

\sphinxAtStartPar
Parameters:
\sphinxhyphen{} red\_band\_path (str): Path to the raster file containing the red band.
\sphinxhyphen{} nir\_band\_path (str): Path to the raster file containing the near\sphinxhyphen{}infrared (NIR) band.
\sphinxhyphen{} output\_path (str): Path to save the generated vegetation mask raster.
\sphinxhyphen{} shapefile\_path (str): Path to the shapefile that defines the area of interest (AOI).
\sphinxhyphen{} threshold (float, optional): NDVI threshold for determining vegetation. Pixels with NDVI less than this threshold are considered vegetation. Default is 0.3.
\sphinxhyphen{} save\_plot (bool, optional): Whether to save a plot of the vegetation mask. Default is False.
\sphinxhyphen{} plot\_path (str, optional): Path to save the plot if save\_plot is True. Default is “plot.png”.

\sphinxAtStartPar
Returns:
None. The vegetation mask is written to output\_path, and optionally a plot is saved.

\sphinxAtStartPar
Note:
The function assumes that the shapefile contains only one geometry.

\end{fulllineitems}

\index{crop\_point\_shapefile\_with\_aoi() (in module akhdefo\_functions.Akhdefo\_utils)@\spxentry{crop\_point\_shapefile\_with\_aoi()}\spxextra{in module akhdefo\_functions.Akhdefo\_utils}}

\begin{fulllineitems}
\phantomsection\label{\detokenize{akhdefo_functions:akhdefo_functions.Akhdefo_utils.crop_point_shapefile_with_aoi}}
\pysigstartsignatures
\pysiglinewithargsret{\sphinxcode{\sphinxupquote{akhdefo\_functions.Akhdefo\_utils.}}\sphinxbfcode{\sphinxupquote{crop\_point\_shapefile\_with\_aoi}}}{\sphinxparam{\DUrole{n}{point\_shapefile}}\sphinxparamcomma \sphinxparam{\DUrole{n}{aoi\_shapefile}}\sphinxparamcomma \sphinxparam{\DUrole{n}{output\_folder}}}{}
\pysigstopsignatures
\sphinxAtStartPar
Crop a point shapefile with an Area of Interest (AOI) polygon shapefile.

\sphinxAtStartPar
Parameters:
\sphinxhyphen{} point\_shapefile (str): Path to the input point shapefile.
\sphinxhyphen{} aoi\_shapefile (str): Path to the AOI polygon shapefile for cropping.
\sphinxhyphen{} output\_folder (str): Path to the folder where the cropped shapefile will be saved.

\sphinxAtStartPar
Returns:
None

\sphinxAtStartPar
This function reads a point shapefile and an AOI polygon shapefile, and performs a spatial
intersection to crop the points within the specified AOI. The resulting GeoDataFrame is then
saved as a new shapefile in the specified output folder. If no points are within the AOI, no
file is created, and a message is printed.

\sphinxAtStartPar
Example:
\sphinxcode{\sphinxupquote{\textasciigrave{}
import akhdefo\_functions,
from akhdefo\_functions import crop\_point\_shapefile\_with\_aoi
\textasciigrave{}}}

\sphinxAtStartPar
\sphinxcode{\sphinxupquote{\textasciigrave{}
point\_file = "path\_to\_points.shp"
aoi\_file = "path\_to\_aoi.shp"
output\_folder = "output\_folder"
\textasciigrave{}}}
\sphinxcode{\sphinxupquote{\textasciigrave{}
crop\_point\_shapefile\_with\_aoi(point\_file, aoi\_file, output\_folder)
\textasciigrave{}}}

\sphinxAtStartPar
Note:
\sphinxhyphen{} Requires the GeoPandas library.
\sphinxhyphen{} The AOI polygon is dissolved into a single geometry for intersection.
\sphinxhyphen{} Both input GeoDataFrames must have the same Coordinate Reference System (CRS).
\sphinxhyphen{} The resulting shapefile is saved in the output folder with “\_clipped” appended to the filename.
\sphinxhyphen{} If no points are within the AOI, no file is created, and a message is printed.

\end{fulllineitems}

\index{crop\_to\_overlap() (in module akhdefo\_functions.Akhdefo\_utils)@\spxentry{crop\_to\_overlap()}\spxextra{in module akhdefo\_functions.Akhdefo\_utils}}

\begin{fulllineitems}
\phantomsection\label{\detokenize{akhdefo_functions:akhdefo_functions.Akhdefo_utils.crop_to_overlap}}
\pysigstartsignatures
\pysiglinewithargsret{\sphinxcode{\sphinxupquote{akhdefo\_functions.Akhdefo\_utils.}}\sphinxbfcode{\sphinxupquote{crop\_to\_overlap}}}{\sphinxparam{\DUrole{n}{folder\_path}}}{}
\pysigstopsignatures
\sphinxAtStartPar
Crop all valid image files in the specified folder to the overlapping area.

\sphinxAtStartPar
This function performs the following steps:
1. Iterate through all files in the folder and identify valid image files.
2. Calculate the overlapping bounding box of all valid images.
3. Crop each valid image to the overlapping bounding box.
4. Overwrite the original image files with the cropped images.

\sphinxAtStartPar
Parameters:
\sphinxhyphen{} folder\_path (str): The path to the folder containing the image files to be cropped.

\sphinxAtStartPar
Notes:
\sphinxhyphen{} Supports ‘.tif’, ‘.jpg’, ‘.png’, and ‘.bmp’ file formats.
\sphinxhyphen{} Requires GDAL, OSR, and Shapely libraries.
\sphinxhyphen{} The folder should contain images that share the same spatial reference system (projection).

\sphinxAtStartPar
Exceptions are handled for permission errors and unexpected errors during the file overwriting process.

\end{fulllineitems}

\index{download\_RTC() (in module akhdefo\_functions.Akhdefo\_utils)@\spxentry{download\_RTC()}\spxextra{in module akhdefo\_functions.Akhdefo\_utils}}

\begin{fulllineitems}
\phantomsection\label{\detokenize{akhdefo_functions:akhdefo_functions.Akhdefo_utils.download_RTC}}
\pysigstartsignatures
\pysiglinewithargsret{\sphinxcode{\sphinxupquote{akhdefo\_functions.Akhdefo\_utils.}}\sphinxbfcode{\sphinxupquote{download\_RTC}}}{\sphinxparam{\DUrole{n}{username}\DUrole{p}{:}\DUrole{w}{ }\DUrole{n}{str}\DUrole{w}{ }\DUrole{o}{=}\DUrole{w}{ }\DUrole{default_value}{\textquotesingle{}\textquotesingle{}}}\sphinxparamcomma \sphinxparam{\DUrole{n}{password}\DUrole{p}{:}\DUrole{w}{ }\DUrole{n}{str}\DUrole{w}{ }\DUrole{o}{=}\DUrole{w}{ }\DUrole{default_value}{\textquotesingle{}\textquotesingle{}}}\sphinxparamcomma \sphinxparam{\DUrole{n}{prompt}\DUrole{o}{=}\DUrole{default_value}{False}}\sphinxparamcomma \sphinxparam{\DUrole{n}{asf\_datapool\_results\_file}\DUrole{p}{:}\DUrole{w}{ }\DUrole{n}{str}\DUrole{w}{ }\DUrole{o}{=}\DUrole{w}{ }\DUrole{default_value}{\textquotesingle{}\textquotesingle{}}}\sphinxparamcomma \sphinxparam{\DUrole{n}{save\_dir}\DUrole{p}{:}\DUrole{w}{ }\DUrole{n}{str}\DUrole{w}{ }\DUrole{o}{=}\DUrole{w}{ }\DUrole{default_value}{\textquotesingle{}\textquotesingle{}}}\sphinxparamcomma \sphinxparam{\DUrole{n}{job\_name}\DUrole{p}{:}\DUrole{w}{ }\DUrole{n}{str}\DUrole{w}{ }\DUrole{o}{=}\DUrole{w}{ }\DUrole{default_value}{\textquotesingle{}rtc\sphinxhyphen{}test\textquotesingle{}}}\sphinxparamcomma \sphinxparam{\DUrole{n}{dem\_matching}\DUrole{p}{:}\DUrole{w}{ }\DUrole{n}{bool}\DUrole{w}{ }\DUrole{o}{=}\DUrole{w}{ }\DUrole{default_value}{True}}\sphinxparamcomma \sphinxparam{\DUrole{n}{include\_dem}\DUrole{p}{:}\DUrole{w}{ }\DUrole{n}{bool}\DUrole{w}{ }\DUrole{o}{=}\DUrole{w}{ }\DUrole{default_value}{True}}\sphinxparamcomma \sphinxparam{\DUrole{n}{include\_inc\_map}\DUrole{p}{:}\DUrole{w}{ }\DUrole{n}{bool}\DUrole{w}{ }\DUrole{o}{=}\DUrole{w}{ }\DUrole{default_value}{True}}\sphinxparamcomma \sphinxparam{\DUrole{n}{include\_rgb}\DUrole{p}{:}\DUrole{w}{ }\DUrole{n}{bool}\DUrole{w}{ }\DUrole{o}{=}\DUrole{w}{ }\DUrole{default_value}{False}}\sphinxparamcomma \sphinxparam{\DUrole{n}{include\_scattering\_area}\DUrole{p}{:}\DUrole{w}{ }\DUrole{n}{bool}\DUrole{w}{ }\DUrole{o}{=}\DUrole{w}{ }\DUrole{default_value}{False}}\sphinxparamcomma \sphinxparam{\DUrole{n}{scale}\DUrole{p}{:}\DUrole{w}{ }\DUrole{n}{str}\DUrole{w}{ }\DUrole{o}{=}\DUrole{w}{ }\DUrole{default_value}{\textquotesingle{}amplitude\textquotesingle{}}}\sphinxparamcomma \sphinxparam{\DUrole{n}{resolution}\DUrole{p}{:}\DUrole{w}{ }\DUrole{n}{int}\DUrole{w}{ }\DUrole{o}{=}\DUrole{w}{ }\DUrole{default_value}{10}}\sphinxparamcomma \sphinxparam{\DUrole{n}{speckle\_filter}\DUrole{p}{:}\DUrole{w}{ }\DUrole{n}{bool}\DUrole{w}{ }\DUrole{o}{=}\DUrole{w}{ }\DUrole{default_value}{True}}\sphinxparamcomma \sphinxparam{\DUrole{n}{radiometry}\DUrole{o}{=}\DUrole{default_value}{\textquotesingle{}gamma0\textquotesingle{}}}\sphinxparamcomma \sphinxparam{\DUrole{n}{dem\_name}\DUrole{o}{=}\DUrole{default_value}{\textquotesingle{}copernicus\textquotesingle{}}}\sphinxparamcomma \sphinxparam{\DUrole{n}{download}\DUrole{p}{:}\DUrole{w}{ }\DUrole{n}{bool}\DUrole{w}{ }\DUrole{o}{=}\DUrole{w}{ }\DUrole{default_value}{False}}\sphinxparamcomma \sphinxparam{\DUrole{n}{limit}\DUrole{p}{:}\DUrole{w}{ }\DUrole{n}{int\DUrole{w}{ }\DUrole{p}{|}\DUrole{w}{ }None}\DUrole{w}{ }\DUrole{o}{=}\DUrole{w}{ }\DUrole{default_value}{None}}\sphinxparamcomma \sphinxparam{\DUrole{n}{path\_number}\DUrole{p}{:}\DUrole{w}{ }\DUrole{n}{int\DUrole{w}{ }\DUrole{p}{|}\DUrole{w}{ }None}\DUrole{w}{ }\DUrole{o}{=}\DUrole{w}{ }\DUrole{default_value}{None}}\sphinxparamcomma \sphinxparam{\DUrole{n}{frame\_number}\DUrole{p}{:}\DUrole{w}{ }\DUrole{n}{int\DUrole{w}{ }\DUrole{p}{|}\DUrole{w}{ }None}\DUrole{w}{ }\DUrole{o}{=}\DUrole{w}{ }\DUrole{default_value}{None}}\sphinxparamcomma \sphinxparam{\DUrole{n}{RTC}\DUrole{p}{:}\DUrole{w}{ }\DUrole{n}{bool}\DUrole{w}{ }\DUrole{o}{=}\DUrole{w}{ }\DUrole{default_value}{False}}\sphinxparamcomma \sphinxparam{\DUrole{n}{autorift}\DUrole{p}{:}\DUrole{w}{ }\DUrole{n}{bool}\DUrole{w}{ }\DUrole{o}{=}\DUrole{w}{ }\DUrole{default_value}{False}}\sphinxparamcomma \sphinxparam{\DUrole{n}{insar}\DUrole{p}{:}\DUrole{w}{ }\DUrole{n}{bool}\DUrole{w}{ }\DUrole{o}{=}\DUrole{w}{ }\DUrole{default_value}{False}}\sphinxparamcomma \sphinxparam{\DUrole{n}{max\_neighbors}\DUrole{p}{:}\DUrole{w}{ }\DUrole{n}{int}\DUrole{w}{ }\DUrole{o}{=}\DUrole{w}{ }\DUrole{default_value}{2}}}{}
\pysigstopsignatures
\sphinxAtStartPar
Initiates the download of Synthetic Aperture Radar (SAR) products from ASF’s HyP3 platform.

\sphinxAtStartPar
This function facilitates the submission and download of SAR processing jobs, including Radiometric Terrain 
Correction (RTC), AutoRIFT, and InSAR products. It filters and selects granules based on the provided ASF datapool 
results file and other specified parameters.
\begin{description}
\sphinxlineitem{Args:}
\sphinxAtStartPar
username (str): ASF HyP3 username. Required unless prompt=True.
password (str): ASF HyP3 password. Required unless prompt=True.
prompt (bool): If True, use interactive prompts for login instead of username/password.
asf\_datapool\_results\_file (str): Path to ASF datapool results CSV file.
save\_dir (str): Directory where downloaded files will be saved.
job\_name (str, optional): Name for the job. Defaults to ‘rtc\sphinxhyphen{}test’.
dem\_matching (bool, optional): If True, perform DEM matching for RTC jobs. Defaults to True.
include\_dem (bool, optional): If True, include Digital Elevation Model in RTC jobs. Defaults to True.
include\_inc\_map (bool, optional): If True, include incidence angle map in RTC jobs. Defaults to True.
include\_rgb (bool, optional): If True, include RGB decomposition in RTC jobs. Defaults to False.
include\_scattering\_area (bool, optional): If True, include scattering area in RTC jobs. Defaults to False.
scale (str, optional): Scale for the image in RTC jobs. Defaults to ‘amplitude’.
resolution (int, optional): Desired resolution in meters for RTC jobs. Defaults to 10.
speckle\_filter (bool, optional): Apply Enhanced Lee speckle filter in RTC jobs. Defaults to True.
radiometry (str): Radiometry normalization (either ‘sigma0’ or ‘gamma0’) for RTC jobs.
dem\_name (str): DEM to use for RTC jobs. ‘copernicus’ or ‘legacy’.
download (bool): If True, submit jobs and download data. Defaults to False.
limit (int, optional): Limit the number of images to download.
path\_number (int, optional): Filter granules by path number.
frame\_number (int, optional): Filter granules by frame number.
RTC (bool): If True, process RTC jobs.
autorift (bool): If True, process AutoRIFT jobs.
insar (bool): If True, process InSAR jobs.
max\_neighbors (int): Max number of neighbors for InSAR job pairing.

\sphinxlineitem{Returns:}
\sphinxAtStartPar
sdk.Batch: A Batch object containing the submitted jobs.

\sphinxlineitem{Raises:}
\sphinxAtStartPar
ValueError: If required arguments are not provided.

\sphinxlineitem{Notes:}\begin{itemize}
\item {} 
\sphinxAtStartPar
This function requires prior installation of the HyP3 SDK and relevant dependencies.

\item {} 
\sphinxAtStartPar
The ASF datapool results file should be in CSV format with specific columns for filtering.

\item {} 
\sphinxAtStartPar
Downloaded files will be saved in the specified directory with additional suffixes based on the job type.

\end{itemize}

\end{description}

\end{fulllineitems}

\index{measure\_displacement\_from\_camera() (in module akhdefo\_functions.Akhdefo\_utils)@\spxentry{measure\_displacement\_from\_camera()}\spxextra{in module akhdefo\_functions.Akhdefo\_utils}}

\begin{fulllineitems}
\phantomsection\label{\detokenize{akhdefo_functions:akhdefo_functions.Akhdefo_utils.measure_displacement_from_camera}}
\pysigstartsignatures
\pysiglinewithargsret{\sphinxcode{\sphinxupquote{akhdefo\_functions.Akhdefo\_utils.}}\sphinxbfcode{\sphinxupquote{measure\_displacement\_from\_camera}}}{\sphinxparam{\DUrole{n}{hls\_url}}\sphinxparamcomma \sphinxparam{\DUrole{n}{alpha}\DUrole{o}{=}\DUrole{default_value}{0.1}}\sphinxparamcomma \sphinxparam{\DUrole{n}{save\_output}\DUrole{o}{=}\DUrole{default_value}{False}}\sphinxparamcomma \sphinxparam{\DUrole{n}{output\_filename}\DUrole{o}{=}\DUrole{default_value}{None}}\sphinxparamcomma \sphinxparam{\DUrole{n}{ssim\_threshold}\DUrole{o}{=}\DUrole{default_value}{0.4}}\sphinxparamcomma \sphinxparam{\DUrole{n}{pyr\_scale}\DUrole{o}{=}\DUrole{default_value}{0.5}}\sphinxparamcomma \sphinxparam{\DUrole{n}{levels}\DUrole{o}{=}\DUrole{default_value}{15}}\sphinxparamcomma \sphinxparam{\DUrole{n}{winsize}\DUrole{o}{=}\DUrole{default_value}{32}}\sphinxparamcomma \sphinxparam{\DUrole{n}{iterations}\DUrole{o}{=}\DUrole{default_value}{3}}\sphinxparamcomma \sphinxparam{\DUrole{n}{poly\_n}\DUrole{o}{=}\DUrole{default_value}{5}}\sphinxparamcomma \sphinxparam{\DUrole{n}{poly\_sigma}\DUrole{o}{=}\DUrole{default_value}{1.5}}\sphinxparamcomma \sphinxparam{\DUrole{n}{flags}\DUrole{o}{=}\DUrole{default_value}{1}}\sphinxparamcomma \sphinxparam{\DUrole{n}{show\_video}\DUrole{o}{=}\DUrole{default_value}{True}}\sphinxparamcomma \sphinxparam{\DUrole{n}{streamer\_option}\DUrole{o}{=}\DUrole{default_value}{\textquotesingle{}mag\textquotesingle{}}}}{}
\pysigstopsignatures
\sphinxAtStartPar
Test data URLs; hls\_url = “\sphinxurl{https://chiefcam.com/resources/video/events/september-2021-rockfall/september-2021-rockfall-1080p.mp4}” or “\sphinxurl{https://chiefcam.com/video/hls/live/1080p/index.m3u8}”


\paragraph{Measure the displacement from the camera feed using Dense Optical Flow.}
\label{\detokenize{akhdefo_functions:measure-the-displacement-from-the-camera-feed-using-dense-optical-flow}}
\sphinxAtStartPar
Parameters:
\begin{description}
\sphinxlineitem{hls\_url: str}
\sphinxAtStartPar
The URL of the HLS video stream. or type 0 to process video from live pc webcam or add path to your video

\sphinxlineitem{alpha: float, optional}
\sphinxAtStartPar
The weight of the image to update the background model, default is 0.1.

\sphinxlineitem{save\_output}{[}bool, optional{]}
\sphinxAtStartPar
Flag to save the output, default is False.

\sphinxlineitem{output\_filename}{[}str, optional{]}
\sphinxAtStartPar
The filename for saving the output video, required if save\_output is True.

\sphinxlineitem{ssim\_threshold: float , default 0.4}
\sphinxAtStartPar
if interetesed to identify rockfalls recommended value is 0.5 or less.

\sphinxlineitem{pyr\_scale: float}
\sphinxAtStartPar
parameter, specifying the image scale (\textless{}1) to build pyramids for each image; pyr\_scale=0.5 means a classical pyramid, where each next layer is twice smaller than the previous one.

\sphinxlineitem{levels: int}
\sphinxAtStartPar
number of pyramid layers including the initial image; levels=1 means that no extra layers are created and only the original images are used.

\sphinxlineitem{winsize: int}
\sphinxAtStartPar
averaging window size; larger values increase the algorithm robustness to image noise and give more chances for fast motion detection, but yield more blurred motion field.

\sphinxlineitem{iterations: int}
\sphinxAtStartPar
number of iterations the algorithm does at each pyramid level.

\sphinxlineitem{poly\_n: int}
\sphinxAtStartPar
size of the pixel neighborhood used to find polynomial expansion in each pixel; 
larger values mean that the image will be approximated with smoother surfaces, 
yielding more robust algorithm and more blurred motion field, typically poly\_n =5 or 7.

\sphinxlineitem{poly\_sigma: float}
\sphinxAtStartPar
standard deviation of the Gaussian that is used to smooth derivatives used as a basis for the polynomial expansion; 
for poly\_n=5, you can set poly\_sigma=1.1, for poly\_n=7, a good value would be poly\_sigma=1.5.

\sphinxlineitem{flags: 0 or 1}
\sphinxAtStartPar
operation flags that can be a combination of the following:
0 OPTFLOW\_USE\_INITIAL\_FLOW uses the input flow as an initial flow approximation.
1 OPTFLOW\_FARNEBACK\_GAUSSIAN uses the Gaussian winsize×winsize filter instead of a box filter of the same size for optical flow estimation; 
usually, this option gives z more accurate flow than with a box filter, at the cost of lower speed; 
normally, winsize for a Gaussian window should be set to a larger value to achieve the same level of robustness.

\sphinxlineitem{Returns: }
\sphinxAtStartPar
Video output with motion vectors and magnitude.

\end{description}

\end{fulllineitems}

\index{measure\_displacement\_from\_camera\_toFile() (in module akhdefo\_functions.Akhdefo\_utils)@\spxentry{measure\_displacement\_from\_camera\_toFile()}\spxextra{in module akhdefo\_functions.Akhdefo\_utils}}

\begin{fulllineitems}
\phantomsection\label{\detokenize{akhdefo_functions:akhdefo_functions.Akhdefo_utils.measure_displacement_from_camera_toFile}}
\pysigstartsignatures
\pysiglinewithargsret{\sphinxcode{\sphinxupquote{akhdefo\_functions.Akhdefo\_utils.}}\sphinxbfcode{\sphinxupquote{measure\_displacement\_from\_camera\_toFile}}}{\sphinxparam{\DUrole{n}{hls\_url}}\sphinxparamcomma \sphinxparam{\DUrole{n}{alpha}\DUrole{o}{=}\DUrole{default_value}{0.1}}\sphinxparamcomma \sphinxparam{\DUrole{n}{save\_output}\DUrole{o}{=}\DUrole{default_value}{False}}\sphinxparamcomma \sphinxparam{\DUrole{n}{output\_filename}\DUrole{o}{=}\DUrole{default_value}{None}}\sphinxparamcomma \sphinxparam{\DUrole{n}{ssim\_threshold}\DUrole{o}{=}\DUrole{default_value}{0.4}}\sphinxparamcomma \sphinxparam{\DUrole{n}{pyr\_scale}\DUrole{o}{=}\DUrole{default_value}{0.5}}\sphinxparamcomma \sphinxparam{\DUrole{n}{levels}\DUrole{o}{=}\DUrole{default_value}{15}}\sphinxparamcomma \sphinxparam{\DUrole{n}{winsize}\DUrole{o}{=}\DUrole{default_value}{32}}\sphinxparamcomma \sphinxparam{\DUrole{n}{iterations}\DUrole{o}{=}\DUrole{default_value}{3}}\sphinxparamcomma \sphinxparam{\DUrole{n}{poly\_n}\DUrole{o}{=}\DUrole{default_value}{5}}\sphinxparamcomma \sphinxparam{\DUrole{n}{poly\_sigma}\DUrole{o}{=}\DUrole{default_value}{1.5}}\sphinxparamcomma \sphinxparam{\DUrole{n}{flags}\DUrole{o}{=}\DUrole{default_value}{1}}\sphinxparamcomma \sphinxparam{\DUrole{n}{show\_video}\DUrole{o}{=}\DUrole{default_value}{True}}}{}
\pysigstopsignatures
\sphinxAtStartPar
Test data URLs; hls\_url = “\sphinxurl{https://chiefcam.com/resources/video/events/september-2021-rockfall/september-2021-rockfall-1080p.mp4}” or “\sphinxurl{https://chiefcam.com/video/hls/live/1080p/index.m3u8}”


\paragraph{Measure the displacement from the camera feed using Dense Optical Flow.}
\label{\detokenize{akhdefo_functions:id25}}
\sphinxAtStartPar
Parameters:
\begin{description}
\sphinxlineitem{hls\_url: str}
\sphinxAtStartPar
The URL of the HLS video stream. or type 0 to process video from live pc webcam or add path to your video

\sphinxlineitem{alpha: float, optional}
\sphinxAtStartPar
The weight of the image to update the background model, default is 0.1.

\sphinxlineitem{save\_output}{[}bool, optional{]}
\sphinxAtStartPar
Flag to save the output, default is False.

\sphinxlineitem{output\_filename}{[}str, optional{]}
\sphinxAtStartPar
The filename for saving the output video, required if save\_output is True.

\sphinxlineitem{ssim\_threshold: float , default 0.4}
\sphinxAtStartPar
if interetesed to identify rockfalls recommended value is 0.5 or less.

\sphinxlineitem{pyr\_scale: float}
\sphinxAtStartPar
parameter, specifying the image scale (\textless{}1) to build pyramids for each image; pyr\_scale=0.5 means a classical pyramid, where each next layer is twice smaller than the previous one.

\sphinxlineitem{levels: int}
\sphinxAtStartPar
number of pyramid layers including the initial image; levels=1 means that no extra layers are created and only the original images are used.

\sphinxlineitem{winsize: int}
\sphinxAtStartPar
averaging window size; larger values increase the algorithm robustness to image noise and give more chances for fast motion detection, but yield more blurred motion field.

\sphinxlineitem{iterations: int}
\sphinxAtStartPar
number of iterations the algorithm does at each pyramid level.

\sphinxlineitem{poly\_n: int}
\sphinxAtStartPar
size of the pixel neighborhood used to find polynomial expansion in each pixel; 
larger values mean that the image will be approximated with smoother surfaces, 
yielding more robust algorithm and more blurred motion field, typically poly\_n =5 or 7.

\sphinxlineitem{poly\_sigma: float}
\sphinxAtStartPar
standard deviation of the Gaussian that is used to smooth derivatives used as a basis for the polynomial expansion; 
for poly\_n=5, you can set poly\_sigma=1.1, for poly\_n=7, a good value would be poly\_sigma=1.5.

\sphinxlineitem{flags: 0 or 1}
\sphinxAtStartPar
operation flags that can be a combination of the following:
0 OPTFLOW\_USE\_INITIAL\_FLOW uses the input flow as an initial flow approximation.
1 OPTFLOW\_FARNEBACK\_GAUSSIAN uses the Gaussian winsize×winsize filter instead of a box filter of the same size for optical flow estimation; 
usually, this option gives z more accurate flow than with a box filter, at the cost of lower speed; 
normally, winsize for a Gaussian window should be set to a larger value to achieve the same level of robustness.

\sphinxlineitem{Returns: }
\sphinxAtStartPar
Video output with motion vectors and magnitude.

\end{description}

\end{fulllineitems}

\index{reproject\_raster\_to\_match\_shapefile() (in module akhdefo\_functions.Akhdefo\_utils)@\spxentry{reproject\_raster\_to\_match\_shapefile()}\spxextra{in module akhdefo\_functions.Akhdefo\_utils}}

\begin{fulllineitems}
\phantomsection\label{\detokenize{akhdefo_functions:akhdefo_functions.Akhdefo_utils.reproject_raster_to_match_shapefile}}
\pysigstartsignatures
\pysiglinewithargsret{\sphinxcode{\sphinxupquote{akhdefo\_functions.Akhdefo\_utils.}}\sphinxbfcode{\sphinxupquote{reproject\_raster\_to\_match\_shapefile}}}{\sphinxparam{\DUrole{n}{src\_path}}\sphinxparamcomma \sphinxparam{\DUrole{n}{dst\_path}}\sphinxparamcomma \sphinxparam{\DUrole{n}{dst\_crs}}}{}
\pysigstopsignatures
\sphinxAtStartPar
Reproject a raster to match the coordinate reference system (CRS) of a shapefile.

\sphinxAtStartPar
Parameters:
\sphinxhyphen{} src\_path (str): Path to the source raster file that needs to be reprojected.
\sphinxhyphen{} dst\_path (str): Path to save the reprojected raster.
\sphinxhyphen{} dst\_crs (CRS or str): Target coordinate reference system.

\sphinxAtStartPar
Returns:
None. The reprojected raster is written to dst\_path.

\end{fulllineitems}



\subsubsection{akhdefo\_functions.Filter\_PreProc module}
\label{\detokenize{akhdefo_functions:module-akhdefo_functions.Filter_PreProc}}\label{\detokenize{akhdefo_functions:akhdefo-functions-filter-preproc-module}}\index{module@\spxentry{module}!akhdefo\_functions.Filter\_PreProc@\spxentry{akhdefo\_functions.Filter\_PreProc}}\index{akhdefo\_functions.Filter\_PreProc@\spxentry{akhdefo\_functions.Filter\_PreProc}!module@\spxentry{module}}\index{Filter\_PreProcess() (in module akhdefo\_functions.Filter\_PreProc)@\spxentry{Filter\_PreProcess()}\spxextra{in module akhdefo\_functions.Filter\_PreProc}}

\begin{fulllineitems}
\phantomsection\label{\detokenize{akhdefo_functions:akhdefo_functions.Filter_PreProc.Filter_PreProcess}}
\pysigstartsignatures
\pysiglinewithargsret{\sphinxcode{\sphinxupquote{akhdefo\_functions.Filter\_PreProc.}}\sphinxbfcode{\sphinxupquote{Filter\_PreProcess}}}{\sphinxparam{\DUrole{n}{unfiltered\_folderPath}\DUrole{o}{=}\DUrole{default_value}{\textquotesingle{}\textquotesingle{}}}\sphinxparamcomma \sphinxparam{\DUrole{n}{UDM2\_maskfolderPath}\DUrole{o}{=}\DUrole{default_value}{\textquotesingle{}\textquotesingle{}}}\sphinxparamcomma \sphinxparam{\DUrole{n}{outpath\_dir}\DUrole{o}{=}\DUrole{default_value}{\textquotesingle{}\textquotesingle{}}}\sphinxparamcomma \sphinxparam{\DUrole{n}{Udm\_Mask\_Option}\DUrole{o}{=}\DUrole{default_value}{False}}\sphinxparamcomma \sphinxparam{\DUrole{n}{plot\_figure}\DUrole{o}{=}\DUrole{default_value}{False}}}{}
\pysigstopsignatures
\sphinxAtStartPar
This program prepare and uses filters to balanace raster image brightness


\paragraph{Parameters}
\label{\detokenize{akhdefo_functions:id26}}
\sphinxAtStartPar
unfiltered\_folderPath:  str

\sphinxAtStartPar
UDM2\_maskfolderPath:    str

\sphinxAtStartPar
outpath\_dir:    str
\begin{description}
\sphinxlineitem{Udm\_Mask\_Option:    bool}
\sphinxAtStartPar
False if True the program uses planetlabs imagery unusable pixel mask to ignore and mask bad image pixels

\sphinxlineitem{plot\_figure:    bool}
\sphinxAtStartPar
True if you want to display output figure directly inside python

\end{description}


\paragraph{Returns}
\label{\detokenize{akhdefo_functions:id27}}\begin{description}
\sphinxlineitem{geotif rasters}
\sphinxAtStartPar
Filtered geotif rasters

\sphinxlineitem{Figures}
\sphinxAtStartPar
plotted filtered rasters and mask for bad pixels

\end{description}

\end{fulllineitems}

\index{Raster\_Correction() (in module akhdefo\_functions.Filter\_PreProc)@\spxentry{Raster\_Correction()}\spxextra{in module akhdefo\_functions.Filter\_PreProc}}

\begin{fulllineitems}
\phantomsection\label{\detokenize{akhdefo_functions:akhdefo_functions.Filter_PreProc.Raster_Correction}}
\pysigstartsignatures
\pysiglinewithargsret{\sphinxcode{\sphinxupquote{akhdefo\_functions.Filter\_PreProc.}}\sphinxbfcode{\sphinxupquote{Raster\_Correction}}}{\sphinxparam{\DUrole{n}{input\_path}}\sphinxparamcomma \sphinxparam{\DUrole{n}{output\_path}}\sphinxparamcomma \sphinxparam{\DUrole{n}{limit}\DUrole{o}{=}\DUrole{default_value}{None}}\sphinxparamcomma \sphinxparam{\DUrole{n}{lowpass\_kernel\_size}\DUrole{o}{=}\DUrole{default_value}{5}}\sphinxparamcomma \sphinxparam{\DUrole{n}{bilateral\_win\_size}\DUrole{o}{=}\DUrole{default_value}{9}}\sphinxparamcomma \sphinxparam{\DUrole{n}{bilateral\_sigma\_color}\DUrole{o}{=}\DUrole{default_value}{75}}\sphinxparamcomma \sphinxparam{\DUrole{n}{bilateral\_sigma\_spatial}\DUrole{o}{=}\DUrole{default_value}{75}}\sphinxparamcomma \sphinxparam{\DUrole{n}{clip\_percentiles}\DUrole{o}{=}\DUrole{default_value}{{[}2, 98{]}}}\sphinxparamcomma \sphinxparam{\DUrole{n}{optical}\DUrole{o}{=}\DUrole{default_value}{False}}\sphinxparamcomma \sphinxparam{\DUrole{n}{scale}\DUrole{o}{=}\DUrole{default_value}{\textquotesingle{}power\textquotesingle{}}}\sphinxparamcomma \sphinxparam{\DUrole{n}{Vegetation\_mask}\DUrole{o}{=}\DUrole{default_value}{None}}\sphinxparamcomma \sphinxparam{\DUrole{n}{CLAHE\_filter}\DUrole{o}{=}\DUrole{default_value}{False}}}{}
\pysigstopsignatures
\sphinxAtStartPar
Performs a series of image correction and enhancement operations on raster images,
specifically tailored for Synthetic Aperture Radar (SAR) data, but can be used for other rasters too.

\sphinxAtStartPar
This function achieves denoising, normalization, and contrast enhancement for raster images
based on user\sphinxhyphen{}specified parameters. The function is designed to work effectively with both 
optical and SAR imagery.
\begin{description}
\sphinxlineitem{Args:}
\sphinxAtStartPar
input\_path (str): Path to the directory containing input raster images.
output\_path (str): Directory where corrected images and plots will be saved.
limit (int, optional): Maximum number of images to process. If None, all images are processed.
lowpass\_kernel\_size (int, optional): Size of the Gaussian low\sphinxhyphen{}pass filter kernel.
bilateral\_win\_size (int): Size of the bilateral filter window.
bilateral\_sigma\_color (int): Standard deviation for color space in bilateral filter.
bilateral\_sigma\_spatial (int): Standard deviation for spatial space in bilateral filter.
clip\_percentiles (list): 2\sphinxhyphen{}element list containing lower and upper percentiles for clipping pixel values.
optical (bool): Indicates if the raster is optical imagery. Activates certain corrections specific to optical data.
scale (str): Mode of scaling (‘power’ for logarithmic normalization and ‘amplitude’ for linear).
Vegetation\_mask (str, optional): Path to a raster file that represents a vegetation mask. Pixels in the input image that correspond to non\sphinxhyphen{}vegetation in the mask will be set to one.

\sphinxlineitem{Returns:}
\sphinxAtStartPar
None. Outputs are saved to the specified directory.

\sphinxlineitem{Outputs:}\begin{itemize}
\item {} 
\sphinxAtStartPar
Corrected raster images saved in GeoTIFF format.

\item {} 
\sphinxAtStartPar
Plots comparing original, filtered, and corrected images along with their histograms.

\end{itemize}

\sphinxlineitem{Usage:}
\sphinxAtStartPar
Suitable for both SAR and optical imagery correction. Adjust filter parameters as per the nature of input raster
and desired output.

\sphinxlineitem{Notes:}\begin{itemize}
\item {} 
\sphinxAtStartPar
Ensure that the GDAL and other necessary libraries are correctly installed.

\item {} 
\sphinxAtStartPar
The function uses tqdm to provide a progress bar, so ensure it’s installed if you want the progress visualization.

\item {} 
\sphinxAtStartPar
The function has been optimized for a balance between performance and quality, but processing large raster datasets
might still be computationally intensive.

\end{itemize}

\end{description}

\end{fulllineitems}



\subsubsection{akhdefo\_functions.Mosaic\_Crop module}
\label{\detokenize{akhdefo_functions:module-akhdefo_functions.Mosaic_Crop}}\label{\detokenize{akhdefo_functions:akhdefo-functions-mosaic-crop-module}}\index{module@\spxentry{module}!akhdefo\_functions.Mosaic\_Crop@\spxentry{akhdefo\_functions.Mosaic\_Crop}}\index{akhdefo\_functions.Mosaic\_Crop@\spxentry{akhdefo\_functions.Mosaic\_Crop}!module@\spxentry{module}}\index{Crop\_to\_AOI() (in module akhdefo\_functions.Mosaic\_Crop)@\spxentry{Crop\_to\_AOI()}\spxextra{in module akhdefo\_functions.Mosaic\_Crop}}

\begin{fulllineitems}
\phantomsection\label{\detokenize{akhdefo_functions:akhdefo_functions.Mosaic_Crop.Crop_to_AOI}}
\pysigstartsignatures
\pysiglinewithargsret{\sphinxcode{\sphinxupquote{akhdefo\_functions.Mosaic\_Crop.}}\sphinxbfcode{\sphinxupquote{Crop\_to\_AOI}}}{\sphinxparam{\DUrole{n}{Path\_to\_WorkingDir}\DUrole{o}{=}\DUrole{default_value}{\textquotesingle{}\textquotesingle{}}}\sphinxparamcomma \sphinxparam{\DUrole{n}{Path\_to\_AOI\_shapefile}\DUrole{o}{=}\DUrole{default_value}{\textquotesingle{}\textquotesingle{}}}\sphinxparamcomma \sphinxparam{\DUrole{n}{output\_CroppedDir}\DUrole{o}{=}\DUrole{default_value}{\textquotesingle{}\textquotesingle{}}}\sphinxparamcomma \sphinxparam{\DUrole{n}{file\_ex}\DUrole{o}{=}\DUrole{default_value}{\textquotesingle{}.tif\textquotesingle{}}}}{}
\pysigstopsignatures
\sphinxAtStartPar
This program used to clip multiple  raster files


\paragraph{Parameters}
\label{\detokenize{akhdefo_functions:id28}}\begin{description}
\sphinxlineitem{Path\_to\_WorkingDir: str}
\sphinxAtStartPar
path to raster working directory

\sphinxlineitem{Path\_to\_AOI\_shapefile: str}
\sphinxAtStartPar
path to Area of interest in shapefile format

\sphinxlineitem{output\_CroppedDir: str }
\sphinxAtStartPar
path to save cropped raster files

\end{description}


\paragraph{Returns}
\label{\detokenize{akhdefo_functions:id29}}
\sphinxAtStartPar
cropped raster files

\end{fulllineitems}

\index{Mosaic() (in module akhdefo\_functions.Mosaic\_Crop)@\spxentry{Mosaic()}\spxextra{in module akhdefo\_functions.Mosaic\_Crop}}

\begin{fulllineitems}
\phantomsection\label{\detokenize{akhdefo_functions:akhdefo_functions.Mosaic_Crop.Mosaic}}
\pysigstartsignatures
\pysiglinewithargsret{\sphinxcode{\sphinxupquote{akhdefo\_functions.Mosaic\_Crop.}}\sphinxbfcode{\sphinxupquote{Mosaic}}}{\sphinxparam{\DUrole{n}{Path\_to\_WorkingDir}\DUrole{o}{=}\DUrole{default_value}{\textquotesingle{}\textquotesingle{}}}\sphinxparamcomma \sphinxparam{\DUrole{n}{output\_MosaicDir}\DUrole{o}{=}\DUrole{default_value}{\textquotesingle{}\textquotesingle{}}}\sphinxparamcomma \sphinxparam{\DUrole{n}{img\_mode}\DUrole{o}{=}\DUrole{default_value}{1}}\sphinxparamcomma \sphinxparam{\DUrole{n}{file\_ex}\DUrole{o}{=}\DUrole{default_value}{\textquotesingle{}.tif\textquotesingle{}}}}{}
\pysigstopsignatures
\sphinxAtStartPar
This program mosiacs raster images in geotif format as well as grab dates of the satellite image taken for further processing. 
The current version only supports PlanetLabs ortho SurfaceReflectance products.


\paragraph{Parameters}
\label{\detokenize{akhdefo_functions:id30}}
\sphinxAtStartPar
Path\_to\_WorkingDir: str

\sphinxAtStartPar
output\_MosaicDir: str
\begin{description}
\sphinxlineitem{img\_mode: int}
\sphinxAtStartPar
if img\_mode=0 the the programs mosaics only the udm maskraster images.

\sphinxAtStartPar
if img\_mode=1 the program mosiacs only  rasters data images

\end{description}


\paragraph{Returns}
\label{\detokenize{akhdefo_functions:id31}}
\sphinxAtStartPar
Mosaiced raster images

\end{fulllineitems}

\index{rasterClip() (in module akhdefo\_functions.Mosaic\_Crop)@\spxentry{rasterClip()}\spxextra{in module akhdefo\_functions.Mosaic\_Crop}}

\begin{fulllineitems}
\phantomsection\label{\detokenize{akhdefo_functions:akhdefo_functions.Mosaic_Crop.rasterClip}}
\pysigstartsignatures
\pysiglinewithargsret{\sphinxcode{\sphinxupquote{akhdefo\_functions.Mosaic\_Crop.}}\sphinxbfcode{\sphinxupquote{rasterClip}}}{\sphinxparam{\DUrole{n}{rasterpath}}\sphinxparamcomma \sphinxparam{\DUrole{n}{aoi}}\sphinxparamcomma \sphinxparam{\DUrole{n}{outfilename}}}{}
\pysigstopsignatures
\sphinxAtStartPar
Clip a raster file using an Area Of Interest (AOI) defined by a shapefile.
Uses GDAL for processing.


\paragraph{Parameters}
\label{\detokenize{akhdefo_functions:id32}}\begin{description}
\sphinxlineitem{rasterpath}{[}str{]}
\sphinxAtStartPar
Path to the input raster file.

\sphinxlineitem{aoi}{[}str{]}
\sphinxAtStartPar
Path to the Area Of Interest shapefile.

\sphinxlineitem{outfilename}{[}str{]}
\sphinxAtStartPar
Path to the output clipped raster file.

\end{description}

\end{fulllineitems}



\subsubsection{akhdefo\_functions.OpticalFlow module}
\label{\detokenize{akhdefo_functions:module-akhdefo_functions.OpticalFlow}}\label{\detokenize{akhdefo_functions:akhdefo-functions-opticalflow-module}}\index{module@\spxentry{module}!akhdefo\_functions.OpticalFlow@\spxentry{akhdefo\_functions.OpticalFlow}}\index{akhdefo\_functions.OpticalFlow@\spxentry{akhdefo\_functions.OpticalFlow}!module@\spxentry{module}}\index{DynamicChangeDetection() (in module akhdefo\_functions.OpticalFlow)@\spxentry{DynamicChangeDetection()}\spxextra{in module akhdefo\_functions.OpticalFlow}}

\begin{fulllineitems}
\phantomsection\label{\detokenize{akhdefo_functions:akhdefo_functions.OpticalFlow.DynamicChangeDetection}}
\pysigstartsignatures
\pysiglinewithargsret{\sphinxcode{\sphinxupquote{akhdefo\_functions.OpticalFlow.}}\sphinxbfcode{\sphinxupquote{DynamicChangeDetection}}}{\sphinxparam{\DUrole{n}{Path\_working\_Directory}\DUrole{o}{=}\DUrole{default_value}{\textquotesingle{}\textquotesingle{}}}\sphinxparamcomma \sphinxparam{\DUrole{n}{Path\_UDM2\_folder}\DUrole{o}{=}\DUrole{default_value}{\textquotesingle{}\textquotesingle{}}}\sphinxparamcomma \sphinxparam{\DUrole{n}{AOI\_shapefile}\DUrole{o}{=}\DUrole{default_value}{\textquotesingle{}\textquotesingle{}}}\sphinxparamcomma \sphinxparam{\DUrole{n}{Path\_to\_DEMFile}\DUrole{o}{=}\DUrole{default_value}{\textquotesingle{}\textquotesingle{}}}\sphinxparamcomma \sphinxparam{\DUrole{n}{out\_dir}\DUrole{o}{=}\DUrole{default_value}{\textquotesingle{}\textquotesingle{}}}\sphinxparamcomma \sphinxparam{\DUrole{n}{Coh\_Thresh}\DUrole{o}{=}\DUrole{default_value}{0.75}}\sphinxparamcomma \sphinxparam{\DUrole{n}{vel\_thresh}\DUrole{o}{=}\DUrole{default_value}{0.063}}\sphinxparamcomma \sphinxparam{\DUrole{n}{image\_sensor\_resolution}\DUrole{o}{=}\DUrole{default_value}{3125.0}}\sphinxparamcomma \sphinxparam{\DUrole{n}{udm\_mask\_option}\DUrole{o}{=}\DUrole{default_value}{False}}\sphinxparamcomma \sphinxparam{\DUrole{n}{cmap}\DUrole{o}{=}\DUrole{default_value}{\textquotesingle{}jet\textquotesingle{}}}\sphinxparamcomma \sphinxparam{\DUrole{n}{Median\_Filter}\DUrole{o}{=}\DUrole{default_value}{False}}\sphinxparamcomma \sphinxparam{\DUrole{n}{Set\_fig\_MinMax}\DUrole{o}{=}\DUrole{default_value}{False}}\sphinxparamcomma \sphinxparam{\DUrole{n}{show\_figure}\DUrole{o}{=}\DUrole{default_value}{False}}\sphinxparamcomma \sphinxparam{\DUrole{n}{plot\_option}\DUrole{o}{=}\DUrole{default_value}{\textquotesingle{}origional\textquotesingle{}}}\sphinxparamcomma \sphinxparam{\DUrole{n}{xres}\DUrole{o}{=}\DUrole{default_value}{10}}\sphinxparamcomma \sphinxparam{\DUrole{n}{yres}\DUrole{o}{=}\DUrole{default_value}{10}}}{}
\pysigstopsignatures
\sphinxAtStartPar
This program calculates optical flow velocity from triplets of daily optical satellite images.
Final Timeseris products will be a shapefile format using Time\_Series function after stackprep step.


\paragraph{Parameters}
\label{\detokenize{akhdefo_functions:id33}}\begin{description}
\sphinxlineitem{Path\_working\_Directory: str}
\sphinxAtStartPar
path to filtered raster images

\sphinxlineitem{Path\_UDM2\_folder: str}
\sphinxAtStartPar
path to planetlabs udm2 mask files

\sphinxlineitem{Path\_to\_DEMFile: str}
\sphinxAtStartPar
path to digital elevation model

\sphinxlineitem{AOI\_shapefile: str}
\sphinxAtStartPar
path to area of interest file in esri shapefile format

\sphinxlineitem{Coh\_Thresh: float}
\sphinxAtStartPar
similarity index threshold

\sphinxlineitem{vel\_thresh: float}
\sphinxAtStartPar
maximum velocity magnitude allowed to be measured; this will help the program to exlude rockfall velocity.
hence, only calculating displacement velocity.

\sphinxlineitem{image\_sensor\_resolution: float}
\sphinxAtStartPar
Resolution of the satallite image raster resolution in millimeters. 
for instance Planetlabs ortho imagery 1 pixel=3125.0 mm

\sphinxlineitem{udm\_mask\_option: bool}
\sphinxAtStartPar
True or False

\sphinxlineitem{cmap: str}
\sphinxAtStartPar
matplotlib colormap such as “jet”, “hsv”, etc…

\sphinxlineitem{Median\_Filter: bool}
\sphinxAtStartPar
True or False

\sphinxlineitem{Set\_fig\_MinMax: bool}
\sphinxAtStartPar
True or False

\sphinxlineitem{show\_figure: bool}
\sphinxAtStartPar
True or False

\sphinxlineitem{plot\_option: str}
\sphinxAtStartPar
“origional”,  “resampled”

\end{description}

\sphinxAtStartPar
xres: int

\sphinxAtStartPar
yres: int


\paragraph{Returns}
\label{\detokenize{akhdefo_functions:id34}}\begin{description}
\sphinxlineitem{Rasters}
\sphinxAtStartPar
velocity in X direction(EW)
Velocity in Y direction(NS)

\sphinxlineitem{Figures  }
\sphinxAtStartPar
Initial Timesereis Figures (those figures are only intermediate products needs calibration)

\end{description}

\end{fulllineitems}

\index{binary\_mask() (in module akhdefo\_functions.OpticalFlow)@\spxentry{binary\_mask()}\spxextra{in module akhdefo\_functions.OpticalFlow}}

\begin{fulllineitems}
\phantomsection\label{\detokenize{akhdefo_functions:akhdefo_functions.OpticalFlow.binary_mask}}
\pysigstartsignatures
\pysiglinewithargsret{\sphinxcode{\sphinxupquote{akhdefo\_functions.OpticalFlow.}}\sphinxbfcode{\sphinxupquote{binary\_mask}}}{\sphinxparam{\DUrole{n}{raster\_path}}\sphinxparamcomma \sphinxparam{\DUrole{n}{shape\_path}}\sphinxparamcomma \sphinxparam{\DUrole{n}{output\_path}}\sphinxparamcomma \sphinxparam{\DUrole{n}{file\_name}}}{}
\pysigstopsignatures
\sphinxAtStartPar
Function that generates a binary mask from a vector file (shp)


\paragraph{Parameters}
\label{\detokenize{akhdefo_functions:id35}}\begin{description}
\sphinxlineitem{raster\_path: str}
\sphinxAtStartPar
path to the .tif;

\sphinxlineitem{shape\_path: str}
\sphinxAtStartPar
path to the shapefile.

\sphinxlineitem{output\_path: str}
\sphinxAtStartPar
Path to save the binary mask.

\sphinxlineitem{file\_name: str}
\sphinxAtStartPar
Name of the file.

\end{description}


\paragraph{Returns}
\label{\detokenize{akhdefo_functions:id36}}\begin{description}
\sphinxlineitem{Raster: ndarray}
\sphinxAtStartPar
Binary Mask in tif format

\end{description}

\end{fulllineitems}



\subsubsection{akhdefo\_functions.Stacked\_Velocity module}
\label{\detokenize{akhdefo_functions:module-akhdefo_functions.Stacked_Velocity}}\label{\detokenize{akhdefo_functions:akhdefo-functions-stacked-velocity-module}}\index{module@\spxentry{module}!akhdefo\_functions.Stacked\_Velocity@\spxentry{akhdefo\_functions.Stacked\_Velocity}}\index{akhdefo\_functions.Stacked\_Velocity@\spxentry{akhdefo\_functions.Stacked\_Velocity}!module@\spxentry{module}}\index{stackprep() (in module akhdefo\_functions.Stacked\_Velocity)@\spxentry{stackprep()}\spxextra{in module akhdefo\_functions.Stacked\_Velocity}}

\begin{fulllineitems}
\phantomsection\label{\detokenize{akhdefo_functions:akhdefo_functions.Stacked_Velocity.stackprep}}
\pysigstartsignatures
\pysiglinewithargsret{\sphinxcode{\sphinxupquote{akhdefo\_functions.Stacked\_Velocity.}}\sphinxbfcode{\sphinxupquote{stackprep}}}{\sphinxparam{\DUrole{n}{path\_to\_flowxnFolder}\DUrole{o}{=}\DUrole{default_value}{\textquotesingle{}\textquotesingle{}}}\sphinxparamcomma \sphinxparam{\DUrole{n}{path\_toFlowynFolder}\DUrole{o}{=}\DUrole{default_value}{\textquotesingle{}\textquotesingle{}}}\sphinxparamcomma \sphinxparam{\DUrole{n}{dem}\DUrole{o}{=}\DUrole{default_value}{\textquotesingle{}\textquotesingle{}}}\sphinxparamcomma \sphinxparam{\DUrole{n}{print\_list}\DUrole{o}{=}\DUrole{default_value}{False}}\sphinxparamcomma \sphinxparam{\DUrole{n}{start\_date}\DUrole{o}{=}\DUrole{default_value}{\textquotesingle{}YYYYMMDD\textquotesingle{}}}\sphinxparamcomma \sphinxparam{\DUrole{n}{end\_date}\DUrole{o}{=}\DUrole{default_value}{\textquotesingle{}YYYYMMDD\textquotesingle{}}}\sphinxparamcomma \sphinxparam{\DUrole{n}{output\_stackedFolder}\DUrole{o}{=}\DUrole{default_value}{\textquotesingle{}\textquotesingle{}}}\sphinxparamcomma \sphinxparam{\DUrole{n}{VEL\_scale}\DUrole{o}{=}\DUrole{default_value}{(\textquotesingle{}month\textquotesingle{}, \textquotesingle{}year\textquotesingle{})}}\sphinxparamcomma \sphinxparam{\DUrole{n}{xres}\DUrole{o}{=}\DUrole{default_value}{3.125}}\sphinxparamcomma \sphinxparam{\DUrole{n}{yres}\DUrole{o}{=}\DUrole{default_value}{3.125}}\sphinxparamcomma \sphinxparam{\DUrole{n}{Velocity\_shapeFile}\DUrole{o}{=}\DUrole{default_value}{False}}\sphinxparamcomma \sphinxparam{\DUrole{n}{Resampling}\DUrole{o}{=}\DUrole{default_value}{True}}\sphinxparamcomma \sphinxparam{\DUrole{n}{Raster\_stack\_correction}\DUrole{o}{=}\DUrole{default_value}{False}}}{}
\pysigstopsignatures
\sphinxAtStartPar
This program collects velocity candiate points for time\sphinxhyphen{}series analysis.


\paragraph{Parameters}
\label{\detokenize{akhdefo_functions:id37}}\begin{description}
\sphinxlineitem{path\_to\_flowxnFolder: str}
\sphinxAtStartPar
path to folder include east\sphinxhyphen{}west velocity files

\sphinxlineitem{path\_toFlowynFolder: str}
\sphinxAtStartPar
path to folder include north\sphinxhyphen{}south velocity files

\sphinxlineitem{dem: str}
\sphinxAtStartPar
path to digital elevation model file will be used to geocode the products

\sphinxlineitem{print\_list: bool}
\sphinxAtStartPar
print list of temporal proceesed dates default is False

\sphinxlineitem{start\_date: str}
\sphinxAtStartPar
“YYYYMMDD”

\sphinxlineitem{end\_date: str}
\sphinxAtStartPar
“YYYYMMDD”

\end{description}

\sphinxAtStartPar
output\_stackedFolder: str
\begin{description}
\sphinxlineitem{VEL\_scale: str}
\sphinxAtStartPar
“month” or “year”) at this stage you can ignore this option; will be removed from future versions

\sphinxlineitem{xres: float}
\sphinxAtStartPar
x resolution

\sphinxlineitem{yres: float}
\sphinxAtStartPar
y resolution

\end{description}

\sphinxAtStartPar
Velocity\_shapeFile: bool
\begin{quote}

\sphinxAtStartPar
set to True if need to generate points for temporal deformation analysis
\end{quote}
\begin{description}
\sphinxlineitem{Resampling: bool}
\sphinxAtStartPar
if True reduce number of measurement points but faster processing

\sphinxlineitem{Raster\_stack\_correction: bool}
\sphinxAtStartPar
if True this feature computes the linearly interpolated pixel values between subsequent time\sphinxhyphen{}slices(bands) in raster stack

\end{description}


\paragraph{Returns}
\label{\detokenize{akhdefo_functions:id38}}\begin{description}
\sphinxlineitem{ESRI Shapefile}
\sphinxAtStartPar
This file include candiate velocity points for timeseries analysis

\end{description}

\end{fulllineitems}



\subsubsection{akhdefo\_functions.Unzip\_CopyFiles module}
\label{\detokenize{akhdefo_functions:module-akhdefo_functions.Unzip_CopyFiles}}\label{\detokenize{akhdefo_functions:akhdefo-functions-unzip-copyfiles-module}}\index{module@\spxentry{module}!akhdefo\_functions.Unzip\_CopyFiles@\spxentry{akhdefo\_functions.Unzip\_CopyFiles}}\index{akhdefo\_functions.Unzip\_CopyFiles@\spxentry{akhdefo\_functions.Unzip\_CopyFiles}!module@\spxentry{module}}\index{copyImage\_Data() (in module akhdefo\_functions.Unzip\_CopyFiles)@\spxentry{copyImage\_Data()}\spxextra{in module akhdefo\_functions.Unzip\_CopyFiles}}

\begin{fulllineitems}
\phantomsection\label{\detokenize{akhdefo_functions:akhdefo_functions.Unzip_CopyFiles.copyImage_Data}}
\pysigstartsignatures
\pysiglinewithargsret{\sphinxcode{\sphinxupquote{akhdefo\_functions.Unzip\_CopyFiles.}}\sphinxbfcode{\sphinxupquote{copyImage\_Data}}}{\sphinxparam{\DUrole{n}{path\_to\_unzipped\_folders}\DUrole{o}{=}\DUrole{default_value}{\textquotesingle{}\textquotesingle{}}}\sphinxparamcomma \sphinxparam{\DUrole{n}{Path\_to\_raster\_tifs}\DUrole{o}{=}\DUrole{default_value}{\textquotesingle{}\textquotesingle{}}}\sphinxparamcomma \sphinxparam{\DUrole{n}{file\_ext}\DUrole{o}{=}\DUrole{default_value}{\textquotesingle{}\textquotesingle{}}}}{}
\pysigstopsignatures
\sphinxAtStartPar
This program copy all the data images into one single folder from subdirectories.


\paragraph{Parameters}
\label{\detokenize{akhdefo_functions:id39}}
\sphinxAtStartPar
path\_to\_unzipped\_folders : str

\sphinxAtStartPar
Path\_to\_raster\_tifs : str
\begin{description}
\sphinxlineitem{file\_ext: str}
\sphinxAtStartPar
ignore this option if you want to copy all the files or
type extension of the files of interestest such as “.tif”

\end{description}


\paragraph{Returns}
\label{\detokenize{akhdefo_functions:id40}}
\sphinxAtStartPar
rasters

\end{fulllineitems}

\index{copyUDM2\_Mask\_Data() (in module akhdefo\_functions.Unzip\_CopyFiles)@\spxentry{copyUDM2\_Mask\_Data()}\spxextra{in module akhdefo\_functions.Unzip\_CopyFiles}}

\begin{fulllineitems}
\phantomsection\label{\detokenize{akhdefo_functions:akhdefo_functions.Unzip_CopyFiles.copyUDM2_Mask_Data}}
\pysigstartsignatures
\pysiglinewithargsret{\sphinxcode{\sphinxupquote{akhdefo\_functions.Unzip\_CopyFiles.}}\sphinxbfcode{\sphinxupquote{copyUDM2\_Mask\_Data}}}{\sphinxparam{\DUrole{n}{path\_to\_unzipped\_folders}\DUrole{o}{=}\DUrole{default_value}{\textquotesingle{}\textquotesingle{}}}\sphinxparamcomma \sphinxparam{\DUrole{n}{Path\_to\_UDM2raster\_tifs}\DUrole{o}{=}\DUrole{default_value}{\textquotesingle{}\textquotesingle{}}}}{}
\pysigstopsignatures
\sphinxAtStartPar
This program copy all  raster masks.


\paragraph{Parameters}
\label{\detokenize{akhdefo_functions:id41}}\begin{description}
\sphinxlineitem{path\_to\_unzipped\_folders}{[}str{]}
\sphinxAtStartPar
file extension must end with udm2\_clip.tif

\end{description}

\sphinxAtStartPar
Path\_to\_UDM2raster\_tifs : str


\paragraph{Returns}
\label{\detokenize{akhdefo_functions:id42}}
\sphinxAtStartPar
rasters

\end{fulllineitems}

\index{read\_data\_prep() (in module akhdefo\_functions.Unzip\_CopyFiles)@\spxentry{read\_data\_prep()}\spxextra{in module akhdefo\_functions.Unzip\_CopyFiles}}

\begin{fulllineitems}
\phantomsection\label{\detokenize{akhdefo_functions:akhdefo_functions.Unzip_CopyFiles.read_data_prep}}
\pysigstartsignatures
\pysiglinewithargsret{\sphinxcode{\sphinxupquote{akhdefo\_functions.Unzip\_CopyFiles.}}\sphinxbfcode{\sphinxupquote{read\_data\_prep}}}{\sphinxparam{\DUrole{n}{zip\_dir}\DUrole{o}{=}\DUrole{default_value}{\textquotesingle{}\textquotesingle{}}}\sphinxparamcomma \sphinxparam{\DUrole{n}{image\_dir}\DUrole{o}{=}\DUrole{default_value}{\textquotesingle{}image\_dir\textquotesingle{}}}\sphinxparamcomma \sphinxparam{\DUrole{n}{ext\_image\_file}\DUrole{o}{=}\DUrole{default_value}{\textquotesingle{}BGRN\_SR\_clip.tif\textquotesingle{}}}\sphinxparamcomma \sphinxparam{\DUrole{n}{udm\_mask\_dir}\DUrole{o}{=}\DUrole{default_value}{\textquotesingle{}udm\_mask\_dir\textquotesingle{}}}\sphinxparamcomma \sphinxparam{\DUrole{n}{ext\_udm\_mask\_file}\DUrole{o}{=}\DUrole{default_value}{\textquotesingle{}udm2\_clip.tif\textquotesingle{}}}\sphinxparamcomma \sphinxparam{\DUrole{n}{search\_string}\DUrole{o}{=}\DUrole{default_value}{None}}}{}
\pysigstopsignatures
\sphinxAtStartPar
This program reads planetlabs orthoimagery in zipfolder format
create different directory for the raster images and corresponding unusable data masks(udm)
the udm mask include snow, haze, etc… see planetslabs udm mask types for further details


\paragraph{Parameters}
\label{\detokenize{akhdefo_functions:id43}}\begin{description}
\sphinxlineitem{zip\_dir: str}
\sphinxAtStartPar
path to directory contains all the zipfiles

\sphinxlineitem{image\_dir: str}
\sphinxAtStartPar
path to folder contains only raster data images

\sphinxlineitem{udm\_mask\_dir: str}
\sphinxAtStartPar
path to folder contains only udm mask ratsers

\end{description}


\paragraph{Returns}
\label{\detokenize{akhdefo_functions:id44}}
\sphinxAtStartPar
unzip folder
image\_dir
udm\_mask\_dir

\end{fulllineitems}

\index{unzip() (in module akhdefo\_functions.Unzip\_CopyFiles)@\spxentry{unzip()}\spxextra{in module akhdefo\_functions.Unzip\_CopyFiles}}

\begin{fulllineitems}
\phantomsection\label{\detokenize{akhdefo_functions:akhdefo_functions.Unzip_CopyFiles.unzip}}
\pysigstartsignatures
\pysiglinewithargsret{\sphinxcode{\sphinxupquote{akhdefo\_functions.Unzip\_CopyFiles.}}\sphinxbfcode{\sphinxupquote{unzip}}}{\sphinxparam{\DUrole{n}{zipdir}}\sphinxparamcomma \sphinxparam{\DUrole{n}{dst\_dir}}}{}
\pysigstopsignatures
\sphinxAtStartPar
This program unzips all the zip products into one folder


\paragraph{Parameters}
\label{\detokenize{akhdefo_functions:id45}}\begin{description}
\sphinxlineitem{zipdir}{[}str{]}
\sphinxAtStartPar
path to directory contains all the zipfiles

\sphinxlineitem{dst\_dir}{[}str{]}
\sphinxAtStartPar
path to destination folder to copy all unzipped products.

\end{description}


\paragraph{Returns}
\label{\detokenize{akhdefo_functions:id46}}
\sphinxAtStartPar
unzip folder

\end{fulllineitems}



\subsubsection{akhdefo\_functions.video\_streamer module}
\label{\detokenize{akhdefo_functions:module-akhdefo_functions.video_streamer}}\label{\detokenize{akhdefo_functions:akhdefo-functions-video-streamer-module}}\index{module@\spxentry{module}!akhdefo\_functions.video\_streamer@\spxentry{akhdefo\_functions.video\_streamer}}\index{akhdefo\_functions.video\_streamer@\spxentry{akhdefo\_functions.video\_streamer}!module@\spxentry{module}}\index{VideoStreamer (class in akhdefo\_functions.video\_streamer)@\spxentry{VideoStreamer}\spxextra{class in akhdefo\_functions.video\_streamer}}

\begin{fulllineitems}
\phantomsection\label{\detokenize{akhdefo_functions:akhdefo_functions.video_streamer.VideoStreamer}}
\pysigstartsignatures
\pysigline{\sphinxbfcode{\sphinxupquote{class\DUrole{w}{ }}}\sphinxcode{\sphinxupquote{akhdefo\_functions.video\_streamer.}}\sphinxbfcode{\sphinxupquote{VideoStreamer}}}
\pysigstopsignatures
\sphinxAtStartPar
Bases: \sphinxcode{\sphinxupquote{object}}

\sphinxAtStartPar
A Flask\sphinxhyphen{}based web application for streaming video from a camera or video file.

\sphinxAtStartPar
This class encapsulates the functionalities of streaming video using the OpenCV library
and custom processing from the ‘akhdefo\_functions’ module. It sets up a Flask server
with routes to handle video streaming and the main index page.
\begin{description}
\sphinxlineitem{Attributes:}
\sphinxAtStartPar
args: Command\sphinxhyphen{}line arguments for configuration.
app: The Flask application instance.

\sphinxlineitem{Methods:}
\sphinxAtStartPar
generate\_frames1(): Generates and yields encoded frames with specific processing parameters.
generate\_frames2(): Similar to generate\_frames1() but with different parameters.
setup\_routes(): Sets up the Flask routes for the application.
run(): Starts the Flask server.

\sphinxlineitem{Examples:}
\sphinxAtStartPar
To run the application, first ensure that all dependencies are installed. Then, execute
the script from the command line with the necessary arguments. For example:
\begin{quote}

\sphinxAtStartPar
python video\_streamer.py \textendash{}port 5000
\end{quote}

\sphinxAtStartPar
This command runs the Flask server on port 5000. You can access the main page at
\sphinxtitleref{http://localhost:5000/}. For streaming video, access \sphinxtitleref{http://localhost:5000/video1}
or \sphinxtitleref{http://localhost:5000/video2}.

\sphinxlineitem{Note:}
\sphinxAtStartPar
Ensure that the ‘akhdefo\_functions’ module and other dependencies are properly installed
and accessible to the script. If using custom modules, they should be in the same directory
as the script or in a location where Python can find them.

\end{description}
\index{generate\_frames1() (akhdefo\_functions.video\_streamer.VideoStreamer method)@\spxentry{generate\_frames1()}\spxextra{akhdefo\_functions.video\_streamer.VideoStreamer method}}

\begin{fulllineitems}
\phantomsection\label{\detokenize{akhdefo_functions:akhdefo_functions.video_streamer.VideoStreamer.generate_frames1}}
\pysigstartsignatures
\pysiglinewithargsret{\sphinxbfcode{\sphinxupquote{generate\_frames1}}}{}{}
\pysigstopsignatures
\end{fulllineitems}

\index{generate\_frames2() (akhdefo\_functions.video\_streamer.VideoStreamer method)@\spxentry{generate\_frames2()}\spxextra{akhdefo\_functions.video\_streamer.VideoStreamer method}}

\begin{fulllineitems}
\phantomsection\label{\detokenize{akhdefo_functions:akhdefo_functions.video_streamer.VideoStreamer.generate_frames2}}
\pysigstartsignatures
\pysiglinewithargsret{\sphinxbfcode{\sphinxupquote{generate\_frames2}}}{}{}
\pysigstopsignatures
\end{fulllineitems}

\index{run() (akhdefo\_functions.video\_streamer.VideoStreamer method)@\spxentry{run()}\spxextra{akhdefo\_functions.video\_streamer.VideoStreamer method}}

\begin{fulllineitems}
\phantomsection\label{\detokenize{akhdefo_functions:akhdefo_functions.video_streamer.VideoStreamer.run}}
\pysigstartsignatures
\pysiglinewithargsret{\sphinxbfcode{\sphinxupquote{run}}}{}{}
\pysigstopsignatures
\end{fulllineitems}

\index{setup\_routes() (akhdefo\_functions.video\_streamer.VideoStreamer method)@\spxentry{setup\_routes()}\spxextra{akhdefo\_functions.video\_streamer.VideoStreamer method}}

\begin{fulllineitems}
\phantomsection\label{\detokenize{akhdefo_functions:akhdefo_functions.video_streamer.VideoStreamer.setup_routes}}
\pysigstartsignatures
\pysiglinewithargsret{\sphinxbfcode{\sphinxupquote{setup\_routes}}}{}{}
\pysigstopsignatures
\end{fulllineitems}


\end{fulllineitems}



\subsubsection{Module contents}
\label{\detokenize{akhdefo_functions:module-akhdefo_functions}}\label{\detokenize{akhdefo_functions:module-contents}}\index{module@\spxentry{module}!akhdefo\_functions@\spxentry{akhdefo\_functions}}\index{akhdefo\_functions@\spxentry{akhdefo\_functions}!module@\spxentry{module}}
\sphinxAtStartPar
akhdefo\_functions

\sphinxAtStartPar
collection of python modules performs geospatial image processing to moniter land deformation


\chapter{Tutorials!}
\label{\detokenize{index:tutorials}}
\sphinxstepscope


\section{Morenny Rock Glacier Optical Imagery Example}
\label{\detokenize{notebooks/RockGlacier_optical:Morenny-Rock-Glacier-Optical-Imagery-Example}}\label{\detokenize{notebooks/RockGlacier_optical::doc}}

\subsection{Download data from planet.com website}
\label{\detokenize{notebooks/RockGlacier_optical:Download-data-from-planet.com-website}}
\sphinxAtStartPar
Create an Account or Log In: To access data, you generally need an account with \sphinxurl{https://www.planet.com} . If you don’t have one, you’ll need to sign up. Otherwise, log in with your existing credentials.

\noindent\sphinxincludegraphics[width=1690\sphinxpxdimen,height=851\sphinxpxdimen]{{planet_screenshot1}.png}


\subsection{Import AkhDefo Packages.}
\label{\detokenize{notebooks/RockGlacier_optical:Import-AkhDefo-Packages.}}
\begin{sphinxuseclass}{nbinput}
\begin{sphinxuseclass}{nblast}
{
\begin{sphinxVerbatim}[commandchars=\\\{\}]
\llap{\color{nbsphinxin}[ ]:\,\hspace{\fboxrule}\hspace{\fboxsep}}
\PYG{c+c1}{\PYGZsh{}Functions for data preparations}
\PYG{k+kn}{import} \PYG{n+nn}{akhdefo\PYGZus{}functions}
\PYG{k+kn}{from} \PYG{n+nn}{akhdefo\PYGZus{}functions} \PYG{k+kn}{import} \PYG{n}{akhdefo\PYGZus{}download\PYGZus{}planet}
\PYG{k+kn}{from} \PYG{n+nn}{akhdefo\PYGZus{}functions} \PYG{k+kn}{import} \PYG{n}{read\PYGZus{}data\PYGZus{}prep}
\PYG{k+kn}{from} \PYG{n+nn}{akhdefo\PYGZus{}functions} \PYG{k+kn}{import} \PYG{n}{move\PYGZus{}files\PYGZus{}with\PYGZus{}string}
\PYG{k+kn}{from} \PYG{n+nn}{akhdefo\PYGZus{}functions} \PYG{k+kn}{import} \PYG{n}{Crop\PYGZus{}to\PYGZus{}AOI}

\PYG{c+c1}{\PYGZsh{}Functions data processing}
\PYG{k+kn}{from} \PYG{n+nn}{akhdefo\PYGZus{}functions} \PYG{k+kn}{import} \PYG{n}{scatter\PYGZus{}area\PYGZus{}mask}
\PYG{k+kn}{from} \PYG{n+nn}{akhdefo\PYGZus{}functions} \PYG{k+kn}{import} \PYG{n}{Raster\PYGZus{}Correction}
\PYG{k+kn}{from} \PYG{n+nn}{akhdefo\PYGZus{}functions} \PYG{k+kn}{import} \PYG{n}{Optical\PYGZus{}flow\PYGZus{}akhdefo}

\PYG{c+c1}{\PYGZsh{}Functions for shapefile point data interpolation to raster geotif}
\PYG{c+c1}{\PYGZsh{}Function Autovariogram use Kriging method it fits the data to different variogram models and automatically use the best fit variogram model (Computationally Heavy but you can process data in chunks)}
\PYG{k+kn}{from} \PYG{n+nn}{akhdefo\PYGZus{}functions} \PYG{k+kn}{import} \PYG{n}{Auto\PYGZus{}Variogram}

\PYG{c+c1}{\PYGZsh{} interpoate\PYGZus{}xyz uses the following interpolation methods (nearest, linear and cubic)}
\PYG{k+kn}{from} \PYG{n+nn}{akhdefo\PYGZus{}functions} \PYG{k+kn}{import} \PYG{n}{interpolate\PYGZus{}xyz}

\PYG{c+c1}{\PYGZsh{}Functions for data Visualization and Analysis}
\PYG{k+kn}{from} \PYG{n+nn}{akhdefo\PYGZus{}functions} \PYG{k+kn}{import} \PYG{n}{MeanProducts\PYGZus{}plot\PYGZus{}ts}
\PYG{k+kn}{from} \PYG{n+nn}{akhdefo\PYGZus{}functions} \PYG{k+kn}{import} \PYG{n}{plot\PYGZus{}stackNetwork}
\PYG{k+kn}{from} \PYG{n+nn}{akhdefo\PYGZus{}functions} \PYG{k+kn}{import} \PYG{n}{akhdefo\PYGZus{}viewer}
\PYG{k+kn}{from} \PYG{n+nn}{akhdefo\PYGZus{}functions} \PYG{k+kn}{import} \PYG{n}{akhdefo\PYGZus{}dashApp}
\end{sphinxVerbatim}
}

\end{sphinxuseclass}
\end{sphinxuseclass}

\subsection{Download Data using Planetlab API Account}
\label{\detokenize{notebooks/RockGlacier_optical:Download-Data-using-Planetlab-API-Account}}
\begin{sphinxuseclass}{nbinput}
\begin{sphinxuseclass}{nblast}
{
\begin{sphinxVerbatim}[commandchars=\\\{\}]
\llap{\color{nbsphinxin}[ ]:\,\hspace{\fboxrule}\hspace{\fboxsep}}
\PYG{l+s+sd}{\PYGZsq{}\PYGZsq{}\PYGZsq{}}

\PYG{l+s+sd}{First Run the below command and make sure \PYGZob{}download=False\PYGZcb{} to see list of available dataset.}

\PYG{l+s+sd}{Its highly recommended to use data frame same path for the entire time\PYGZhy{}series for more accurate results}

\PYG{l+s+sd}{\PYGZsq{}\PYGZsq{}\PYGZsq{}}

\PYG{c+c1}{\PYGZsh{}uncomment below to download data using planet lab api key}

\PYG{l+s+sd}{\PYGZsq{}\PYGZsq{}\PYGZsq{}}
\PYG{l+s+sd}{akhdefo\PYGZus{}download\PYGZus{}planet(planet\PYGZus{}api\PYGZus{}key=\PYGZsq{}\PYGZsq{}, AOI=\PYGZsq{}./data/morenny/AOI.shp\PYGZsq{}, start\PYGZus{}date=\PYGZsq{}May 1, 2018\PYGZsq{}, end\PYGZus{}date=\PYGZsq{} September 30 2022\PYGZsq{}, limit=5, item\PYGZus{}type=\PYGZsq{}PSScene\PYGZsq{},}
\PYG{l+s+sd}{                        product\PYGZus{}bundle=\PYGZsq{}analytic\PYGZus{}sr\PYGZus{}udm2\PYGZsq{},}
\PYG{l+s+sd}{                        clear\PYGZus{}percent=90, cloud\PYGZus{}filter=0.1,}
\PYG{l+s+sd}{                        output\PYGZus{}folder=\PYGZsq{}raw\PYGZus{}data\PYGZsq{},}
\PYG{l+s+sd}{                        clip\PYGZus{}flag=True, download\PYGZus{}data=False}

\PYG{l+s+sd}{\PYGZsq{}\PYGZsq{}\PYGZsq{}}
\PYG{c+c1}{\PYGZsh{} for this excecise we download data from planet labs explorer web application}
\PYG{c+c1}{\PYGZsh{}Lets Navigate to the zip folder directory}

\PYG{k+kn}{import} \PYG{n+nn}{os}

\PYG{c+c1}{\PYGZsh{} Replace \PYGZsq{}your\PYGZus{}directory\PYGZus{}path\PYGZsq{} with the path of your directory}
\PYG{n}{zip\PYGZus{}dir} \PYG{o}{=} \PYG{l+s+s1}{\PYGZsq{}}\PYG{l+s+s1}{./data/morenny/optical/zip\PYGZus{}folder/}\PYG{l+s+s1}{\PYGZsq{}}

\PYG{c+c1}{\PYGZsh{} List all files and directories in the specified path}
\PYG{n}{contents} \PYG{o}{=} \PYG{n}{os}\PYG{o}{.}\PYG{n}{listdir}\PYG{p}{(}\PYG{n}{zip\PYGZus{}dir}\PYG{p}{)}

\PYG{c+c1}{\PYGZsh{} Print the contents}
\PYG{k}{for} \PYG{n}{item} \PYG{o+ow}{in} \PYG{n}{contents}\PYG{p}{:}
    \PYG{n+nb}{print}\PYG{p}{(}\PYG{n}{item}\PYG{p}{)}

\PYG{c+c1}{\PYGZsh{} The zip folder inside the zip folder directory inclue cropped planet lab imagery between 2018 to 2022. lets extract the data and look at the images}
\end{sphinxVerbatim}
}

\end{sphinxuseclass}
\end{sphinxuseclass}

\subsection{Extract downloaded zip files}
\label{\detokenize{notebooks/RockGlacier_optical:Extract-downloaded-zip-files}}
\begin{sphinxuseclass}{nbinput}
\begin{sphinxuseclass}{nblast}
{
\begin{sphinxVerbatim}[commandchars=\\\{\}]
\llap{\color{nbsphinxin}[ ]:\,\hspace{\fboxrule}\hspace{\fboxsep}}\PYG{c+c1}{\PYGZsh{}This functions extracts data from all the granuples zip directory.}
\PYG{n}{read\PYGZus{}data\PYGZus{}prep}\PYG{p}{(}\PYG{n}{zip\PYGZus{}dir}\PYG{o}{=}\PYG{n}{zip\PYGZus{}dir}\PYG{p}{,}
                                 \PYG{n}{image\PYGZus{}dir}\PYG{o}{=}\PYG{l+s+s1}{\PYGZsq{}}\PYG{l+s+s1}{./data/morenny/optical/image\PYGZus{}dir}\PYG{l+s+s1}{\PYGZsq{}}\PYG{p}{,}
                                 \PYG{n}{ext\PYGZus{}image\PYGZus{}file}\PYG{o}{=}\PYG{l+s+s1}{\PYGZsq{}}\PYG{l+s+s1}{\PYGZus{}SR\PYGZus{}clip.tif}\PYG{l+s+s1}{\PYGZsq{}}\PYG{p}{,} \PYG{n}{udm\PYGZus{}mask\PYGZus{}dir}\PYG{o}{=}\PYG{l+s+s1}{\PYGZsq{}}\PYG{l+s+s1}{./data/morenny/optical/udm\PYGZus{}mask\PYGZus{}dir}\PYG{l+s+s1}{\PYGZsq{}}\PYG{p}{,} \PYG{n}{ext\PYGZus{}udm\PYGZus{}mask\PYGZus{}file}\PYG{o}{=}\PYG{l+s+s1}{\PYGZsq{}}\PYG{l+s+s1}{udm2\PYGZus{}clip.tif}\PYG{l+s+s1}{\PYGZsq{}}\PYG{p}{)}
\end{sphinxVerbatim}
}

\end{sphinxuseclass}
\end{sphinxuseclass}
\begin{sphinxuseclass}{nbinput}
\begin{sphinxuseclass}{nblast}
{
\begin{sphinxVerbatim}[commandchars=\\\{\}]
\llap{\color{nbsphinxin}[ ]:\,\hspace{\fboxrule}\hspace{\fboxsep}}\PYG{c+c1}{\PYGZsh{} This function will move all the data from each granules subdir to a single folder}
\PYG{c+c1}{\PYGZsh{} move\PYGZus{}files\PYGZus{}with\PYGZus{}string(source\PYGZus{}dir=\PYGZsq{}./data/morenny/optical/RS\PYGZus{}dir\PYGZus{}path129/\PYGZsq{}, dest\PYGZus{}dir=\PYGZsq{}./data/morenny/optical/RS\PYGZus{}path129\PYGZsq{}, search\PYGZus{}string=\PYGZsq{}.tif\PYGZsq{})}
\PYG{c+c1}{\PYGZsh{} move\PYGZus{}files\PYGZus{}with\PYGZus{}string(source\PYGZus{}dir=\PYGZsq{}./data/morenny/optical/scatter\PYGZus{}dir/\PYGZsq{}, dest\PYGZus{}dir=\PYGZsq{}./data/morenny/optical/scatter\PYGZus{}rs\PYGZsq{}, search\PYGZus{}string=\PYGZsq{}.tif\PYGZsq{})}

\end{sphinxVerbatim}
}

\end{sphinxuseclass}
\end{sphinxuseclass}
\begin{sphinxuseclass}{nbinput}
\begin{sphinxuseclass}{nblast}
{
\begin{sphinxVerbatim}[commandchars=\\\{\}]
\llap{\color{nbsphinxin}[ ]:\,\hspace{\fboxrule}\hspace{\fboxsep}}
\PYG{l+s+sd}{\PYGZsq{}\PYGZsq{}\PYGZsq{}}
\PYG{l+s+sd}{Crop all raster to Area of interest bounding box}


\PYG{l+s+sd}{\PYGZsq{}\PYGZsq{}\PYGZsq{}}
\PYG{n}{Crop\PYGZus{}to\PYGZus{}AOI}\PYG{p}{(}\PYG{n}{Path\PYGZus{}to\PYGZus{}WorkingDir}\PYG{o}{=}\PYG{l+s+s1}{\PYGZsq{}}\PYG{l+s+s1}{./data/morenny/optical/image\PYGZus{}dir}\PYG{l+s+s1}{\PYGZsq{}}\PYG{p}{,}
                              \PYG{n}{Path\PYGZus{}to\PYGZus{}AOI\PYGZus{}shapefile}\PYG{o}{=}\PYG{l+s+s1}{\PYGZsq{}}\PYG{l+s+s1}{./data/morenny/AOI.shp}\PYG{l+s+s1}{\PYGZsq{}}\PYG{p}{,}
                              \PYG{n}{output\PYGZus{}CroppedDir}\PYG{o}{=}\PYG{l+s+s1}{\PYGZsq{}}\PYG{l+s+s1}{./data/morenny/optical/image\PYGZus{}dir\PYGZus{}aoi}\PYG{l+s+s1}{\PYGZsq{}}\PYG{p}{,} \PYG{n}{file\PYGZus{}ex}\PYG{o}{=}\PYG{l+s+s1}{\PYGZsq{}}\PYG{l+s+s1}{.tif}\PYG{l+s+s1}{\PYGZsq{}}\PYG{p}{)}


\end{sphinxVerbatim}
}

\end{sphinxuseclass}
\end{sphinxuseclass}

\subsection{Preparing raster for optical flow and displacement calculation!}
\label{\detokenize{notebooks/RockGlacier_optical:Preparing-raster-for-optical-flow-and-displacement-calculation!}}
\sphinxAtStartPar
The data needs to be converted to “uint8” format and be calibrated and normalized.

\begin{sphinxuseclass}{nbinput}
\begin{sphinxuseclass}{nblast}
{
\begin{sphinxVerbatim}[commandchars=\\\{\}]
\llap{\color{nbsphinxin}[ ]:\,\hspace{\fboxrule}\hspace{\fboxsep}}\PYG{k+kn}{import} \PYG{n+nn}{warnings}
\PYG{n}{warnings}\PYG{o}{.}\PYG{n}{filterwarnings}\PYG{p}{(}\PYG{l+s+s2}{\PYGZdq{}}\PYG{l+s+s2}{ignore}\PYG{l+s+s2}{\PYGZdq{}}\PYG{p}{)}


\PYG{n}{Raster\PYGZus{}Correction}\PYG{p}{(}\PYG{n}{input\PYGZus{}path}\PYG{o}{=}\PYG{l+s+s2}{\PYGZdq{}}\PYG{l+s+s2}{./data/morenny/optical/image\PYGZus{}dir\PYGZus{}aoi}\PYG{l+s+s2}{\PYGZdq{}}\PYG{p}{,} \PYG{n}{output\PYGZus{}path}\PYG{o}{=}\PYG{l+s+s2}{\PYGZdq{}}\PYG{l+s+s2}{./data/morenny/optical/image\PYGZus{}dir\PYGZus{}aoi\PYGZus{}filt}\PYG{l+s+s2}{\PYGZdq{}}\PYG{p}{,} \PYG{n}{limit}\PYG{o}{=}\PYG{k+kc}{None}\PYG{p}{,}
                                    \PYG{n}{lowpass\PYGZus{}kernel\PYGZus{}size}\PYG{o}{=}\PYG{k+kc}{None}\PYG{p}{,} \PYG{n}{bilateral\PYGZus{}win\PYGZus{}size}\PYG{o}{=}\PYG{l+m+mi}{7}\PYG{p}{,} \PYG{n}{bilateral\PYGZus{}sigma\PYGZus{}color}\PYG{o}{=}\PYG{l+m+mi}{75}\PYG{p}{,}
                                    \PYG{n}{bilateral\PYGZus{}sigma\PYGZus{}spatial}\PYG{o}{=}\PYG{l+m+mi}{75}\PYG{p}{,}
                                    \PYG{n}{clip\PYGZus{}percentiles}\PYG{o}{=}\PYG{p}{[}\PYG{l+m+mi}{2}\PYG{p}{,} \PYG{l+m+mi}{98}\PYG{p}{]}\PYG{p}{,} \PYG{n}{optical}\PYG{o}{=}\PYG{k+kc}{True}\PYG{p}{,}
                                    \PYG{n}{scale}\PYG{o}{=}\PYG{k+kc}{None}\PYG{p}{,} \PYG{n}{Vegetation\PYGZus{}mask}\PYG{o}{=}\PYG{k+kc}{None}\PYG{p}{)}
\end{sphinxVerbatim}
}

\end{sphinxuseclass}
\end{sphinxuseclass}

\subsection{Applying optical flow and producing time\sphinxhyphen{}series deformation products.}
\label{\detokenize{notebooks/RockGlacier_optical:Applying-optical-flow-and-producing-time-series-deformation-products.}}
\begin{sphinxuseclass}{nbinput}
\begin{sphinxuseclass}{nblast}
{
\begin{sphinxVerbatim}[commandchars=\\\{\}]
\llap{\color{nbsphinxin}[ ]:\,\hspace{\fboxrule}\hspace{\fboxsep}}\PYG{k+kn}{import} \PYG{n+nn}{warnings}
\PYG{n}{warnings}\PYG{o}{.}\PYG{n}{filterwarnings}\PYG{p}{(}\PYG{l+s+s2}{\PYGZdq{}}\PYG{l+s+s2}{ignore}\PYG{l+s+s2}{\PYGZdq{}}\PYG{p}{)}

\PYG{n}{Optical\PYGZus{}flow\PYGZus{}akhdefo}\PYG{p}{(}\PYG{n}{input\PYGZus{}dir}\PYG{o}{=}\PYG{l+s+s1}{\PYGZsq{}}\PYG{l+s+s1}{./data/morenny/optical/image\PYGZus{}dir\PYGZus{}aoi\PYGZus{}filt}\PYG{l+s+s1}{\PYGZsq{}}\PYG{p}{,} \PYG{n}{output\PYGZus{}dir}\PYG{o}{=}\PYG{l+s+s1}{\PYGZsq{}}\PYG{l+s+s1}{./data/morenny/optical/geo}\PYG{l+s+s1}{\PYGZsq{}}\PYG{p}{,} \PYG{n}{AOI}\PYG{o}{=}\PYG{l+s+s1}{\PYGZsq{}}\PYG{l+s+s1}{./data/morenny/AOI.shp}\PYG{l+s+s1}{\PYGZsq{}}\PYG{p}{,} \PYG{n}{zscore\PYGZus{}threshold}\PYG{o}{=}\PYG{l+m+mi}{2}\PYG{p}{,}
                                       \PYG{n}{ssim\PYGZus{}thresh}\PYG{o}{=}\PYG{l+m+mf}{0.65}\PYG{p}{,} \PYG{n}{image\PYGZus{}resolution}\PYG{o}{=}\PYG{l+s+s1}{\PYGZsq{}}\PYG{l+s+s1}{3.125m}\PYG{l+s+s1}{\PYGZsq{}}\PYG{p}{,} \PYG{n}{interpolate}\PYG{o}{=}\PYG{l+s+s1}{\PYGZsq{}}\PYG{l+s+s1}{kriging}\PYG{l+s+s1}{\PYGZsq{}}\PYG{p}{,}
                                       \PYG{n}{show\PYGZus{}figure}\PYG{o}{=}\PYG{k+kc}{False}\PYG{p}{,} \PYG{n}{point\PYGZus{}size}\PYG{o}{=}\PYG{l+m+mi}{2}\PYG{p}{,} \PYG{n}{dem\PYGZus{}path}\PYG{o}{=}\PYG{l+s+s1}{\PYGZsq{}}\PYG{l+s+s1}{./data/morenny/dem.tif}\PYG{l+s+s1}{\PYGZsq{}}\PYG{p}{,} \PYG{n}{smoothing\PYGZus{}kernel\PYGZus{}size}\PYG{o}{=}\PYG{l+m+mi}{2}\PYG{p}{,}
                                       \PYG{n}{Vegetation\PYGZus{}mask}\PYG{o}{=}\PYG{k+kc}{None}\PYG{p}{,} \PYG{n}{VEL\PYGZus{}scale}\PYG{o}{=}\PYG{l+s+s1}{\PYGZsq{}}\PYG{l+s+s1}{year}\PYG{l+s+s1}{\PYGZsq{}}\PYG{p}{,} \PYG{n}{VEL\PYGZus{}Mode}\PYG{o}{=}\PYG{l+s+s1}{\PYGZsq{}}\PYG{l+s+s1}{linear}\PYG{l+s+s1}{\PYGZsq{}}\PYG{p}{,} \PYG{n}{good\PYGZus{}match\PYGZus{}option}\PYG{o}{=}\PYG{l+m+mf}{0.75}\PYG{p}{,}
                                       \PYG{n}{hillshade\PYGZus{}option}\PYG{o}{=}\PYG{k+kc}{True}\PYG{p}{,} \PYG{n}{shapefile\PYGZus{}output}\PYG{o}{=}\PYG{k+kc}{True}\PYG{p}{,} \PYG{n}{max\PYGZus{}triplet\PYGZus{}interval}\PYG{o}{=}\PYG{l+m+mi}{300}\PYG{p}{,} \PYG{n}{pixel\PYGZus{}size}\PYG{o}{=}\PYG{l+m+mi}{10}\PYG{p}{,} \PYG{n}{num\PYGZus{}chunks}\PYG{o}{=}\PYG{l+m+mi}{10}\PYG{p}{,} \PYG{n}{overlap\PYGZus{}percentage}\PYG{o}{=}\PYG{l+m+mi}{0}\PYG{p}{,} \PYG{n}{pyr\PYGZus{}scale}\PYG{o}{=}\PYG{l+m+mf}{0.5}\PYG{p}{,} \PYG{n}{levels}\PYG{o}{=}\PYG{l+m+mi}{15}\PYG{p}{,}
                                       \PYG{n}{winsize}\PYG{o}{=}\PYG{l+m+mi}{128}\PYG{p}{,} \PYG{n}{iterations}\PYG{o}{=}\PYG{l+m+mi}{7}\PYG{p}{,} \PYG{n}{poly\PYGZus{}n}\PYG{o}{=}\PYG{l+m+mi}{7}\PYG{p}{,} \PYG{n}{poly\PYGZus{}sigma}\PYG{o}{=}\PYG{l+m+mf}{1.5}\PYG{p}{,}
                                       \PYG{n}{flags}\PYG{o}{=}\PYG{l+m+mi}{1}\PYG{p}{,} \PYG{n}{master\PYGZus{}reference}\PYG{o}{=}\PYG{l+s+s1}{\PYGZsq{}}\PYG{l+s+s1}{single}\PYG{l+s+s1}{\PYGZsq{}}\PYG{p}{,}
                                       \PYG{n}{selection\PYGZus{}Mode}\PYG{o}{=}\PYG{l+s+s1}{\PYGZsq{}}\PYG{l+s+s1}{pair}\PYG{l+s+s1}{\PYGZsq{}}\PYG{p}{,} \PYG{n}{start\PYGZus{}date}\PYG{o}{=}\PYG{k+kc}{None}\PYG{p}{,} \PYG{n}{end\PYGZus{}date}\PYG{o}{=}\PYG{k+kc}{None}\PYG{p}{,} \PYG{n}{krig\PYGZus{}method}\PYG{o}{=}\PYG{l+s+s1}{\PYGZsq{}}\PYG{l+s+s1}{universal}\PYG{l+s+s1}{\PYGZsq{}}\PYG{p}{,}\PYG{n}{spatial\PYGZus{}ref}\PYG{o}{=}\PYG{k+kc}{True}\PYG{p}{)}
\end{sphinxVerbatim}
}

\end{sphinxuseclass}
\end{sphinxuseclass}
\begin{sphinxuseclass}{nbinput}
{
\begin{sphinxVerbatim}[commandchars=\\\{\}]
\llap{\color{nbsphinxin}[1]:\,\hspace{\fboxrule}\hspace{\fboxsep}}\PYG{k+kn}{import} \PYG{n+nn}{akhdefo\PYGZus{}functions}
\PYG{k+kn}{from} \PYG{n+nn}{akhdefo\PYGZus{}functions} \PYG{k+kn}{import} \PYG{n}{Auto\PYGZus{}Variogram}

\PYG{c+c1}{\PYGZsh{} Auto\PYGZus{}Variogram(data=\PYGZsq{}./data/morenny/optical/geo/temp\PYGZus{}shapefile\PYGZus{}dir/updated\PYGZus{}shapefiles/20180627\PYGZus{}20220926\PYGZus{}20220926\PYGZus{}N.shp\PYGZsq{}, column\PYGZus{}attribute=\PYGZsq{}VEL\PYGZsq{}}
\PYG{c+c1}{\PYGZsh{}                , pixel\PYGZus{}size=20, num\PYGZus{}chunks=1, out\PYGZus{}fileName=\PYGZsq{}N\PYGZsq{}, geo\PYGZus{}folder=\PYGZsq{}./data/morenny/optical/VEL\PYGZus{}Folder\PYGZsq{}, plot\PYGZus{}folder=\PYGZsq{}./data/morenny/optical/VEL\PYGZus{}Folder\PYGZsq{}}
\PYG{c+c1}{\PYGZsh{}                , smoothing\PYGZus{}kernel=2, latlon=False, aoi\PYGZus{}shapefile=\PYGZsq{}./data/morenny/AOI.shp\PYGZsq{} , krig\PYGZus{}method=\PYGZsq{}universal\PYGZsq{})}


\PYG{c+c1}{\PYGZsh{} Auto\PYGZus{}Variogram(data=\PYGZsq{}./data/morenny/optical/geo/temp\PYGZus{}shapefile\PYGZus{}dir/updated\PYGZus{}shapefiles/20180627\PYGZus{}20220926\PYGZus{}20220926\PYGZus{}E.shp\PYGZsq{}, column\PYGZus{}attribute=\PYGZsq{}VEL\PYGZsq{}}
\PYG{c+c1}{\PYGZsh{}                , pixel\PYGZus{}size=20, num\PYGZus{}chunks=1, out\PYGZus{}fileName=\PYGZsq{}E\PYGZsq{}, geo\PYGZus{}folder=\PYGZsq{}./data/morenny/optical/VEL\PYGZus{}Folder\PYGZsq{}, plot\PYGZus{}folder=\PYGZsq{}./data/morenny/optical/VEL\PYGZus{}Folder\PYGZsq{}}
\PYG{c+c1}{\PYGZsh{}                , smoothing\PYGZus{}kernel=2, latlon=False, aoi\PYGZus{}shapefile=\PYGZsq{}./data/morenny/AOI.shp\PYGZsq{} , krig\PYGZus{}method=\PYGZsq{}universal\PYGZsq{})}



\PYG{c+c1}{\PYGZsh{} Auto\PYGZus{}Variogram(data=\PYGZsq{}./data/morenny/optical/geo/temp\PYGZus{}shapefile\PYGZus{}dir/updated\PYGZus{}shapefiles/20180627\PYGZus{}20220926\PYGZus{}20220926\PYGZus{}2DVEL.shp\PYGZsq{}, column\PYGZus{}attribute=\PYGZsq{}VEL\PYGZsq{}}
\PYG{c+c1}{\PYGZsh{}                , pixel\PYGZus{}size=20, num\PYGZus{}chunks=1, out\PYGZus{}fileName=\PYGZsq{}2DVEL\PYGZsq{}, geo\PYGZus{}folder=\PYGZsq{}./data/morenny/optical/VEL\PYGZus{}Folder\PYGZsq{}, plot\PYGZus{}folder=\PYGZsq{}./data/morenny/optical/VEL\PYGZus{}Folder\PYGZsq{}}
\PYG{c+c1}{\PYGZsh{}                , smoothing\PYGZus{}kernel=2, latlon=False, aoi\PYGZus{}shapefile=\PYGZsq{}./data/morenny/AOI.shp\PYGZsq{} , krig\PYGZus{}method=\PYGZsq{}universal\PYGZsq{})}



\PYG{n}{Auto\PYGZus{}Variogram}\PYG{p}{(}\PYG{n}{data}\PYG{o}{=}\PYG{l+s+s1}{\PYGZsq{}}\PYG{l+s+s1}{./data/morenny/optical/geo/temp\PYGZus{}shapefile\PYGZus{}dir/updated\PYGZus{}shapefiles/20180627\PYGZus{}20220926\PYGZus{}20220926\PYGZus{}2DVEL.shp}\PYG{l+s+s1}{\PYGZsq{}}\PYG{p}{,} \PYG{n}{column\PYGZus{}attribute}\PYG{o}{=}\PYG{l+s+s1}{\PYGZsq{}}\PYG{l+s+s1}{aspect}\PYG{l+s+s1}{\PYGZsq{}}
               \PYG{p}{,} \PYG{n}{pixel\PYGZus{}size}\PYG{o}{=}\PYG{l+m+mi}{20}\PYG{p}{,} \PYG{n}{num\PYGZus{}chunks}\PYG{o}{=}\PYG{l+m+mi}{1}\PYG{p}{,} \PYG{n}{out\PYGZus{}fileName}\PYG{o}{=}\PYG{l+s+s1}{\PYGZsq{}}\PYG{l+s+s1}{aspect}\PYG{l+s+s1}{\PYGZsq{}}\PYG{p}{,} \PYG{n}{geo\PYGZus{}folder}\PYG{o}{=}\PYG{l+s+s1}{\PYGZsq{}}\PYG{l+s+s1}{./data/morenny/optical/VEL\PYGZus{}Folder}\PYG{l+s+s1}{\PYGZsq{}}\PYG{p}{,} \PYG{n}{plot\PYGZus{}folder}\PYG{o}{=}\PYG{l+s+s1}{\PYGZsq{}}\PYG{l+s+s1}{./data/morenny/optical/VEL\PYGZus{}Folder}\PYG{l+s+s1}{\PYGZsq{}}
               \PYG{p}{,} \PYG{n}{smoothing\PYGZus{}kernel}\PYG{o}{=}\PYG{l+m+mi}{2}\PYG{p}{,} \PYG{n}{latlon}\PYG{o}{=}\PYG{k+kc}{False}\PYG{p}{,} \PYG{n}{aoi\PYGZus{}shapefile}\PYG{o}{=}\PYG{l+s+s1}{\PYGZsq{}}\PYG{l+s+s1}{./data/morenny/AOI.shp}\PYG{l+s+s1}{\PYGZsq{}} \PYG{p}{,} \PYG{n}{krig\PYGZus{}method}\PYG{o}{=}\PYG{l+s+s1}{\PYGZsq{}}\PYG{l+s+s1}{simple}\PYG{l+s+s1}{\PYGZsq{}}\PYG{p}{)}
\end{sphinxVerbatim}
}

\end{sphinxuseclass}
\begin{sphinxuseclass}{nboutput}
\begin{sphinxuseclass}{nblast}
{

\kern-\sphinxverbatimsmallskipamount\kern-\baselineskip
\kern+\FrameHeightAdjust\kern-\fboxrule
\vspace{\nbsphinxcodecellspacing}

\sphinxsetup{VerbatimColor={named}{nbsphinx-stderr}}
\begin{sphinxuseclass}{output_area}
\begin{sphinxuseclass}{stderr}


\begin{sphinxVerbatim}[commandchars=\\\{\}]
c:\textbackslash{}Users\textbackslash{}mahmud\textbackslash{}anaconda3\textbackslash{}envs\textbackslash{}akhdefov2\textbackslash{}lib\textbackslash{}site-packages\textbackslash{}gstools\textbackslash{}covmodel\textbackslash{}plot.py:114: UserWarning:

Matplotlib is currently using module://matplotlib\_inline.backend\_inline, which is a non-GUI backend, so cannot show the figure.

\end{sphinxVerbatim}



\end{sphinxuseclass}
\end{sphinxuseclass}
}

\end{sphinxuseclass}
\end{sphinxuseclass}

\subsection{Visualize Displacement Products}
\label{\detokenize{notebooks/RockGlacier_optical:Visualize-Displacement-Products}}
\begin{sphinxuseclass}{nbinput}
\begin{sphinxuseclass}{nblast}
{
\begin{sphinxVerbatim}[commandchars=\\\{\}]
\llap{\color{nbsphinxin}[ ]:\,\hspace{\fboxrule}\hspace{\fboxsep}}\PYG{k+kn}{import} \PYG{n+nn}{akhdefo\PYGZus{}functions}
\PYG{k+kn}{import} \PYG{n+nn}{cmocean}\PYG{n+nn}{.}\PYG{n+nn}{cm} \PYG{k}{as} \PYG{n+nn}{cmo}

\PYG{n}{akhdefo\PYGZus{}functions}\PYG{o}{.}\PYG{n}{akhdefo\PYGZus{}viewer}\PYG{p}{(}\PYG{n}{path\PYGZus{}to\PYGZus{}dem\PYGZus{}file}\PYG{o}{=}\PYG{l+s+s1}{\PYGZsq{}}\PYG{l+s+s1}{./data/morenny/basemap.tif}\PYG{l+s+s1}{\PYGZsq{}}\PYG{p}{,} \PYG{n}{raster\PYGZus{}file}\PYG{o}{=}\PYG{l+s+s1}{\PYGZsq{}}\PYG{l+s+s1}{./data/morenny/optical/VEL\PYGZus{}Folder/2DVEL.tif}\PYG{l+s+s1}{\PYGZsq{}}\PYG{p}{,} \PYG{n}{output\PYGZus{}folder}\PYG{o}{=}\PYG{l+s+s1}{\PYGZsq{}}\PYG{l+s+s1}{./data/morenny/optical/Figs\PYGZus{}analysis/}\PYG{l+s+s1}{\PYGZsq{}}\PYG{p}{,} \PYG{n}{title}\PYG{o}{=}\PYG{l+s+s1}{\PYGZsq{}}\PYG{l+s+s1}{2D Velocity with Vector Motion}\PYG{l+s+s1}{\PYGZsq{}}\PYG{p}{,}
                   \PYG{n}{pixel\PYGZus{}resolution\PYGZus{}meters}\PYG{o}{=}\PYG{k+kc}{None}\PYG{p}{,} \PYG{n}{output\PYGZus{}file\PYGZus{}name}\PYG{o}{=}\PYG{l+s+s2}{\PYGZdq{}}\PYG{l+s+s2}{2DVEL\PYGZus{}WithVectorMotion.png}\PYG{l+s+s2}{\PYGZdq{}}\PYG{p}{,}
                   \PYG{n}{alpha}\PYG{o}{=}\PYG{l+m+mf}{0.4}\PYG{p}{,} \PYG{n}{unit\PYGZus{}conversion}\PYG{o}{=}\PYG{k+kc}{None}\PYG{p}{,} \PYG{n}{no\PYGZus{}data\PYGZus{}mask}\PYG{o}{=}\PYG{k+kc}{True}\PYG{p}{,}
                   \PYG{n}{colormap}\PYG{o}{=}\PYG{n}{cmo}\PYG{o}{.}\PYG{n}{thermal}
                   \PYG{p}{,} \PYG{n}{min\PYGZus{}value}\PYG{o}{=}\PYG{k+kc}{None}\PYG{p}{,} \PYG{n}{max\PYGZus{}value}\PYG{o}{=}\PYG{k+kc}{None}\PYG{p}{,}
                   \PYG{n}{normalize}\PYG{o}{=}\PYG{k+kc}{True}\PYG{p}{,} \PYG{n}{colorbar\PYGZus{}label}\PYG{o}{=}\PYG{l+s+s1}{\PYGZsq{}}\PYG{l+s+s1}{meter/year}\PYG{l+s+s1}{\PYGZsq{}}\PYG{p}{,} \PYG{n}{show\PYGZus{}figure}\PYG{o}{=}\PYG{k+kc}{True}\PYG{p}{,} \PYG{n}{aspect\PYGZus{}raster}\PYG{o}{=}\PYG{l+s+s1}{\PYGZsq{}}\PYG{l+s+s1}{./data/morenny/optical/VEL\PYGZus{}Folder/aspect.tif}\PYG{l+s+s1}{\PYGZsq{}}\PYG{p}{,} \PYG{n}{cmap\PYGZus{}aspect}\PYG{o}{=}\PYG{k+kc}{None} \PYG{p}{,} \PYG{n}{step}\PYG{o}{=}\PYG{l+m+mi}{5}\PYG{p}{)}
\end{sphinxVerbatim}
}

\end{sphinxuseclass}
\end{sphinxuseclass}
\begin{sphinxuseclass}{nbinput}
\begin{sphinxuseclass}{nblast}
{
\begin{sphinxVerbatim}[commandchars=\\\{\}]
\llap{\color{nbsphinxin}[ ]:\,\hspace{\fboxrule}\hspace{\fboxsep}}\PYG{c+c1}{\PYGZsh{}Visualize Mean Velocity East\PYGZhy{}West}
\PYG{c+c1}{\PYGZsh{} if set batch\PYGZus{}plot=True it will plot static time series figure for each date acquisition}

\PYG{n}{MeanProducts\PYGZus{}plot\PYGZus{}ts}\PYG{p}{(}\PYG{n}{path\PYGZus{}to\PYGZus{}shapefile}\PYG{o}{=}\PYG{l+s+s1}{\PYGZsq{}}\PYG{l+s+s1}{./data/morenny/optical/geo/temp\PYGZus{}shapefile\PYGZus{}dir/updated\PYGZus{}shapefiles/20180627\PYGZus{}20220926\PYGZus{}20220926\PYGZus{}E.shp}\PYG{l+s+s1}{\PYGZsq{}}\PYG{p}{,}
                                       \PYG{n}{dem\PYGZus{}path}\PYG{o}{=}\PYG{l+s+s1}{\PYGZsq{}}\PYG{l+s+s1}{./data/morenny/dem.tif}\PYG{l+s+s1}{\PYGZsq{}}\PYG{p}{,} \PYG{n}{out\PYGZus{}folder}\PYG{o}{=}\PYG{l+s+s1}{\PYGZsq{}}\PYG{l+s+s1}{./data/morenny/optical/Figs\PYGZus{}analysis/E}\PYG{l+s+s1}{\PYGZsq{}}\PYG{p}{,}
                                       \PYG{n}{color\PYGZus{}field}\PYG{o}{=}\PYG{l+s+s2}{\PYGZdq{}}\PYG{l+s+s2}{VEL}\PYG{l+s+s2}{\PYGZdq{}}\PYG{p}{,} \PYG{n}{Set\PYGZus{}fig\PYGZus{}MinMax}\PYG{o}{=}\PYG{k+kc}{False}\PYG{p}{,}
                                       \PYG{n}{MinMaxRange}\PYG{o}{=}\PYG{p}{[}\PYG{l+m+mi}{0}\PYG{p}{,} \PYG{l+m+mf}{1.5}\PYG{p}{]}\PYG{p}{,} \PYG{n}{opacity}\PYG{o}{=}\PYG{l+m+mf}{0.7}\PYG{p}{,}
                                       \PYG{n}{cmap}\PYG{o}{=}\PYG{l+s+s1}{\PYGZsq{}}\PYG{l+s+s1}{bwr}\PYG{l+s+s1}{\PYGZsq{}}\PYG{p}{,} \PYG{n}{point\PYGZus{}size}\PYG{o}{=}\PYG{l+m+mi}{10}\PYG{p}{,}
                                       \PYG{n}{cbar\PYGZus{}label}\PYG{o}{=}\PYG{l+s+s1}{\PYGZsq{}}\PYG{l+s+s1}{East\PYGZhy{}West (m/year)}\PYG{l+s+s1}{\PYGZsq{}}\PYG{p}{,}
                                       \PYG{n}{batch\PYGZus{}plot}\PYG{o}{=}\PYG{k+kc}{False}\PYG{p}{)}

\PYG{c+c1}{\PYGZsh{}Visualize Mean Velocity North\PYGZhy{}South}
\PYG{n}{MeanProducts\PYGZus{}plot\PYGZus{}ts}\PYG{p}{(}\PYG{n}{path\PYGZus{}to\PYGZus{}shapefile}\PYG{o}{=}\PYG{l+s+s1}{\PYGZsq{}}\PYG{l+s+s1}{./data/morenny/optical/geo/temp\PYGZus{}shapefile\PYGZus{}dir/updated\PYGZus{}shapefiles/20180627\PYGZus{}20220926\PYGZus{}20220926\PYGZus{}N.shp}\PYG{l+s+s1}{\PYGZsq{}}\PYG{p}{,}
                                       \PYG{n}{dem\PYGZus{}path}\PYG{o}{=}\PYG{l+s+s1}{\PYGZsq{}}\PYG{l+s+s1}{./data/morenny/dem.tif}\PYG{l+s+s1}{\PYGZsq{}}\PYG{p}{,} \PYG{n}{out\PYGZus{}folder}\PYG{o}{=}\PYG{l+s+s1}{\PYGZsq{}}\PYG{l+s+s1}{./data/morenny/optical/Figs\PYGZus{}analysis/N}\PYG{l+s+s1}{\PYGZsq{}}\PYG{p}{,}
                                       \PYG{n}{color\PYGZus{}field}\PYG{o}{=}\PYG{l+s+s2}{\PYGZdq{}}\PYG{l+s+s2}{VEL}\PYG{l+s+s2}{\PYGZdq{}}\PYG{p}{,} \PYG{n}{Set\PYGZus{}fig\PYGZus{}MinMax}\PYG{o}{=}\PYG{k+kc}{False}\PYG{p}{,}
                                       \PYG{n}{MinMaxRange}\PYG{o}{=}\PYG{p}{[}\PYG{l+m+mi}{0}\PYG{p}{,} \PYG{l+m+mf}{1.5}\PYG{p}{]}\PYG{p}{,} \PYG{n}{opacity}\PYG{o}{=}\PYG{l+m+mf}{0.7}\PYG{p}{,}
                                       \PYG{n}{cmap}\PYG{o}{=}\PYG{l+s+s1}{\PYGZsq{}}\PYG{l+s+s1}{bwr}\PYG{l+s+s1}{\PYGZsq{}}\PYG{p}{,} \PYG{n}{point\PYGZus{}size}\PYG{o}{=}\PYG{l+m+mi}{10}\PYG{p}{,}
                                       \PYG{n}{cbar\PYGZus{}label}\PYG{o}{=}\PYG{l+s+s1}{\PYGZsq{}}\PYG{l+s+s1}{North\PYGZhy{}South (m/year)}\PYG{l+s+s1}{\PYGZsq{}}\PYG{p}{,}
                                       \PYG{n}{batch\PYGZus{}plot}\PYG{o}{=}\PYG{k+kc}{False}\PYG{p}{)}

\PYG{c+c1}{\PYGZsh{}Visualize Mean Velocity 2D Velocity}
\PYG{n}{MeanProducts\PYGZus{}plot\PYGZus{}ts}\PYG{p}{(}\PYG{n}{path\PYGZus{}to\PYGZus{}shapefile}\PYG{o}{=}\PYG{l+s+s1}{\PYGZsq{}}\PYG{l+s+s1}{./data/morenny/optical/geo/temp\PYGZus{}shapefile\PYGZus{}dir/updated\PYGZus{}shapefiles/20180627\PYGZus{}20220926\PYGZus{}20220926\PYGZus{}2DVEL.shp}\PYG{l+s+s1}{\PYGZsq{}}\PYG{p}{,}
                                       \PYG{n}{dem\PYGZus{}path}\PYG{o}{=}\PYG{l+s+s1}{\PYGZsq{}}\PYG{l+s+s1}{./data/morenny/dem.tif}\PYG{l+s+s1}{\PYGZsq{}}\PYG{p}{,} \PYG{n}{out\PYGZus{}folder}\PYG{o}{=}\PYG{l+s+s1}{\PYGZsq{}}\PYG{l+s+s1}{./data/morenny/optical/Figs\PYGZus{}analysis/2D}\PYG{l+s+s1}{\PYGZsq{}}\PYG{p}{,}
                                       \PYG{n}{color\PYGZus{}field}\PYG{o}{=}\PYG{l+s+s2}{\PYGZdq{}}\PYG{l+s+s2}{VEL}\PYG{l+s+s2}{\PYGZdq{}}\PYG{p}{,} \PYG{n}{Set\PYGZus{}fig\PYGZus{}MinMax}\PYG{o}{=}\PYG{k+kc}{False}\PYG{p}{,}
                                       \PYG{n}{MinMaxRange}\PYG{o}{=}\PYG{p}{[}\PYG{l+m+mi}{0}\PYG{p}{,} \PYG{l+m+mf}{1.5}\PYG{p}{]}\PYG{p}{,} \PYG{n}{opacity}\PYG{o}{=}\PYG{l+m+mf}{0.7}\PYG{p}{,}
                                       \PYG{n}{cmap}\PYG{o}{=}\PYG{l+s+s1}{\PYGZsq{}}\PYG{l+s+s1}{plasma}\PYG{l+s+s1}{\PYGZsq{}}\PYG{p}{,} \PYG{n}{point\PYGZus{}size}\PYG{o}{=}\PYG{l+m+mi}{10}\PYG{p}{,}
                                       \PYG{n}{cbar\PYGZus{}label}\PYG{o}{=}\PYG{l+s+s1}{\PYGZsq{}}\PYG{l+s+s1}{2D\PYGZhy{}VEL (meter/year)}\PYG{l+s+s1}{\PYGZsq{}}\PYG{p}{,}
                                       \PYG{n}{batch\PYGZus{}plot}\PYG{o}{=}\PYG{k+kc}{False}\PYG{p}{)}
\end{sphinxVerbatim}
}

\end{sphinxuseclass}
\end{sphinxuseclass}

\subsection{Interactive Time Series Analysis in a Dash Web App}
\label{\detokenize{notebooks/RockGlacier_optical:Interactive-Time-Series-Analysis-in-a-Dash-Web-App}}
\sphinxAtStartPar
This plot is part of an interactive web application. Users have the ability to select groups of data points using either the box select or lasso tool, which are superimposed on a satellite image base map. The selected data will then be reflected in the time series chart. Additionally, users can customize the axis labels to suit their dataset; for example, if the dataset measures displacement or velocity in meters, the y\sphinxhyphen{}axis label can be adjusted accordingly from the default mm. Furthermore,
users can download their analysis and the generated figures directly from the web application.

\noindent\sphinxincludegraphics[width=1736\sphinxpxdimen,height=942\sphinxpxdimen]{{dash2d1}.png}

\noindent\sphinxincludegraphics[width=1833\sphinxpxdimen,height=450\sphinxpxdimen]{{newplot10}.png}

\noindent\sphinxincludegraphics[width=1000\sphinxpxdimen,height=800\sphinxpxdimen]{{newplot9}.png}

\begin{sphinxuseclass}{nbinput}
\begin{sphinxuseclass}{nblast}
{
\begin{sphinxVerbatim}[commandchars=\\\{\}]
\llap{\color{nbsphinxin}[ ]:\,\hspace{\fboxrule}\hspace{\fboxsep}}\PYG{k+kn}{import} \PYG{n+nn}{akhdefo\PYGZus{}functions}

\PYG{n}{akhdefo\PYGZus{}functions}\PYG{o}{.}\PYG{n}{akhdefo\PYGZus{}dashApp}\PYG{p}{(}
\PYG{n}{Path\PYGZus{}to\PYGZus{}Shapefile}\PYG{o}{=}\PYG{l+s+s1}{\PYGZsq{}}\PYG{l+s+s1}{./data/morenny/optical/geo/temp\PYGZus{}shapefile\PYGZus{}dir/updated\PYGZus{}shapefiles/20180627\PYGZus{}20220926\PYGZus{}20220926\PYGZus{}2DVEL.shp}\PYG{l+s+s1}{\PYGZsq{}}\PYG{p}{,}
                                  \PYG{n}{port}\PYG{o}{=}\PYG{l+m+mi}{8047}\PYG{p}{,} \PYG{n}{BaseMap}\PYG{o}{=}\PYG{k+kc}{True}\PYG{p}{,} \PYG{n}{basemap\PYGZus{}type}\PYG{o}{=}\PYG{l+s+s1}{\PYGZsq{}}\PYG{l+s+s1}{image}\PYG{l+s+s1}{\PYGZsq{}}\PYG{p}{)}
\end{sphinxVerbatim}
}

\end{sphinxuseclass}
\end{sphinxuseclass}
\begin{sphinxuseclass}{nbinput}
\begin{sphinxuseclass}{nblast}
{
\begin{sphinxVerbatim}[commandchars=\\\{\}]
\llap{\color{nbsphinxin}[ ]:\,\hspace{\fboxrule}\hspace{\fboxsep}}
\PYG{n}{akhdefo\PYGZus{}dashApp}\PYG{p}{(}
\PYG{n}{Path\PYGZus{}to\PYGZus{}Shapefile}\PYG{o}{=}\PYG{l+s+s1}{\PYGZsq{}}\PYG{l+s+s1}{../data/morenny/optical/geo/temp\PYGZus{}shapefile\PYGZus{}dir/updated\PYGZus{}shapefiles/20180627\PYGZus{}20220926\PYGZus{}20220926\PYGZus{}N.shp}\PYG{l+s+s1}{\PYGZsq{}}\PYG{p}{,}
                                  \PYG{n}{port}\PYG{o}{=}\PYG{l+m+mi}{8056}\PYG{p}{,} \PYG{n}{BaseMap}\PYG{o}{=}\PYG{k+kc}{True}\PYG{p}{,} \PYG{n}{basemap\PYGZus{}type}\PYG{o}{=}\PYG{l+s+s1}{\PYGZsq{}}\PYG{l+s+s1}{image}\PYG{l+s+s1}{\PYGZsq{}}\PYG{p}{)}
\end{sphinxVerbatim}
}

\end{sphinxuseclass}
\end{sphinxuseclass}
\begin{sphinxuseclass}{nbinput}
\begin{sphinxuseclass}{nblast}
{
\begin{sphinxVerbatim}[commandchars=\\\{\}]
\llap{\color{nbsphinxin}[ ]:\,\hspace{\fboxrule}\hspace{\fboxsep}}
\PYG{n}{akhdefo\PYGZus{}dashApp}\PYG{p}{(}
\PYG{n}{Path\PYGZus{}to\PYGZus{}Shapefile}\PYG{o}{=}\PYG{l+s+s1}{\PYGZsq{}}\PYG{l+s+s1}{./data/morenny/optical/geo/temp\PYGZus{}shapefile\PYGZus{}dir/updated\PYGZus{}shapefiles/20180627\PYGZus{}20220926\PYGZus{}20220926\PYGZus{}E.shp}\PYG{l+s+s1}{\PYGZsq{}}\PYG{p}{,}
                                  \PYG{n}{port}\PYG{o}{=}\PYG{l+m+mi}{8055}\PYG{p}{,} \PYG{n}{BaseMap}\PYG{o}{=}\PYG{k+kc}{True}\PYG{p}{,} \PYG{n}{basemap\PYGZus{}type}\PYG{o}{=}\PYG{l+s+s1}{\PYGZsq{}}\PYG{l+s+s1}{image}\PYG{l+s+s1}{\PYGZsq{}}\PYG{p}{)}
\end{sphinxVerbatim}
}

\end{sphinxuseclass}
\end{sphinxuseclass}
\sphinxstepscope


\section{Morenney Rock Glacier Radar Imagery Example}
\label{\detokenize{notebooks/RockGlacier_radar:Morenney-Rock-Glacier-Radar-Imagery-Example}}\label{\detokenize{notebooks/RockGlacier_radar::doc}}

\subsection{Prepare Data for Bulk Download}
\label{\detokenize{notebooks/RockGlacier_radar:Prepare-Data-for-Bulk-Download}}
\sphinxAtStartPar
Akhdefo processes radar imagery that has been corrected for radiometric and terrain distortions. Its function, ‘download\_RTC,’ enables the downloading of Synthetic Aperture Radar (SAR) products from ASF’s HyP3 platform. Users can easily order RTC, AutoRIFT, and InSAR products with just a few lines of Python script. Firstly, users are advised to visit \sphinxhref{https://search.asf.alaska.edu/\#/?maxResults=250}{ASF’s website} to download metadata for their chosen time\sphinxhyphen{}series product granules. To assist
users, example screenshots are provided below as a guide to navigate through the process.

\noindent\sphinxincludegraphics[width=1865\sphinxpxdimen,height=957\sphinxpxdimen]{{asf1}.png}

\noindent\sphinxincludegraphics[width=1727\sphinxpxdimen,height=895\sphinxpxdimen]{{asf2}.png}


\subsection{Import AkhDefo Packages.}
\label{\detokenize{notebooks/RockGlacier_radar:Import-AkhDefo-Packages.}}
\begin{sphinxuseclass}{nbinput}
\begin{sphinxuseclass}{nblast}
{
\begin{sphinxVerbatim}[commandchars=\\\{\}]
\llap{\color{nbsphinxin}[ ]:\,\hspace{\fboxrule}\hspace{\fboxsep}}
\PYG{c+c1}{\PYGZsh{}Functions for data preparations}
\PYG{k+kn}{import} \PYG{n+nn}{akhdefo\PYGZus{}functions}
\PYG{k+kn}{from} \PYG{n+nn}{akhdefo\PYGZus{}functions} \PYG{k+kn}{import} \PYG{n}{download\PYGZus{}RTC}
\PYG{k+kn}{from} \PYG{n+nn}{akhdefo\PYGZus{}functions} \PYG{k+kn}{import} \PYG{n}{read\PYGZus{}data\PYGZus{}prep}
\PYG{k+kn}{from} \PYG{n+nn}{akhdefo\PYGZus{}functions} \PYG{k+kn}{import} \PYG{n}{move\PYGZus{}files\PYGZus{}with\PYGZus{}string}
\PYG{k+kn}{from} \PYG{n+nn}{akhdefo\PYGZus{}functions} \PYG{k+kn}{import} \PYG{n}{Crop\PYGZus{}to\PYGZus{}AOI}

\PYG{c+c1}{\PYGZsh{}Functions data processing}
\PYG{k+kn}{from} \PYG{n+nn}{akhdefo\PYGZus{}functions} \PYG{k+kn}{import} \PYG{n}{scatter\PYGZus{}area\PYGZus{}mask}
\PYG{k+kn}{from} \PYG{n+nn}{akhdefo\PYGZus{}functions} \PYG{k+kn}{import} \PYG{n}{Raster\PYGZus{}Correction}
\PYG{k+kn}{from} \PYG{n+nn}{akhdefo\PYGZus{}functions} \PYG{k+kn}{import} \PYG{n}{Optical\PYGZus{}flow\PYGZus{}akhdefo}

\PYG{c+c1}{\PYGZsh{}Functions for shapefile point data interpolation to raster geotif}
\PYG{c+c1}{\PYGZsh{}Function Autovariogram use Kriging method it fits the data to different variogram models and automatically use the best fit variogram model (Computationally Heavy but you can process data in chunks)}
\PYG{k+kn}{from} \PYG{n+nn}{akhdefo\PYGZus{}functions} \PYG{k+kn}{import} \PYG{n}{Auto\PYGZus{}Variogram}

\PYG{c+c1}{\PYGZsh{} interpoate\PYGZus{}xyz uses the following interpolation methods (nearest, linear and cubic)}
\PYG{k+kn}{from} \PYG{n+nn}{akhdefo\PYGZus{}functions} \PYG{k+kn}{import} \PYG{n}{interpolate\PYGZus{}xyz}

\PYG{c+c1}{\PYGZsh{}Functions for data Visualization and Analysis}
\PYG{k+kn}{from} \PYG{n+nn}{akhdefo\PYGZus{}functions} \PYG{k+kn}{import} \PYG{n}{MeanProducts\PYGZus{}plot\PYGZus{}ts}
\PYG{k+kn}{from} \PYG{n+nn}{akhdefo\PYGZus{}functions} \PYG{k+kn}{import} \PYG{n}{plot\PYGZus{}stackNetwork}
\PYG{k+kn}{from} \PYG{n+nn}{akhdefo\PYGZus{}functions} \PYG{k+kn}{import} \PYG{n}{akhdefo\PYGZus{}viewer}
\PYG{k+kn}{from} \PYG{n+nn}{akhdefo\PYGZus{}functions} \PYG{k+kn}{import} \PYG{n}{akhdefo\PYGZus{}dashApp}
\end{sphinxVerbatim}
}

\end{sphinxuseclass}
\end{sphinxuseclass}

\subsection{Download Data using EarthData ASF Account}
\label{\detokenize{notebooks/RockGlacier_radar:Download-Data-using-EarthData-ASF-Account}}
\begin{sphinxuseclass}{nbinput}
\begin{sphinxuseclass}{nblast}
{
\begin{sphinxVerbatim}[commandchars=\\\{\}]
\llap{\color{nbsphinxin}[ ]:\,\hspace{\fboxrule}\hspace{\fboxsep}}
\PYG{l+s+sd}{\PYGZsq{}\PYGZsq{}\PYGZsq{}}

\PYG{l+s+sd}{First Run the below command and make sure \PYGZob{}download=False\PYGZcb{} to see the granule paths and frames.}

\PYG{l+s+sd}{Its highly recommended to use data frame same path for the entire time\PYGZhy{}series for more accurate results}

\PYG{l+s+sd}{\PYGZsq{}\PYGZsq{}\PYGZsq{}}


\PYG{c+c1}{\PYGZsh{} download\PYGZus{}RTC( prompt=True, asf\PYGZus{}datapool\PYGZus{}results\PYGZus{}file= \PYGZsq{}./data/morenny/radar/asf\PYGZhy{}datapool\PYGZhy{}results\PYGZhy{}2023\PYGZhy{}12\PYGZhy{}19\PYGZus{}04\PYGZhy{}01\PYGZhy{}18.csv\PYGZsq{},}
\PYG{c+c1}{\PYGZsh{}             save\PYGZus{}dir= \PYGZsq{}./data/morenny/radar/path129\PYGZsq{}, job\PYGZus{}name= \PYGZsq{}./data/morenny/radar/RockGlacier\PYGZus{}asc\PYGZsq{},}
\PYG{c+c1}{\PYGZsh{}             dem\PYGZus{}matching= False,}
\PYG{c+c1}{\PYGZsh{}             include\PYGZus{}dem= False,}
\PYG{c+c1}{\PYGZsh{}             include\PYGZus{}inc\PYGZus{}map= True,}
\PYG{c+c1}{\PYGZsh{}             include\PYGZus{}rgb = False,}
\PYG{c+c1}{\PYGZsh{}             include\PYGZus{}scattering\PYGZus{}area= True,}
\PYG{c+c1}{\PYGZsh{}             scale = \PYGZsq{}power\PYGZsq{},}
\PYG{c+c1}{\PYGZsh{}             resolution = 20,}
\PYG{c+c1}{\PYGZsh{}             speckle\PYGZus{}filter= False,}
\PYG{c+c1}{\PYGZsh{}             radiometry=\PYGZsq{}gamma0\PYGZsq{},}
\PYG{c+c1}{\PYGZsh{}             dem\PYGZus{}name=\PYGZsq{}copernicus\PYGZsq{}, limit=None , download=False , path\PYGZus{}number=129, RTC=False)}
\end{sphinxVerbatim}
}

\end{sphinxuseclass}
\end{sphinxuseclass}

\subsection{Extract downloaded zip files}
\label{\detokenize{notebooks/RockGlacier_radar:Extract-downloaded-zip-files}}
\begin{sphinxuseclass}{nbinput}
\begin{sphinxuseclass}{nblast}
{
\begin{sphinxVerbatim}[commandchars=\\\{\}]
\llap{\color{nbsphinxin}[ ]:\,\hspace{\fboxrule}\hspace{\fboxsep}}\PYG{c+c1}{\PYGZsh{}This functions extracts data from all the granuples zip directory.}
\PYG{n}{read\PYGZus{}data\PYGZus{}prep}\PYG{p}{(}\PYG{n}{zip\PYGZus{}dir}\PYG{o}{=}\PYG{l+s+s1}{\PYGZsq{}}\PYG{l+s+s1}{./data/morenny/radar/path129\PYGZus{}rtc/}\PYG{l+s+s1}{\PYGZsq{}}\PYG{p}{,}
                                 \PYG{n}{image\PYGZus{}dir}\PYG{o}{=}\PYG{l+s+s1}{\PYGZsq{}}\PYG{l+s+s1}{./data/morenny/radar/RS\PYGZus{}dir\PYGZus{}path129}\PYG{l+s+s1}{\PYGZsq{}}\PYG{p}{,}
                                 \PYG{n}{ext\PYGZus{}image\PYGZus{}file}\PYG{o}{=}\PYG{l+s+s1}{\PYGZsq{}}\PYG{l+s+s1}{VV.tif}\PYG{l+s+s1}{\PYGZsq{}}\PYG{p}{,} \PYG{n}{udm\PYGZus{}mask\PYGZus{}dir}\PYG{o}{=}\PYG{l+s+s1}{\PYGZsq{}}\PYG{l+s+s1}{./data/morenny/radar/scatter\PYGZus{}dir}\PYG{l+s+s1}{\PYGZsq{}}\PYG{p}{,} \PYG{n}{ext\PYGZus{}udm\PYGZus{}mask\PYGZus{}file}\PYG{o}{=}\PYG{l+s+s1}{\PYGZsq{}}\PYG{l+s+s1}{area.tif}\PYG{l+s+s1}{\PYGZsq{}}\PYG{p}{)}
\end{sphinxVerbatim}
}

\end{sphinxuseclass}
\end{sphinxuseclass}
\begin{sphinxuseclass}{nbinput}
\begin{sphinxuseclass}{nblast}
{
\begin{sphinxVerbatim}[commandchars=\\\{\}]
\llap{\color{nbsphinxin}[ ]:\,\hspace{\fboxrule}\hspace{\fboxsep}}\PYG{c+c1}{\PYGZsh{} This function will move all the data from each granules subdir to a single folder}
\PYG{n}{move\PYGZus{}files\PYGZus{}with\PYGZus{}string}\PYG{p}{(}\PYG{n}{source\PYGZus{}dir}\PYG{o}{=}\PYG{l+s+s1}{\PYGZsq{}}\PYG{l+s+s1}{./data/morenny/radar/RS\PYGZus{}dir\PYGZus{}path129}\PYG{l+s+s1}{\PYGZsq{}}\PYG{p}{,} \PYG{n}{dest\PYGZus{}dir}\PYG{o}{=}\PYG{l+s+s1}{\PYGZsq{}}\PYG{l+s+s1}{./data/morenny/radar/RS\PYGZus{}path129}\PYG{l+s+s1}{\PYGZsq{}}\PYG{p}{,} \PYG{n}{search\PYGZus{}string}\PYG{o}{=}\PYG{l+s+s1}{\PYGZsq{}}\PYG{l+s+s1}{.tif}\PYG{l+s+s1}{\PYGZsq{}}\PYG{p}{)}
\PYG{n}{move\PYGZus{}files\PYGZus{}with\PYGZus{}string}\PYG{p}{(}\PYG{n}{source\PYGZus{}dir}\PYG{o}{=}\PYG{l+s+s1}{\PYGZsq{}}\PYG{l+s+s1}{./data/morenny/radar/scatter\PYGZus{}dir}\PYG{l+s+s1}{\PYGZsq{}}\PYG{p}{,} \PYG{n}{dest\PYGZus{}dir}\PYG{o}{=}\PYG{l+s+s1}{\PYGZsq{}}\PYG{l+s+s1}{./data/morenny/radar/scatter\PYGZus{}rs}\PYG{l+s+s1}{\PYGZsq{}}\PYG{p}{,} \PYG{n}{search\PYGZus{}string}\PYG{o}{=}\PYG{l+s+s1}{\PYGZsq{}}\PYG{l+s+s1}{area.tif}\PYG{l+s+s1}{\PYGZsq{}}\PYG{p}{)}

\end{sphinxVerbatim}
}

\end{sphinxuseclass}
\end{sphinxuseclass}
\begin{sphinxuseclass}{nbinput}
\begin{sphinxuseclass}{nblast}
{
\begin{sphinxVerbatim}[commandchars=\\\{\}]
\llap{\color{nbsphinxin}[ ]:\,\hspace{\fboxrule}\hspace{\fboxsep}}\PYG{l+s+sd}{\PYGZsq{}\PYGZsq{}\PYGZsq{}}
\PYG{l+s+sd}{Optional:}

\PYG{l+s+sd}{Create an accumulated scatter area mask from a set of raster images based on a given threshold.}
\PYG{l+s+sd}{the input dataset is taken from ASF RTC processing.}
\PYG{l+s+sd}{The scattering area for each pixel in the RTC image in square meters.}
\PYG{l+s+sd}{The values are calculated based on the effectively illuminated gamma\PYGZhy{}0 terrain surface using a digital elevation model,}
\PYG{l+s+sd}{the local incidence angle map, and the layover\PYGZhy{}shadow map. see detailes}
\PYG{l+s+sd}{at the following website https://hyp3\PYGZhy{}docs.asf.alaska.edu/guides/rtc\PYGZus{}product\PYGZus{}guide/\PYGZsh{}scattering\PYGZhy{}area\PYGZhy{}map}
\PYG{l+s+sd}{The function processes each raster image in the input folder, crops it based on the provided AOI from the shapefile,}
\PYG{l+s+sd}{normalizes the cropped raster, and then converts the normalized image to a binary mask based on the scatter\PYGZus{}percentageArea\PYGZus{}threshold.}
\PYG{l+s+sd}{The binary masks from each raster are then accumulated to generate the final scatter area mask.}

\PYG{l+s+sd}{\PYGZsq{}\PYGZsq{}\PYGZsq{}}

\PYG{n}{input\PYGZus{}folder} \PYG{o}{=} \PYG{l+s+s2}{\PYGZdq{}}\PYG{l+s+s2}{./data/morenny/radar/scatter\PYGZus{}rs}\PYG{l+s+s2}{\PYGZdq{}}
\PYG{n}{output\PYGZus{}folder} \PYG{o}{=} \PYG{l+s+s2}{\PYGZdq{}}\PYG{l+s+s2}{./data/morenny/radar/scatter\PYGZus{}masks\PYGZus{}20}\PYG{l+s+s2}{\PYGZdq{}}
\PYG{n}{plot\PYGZus{}folder} \PYG{o}{=} \PYG{l+s+s2}{\PYGZdq{}}\PYG{l+s+s2}{./data/morenny/radar/scatter\PYGZus{}mask\PYGZus{}plots\PYGZus{}20}\PYG{l+s+s2}{\PYGZdq{}}
\PYG{n}{shapefile\PYGZus{}path} \PYG{o}{=} \PYG{l+s+s2}{\PYGZdq{}}\PYG{l+s+s2}{./data/morenny/AOI.shp}\PYG{l+s+s2}{\PYGZdq{}}
\PYG{n}{scatter\PYGZus{}Area\PYGZus{}threshold}\PYG{o}{=}\PYG{l+m+mi}{5}
\PYG{n}{vegetation\PYGZus{}mask\PYGZus{}path}\PYG{o}{=}\PYG{k+kc}{None}


\PYG{c+c1}{\PYGZsh{}scatter\PYGZus{}area\PYGZus{}mask(input\PYGZus{}folder, output\PYGZus{}folder, plot\PYGZus{}folder, shapefile\PYGZus{}path, scatter\PYGZus{}Area\PYGZus{}threshold, vegetation\PYGZus{}mask\PYGZus{}path)}


\end{sphinxVerbatim}
}

\end{sphinxuseclass}
\end{sphinxuseclass}
\begin{sphinxuseclass}{nbinput}
\begin{sphinxuseclass}{nblast}
{
\begin{sphinxVerbatim}[commandchars=\\\{\}]
\llap{\color{nbsphinxin}[ ]:\,\hspace{\fboxrule}\hspace{\fboxsep}}
\PYG{l+s+sd}{\PYGZsq{}\PYGZsq{}\PYGZsq{}}
\PYG{l+s+sd}{Crop all raster to Area of interest bounding box}


\PYG{l+s+sd}{\PYGZsq{}\PYGZsq{}\PYGZsq{}}
\PYG{n}{Crop\PYGZus{}to\PYGZus{}AOI}\PYG{p}{(}\PYG{n}{Path\PYGZus{}to\PYGZus{}WorkingDir}\PYG{o}{=}\PYG{l+s+s1}{\PYGZsq{}}\PYG{l+s+s1}{./data/morenny/radar/RS\PYGZus{}path129/}\PYG{l+s+s1}{\PYGZsq{}}\PYG{p}{,}
                              \PYG{n}{Path\PYGZus{}to\PYGZus{}AOI\PYGZus{}shapefile}\PYG{o}{=}\PYG{l+s+s1}{\PYGZsq{}}\PYG{l+s+s1}{./data/morenny/AOI.shp}\PYG{l+s+s1}{\PYGZsq{}}\PYG{p}{,}
                              \PYG{n}{output\PYGZus{}CroppedDir}\PYG{o}{=}\PYG{l+s+s1}{\PYGZsq{}}\PYG{l+s+s1}{./data/morenny/radar/RS\PYGZus{}cropped\PYGZus{}path129}\PYG{l+s+s1}{\PYGZsq{}}\PYG{p}{,} \PYG{n}{file\PYGZus{}ex}\PYG{o}{=}\PYG{l+s+s1}{\PYGZsq{}}\PYG{l+s+s1}{.tif}\PYG{l+s+s1}{\PYGZsq{}}\PYG{p}{)}


\end{sphinxVerbatim}
}

\end{sphinxuseclass}
\end{sphinxuseclass}

\subsection{Preparing raster for optical flow and displacement calculation!}
\label{\detokenize{notebooks/RockGlacier_radar:Preparing-raster-for-optical-flow-and-displacement-calculation!}}
\sphinxAtStartPar
By deafult RTC data comes in the below three formats. The below raster correction function converts the below RTC data formats to Unassigned 8 bit integer(“uint8”).
\begin{itemize}
\item {} 
\sphinxAtStartPar
Power Scale: SAR images in the power scale represent the square of the amplitude. Power is a measure of the energy of the radar signal returned from each point on the ground. It’s more intuitive in terms of energy interpretation but less used for visual interpretation due to its non\sphinxhyphen{}linear nature.

\item {} 
\sphinxAtStartPar
Amplitude Scale: The amplitude scale is a direct representation of the radar signal’s strength received from each point. It’s more commonly used for visual interpretation since it’s more linear than the power scale. Amplitude images are easier to interpret but can be sensitive to noise.

\item {} 
\sphinxAtStartPar
Decibel (dB) Scale: The decibel scale is a logarithmic scale used to express SAR data. It’s derived from the amplitude or power scale and is used to compress the dynamic range of the SAR data. The dB scale is beneficial for enhancing certain features in the data, especially in areas with very high or very low backscatter.

\end{itemize}

\begin{sphinxuseclass}{nbinput}
\begin{sphinxuseclass}{nblast}
{
\begin{sphinxVerbatim}[commandchars=\\\{\}]
\llap{\color{nbsphinxin}[ ]:\,\hspace{\fboxrule}\hspace{\fboxsep}}\PYG{k+kn}{import} \PYG{n+nn}{warnings}
\PYG{n}{warnings}\PYG{o}{.}\PYG{n}{filterwarnings}\PYG{p}{(}\PYG{l+s+s2}{\PYGZdq{}}\PYG{l+s+s2}{ignore}\PYG{l+s+s2}{\PYGZdq{}}\PYG{p}{)}


\PYG{n}{Raster\PYGZus{}Correction}\PYG{p}{(}\PYG{n}{input\PYGZus{}path}\PYG{o}{=}\PYG{l+s+s2}{\PYGZdq{}}\PYG{l+s+s2}{./data/morenny/radar/RS\PYGZus{}cropped\PYGZus{}path129}\PYG{l+s+s2}{\PYGZdq{}}\PYG{p}{,} \PYG{n}{output\PYGZus{}path}\PYG{o}{=}\PYG{l+s+s2}{\PYGZdq{}}\PYG{l+s+s2}{./data/morenny/radar/RS\PYGZus{}cropped\PYGZus{}path129\PYGZus{}filt}\PYG{l+s+s2}{\PYGZdq{}}\PYG{p}{,} \PYG{n}{limit}\PYG{o}{=}\PYG{k+kc}{None}\PYG{p}{,}
                                    \PYG{n}{lowpass\PYGZus{}kernel\PYGZus{}size}\PYG{o}{=}\PYG{k+kc}{None}\PYG{p}{,} \PYG{n}{bilateral\PYGZus{}win\PYGZus{}size}\PYG{o}{=}\PYG{l+m+mi}{7}\PYG{p}{,} \PYG{n}{bilateral\PYGZus{}sigma\PYGZus{}color}\PYG{o}{=}\PYG{l+m+mi}{75}\PYG{p}{,}
                                    \PYG{n}{bilateral\PYGZus{}sigma\PYGZus{}spatial}\PYG{o}{=}\PYG{l+m+mi}{75}\PYG{p}{,}
                                    \PYG{n}{clip\PYGZus{}percentiles}\PYG{o}{=}\PYG{p}{[}\PYG{l+m+mi}{2}\PYG{p}{,} \PYG{l+m+mi}{98}\PYG{p}{]}\PYG{p}{,} \PYG{n}{optical}\PYG{o}{=}\PYG{k+kc}{False}\PYG{p}{,}
                                    \PYG{n}{scale}\PYG{o}{=}\PYG{l+s+s1}{\PYGZsq{}}\PYG{l+s+s1}{power}\PYG{l+s+s1}{\PYGZsq{}}\PYG{p}{,} \PYG{n}{Vegetation\PYGZus{}mask}\PYG{o}{=}\PYG{k+kc}{None}\PYG{p}{)}
\end{sphinxVerbatim}
}

\end{sphinxuseclass}
\end{sphinxuseclass}

\subsection{Applying optical flow and producing time\sphinxhyphen{}series deformation products.}
\label{\detokenize{notebooks/RockGlacier_radar:Applying-optical-flow-and-producing-time-series-deformation-products.}}
\begin{sphinxuseclass}{nbinput}
\begin{sphinxuseclass}{nblast}
{
\begin{sphinxVerbatim}[commandchars=\\\{\}]
\llap{\color{nbsphinxin}[ ]:\,\hspace{\fboxrule}\hspace{\fboxsep}}\PYG{k+kn}{import} \PYG{n+nn}{warnings}
\PYG{n}{warnings}\PYG{o}{.}\PYG{n}{filterwarnings}\PYG{p}{(}\PYG{l+s+s2}{\PYGZdq{}}\PYG{l+s+s2}{ignore}\PYG{l+s+s2}{\PYGZdq{}}\PYG{p}{)}
\PYG{n}{Optical\PYGZus{}flow\PYGZus{}akhdefo}\PYG{p}{(}\PYG{n}{input\PYGZus{}dir}\PYG{o}{=}\PYG{l+s+s1}{\PYGZsq{}}\PYG{l+s+s1}{./data/morenny/radar/RS\PYGZus{}cropped\PYGZus{}path129\PYGZus{}filt}\PYG{l+s+s1}{\PYGZsq{}}\PYG{p}{,} \PYG{n}{output\PYGZus{}dir}\PYG{o}{=}\PYG{l+s+s1}{\PYGZsq{}}\PYG{l+s+s1}{./data/morenny/radar/geo\PYGZus{}path129}\PYG{l+s+s1}{\PYGZsq{}}\PYG{p}{,} \PYG{n}{AOI}\PYG{o}{=}\PYG{l+s+s1}{\PYGZsq{}}\PYG{l+s+s1}{./data/morenny/AOI.shp}\PYG{l+s+s1}{\PYGZsq{}}\PYG{p}{,} \PYG{n}{zscore\PYGZus{}threshold}\PYG{o}{=}\PYG{l+m+mi}{2}\PYG{p}{,}
                                       \PYG{n}{ssim\PYGZus{}thresh}\PYG{o}{=}\PYG{l+m+mf}{0.6}\PYG{p}{,} \PYG{n}{image\PYGZus{}resolution}\PYG{o}{=}\PYG{l+s+s1}{\PYGZsq{}}\PYG{l+s+s1}{20m}\PYG{l+s+s1}{\PYGZsq{}}\PYG{p}{,} \PYG{n}{interpolate}\PYG{o}{=}\PYG{l+s+s1}{\PYGZsq{}}\PYG{l+s+s1}{kriging}\PYG{l+s+s1}{\PYGZsq{}}\PYG{p}{,}
                                       \PYG{n}{show\PYGZus{}figure}\PYG{o}{=}\PYG{k+kc}{False}\PYG{p}{,} \PYG{n}{point\PYGZus{}size}\PYG{o}{=}\PYG{l+m+mi}{2}\PYG{p}{,} \PYG{n}{dem\PYGZus{}path}\PYG{o}{=}\PYG{l+s+s1}{\PYGZsq{}}\PYG{l+s+s1}{./data/morenny/dem.tif}\PYG{l+s+s1}{\PYGZsq{}}\PYG{p}{,} \PYG{n}{smoothing\PYGZus{}kernel\PYGZus{}size}\PYG{o}{=}\PYG{k+kc}{None}\PYG{p}{,}
                                       \PYG{n}{Vegetation\PYGZus{}mask}\PYG{o}{=}\PYG{k+kc}{None}\PYG{p}{,} \PYG{n}{VEL\PYGZus{}scale}\PYG{o}{=}\PYG{l+s+s1}{\PYGZsq{}}\PYG{l+s+s1}{year}\PYG{l+s+s1}{\PYGZsq{}}\PYG{p}{,} \PYG{n}{VEL\PYGZus{}Mode}\PYG{o}{=}\PYG{l+s+s1}{\PYGZsq{}}\PYG{l+s+s1}{linear}\PYG{l+s+s1}{\PYGZsq{}}\PYG{p}{,} \PYG{n}{good\PYGZus{}match\PYGZus{}option}\PYG{o}{=}\PYG{l+m+mf}{0.75}\PYG{p}{,}
                                       \PYG{n}{hillshade\PYGZus{}option}\PYG{o}{=}\PYG{k+kc}{True}\PYG{p}{,} \PYG{n}{shapefile\PYGZus{}output}\PYG{o}{=}\PYG{k+kc}{True}\PYG{p}{,} \PYG{n}{max\PYGZus{}triplet\PYGZus{}interval}\PYG{o}{=}\PYG{l+m+mi}{300}\PYG{p}{,} \PYG{n}{pixel\PYGZus{}size}\PYG{o}{=}\PYG{l+m+mi}{20}\PYG{p}{,} \PYG{n}{num\PYGZus{}chunks}\PYG{o}{=}\PYG{l+m+mi}{1}\PYG{p}{,} \PYG{n}{overlap\PYGZus{}percentage}\PYG{o}{=}\PYG{l+m+mi}{0}\PYG{p}{,} \PYG{n}{pyr\PYGZus{}scale}\PYG{o}{=}\PYG{l+m+mf}{0.5}\PYG{p}{,} \PYG{n}{levels}\PYG{o}{=}\PYG{l+m+mi}{15}\PYG{p}{,}
                                       \PYG{n}{winsize}\PYG{o}{=}\PYG{l+m+mi}{128}\PYG{p}{,} \PYG{n}{iterations}\PYG{o}{=}\PYG{l+m+mi}{7}\PYG{p}{,} \PYG{n}{poly\PYGZus{}n}\PYG{o}{=}\PYG{l+m+mi}{7}\PYG{p}{,} \PYG{n}{poly\PYGZus{}sigma}\PYG{o}{=}\PYG{l+m+mf}{1.5}\PYG{p}{,}
                                       \PYG{n}{flags}\PYG{o}{=}\PYG{l+m+mi}{1}\PYG{p}{,} \PYG{n}{master\PYGZus{}reference}\PYG{o}{=}\PYG{l+s+s1}{\PYGZsq{}}\PYG{l+s+s1}{single}\PYG{l+s+s1}{\PYGZsq{}}\PYG{p}{,}
                                       \PYG{n}{selection\PYGZus{}Mode}\PYG{o}{=}\PYG{l+s+s1}{\PYGZsq{}}\PYG{l+s+s1}{pair}\PYG{l+s+s1}{\PYGZsq{}}\PYG{p}{,} \PYG{n}{start\PYGZus{}date}\PYG{o}{=}\PYG{k+kc}{None}\PYG{p}{,} \PYG{n}{end\PYGZus{}date}\PYG{o}{=}\PYG{k+kc}{None}\PYG{p}{,} \PYG{n}{krig\PYGZus{}method}\PYG{o}{=}\PYG{l+s+s1}{\PYGZsq{}}\PYG{l+s+s1}{universal}\PYG{l+s+s1}{\PYGZsq{}}\PYG{p}{,} \PYG{n}{spatial\PYGZus{}ref}\PYG{o}{=}\PYG{k+kc}{True}\PYG{p}{)}
\end{sphinxVerbatim}
}

\end{sphinxuseclass}
\end{sphinxuseclass}

\subsection{Interpolate and Create raster Velocities}
\label{\detokenize{notebooks/RockGlacier_radar:Interpolate-and-Create-raster-Velocities}}
\begin{sphinxuseclass}{nbinput}
\begin{sphinxuseclass}{nblast}
{
\begin{sphinxVerbatim}[commandchars=\\\{\}]
\llap{\color{nbsphinxin}[ ]:\,\hspace{\fboxrule}\hspace{\fboxsep}}\PYG{k+kn}{import} \PYG{n+nn}{akhdefo\PYGZus{}functions}
\PYG{k+kn}{from} \PYG{n+nn}{akhdefo\PYGZus{}functions} \PYG{k+kn}{import} \PYG{n}{Auto\PYGZus{}Variogram}

\PYG{n}{Auto\PYGZus{}Variogram}\PYG{p}{(}\PYG{n}{data}\PYG{o}{=}\PYG{l+s+s1}{\PYGZsq{}}\PYG{l+s+s1}{./data/morenny/radar/geo\PYGZus{}path129/temp\PYGZus{}shapefile\PYGZus{}dir/updated\PYGZus{}shapefiles/20150717\PYGZus{}20230921\PYGZus{}20230921\PYGZus{}N.shp}\PYG{l+s+s1}{\PYGZsq{}}\PYG{p}{,} \PYG{n}{column\PYGZus{}attribute}\PYG{o}{=}\PYG{l+s+s1}{\PYGZsq{}}\PYG{l+s+s1}{VEL}\PYG{l+s+s1}{\PYGZsq{}}
               \PYG{p}{,} \PYG{n}{pixel\PYGZus{}size}\PYG{o}{=}\PYG{l+m+mi}{20}\PYG{p}{,} \PYG{n}{num\PYGZus{}chunks}\PYG{o}{=}\PYG{l+m+mi}{1}\PYG{p}{,} \PYG{n}{out\PYGZus{}fileName}\PYG{o}{=}\PYG{l+s+s1}{\PYGZsq{}}\PYG{l+s+s1}{N}\PYG{l+s+s1}{\PYGZsq{}}\PYG{p}{,} \PYG{n}{geo\PYGZus{}folder}\PYG{o}{=}\PYG{l+s+s1}{\PYGZsq{}}\PYG{l+s+s1}{./data/morenny/radar/VEL\PYGZus{}Folder}\PYG{l+s+s1}{\PYGZsq{}}\PYG{p}{,} \PYG{n}{plot\PYGZus{}folder}\PYG{o}{=}\PYG{l+s+s1}{\PYGZsq{}}\PYG{l+s+s1}{./data/morenny/radar/VEL\PYGZus{}Folder}\PYG{l+s+s1}{\PYGZsq{}}
               \PYG{p}{,} \PYG{n}{smoothing\PYGZus{}kernel}\PYG{o}{=}\PYG{l+m+mi}{3}\PYG{p}{,} \PYG{n}{latlon}\PYG{o}{=}\PYG{k+kc}{False}\PYG{p}{,} \PYG{n}{aoi\PYGZus{}shapefile}\PYG{o}{=}\PYG{l+s+s1}{\PYGZsq{}}\PYG{l+s+s1}{./data/morenny/AOI.shp}\PYG{l+s+s1}{\PYGZsq{}} \PYG{p}{,}  \PYG{n}{krig\PYGZus{}method}\PYG{o}{=}\PYG{l+s+s1}{\PYGZsq{}}\PYG{l+s+s1}{simple}\PYG{l+s+s1}{\PYGZsq{}}\PYG{p}{)}


\PYG{n}{Auto\PYGZus{}Variogram}\PYG{p}{(}\PYG{n}{data}\PYG{o}{=}\PYG{l+s+s1}{\PYGZsq{}}\PYG{l+s+s1}{./data/morenny/radar/geo\PYGZus{}path129/temp\PYGZus{}shapefile\PYGZus{}dir/updated\PYGZus{}shapefiles/20150717\PYGZus{}20230921\PYGZus{}20230921\PYGZus{}E.shp}\PYG{l+s+s1}{\PYGZsq{}}\PYG{p}{,} \PYG{n}{column\PYGZus{}attribute}\PYG{o}{=}\PYG{l+s+s1}{\PYGZsq{}}\PYG{l+s+s1}{VEL}\PYG{l+s+s1}{\PYGZsq{}}
               \PYG{p}{,} \PYG{n}{pixel\PYGZus{}size}\PYG{o}{=}\PYG{l+m+mi}{20}\PYG{p}{,} \PYG{n}{num\PYGZus{}chunks}\PYG{o}{=}\PYG{l+m+mi}{1}\PYG{p}{,} \PYG{n}{out\PYGZus{}fileName}\PYG{o}{=}\PYG{l+s+s1}{\PYGZsq{}}\PYG{l+s+s1}{E}\PYG{l+s+s1}{\PYGZsq{}}\PYG{p}{,} \PYG{n}{geo\PYGZus{}folder}\PYG{o}{=}\PYG{l+s+s1}{\PYGZsq{}}\PYG{l+s+s1}{./data/morenny/radar/VEL\PYGZus{}Folder}\PYG{l+s+s1}{\PYGZsq{}}\PYG{p}{,} \PYG{n}{plot\PYGZus{}folder}\PYG{o}{=}\PYG{l+s+s1}{\PYGZsq{}}\PYG{l+s+s1}{./data/morenny/radar/VEL\PYGZus{}Folder}\PYG{l+s+s1}{\PYGZsq{}}
               \PYG{p}{,} \PYG{n}{smoothing\PYGZus{}kernel}\PYG{o}{=}\PYG{l+m+mi}{3}\PYG{p}{,} \PYG{n}{latlon}\PYG{o}{=}\PYG{k+kc}{False}\PYG{p}{,} \PYG{n}{aoi\PYGZus{}shapefile}\PYG{o}{=}\PYG{l+s+s1}{\PYGZsq{}}\PYG{l+s+s1}{./data/morenny/AOI.shp}\PYG{l+s+s1}{\PYGZsq{}}\PYG{p}{,}  \PYG{n}{krig\PYGZus{}method}\PYG{o}{=}\PYG{l+s+s1}{\PYGZsq{}}\PYG{l+s+s1}{simple}\PYG{l+s+s1}{\PYGZsq{}}\PYG{p}{)}



\PYG{n}{Auto\PYGZus{}Variogram}\PYG{p}{(}\PYG{n}{data}\PYG{o}{=}\PYG{l+s+s1}{\PYGZsq{}}\PYG{l+s+s1}{./data/morenny/radar/geo\PYGZus{}path129/temp\PYGZus{}shapefile\PYGZus{}dir/updated\PYGZus{}shapefiles/20150717\PYGZus{}20230921\PYGZus{}20230921\PYGZus{}2DVEL.shp}\PYG{l+s+s1}{\PYGZsq{}}\PYG{p}{,} \PYG{n}{column\PYGZus{}attribute}\PYG{o}{=}\PYG{l+s+s1}{\PYGZsq{}}\PYG{l+s+s1}{VEL}\PYG{l+s+s1}{\PYGZsq{}}
               \PYG{p}{,} \PYG{n}{pixel\PYGZus{}size}\PYG{o}{=}\PYG{l+m+mi}{20}\PYG{p}{,} \PYG{n}{num\PYGZus{}chunks}\PYG{o}{=}\PYG{l+m+mi}{1}\PYG{p}{,} \PYG{n}{out\PYGZus{}fileName}\PYG{o}{=}\PYG{l+s+s1}{\PYGZsq{}}\PYG{l+s+s1}{2DVEL}\PYG{l+s+s1}{\PYGZsq{}}\PYG{p}{,} \PYG{n}{geo\PYGZus{}folder}\PYG{o}{=}\PYG{l+s+s1}{\PYGZsq{}}\PYG{l+s+s1}{./data/morenny/radar/VEL\PYGZus{}Folder}\PYG{l+s+s1}{\PYGZsq{}}\PYG{p}{,} \PYG{n}{plot\PYGZus{}folder}\PYG{o}{=}\PYG{l+s+s1}{\PYGZsq{}}\PYG{l+s+s1}{./data/morenny/radar/VEL\PYGZus{}Folder}\PYG{l+s+s1}{\PYGZsq{}}
               \PYG{p}{,} \PYG{n}{smoothing\PYGZus{}kernel}\PYG{o}{=}\PYG{l+m+mi}{3}\PYG{p}{,} \PYG{n}{latlon}\PYG{o}{=}\PYG{k+kc}{False}\PYG{p}{,} \PYG{n}{aoi\PYGZus{}shapefile}\PYG{o}{=}\PYG{l+s+s1}{\PYGZsq{}}\PYG{l+s+s1}{./data/morenny/AOI.shp}\PYG{l+s+s1}{\PYGZsq{}}\PYG{p}{,}  \PYG{n}{krig\PYGZus{}method}\PYG{o}{=}\PYG{l+s+s1}{\PYGZsq{}}\PYG{l+s+s1}{simple}\PYG{l+s+s1}{\PYGZsq{}}\PYG{p}{)}



\PYG{n}{Auto\PYGZus{}Variogram}\PYG{p}{(}\PYG{n}{data}\PYG{o}{=}\PYG{l+s+s1}{\PYGZsq{}}\PYG{l+s+s1}{./data/morenny/radar/geo\PYGZus{}path129/temp\PYGZus{}shapefile\PYGZus{}dir/updated\PYGZus{}shapefiles/20150717\PYGZus{}20230921\PYGZus{}20230921\PYGZus{}2DVEL.shp}\PYG{l+s+s1}{\PYGZsq{}}\PYG{p}{,} \PYG{n}{column\PYGZus{}attribute}\PYG{o}{=}\PYG{l+s+s1}{\PYGZsq{}}\PYG{l+s+s1}{aspect}\PYG{l+s+s1}{\PYGZsq{}}
               \PYG{p}{,} \PYG{n}{pixel\PYGZus{}size}\PYG{o}{=}\PYG{l+m+mi}{20}\PYG{p}{,} \PYG{n}{num\PYGZus{}chunks}\PYG{o}{=}\PYG{l+m+mi}{1}\PYG{p}{,} \PYG{n}{out\PYGZus{}fileName}\PYG{o}{=}\PYG{l+s+s1}{\PYGZsq{}}\PYG{l+s+s1}{aspect}\PYG{l+s+s1}{\PYGZsq{}}\PYG{p}{,} \PYG{n}{geo\PYGZus{}folder}\PYG{o}{=}\PYG{l+s+s1}{\PYGZsq{}}\PYG{l+s+s1}{./data/morenny/radar/VEL\PYGZus{}Folder}\PYG{l+s+s1}{\PYGZsq{}}\PYG{p}{,} \PYG{n}{plot\PYGZus{}folder}\PYG{o}{=}\PYG{l+s+s1}{\PYGZsq{}}\PYG{l+s+s1}{./data/morenny/radar/VEL\PYGZus{}Folder}\PYG{l+s+s1}{\PYGZsq{}}
               \PYG{p}{,} \PYG{n}{smoothing\PYGZus{}kernel}\PYG{o}{=}\PYG{l+m+mi}{3}\PYG{p}{,} \PYG{n}{latlon}\PYG{o}{=}\PYG{k+kc}{False}\PYG{p}{,} \PYG{n}{aoi\PYGZus{}shapefile}\PYG{o}{=}\PYG{l+s+s1}{\PYGZsq{}}\PYG{l+s+s1}{./data/morenny/AOI.shp}\PYG{l+s+s1}{\PYGZsq{}}\PYG{p}{,}  \PYG{n}{krig\PYGZus{}method}\PYG{o}{=}\PYG{l+s+s1}{\PYGZsq{}}\PYG{l+s+s1}{simple}\PYG{l+s+s1}{\PYGZsq{}}\PYG{p}{)}
\end{sphinxVerbatim}
}

\end{sphinxuseclass}
\end{sphinxuseclass}

\subsection{Visualize Displacement Products}
\label{\detokenize{notebooks/RockGlacier_radar:Visualize-Displacement-Products}}
\begin{sphinxuseclass}{nbinput}
{
\begin{sphinxVerbatim}[commandchars=\\\{\}]
\llap{\color{nbsphinxin}[4]:\,\hspace{\fboxrule}\hspace{\fboxsep}}\PYG{k+kn}{import} \PYG{n+nn}{akhdefo\PYGZus{}functions}
\PYG{k+kn}{import} \PYG{n+nn}{cmocean}\PYG{n+nn}{.}\PYG{n+nn}{cm} \PYG{k}{as} \PYG{n+nn}{cmo}

\PYG{n}{akhdefo\PYGZus{}functions}\PYG{o}{.}\PYG{n}{akhdefo\PYGZus{}viewer}\PYG{p}{(}\PYG{n}{path\PYGZus{}to\PYGZus{}dem\PYGZus{}file}\PYG{o}{=}\PYG{l+s+s1}{\PYGZsq{}}\PYG{l+s+s1}{./data/morenny/basemap.tif}\PYG{l+s+s1}{\PYGZsq{}}\PYG{p}{,} \PYG{n}{raster\PYGZus{}file}\PYG{o}{=}\PYG{l+s+s1}{\PYGZsq{}}\PYG{l+s+s1}{./data/morenny/radar/VEL\PYGZus{}Folder/2DVEL.tif}\PYG{l+s+s1}{\PYGZsq{}}\PYG{p}{,} \PYG{n}{output\PYGZus{}folder}\PYG{o}{=}\PYG{l+s+s1}{\PYGZsq{}}\PYG{l+s+s1}{./data/morenny/radar/Figs\PYGZus{}analysis/}\PYG{l+s+s1}{\PYGZsq{}}\PYG{p}{,} \PYG{n}{title}\PYG{o}{=}\PYG{l+s+s1}{\PYGZsq{}}\PYG{l+s+s1}{2D Velocity with Vector Motion}\PYG{l+s+s1}{\PYGZsq{}}\PYG{p}{,}
                   \PYG{n}{pixel\PYGZus{}resolution\PYGZus{}meters}\PYG{o}{=}\PYG{k+kc}{None}\PYG{p}{,} \PYG{n}{output\PYGZus{}file\PYGZus{}name}\PYG{o}{=}\PYG{l+s+s2}{\PYGZdq{}}\PYG{l+s+s2}{2DVEL\PYGZus{}WithVectorMotion.png}\PYG{l+s+s2}{\PYGZdq{}}\PYG{p}{,}
                   \PYG{n}{alpha}\PYG{o}{=}\PYG{l+m+mf}{0.4}\PYG{p}{,} \PYG{n}{unit\PYGZus{}conversion}\PYG{o}{=}\PYG{k+kc}{None}\PYG{p}{,} \PYG{n}{no\PYGZus{}data\PYGZus{}mask}\PYG{o}{=}\PYG{k+kc}{True}\PYG{p}{,}
                   \PYG{n}{colormap}\PYG{o}{=}\PYG{l+s+s1}{\PYGZsq{}}\PYG{l+s+s1}{hot\PYGZus{}r}\PYG{l+s+s1}{\PYGZsq{}}\PYG{p}{,} \PYG{n}{min\PYGZus{}value}\PYG{o}{=}\PYG{k+kc}{None}\PYG{p}{,} \PYG{n}{max\PYGZus{}value}\PYG{o}{=}\PYG{k+kc}{None}\PYG{p}{,}
                   \PYG{n}{normalize}\PYG{o}{=}\PYG{k+kc}{True}\PYG{p}{,} \PYG{n}{colorbar\PYGZus{}label}\PYG{o}{=}\PYG{l+s+s1}{\PYGZsq{}}\PYG{l+s+s1}{meter/year}\PYG{l+s+s1}{\PYGZsq{}}\PYG{p}{,} \PYG{n}{show\PYGZus{}figure}\PYG{o}{=}\PYG{k+kc}{True}\PYG{p}{,} \PYG{n}{aspect\PYGZus{}raster}\PYG{o}{=}\PYG{l+s+s1}{\PYGZsq{}}\PYG{l+s+s1}{./data/morenny/radar/VEL\PYGZus{}Folder/aspect.tif}\PYG{l+s+s1}{\PYGZsq{}}\PYG{p}{,} \PYG{n}{cmap\PYGZus{}aspect}\PYG{o}{=}\PYG{l+s+s1}{\PYGZsq{}}\PYG{l+s+s1}{hsv}\PYG{l+s+s1}{\PYGZsq{}} \PYG{p}{,} \PYG{n}{step}\PYG{o}{=}\PYG{l+m+mi}{5}\PYG{p}{)}
\end{sphinxVerbatim}
}

\end{sphinxuseclass}
\begin{sphinxuseclass}{nboutput}
\begin{sphinxuseclass}{nblast}
\hrule height -\fboxrule\relax
\vspace{\nbsphinxcodecellspacing}

\makeatletter\setbox\nbsphinxpromptbox\box\voidb@x\makeatother

\begin{nbsphinxfancyoutput}

\begin{sphinxuseclass}{output_area}
\begin{sphinxuseclass}{}
\noindent\sphinxincludegraphics[width=876\sphinxpxdimen,height=923\sphinxpxdimen]{{notebooks_RockGlacier_radar_18_0}.png}

\end{sphinxuseclass}
\end{sphinxuseclass}
\end{nbsphinxfancyoutput}

\end{sphinxuseclass}
\end{sphinxuseclass}
\begin{sphinxuseclass}{nbinput}
{
\begin{sphinxVerbatim}[commandchars=\\\{\}]
\llap{\color{nbsphinxin}[8]:\,\hspace{\fboxrule}\hspace{\fboxsep}}\PYG{c+c1}{\PYGZsh{}Visualize Mean Velocity East\PYGZhy{}West}
\PYG{c+c1}{\PYGZsh{} if set batch\PYGZus{}plot=True it will plot static time series figure for each date acquisition}

\PYG{n}{akhdefo\PYGZus{}functions}\PYG{o}{.}\PYG{n}{MeanProducts\PYGZus{}plot\PYGZus{}ts}\PYG{p}{(}\PYG{n}{path\PYGZus{}to\PYGZus{}shapefile}\PYG{o}{=}\PYG{l+s+s1}{\PYGZsq{}}\PYG{l+s+s1}{./data/morenny/radar/geo\PYGZus{}path129/temp\PYGZus{}shapefile\PYGZus{}dir/updated\PYGZus{}shapefiles/20150717\PYGZus{}20230921\PYGZus{}20230921\PYGZus{}E.shp}\PYG{l+s+s1}{\PYGZsq{}}\PYG{p}{,}
                                       \PYG{n}{dem\PYGZus{}path}\PYG{o}{=}\PYG{l+s+s1}{\PYGZsq{}}\PYG{l+s+s1}{./data/morenny/dem.tif}\PYG{l+s+s1}{\PYGZsq{}}\PYG{p}{,} \PYG{n}{out\PYGZus{}folder}\PYG{o}{=}\PYG{l+s+s1}{\PYGZsq{}}\PYG{l+s+s1}{./data/morenny/radar/Figs\PYGZus{}analysis/E}\PYG{l+s+s1}{\PYGZsq{}}\PYG{p}{,}
                                       \PYG{n}{color\PYGZus{}field}\PYG{o}{=}\PYG{l+s+s2}{\PYGZdq{}}\PYG{l+s+s2}{VEL}\PYG{l+s+s2}{\PYGZdq{}}\PYG{p}{,} \PYG{n}{Set\PYGZus{}fig\PYGZus{}MinMax}\PYG{o}{=}\PYG{k+kc}{True}\PYG{p}{,}
                                       \PYG{n}{MinMaxRange}\PYG{o}{=}\PYG{p}{[}\PYG{o}{\PYGZhy{}}\PYG{l+m+mf}{0.2}\PYG{p}{,} \PYG{l+m+mf}{0.2}\PYG{p}{]}\PYG{p}{,} \PYG{n}{opacity}\PYG{o}{=}\PYG{l+m+mf}{0.7}\PYG{p}{,}
                                       \PYG{n}{cmap}\PYG{o}{=}\PYG{l+s+s1}{\PYGZsq{}}\PYG{l+s+s1}{bwr}\PYG{l+s+s1}{\PYGZsq{}}\PYG{p}{,} \PYG{n}{point\PYGZus{}size}\PYG{o}{=}\PYG{l+m+mi}{10}\PYG{p}{,}
                                       \PYG{n}{cbar\PYGZus{}label}\PYG{o}{=}\PYG{l+s+s1}{\PYGZsq{}}\PYG{l+s+s1}{East\PYGZhy{}West (m/year)}\PYG{l+s+s1}{\PYGZsq{}}\PYG{p}{,}
                                       \PYG{n}{batch\PYGZus{}plot}\PYG{o}{=}\PYG{k+kc}{False}\PYG{p}{)}

\PYG{c+c1}{\PYGZsh{}Visualize Mean Velocity North\PYGZhy{}South}
\PYG{n}{akhdefo\PYGZus{}functions}\PYG{o}{.}\PYG{n}{MeanProducts\PYGZus{}plot\PYGZus{}ts}\PYG{p}{(}\PYG{n}{path\PYGZus{}to\PYGZus{}shapefile}\PYG{o}{=}\PYG{l+s+s1}{\PYGZsq{}}\PYG{l+s+s1}{./data/morenny/radar/geo\PYGZus{}path129/temp\PYGZus{}shapefile\PYGZus{}dir/updated\PYGZus{}shapefiles/20150717\PYGZus{}20230921\PYGZus{}20230921\PYGZus{}N.shp}\PYG{l+s+s1}{\PYGZsq{}}\PYG{p}{,}
                                       \PYG{n}{dem\PYGZus{}path}\PYG{o}{=}\PYG{l+s+s1}{\PYGZsq{}}\PYG{l+s+s1}{./data/morenny/dem.tif}\PYG{l+s+s1}{\PYGZsq{}}\PYG{p}{,} \PYG{n}{out\PYGZus{}folder}\PYG{o}{=}\PYG{l+s+s1}{\PYGZsq{}}\PYG{l+s+s1}{./data/morenny/radar/Figs\PYGZus{}analysis/N}\PYG{l+s+s1}{\PYGZsq{}}\PYG{p}{,}
                                       \PYG{n}{color\PYGZus{}field}\PYG{o}{=}\PYG{l+s+s2}{\PYGZdq{}}\PYG{l+s+s2}{VEL}\PYG{l+s+s2}{\PYGZdq{}}\PYG{p}{,} \PYG{n}{Set\PYGZus{}fig\PYGZus{}MinMax}\PYG{o}{=}\PYG{k+kc}{False}\PYG{p}{,}
                                       \PYG{n}{MinMaxRange}\PYG{o}{=}\PYG{p}{[}\PYG{l+m+mi}{0}\PYG{p}{,} \PYG{l+m+mf}{1.5}\PYG{p}{]}\PYG{p}{,} \PYG{n}{opacity}\PYG{o}{=}\PYG{l+m+mf}{0.7}\PYG{p}{,}
                                       \PYG{n}{cmap}\PYG{o}{=}\PYG{l+s+s1}{\PYGZsq{}}\PYG{l+s+s1}{bwr}\PYG{l+s+s1}{\PYGZsq{}}\PYG{p}{,} \PYG{n}{point\PYGZus{}size}\PYG{o}{=}\PYG{l+m+mi}{10}\PYG{p}{,}
                                       \PYG{n}{cbar\PYGZus{}label}\PYG{o}{=}\PYG{l+s+s1}{\PYGZsq{}}\PYG{l+s+s1}{North\PYGZhy{}South (m/year)}\PYG{l+s+s1}{\PYGZsq{}}\PYG{p}{,}
                                       \PYG{n}{batch\PYGZus{}plot}\PYG{o}{=}\PYG{k+kc}{False}\PYG{p}{)}

\PYG{c+c1}{\PYGZsh{}Visualize Mean Velocity 2D Velocity}
\PYG{n}{akhdefo\PYGZus{}functions}\PYG{o}{.}\PYG{n}{MeanProducts\PYGZus{}plot\PYGZus{}ts}\PYG{p}{(}\PYG{n}{path\PYGZus{}to\PYGZus{}shapefile}\PYG{o}{=}\PYG{l+s+s1}{\PYGZsq{}}\PYG{l+s+s1}{./data/morenny/radar/geo\PYGZus{}path129/temp\PYGZus{}shapefile\PYGZus{}dir/updated\PYGZus{}shapefiles/20150717\PYGZus{}20230921\PYGZus{}20230921\PYGZus{}2DVEL.shp}\PYG{l+s+s1}{\PYGZsq{}}\PYG{p}{,}
                                       \PYG{n}{dem\PYGZus{}path}\PYG{o}{=}\PYG{l+s+s1}{\PYGZsq{}}\PYG{l+s+s1}{./data/morenny/dem.tif}\PYG{l+s+s1}{\PYGZsq{}}\PYG{p}{,} \PYG{n}{out\PYGZus{}folder}\PYG{o}{=}\PYG{l+s+s1}{\PYGZsq{}}\PYG{l+s+s1}{./data/morenny/radar/Figs\PYGZus{}analysis/2D}\PYG{l+s+s1}{\PYGZsq{}}\PYG{p}{,}
                                       \PYG{n}{color\PYGZus{}field}\PYG{o}{=}\PYG{l+s+s2}{\PYGZdq{}}\PYG{l+s+s2}{VEL}\PYG{l+s+s2}{\PYGZdq{}}\PYG{p}{,} \PYG{n}{Set\PYGZus{}fig\PYGZus{}MinMax}\PYG{o}{=}\PYG{k+kc}{False}\PYG{p}{,}
                                       \PYG{n}{MinMaxRange}\PYG{o}{=}\PYG{p}{[}\PYG{o}{\PYGZhy{}}\PYG{l+m+mf}{0.5}\PYG{p}{,} \PYG{l+m+mf}{0.5}\PYG{p}{]}\PYG{p}{,} \PYG{n}{opacity}\PYG{o}{=}\PYG{l+m+mf}{0.7}\PYG{p}{,}
                                       \PYG{n}{cmap}\PYG{o}{=}\PYG{l+s+s1}{\PYGZsq{}}\PYG{l+s+s1}{plasma}\PYG{l+s+s1}{\PYGZsq{}}\PYG{p}{,} \PYG{n}{point\PYGZus{}size}\PYG{o}{=}\PYG{l+m+mi}{10}\PYG{p}{,}
                                       \PYG{n}{cbar\PYGZus{}label}\PYG{o}{=}\PYG{l+s+s1}{\PYGZsq{}}\PYG{l+s+s1}{2D\PYGZhy{}VEL (meter/year)}\PYG{l+s+s1}{\PYGZsq{}}\PYG{p}{,}
                                       \PYG{n}{batch\PYGZus{}plot}\PYG{o}{=}\PYG{k+kc}{False}\PYG{p}{)}
\end{sphinxVerbatim}
}

\end{sphinxuseclass}
\begin{sphinxuseclass}{nboutput}
\hrule height -\fboxrule\relax
\vspace{\nbsphinxcodecellspacing}

\makeatletter\setbox\nbsphinxpromptbox\box\voidb@x\makeatother

\begin{nbsphinxfancyoutput}

\begin{sphinxuseclass}{output_area}
\begin{sphinxuseclass}{}
\noindent\sphinxincludegraphics[width=385\sphinxpxdimen,height=690\sphinxpxdimen]{{notebooks_RockGlacier_radar_19_0}.png}

\end{sphinxuseclass}
\end{sphinxuseclass}
\end{nbsphinxfancyoutput}

\end{sphinxuseclass}
\begin{sphinxuseclass}{nboutput}
\hrule height -\fboxrule\relax
\vspace{\nbsphinxcodecellspacing}

\makeatletter\setbox\nbsphinxpromptbox\box\voidb@x\makeatother

\begin{nbsphinxfancyoutput}

\begin{sphinxuseclass}{output_area}
\begin{sphinxuseclass}{}
\noindent\sphinxincludegraphics[width=384\sphinxpxdimen,height=690\sphinxpxdimen]{{notebooks_RockGlacier_radar_19_1}.png}

\end{sphinxuseclass}
\end{sphinxuseclass}
\end{nbsphinxfancyoutput}

\end{sphinxuseclass}
\begin{sphinxuseclass}{nboutput}
\begin{sphinxuseclass}{nblast}
\hrule height -\fboxrule\relax
\vspace{\nbsphinxcodecellspacing}

\makeatletter\setbox\nbsphinxpromptbox\box\voidb@x\makeatother

\begin{nbsphinxfancyoutput}

\begin{sphinxuseclass}{output_area}
\begin{sphinxuseclass}{}
\noindent\sphinxincludegraphics[width=384\sphinxpxdimen,height=690\sphinxpxdimen]{{notebooks_RockGlacier_radar_19_2}.png}

\end{sphinxuseclass}
\end{sphinxuseclass}
\end{nbsphinxfancyoutput}

\end{sphinxuseclass}
\end{sphinxuseclass}

\subsection{Interactive Time Series Analysis in a Dash Web App}
\label{\detokenize{notebooks/RockGlacier_radar:Interactive-Time-Series-Analysis-in-a-Dash-Web-App}}
\sphinxAtStartPar
This plot is part of an interactive web application. Users have the ability to select groups of data points using either the box select or lasso tool, which are superimposed on a satellite image base map. The selected data will then be reflected in the time series chart. Additionally, users can customize the axis labels to suit their dataset; for example, if the dataset measures displacement or velocity in meters, the y\sphinxhyphen{}axis label can be adjusted accordingly from the default mm. Furthermore,
users can download their analysis and the generated figures directly from the web application.

\noindent\sphinxincludegraphics[width=1849\sphinxpxdimen,height=944\sphinxpxdimen]{{dash2d11}.png}

\noindent\sphinxincludegraphics[width=1833\sphinxpxdimen,height=450\sphinxpxdimen]{{newplot111}.png}

\noindent\sphinxincludegraphics[width=1000\sphinxpxdimen,height=800\sphinxpxdimen]{{newplot12}.png}

\begin{sphinxuseclass}{nbinput}
\begin{sphinxuseclass}{nblast}
{
\begin{sphinxVerbatim}[commandchars=\\\{\}]
\llap{\color{nbsphinxin}[ ]:\,\hspace{\fboxrule}\hspace{\fboxsep}}

\PYG{n}{akhdefo\PYGZus{}dashApp}\PYG{p}{(}
\PYG{n}{Path\PYGZus{}to\PYGZus{}Shapefile}\PYG{o}{=}\PYG{l+s+s1}{\PYGZsq{}}\PYG{l+s+s1}{./data/morenny/radar/geo\PYGZus{}path129/temp\PYGZus{}shapefile\PYGZus{}dir/updated\PYGZus{}shapefiles/20150717\PYGZus{}20230921\PYGZus{}20230921\PYGZus{}2DVEL.shp}\PYG{l+s+s1}{\PYGZsq{}}\PYG{p}{,}
                                  \PYG{n}{port}\PYG{o}{=}\PYG{l+m+mi}{8057}\PYG{p}{,} \PYG{n}{BaseMap}\PYG{o}{=}\PYG{k+kc}{True}\PYG{p}{,} \PYG{n}{basemap\PYGZus{}type}\PYG{o}{=}\PYG{l+s+s1}{\PYGZsq{}}\PYG{l+s+s1}{image}\PYG{l+s+s1}{\PYGZsq{}}\PYG{p}{)}
\end{sphinxVerbatim}
}

\end{sphinxuseclass}
\end{sphinxuseclass}
\begin{sphinxuseclass}{nbinput}
\begin{sphinxuseclass}{nblast}
{
\begin{sphinxVerbatim}[commandchars=\\\{\}]
\llap{\color{nbsphinxin}[ ]:\,\hspace{\fboxrule}\hspace{\fboxsep}}
\PYG{n}{akhdefo\PYGZus{}dashApp}\PYG{p}{(}
\PYG{n}{Path\PYGZus{}to\PYGZus{}Shapefile}\PYG{o}{=}\PYG{l+s+s1}{\PYGZsq{}}\PYG{l+s+s1}{./data/morenny/radar/geo\PYGZus{}path129/temp\PYGZus{}shapefile\PYGZus{}dir/updated\PYGZus{}shapefiles/20150717\PYGZus{}20230921\PYGZus{}20230921\PYGZus{}N.shp}\PYG{l+s+s1}{\PYGZsq{}}\PYG{p}{,}
                                  \PYG{n}{port}\PYG{o}{=}\PYG{l+m+mi}{8056}\PYG{p}{,} \PYG{n}{BaseMap}\PYG{o}{=}\PYG{k+kc}{True}\PYG{p}{,} \PYG{n}{basemap\PYGZus{}type}\PYG{o}{=}\PYG{l+s+s1}{\PYGZsq{}}\PYG{l+s+s1}{image}\PYG{l+s+s1}{\PYGZsq{}}\PYG{p}{)}
\end{sphinxVerbatim}
}

\end{sphinxuseclass}
\end{sphinxuseclass}
\begin{sphinxuseclass}{nbinput}
\begin{sphinxuseclass}{nblast}
{
\begin{sphinxVerbatim}[commandchars=\\\{\}]
\llap{\color{nbsphinxin}[ ]:\,\hspace{\fboxrule}\hspace{\fboxsep}}
\PYG{n}{akhdefo\PYGZus{}dashApp}\PYG{p}{(}
\PYG{n}{Path\PYGZus{}to\PYGZus{}Shapefile}\PYG{o}{=}\PYG{l+s+s1}{\PYGZsq{}}\PYG{l+s+s1}{./data/morenny/radar/geo\PYGZus{}path129/temp\PYGZus{}shapefile\PYGZus{}dir/updated\PYGZus{}shapefiles/20150717\PYGZus{}20230921\PYGZus{}20230921\PYGZus{}E.shp}\PYG{l+s+s1}{\PYGZsq{}}\PYG{p}{,}
                                  \PYG{n}{port}\PYG{o}{=}\PYG{l+m+mi}{8055}\PYG{p}{,} \PYG{n}{BaseMap}\PYG{o}{=}\PYG{k+kc}{True}\PYG{p}{,} \PYG{n}{basemap\PYGZus{}type}\PYG{o}{=}\PYG{l+s+s1}{\PYGZsq{}}\PYG{l+s+s1}{image}\PYG{l+s+s1}{\PYGZsq{}}\PYG{p}{)}
\end{sphinxVerbatim}
}

\end{sphinxuseclass}
\end{sphinxuseclass}
\sphinxstepscope


\section{MudCreek 2017 Landslide Optical Imagery Example}
\label{\detokenize{notebooks/MudCreek_Landslide_optical:MudCreek-2017-Landslide-Optical-Imagery-Example}}\label{\detokenize{notebooks/MudCreek_Landslide_optical::doc}}

\subsection{Download data from planet.com website}
\label{\detokenize{notebooks/MudCreek_Landslide_optical:Download-data-from-planet.com-website}}
\sphinxAtStartPar
Create an Account or Log In: To access data, you generally need an account with \sphinxurl{https://www.planet.com} . If you don’t have one, you’ll need to sign up. Otherwise, log in with your existing credentials.

\noindent\sphinxincludegraphics[width=1690\sphinxpxdimen,height=851\sphinxpxdimen]{{planet_screenshot11}.png}


\subsection{Import AkhDefo Packages.}
\label{\detokenize{notebooks/MudCreek_Landslide_optical:Import-AkhDefo-Packages.}}
\begin{sphinxuseclass}{nbinput}
\begin{sphinxuseclass}{nblast}
{
\begin{sphinxVerbatim}[commandchars=\\\{\}]
\llap{\color{nbsphinxin}[1]:\,\hspace{\fboxrule}\hspace{\fboxsep}}
\PYG{c+c1}{\PYGZsh{}Functions for data preparations}
\PYG{k+kn}{import} \PYG{n+nn}{akhdefo\PYGZus{}functions}
\PYG{k+kn}{from} \PYG{n+nn}{akhdefo\PYGZus{}functions} \PYG{k+kn}{import} \PYG{n}{akhdefo\PYGZus{}download\PYGZus{}planet}
\PYG{k+kn}{from} \PYG{n+nn}{akhdefo\PYGZus{}functions} \PYG{k+kn}{import} \PYG{n}{read\PYGZus{}data\PYGZus{}prep}
\PYG{k+kn}{from} \PYG{n+nn}{akhdefo\PYGZus{}functions} \PYG{k+kn}{import} \PYG{n}{move\PYGZus{}files\PYGZus{}with\PYGZus{}string}
\PYG{k+kn}{from} \PYG{n+nn}{akhdefo\PYGZus{}functions} \PYG{k+kn}{import} \PYG{n}{Crop\PYGZus{}to\PYGZus{}AOI}

\PYG{c+c1}{\PYGZsh{}Functions data processing}
\PYG{k+kn}{from} \PYG{n+nn}{akhdefo\PYGZus{}functions} \PYG{k+kn}{import} \PYG{n}{scatter\PYGZus{}area\PYGZus{}mask}
\PYG{k+kn}{from} \PYG{n+nn}{akhdefo\PYGZus{}functions} \PYG{k+kn}{import} \PYG{n}{Raster\PYGZus{}Correction}
\PYG{k+kn}{from} \PYG{n+nn}{akhdefo\PYGZus{}functions} \PYG{k+kn}{import} \PYG{n}{Optical\PYGZus{}flow\PYGZus{}akhdefo}

\PYG{c+c1}{\PYGZsh{}Functions for shapefile point data interpolation to raster geotif}
\PYG{c+c1}{\PYGZsh{}Function Autovariogram use Kriging method it fits the data to different variogram models and automatically use the best fit variogram model (Computationally Heavy but you can process data in chunks)}
\PYG{k+kn}{from} \PYG{n+nn}{akhdefo\PYGZus{}functions} \PYG{k+kn}{import} \PYG{n}{Auto\PYGZus{}Variogram}

\PYG{c+c1}{\PYGZsh{} interpoate\PYGZus{}xyz uses the following interpolation methods (nearest, linear and cubic)}
\PYG{k+kn}{from} \PYG{n+nn}{akhdefo\PYGZus{}functions} \PYG{k+kn}{import} \PYG{n}{interpolate\PYGZus{}xyz}

\PYG{c+c1}{\PYGZsh{}Functions for data Visualization and Analysis}
\PYG{k+kn}{from} \PYG{n+nn}{akhdefo\PYGZus{}functions} \PYG{k+kn}{import} \PYG{n}{MeanProducts\PYGZus{}plot\PYGZus{}ts}
\PYG{k+kn}{from} \PYG{n+nn}{akhdefo\PYGZus{}functions} \PYG{k+kn}{import} \PYG{n}{plot\PYGZus{}stackNetwork}
\PYG{k+kn}{from} \PYG{n+nn}{akhdefo\PYGZus{}functions} \PYG{k+kn}{import} \PYG{n}{akhdefo\PYGZus{}viewer}
\PYG{k+kn}{from} \PYG{n+nn}{akhdefo\PYGZus{}functions} \PYG{k+kn}{import} \PYG{n}{akhdefo\PYGZus{}dashApp}
\end{sphinxVerbatim}
}

\end{sphinxuseclass}
\end{sphinxuseclass}

\subsection{Download Data using Planetlab API Account}
\label{\detokenize{notebooks/MudCreek_Landslide_optical:Download-Data-using-Planetlab-API-Account}}
\begin{sphinxuseclass}{nbinput}
\begin{sphinxuseclass}{nblast}
{
\begin{sphinxVerbatim}[commandchars=\\\{\}]
\llap{\color{nbsphinxin}[ ]:\,\hspace{\fboxrule}\hspace{\fboxsep}}
\PYG{l+s+sd}{\PYGZsq{}\PYGZsq{}\PYGZsq{}}

\PYG{l+s+sd}{First Run the below command and make sure \PYGZob{}download=False\PYGZcb{} to see list of available dataset.}

\PYG{l+s+sd}{Its highly recommended to use images from same frame number and same path number for the entire time\PYGZhy{}series for more accurate results}

\PYG{l+s+sd}{\PYGZsq{}\PYGZsq{}\PYGZsq{}}

\PYG{c+c1}{\PYGZsh{}uncomment below to download data using planet lab api key}

\PYG{l+s+sd}{\PYGZsq{}\PYGZsq{}\PYGZsq{}}
\PYG{l+s+sd}{akhdefo\PYGZus{}download\PYGZus{}planet(planet\PYGZus{}api\PYGZus{}key=\PYGZsq{}\PYGZsq{}, AOI=\PYGZsq{}./data/mudcreek/AOI.shp\PYGZsq{}, start\PYGZus{}date=\PYGZsq{}May 1, 2018\PYGZsq{}, end\PYGZus{}date=\PYGZsq{} September 30 2022\PYGZsq{}, limit=5, item\PYGZus{}type=\PYGZsq{}PSScene\PYGZsq{},}
\PYG{l+s+sd}{                        product\PYGZus{}bundle=\PYGZsq{}analytic\PYGZus{}sr\PYGZus{}udm2\PYGZsq{},}
\PYG{l+s+sd}{                        clear\PYGZus{}percent=90, cloud\PYGZus{}filter=0.1,}
\PYG{l+s+sd}{                        output\PYGZus{}folder=\PYGZsq{}raw\PYGZus{}data\PYGZsq{},}
\PYG{l+s+sd}{                        clip\PYGZus{}flag=True, download\PYGZus{}data=False}

\PYG{l+s+sd}{\PYGZsq{}\PYGZsq{}\PYGZsq{}}
\PYG{c+c1}{\PYGZsh{} for this excecise we download data from planet labs explorer web application}
\PYG{c+c1}{\PYGZsh{}Lets Navigate to the zip folder directory}

\PYG{k+kn}{import} \PYG{n+nn}{os}

\PYG{c+c1}{\PYGZsh{} Replace \PYGZsq{}your\PYGZus{}directory\PYGZus{}path\PYGZsq{} with the path of your directory}
\PYG{n}{zip\PYGZus{}dir} \PYG{o}{=} \PYG{l+s+s1}{\PYGZsq{}}\PYG{l+s+s1}{./data/mudcreek/optical/zip\PYGZus{}folder/}\PYG{l+s+s1}{\PYGZsq{}}

\PYG{c+c1}{\PYGZsh{} List all files and directories in the specified path}
\PYG{n}{contents} \PYG{o}{=} \PYG{n}{os}\PYG{o}{.}\PYG{n}{listdir}\PYG{p}{(}\PYG{n}{zip\PYGZus{}dir}\PYG{p}{)}

\PYG{c+c1}{\PYGZsh{} Print the contents}
\PYG{k}{for} \PYG{n}{item} \PYG{o+ow}{in} \PYG{n}{contents}\PYG{p}{:}
    \PYG{n+nb}{print}\PYG{p}{(}\PYG{n}{item}\PYG{p}{)}

\PYG{c+c1}{\PYGZsh{} The zip folder inside the zip folder directory inclue cropped planet lab imagery between 2018 to 2022. lets extract the data and look at the images}
\end{sphinxVerbatim}
}

\end{sphinxuseclass}
\end{sphinxuseclass}

\subsection{Extract downloaded zip files}
\label{\detokenize{notebooks/MudCreek_Landslide_optical:Extract-downloaded-zip-files}}
\begin{sphinxuseclass}{nbinput}
\begin{sphinxuseclass}{nblast}
{
\begin{sphinxVerbatim}[commandchars=\\\{\}]
\llap{\color{nbsphinxin}[ ]:\,\hspace{\fboxrule}\hspace{\fboxsep}}\PYG{c+c1}{\PYGZsh{}This functions extracts data from all the granuples zip directory.}
\PYG{n}{read\PYGZus{}data\PYGZus{}prep}\PYG{p}{(}\PYG{n}{zip\PYGZus{}dir}\PYG{o}{=}\PYG{l+s+s1}{\PYGZsq{}}\PYG{l+s+s1}{./data/mudcreek/optical/zip\PYGZus{}dir/}\PYG{l+s+s1}{\PYGZsq{}}\PYG{p}{,}
                                 \PYG{n}{image\PYGZus{}dir}\PYG{o}{=}\PYG{l+s+s1}{\PYGZsq{}}\PYG{l+s+s1}{./data/mudcreek/optical/image\PYGZus{}dir}\PYG{l+s+s1}{\PYGZsq{}}\PYG{p}{,}
                                 \PYG{n}{ext\PYGZus{}image\PYGZus{}file}\PYG{o}{=}\PYG{l+s+s1}{\PYGZsq{}}\PYG{l+s+s1}{harmonized\PYGZus{}clip.tif}\PYG{l+s+s1}{\PYGZsq{}}\PYG{p}{,} \PYG{n}{udm\PYGZus{}mask\PYGZus{}dir}\PYG{o}{=}\PYG{l+s+s1}{\PYGZsq{}}\PYG{l+s+s1}{./data/mudcreek/optical/udm\PYGZus{}mask\PYGZus{}dir}\PYG{l+s+s1}{\PYGZsq{}}\PYG{p}{,} \PYG{n}{ext\PYGZus{}udm\PYGZus{}mask\PYGZus{}file}\PYG{o}{=}\PYG{l+s+s1}{\PYGZsq{}}\PYG{l+s+s1}{udm2\PYGZus{}clip.tif}\PYG{l+s+s1}{\PYGZsq{}}\PYG{p}{)}
\end{sphinxVerbatim}
}

\end{sphinxuseclass}
\end{sphinxuseclass}
\begin{sphinxuseclass}{nbinput}
\begin{sphinxuseclass}{nblast}
{
\begin{sphinxVerbatim}[commandchars=\\\{\}]
\llap{\color{nbsphinxin}[ ]:\,\hspace{\fboxrule}\hspace{\fboxsep}}\PYG{c+c1}{\PYGZsh{} This function will move all the data from each granules subdir to a single folder}
\PYG{c+c1}{\PYGZsh{} move\PYGZus{}files\PYGZus{}with\PYGZus{}string(source\PYGZus{}dir=\PYGZsq{}./data/mudcreek/optical/RS\PYGZus{}dir\PYGZus{}path129/\PYGZsq{}, dest\PYGZus{}dir=\PYGZsq{}./data/mudcreek/optical/RS\PYGZus{}path129\PYGZsq{}, search\PYGZus{}string=\PYGZsq{}.tif\PYGZsq{})}
\PYG{c+c1}{\PYGZsh{} move\PYGZus{}files\PYGZus{}with\PYGZus{}string(source\PYGZus{}dir=\PYGZsq{}./data/mudcreek/optical/scatter\PYGZus{}dir/\PYGZsq{}, dest\PYGZus{}dir=\PYGZsq{}./data/mudcreek/optical/scatter\PYGZus{}rs\PYGZsq{}, search\PYGZus{}string=\PYGZsq{}.tif\PYGZsq{})}

\end{sphinxVerbatim}
}

\end{sphinxuseclass}
\end{sphinxuseclass}
\begin{sphinxuseclass}{nbinput}
\begin{sphinxuseclass}{nblast}
{
\begin{sphinxVerbatim}[commandchars=\\\{\}]
\llap{\color{nbsphinxin}[ ]:\,\hspace{\fboxrule}\hspace{\fboxsep}}
\PYG{l+s+sd}{\PYGZsq{}\PYGZsq{}\PYGZsq{}}
\PYG{l+s+sd}{Crop all raster to Area of interest bounding box}


\PYG{l+s+sd}{\PYGZsq{}\PYGZsq{}\PYGZsq{}}
\PYG{n}{Crop\PYGZus{}to\PYGZus{}AOI}\PYG{p}{(}\PYG{n}{Path\PYGZus{}to\PYGZus{}WorkingDir}\PYG{o}{=}\PYG{l+s+s1}{\PYGZsq{}}\PYG{l+s+s1}{./data/mudcreek/optical/image\PYGZus{}dir}\PYG{l+s+s1}{\PYGZsq{}}\PYG{p}{,}
                              \PYG{n}{Path\PYGZus{}to\PYGZus{}AOI\PYGZus{}shapefile}\PYG{o}{=}\PYG{l+s+s1}{\PYGZsq{}}\PYG{l+s+s1}{./data/mudcreek/AOI.shp}\PYG{l+s+s1}{\PYGZsq{}}\PYG{p}{,}
                              \PYG{n}{output\PYGZus{}CroppedDir}\PYG{o}{=}\PYG{l+s+s1}{\PYGZsq{}}\PYG{l+s+s1}{./data/mudcreek/optical/image\PYGZus{}dir\PYGZus{}aoi}\PYG{l+s+s1}{\PYGZsq{}}\PYG{p}{,} \PYG{n}{file\PYGZus{}ex}\PYG{o}{=}\PYG{l+s+s1}{\PYGZsq{}}\PYG{l+s+s1}{.tif}\PYG{l+s+s1}{\PYGZsq{}}\PYG{p}{)}


\end{sphinxVerbatim}
}

\end{sphinxuseclass}
\end{sphinxuseclass}

\subsection{Preparing raster for optical flow and displacement calculation!}
\label{\detokenize{notebooks/MudCreek_Landslide_optical:Preparing-raster-for-optical-flow-and-displacement-calculation!}}
\sphinxAtStartPar
The data needs to be converted to “uint8” format and be calibrated and normalized.

\begin{sphinxuseclass}{nbinput}
\begin{sphinxuseclass}{nblast}
{
\begin{sphinxVerbatim}[commandchars=\\\{\}]
\llap{\color{nbsphinxin}[ ]:\,\hspace{\fboxrule}\hspace{\fboxsep}}\PYG{k+kn}{import} \PYG{n+nn}{warnings}
\PYG{n}{warnings}\PYG{o}{.}\PYG{n}{filterwarnings}\PYG{p}{(}\PYG{l+s+s2}{\PYGZdq{}}\PYG{l+s+s2}{ignore}\PYG{l+s+s2}{\PYGZdq{}}\PYG{p}{)}


\PYG{n}{Raster\PYGZus{}Correction}\PYG{p}{(}\PYG{n}{input\PYGZus{}path}\PYG{o}{=}\PYG{l+s+s2}{\PYGZdq{}}\PYG{l+s+s2}{./data/mudcreek/optical/image\PYGZus{}dir\PYGZus{}aoi}\PYG{l+s+s2}{\PYGZdq{}}\PYG{p}{,} \PYG{n}{output\PYGZus{}path}\PYG{o}{=}\PYG{l+s+s2}{\PYGZdq{}}\PYG{l+s+s2}{./data/mudcreek/optical/image\PYGZus{}dir\PYGZus{}aoi\PYGZus{}filt}\PYG{l+s+s2}{\PYGZdq{}}\PYG{p}{,} \PYG{n}{limit}\PYG{o}{=}\PYG{k+kc}{None}\PYG{p}{,}
                                    \PYG{n}{lowpass\PYGZus{}kernel\PYGZus{}size}\PYG{o}{=}\PYG{k+kc}{None}\PYG{p}{,} \PYG{n}{bilateral\PYGZus{}win\PYGZus{}size}\PYG{o}{=}\PYG{l+m+mi}{7}\PYG{p}{,} \PYG{n}{bilateral\PYGZus{}sigma\PYGZus{}color}\PYG{o}{=}\PYG{l+m+mi}{75}\PYG{p}{,}
                                    \PYG{n}{bilateral\PYGZus{}sigma\PYGZus{}spatial}\PYG{o}{=}\PYG{l+m+mi}{75}\PYG{p}{,}
                                    \PYG{n}{clip\PYGZus{}percentiles}\PYG{o}{=}\PYG{p}{[}\PYG{l+m+mi}{2}\PYG{p}{,} \PYG{l+m+mi}{98}\PYG{p}{]}\PYG{p}{,} \PYG{n}{optical}\PYG{o}{=}\PYG{k+kc}{True}\PYG{p}{,}
                                    \PYG{n}{scale}\PYG{o}{=}\PYG{k+kc}{None}\PYG{p}{,} \PYG{n}{Vegetation\PYGZus{}mask}\PYG{o}{=}\PYG{k+kc}{None}\PYG{p}{)}
\end{sphinxVerbatim}
}

\end{sphinxuseclass}
\end{sphinxuseclass}

\subsection{Applying optical flow and producing time\sphinxhyphen{}series deformation products.}
\label{\detokenize{notebooks/MudCreek_Landslide_optical:Applying-optical-flow-and-producing-time-series-deformation-products.}}
\begin{sphinxuseclass}{nbinput}
\begin{sphinxuseclass}{nblast}
{
\begin{sphinxVerbatim}[commandchars=\\\{\}]
\llap{\color{nbsphinxin}[ ]:\,\hspace{\fboxrule}\hspace{\fboxsep}}\PYG{k+kn}{import} \PYG{n+nn}{warnings}
\PYG{n}{warnings}\PYG{o}{.}\PYG{n}{filterwarnings}\PYG{p}{(}\PYG{l+s+s2}{\PYGZdq{}}\PYG{l+s+s2}{ignore}\PYG{l+s+s2}{\PYGZdq{}}\PYG{p}{)}

\PYG{k+kn}{import} \PYG{n+nn}{akhdefo\PYGZus{}functions}
\PYG{k+kn}{from} \PYG{n+nn}{akhdefo\PYGZus{}functions} \PYG{k+kn}{import} \PYG{n}{Optical\PYGZus{}flow\PYGZus{}akhdefo}


\PYG{n}{Optical\PYGZus{}flow\PYGZus{}akhdefo}\PYG{p}{(}\PYG{n}{input\PYGZus{}dir}\PYG{o}{=}\PYG{l+s+s1}{\PYGZsq{}}\PYG{l+s+s1}{./data/mudcreek/optical/image\PYGZus{}dir\PYGZus{}aoi\PYGZus{}filt}\PYG{l+s+s1}{\PYGZsq{}}\PYG{p}{,} \PYG{n}{output\PYGZus{}dir}\PYG{o}{=}\PYG{l+s+s1}{\PYGZsq{}}\PYG{l+s+s1}{./data/mudcreek/optical/geo}\PYG{l+s+s1}{\PYGZsq{}}\PYG{p}{,} \PYG{n}{AOI}\PYG{o}{=}\PYG{l+s+s1}{\PYGZsq{}}\PYG{l+s+s1}{./data/mudcreek/AOI.shp}\PYG{l+s+s1}{\PYGZsq{}}\PYG{p}{,} \PYG{n}{zscore\PYGZus{}threshold}\PYG{o}{=}\PYG{l+m+mi}{2}\PYG{p}{,}
                                       \PYG{n}{ssim\PYGZus{}thresh}\PYG{o}{=}\PYG{l+m+mf}{0.6}\PYG{p}{,} \PYG{n}{image\PYGZus{}resolution}\PYG{o}{=}\PYG{l+s+s1}{\PYGZsq{}}\PYG{l+s+s1}{3.125m}\PYG{l+s+s1}{\PYGZsq{}}\PYG{p}{,} \PYG{n}{interpolate}\PYG{o}{=}\PYG{l+s+s1}{\PYGZsq{}}\PYG{l+s+s1}{kriging}\PYG{l+s+s1}{\PYGZsq{}}\PYG{p}{,}
                                       \PYG{n}{show\PYGZus{}figure}\PYG{o}{=}\PYG{k+kc}{False}\PYG{p}{,} \PYG{n}{point\PYGZus{}size}\PYG{o}{=}\PYG{l+m+mi}{2}\PYG{p}{,} \PYG{n}{dem\PYGZus{}path}\PYG{o}{=}\PYG{l+s+s1}{\PYGZsq{}}\PYG{l+s+s1}{./data/mudcreek/dem.tif}\PYG{l+s+s1}{\PYGZsq{}}\PYG{p}{,} \PYG{n}{smoothing\PYGZus{}kernel\PYGZus{}size}\PYG{o}{=}\PYG{l+m+mi}{2}\PYG{p}{,}
                                       \PYG{n}{Vegetation\PYGZus{}mask}\PYG{o}{=}\PYG{k+kc}{None}\PYG{p}{,} \PYG{n}{VEL\PYGZus{}scale}\PYG{o}{=}\PYG{l+s+s1}{\PYGZsq{}}\PYG{l+s+s1}{year}\PYG{l+s+s1}{\PYGZsq{}}\PYG{p}{,} \PYG{n}{VEL\PYGZus{}Mode}\PYG{o}{=}\PYG{l+s+s1}{\PYGZsq{}}\PYG{l+s+s1}{linear}\PYG{l+s+s1}{\PYGZsq{}}\PYG{p}{,} \PYG{n}{good\PYGZus{}match\PYGZus{}option}\PYG{o}{=}\PYG{l+m+mf}{0.75}\PYG{p}{,}
                                       \PYG{n}{hillshade\PYGZus{}option}\PYG{o}{=}\PYG{k+kc}{True}\PYG{p}{,} \PYG{n}{shapefile\PYGZus{}output}\PYG{o}{=}\PYG{k+kc}{True}\PYG{p}{,} \PYG{n}{max\PYGZus{}triplet\PYGZus{}interval}\PYG{o}{=}\PYG{l+m+mi}{300}\PYG{p}{,} \PYG{n}{pixel\PYGZus{}size}\PYG{o}{=}\PYG{l+m+mi}{10}\PYG{p}{,} \PYG{n}{num\PYGZus{}chunks}\PYG{o}{=}\PYG{l+m+mi}{10}\PYG{p}{,} \PYG{n}{overlap\PYGZus{}percentage}\PYG{o}{=}\PYG{l+m+mi}{0}\PYG{p}{,} \PYG{n}{pyr\PYGZus{}scale}\PYG{o}{=}\PYG{l+m+mf}{0.5}\PYG{p}{,} \PYG{n}{levels}\PYG{o}{=}\PYG{l+m+mi}{15}\PYG{p}{,}
                                       \PYG{n}{winsize}\PYG{o}{=}\PYG{l+m+mi}{148}\PYG{p}{,} \PYG{n}{iterations}\PYG{o}{=}\PYG{l+m+mi}{7}\PYG{p}{,} \PYG{n}{poly\PYGZus{}n}\PYG{o}{=}\PYG{l+m+mi}{7}\PYG{p}{,} \PYG{n}{poly\PYGZus{}sigma}\PYG{o}{=}\PYG{l+m+mf}{1.5}\PYG{p}{,}
                                       \PYG{n}{flags}\PYG{o}{=}\PYG{l+m+mi}{1}\PYG{p}{,} \PYG{n}{master\PYGZus{}reference}\PYG{o}{=}\PYG{l+s+s1}{\PYGZsq{}}\PYG{l+s+s1}{single}\PYG{l+s+s1}{\PYGZsq{}}\PYG{p}{,}
                                       \PYG{n}{selection\PYGZus{}Mode}\PYG{o}{=}\PYG{l+s+s1}{\PYGZsq{}}\PYG{l+s+s1}{pair}\PYG{l+s+s1}{\PYGZsq{}}\PYG{p}{,} \PYG{n}{start\PYGZus{}date}\PYG{o}{=}\PYG{l+s+s1}{\PYGZsq{}}\PYG{l+s+s1}{20160101}\PYG{l+s+s1}{\PYGZsq{}}\PYG{p}{,} \PYG{n}{end\PYGZus{}date}\PYG{o}{=}\PYG{l+s+s1}{\PYGZsq{}}\PYG{l+s+s1}{20170518}\PYG{l+s+s1}{\PYGZsq{}}\PYG{p}{,} \PYG{n}{krig\PYGZus{}method}\PYG{o}{=}\PYG{l+s+s1}{\PYGZsq{}}\PYG{l+s+s1}{universal}\PYG{l+s+s1}{\PYGZsq{}}\PYG{p}{,} \PYG{n}{spatial\PYGZus{}ref}\PYG{o}{=}\PYG{k+kc}{True}\PYG{p}{)}
\end{sphinxVerbatim}
}

\end{sphinxuseclass}
\end{sphinxuseclass}

\subsection{Visualize Displacement Products}
\label{\detokenize{notebooks/MudCreek_Landslide_optical:Visualize-Displacement-Products}}
\begin{sphinxuseclass}{nbinput}
{
\begin{sphinxVerbatim}[commandchars=\\\{\}]
\llap{\color{nbsphinxin}[2]:\,\hspace{\fboxrule}\hspace{\fboxsep}}\PYG{c+c1}{\PYGZsh{}Visualize Mean Velocity East\PYGZhy{}West}
\PYG{c+c1}{\PYGZsh{} if set batch\PYGZus{}plot=True it will plot static time series figure for each date acquisition}

\PYG{n}{MeanProducts\PYGZus{}plot\PYGZus{}ts}\PYG{p}{(}\PYG{n}{path\PYGZus{}to\PYGZus{}shapefile}\PYG{o}{=}\PYG{l+s+s1}{\PYGZsq{}}\PYG{l+s+s1}{./data/mudcreek/optical/geo/temp\PYGZus{}shapefile\PYGZus{}dir/updated\PYGZus{}shapefiles/20161002\PYGZus{}20170516\PYGZus{}20170516\PYGZus{}E.shp}\PYG{l+s+s1}{\PYGZsq{}}\PYG{p}{,}
                                       \PYG{n}{dem\PYGZus{}path}\PYG{o}{=}\PYG{l+s+s1}{\PYGZsq{}}\PYG{l+s+s1}{./data/mudcreek/dem.tif}\PYG{l+s+s1}{\PYGZsq{}}\PYG{p}{,} \PYG{n}{out\PYGZus{}folder}\PYG{o}{=}\PYG{l+s+s1}{\PYGZsq{}}\PYG{l+s+s1}{./data/mudcreek/optical/Figs\PYGZus{}analysis/E}\PYG{l+s+s1}{\PYGZsq{}}\PYG{p}{,}
                                       \PYG{n}{color\PYGZus{}field}\PYG{o}{=}\PYG{l+s+s2}{\PYGZdq{}}\PYG{l+s+s2}{VEL}\PYG{l+s+s2}{\PYGZdq{}}\PYG{p}{,} \PYG{n}{Set\PYGZus{}fig\PYGZus{}MinMax}\PYG{o}{=}\PYG{k+kc}{False}\PYG{p}{,}
                                       \PYG{n}{MinMaxRange}\PYG{o}{=}\PYG{p}{[}\PYG{l+m+mi}{0}\PYG{p}{,} \PYG{l+m+mf}{1.5}\PYG{p}{]}\PYG{p}{,} \PYG{n}{opacity}\PYG{o}{=}\PYG{l+m+mf}{0.7}\PYG{p}{,}
                                       \PYG{n}{cmap}\PYG{o}{=}\PYG{l+s+s1}{\PYGZsq{}}\PYG{l+s+s1}{bwr}\PYG{l+s+s1}{\PYGZsq{}}\PYG{p}{,} \PYG{n}{point\PYGZus{}size}\PYG{o}{=}\PYG{l+m+mi}{10}\PYG{p}{,}
                                       \PYG{n}{cbar\PYGZus{}label}\PYG{o}{=}\PYG{l+s+s1}{\PYGZsq{}}\PYG{l+s+s1}{East\PYGZhy{}West (m/year)}\PYG{l+s+s1}{\PYGZsq{}}\PYG{p}{,}
                                       \PYG{n}{batch\PYGZus{}plot}\PYG{o}{=}\PYG{k+kc}{False}\PYG{p}{)}

\PYG{c+c1}{\PYGZsh{}Visualize Mean Velocity North\PYGZhy{}South}
\PYG{n}{MeanProducts\PYGZus{}plot\PYGZus{}ts}\PYG{p}{(}\PYG{n}{path\PYGZus{}to\PYGZus{}shapefile}\PYG{o}{=}\PYG{l+s+s1}{\PYGZsq{}}\PYG{l+s+s1}{./data/mudcreek/optical/geo/temp\PYGZus{}shapefile\PYGZus{}dir/updated\PYGZus{}shapefiles/20161002\PYGZus{}20170516\PYGZus{}20170516\PYGZus{}N.shp}\PYG{l+s+s1}{\PYGZsq{}}\PYG{p}{,}
                                       \PYG{n}{dem\PYGZus{}path}\PYG{o}{=}\PYG{l+s+s1}{\PYGZsq{}}\PYG{l+s+s1}{./data/mudcreek/dem.tif}\PYG{l+s+s1}{\PYGZsq{}}\PYG{p}{,} \PYG{n}{out\PYGZus{}folder}\PYG{o}{=}\PYG{l+s+s1}{\PYGZsq{}}\PYG{l+s+s1}{./data/mudcreek/optical/Figs\PYGZus{}analysis/N}\PYG{l+s+s1}{\PYGZsq{}}\PYG{p}{,}
                                       \PYG{n}{color\PYGZus{}field}\PYG{o}{=}\PYG{l+s+s2}{\PYGZdq{}}\PYG{l+s+s2}{VEL}\PYG{l+s+s2}{\PYGZdq{}}\PYG{p}{,} \PYG{n}{Set\PYGZus{}fig\PYGZus{}MinMax}\PYG{o}{=}\PYG{k+kc}{False}\PYG{p}{,}
                                       \PYG{n}{MinMaxRange}\PYG{o}{=}\PYG{p}{[}\PYG{l+m+mi}{0}\PYG{p}{,} \PYG{l+m+mf}{1.5}\PYG{p}{]}\PYG{p}{,} \PYG{n}{opacity}\PYG{o}{=}\PYG{l+m+mf}{0.7}\PYG{p}{,}
                                       \PYG{n}{cmap}\PYG{o}{=}\PYG{l+s+s1}{\PYGZsq{}}\PYG{l+s+s1}{bwr}\PYG{l+s+s1}{\PYGZsq{}}\PYG{p}{,} \PYG{n}{point\PYGZus{}size}\PYG{o}{=}\PYG{l+m+mi}{10}\PYG{p}{,}
                                       \PYG{n}{cbar\PYGZus{}label}\PYG{o}{=}\PYG{l+s+s1}{\PYGZsq{}}\PYG{l+s+s1}{North\PYGZhy{}South (m/year)}\PYG{l+s+s1}{\PYGZsq{}}\PYG{p}{,}
                                       \PYG{n}{batch\PYGZus{}plot}\PYG{o}{=}\PYG{k+kc}{False}\PYG{p}{)}

\PYG{c+c1}{\PYGZsh{}Visualize Mean Velocity 2D Velocity}
\PYG{n}{MeanProducts\PYGZus{}plot\PYGZus{}ts}\PYG{p}{(}\PYG{n}{path\PYGZus{}to\PYGZus{}shapefile}\PYG{o}{=}\PYG{l+s+s1}{\PYGZsq{}}\PYG{l+s+s1}{./data/mudcreek/optical/geo/temp\PYGZus{}shapefile\PYGZus{}dir/updated\PYGZus{}shapefiles/20161002\PYGZus{}20170516\PYGZus{}20170516\PYGZus{}2DVEL.shp}\PYG{l+s+s1}{\PYGZsq{}}\PYG{p}{,}
                                       \PYG{n}{dem\PYGZus{}path}\PYG{o}{=}\PYG{l+s+s1}{\PYGZsq{}}\PYG{l+s+s1}{./data/mudcreek/dem.tif}\PYG{l+s+s1}{\PYGZsq{}}\PYG{p}{,} \PYG{n}{out\PYGZus{}folder}\PYG{o}{=}\PYG{l+s+s1}{\PYGZsq{}}\PYG{l+s+s1}{./data/mudcreek/optical/Figs\PYGZus{}analysis/2D}\PYG{l+s+s1}{\PYGZsq{}}\PYG{p}{,}
                                       \PYG{n}{color\PYGZus{}field}\PYG{o}{=}\PYG{l+s+s2}{\PYGZdq{}}\PYG{l+s+s2}{VEL}\PYG{l+s+s2}{\PYGZdq{}}\PYG{p}{,} \PYG{n}{Set\PYGZus{}fig\PYGZus{}MinMax}\PYG{o}{=}\PYG{k+kc}{False}\PYG{p}{,}
                                       \PYG{n}{MinMaxRange}\PYG{o}{=}\PYG{p}{[}\PYG{l+m+mi}{0}\PYG{p}{,} \PYG{l+m+mf}{1.5}\PYG{p}{]}\PYG{p}{,} \PYG{n}{opacity}\PYG{o}{=}\PYG{l+m+mf}{0.7}\PYG{p}{,}
                                       \PYG{n}{cmap}\PYG{o}{=}\PYG{l+s+s1}{\PYGZsq{}}\PYG{l+s+s1}{rainbow}\PYG{l+s+s1}{\PYGZsq{}}\PYG{p}{,} \PYG{n}{point\PYGZus{}size}\PYG{o}{=}\PYG{l+m+mi}{10}\PYG{p}{,}
                                       \PYG{n}{cbar\PYGZus{}label}\PYG{o}{=}\PYG{l+s+s1}{\PYGZsq{}}\PYG{l+s+s1}{2D\PYGZhy{}VEL (meter/year)}\PYG{l+s+s1}{\PYGZsq{}}\PYG{p}{,}
                                       \PYG{n}{batch\PYGZus{}plot}\PYG{o}{=}\PYG{k+kc}{False}\PYG{p}{)}
\end{sphinxVerbatim}
}

\end{sphinxuseclass}
\begin{sphinxuseclass}{nboutput}
\hrule height -\fboxrule\relax
\vspace{\nbsphinxcodecellspacing}

\makeatletter\setbox\nbsphinxpromptbox\box\voidb@x\makeatother

\begin{nbsphinxfancyoutput}

\begin{sphinxuseclass}{output_area}
\begin{sphinxuseclass}{}
\noindent\sphinxincludegraphics[width=690\sphinxpxdimen,height=697\sphinxpxdimen]{{notebooks_MudCreek_Landslide_optical_15_0}.png}

\end{sphinxuseclass}
\end{sphinxuseclass}
\end{nbsphinxfancyoutput}

\end{sphinxuseclass}
\begin{sphinxuseclass}{nboutput}
\hrule height -\fboxrule\relax
\vspace{\nbsphinxcodecellspacing}

\makeatletter\setbox\nbsphinxpromptbox\box\voidb@x\makeatother

\begin{nbsphinxfancyoutput}

\begin{sphinxuseclass}{output_area}
\begin{sphinxuseclass}{}
\noindent\sphinxincludegraphics[width=690\sphinxpxdimen,height=697\sphinxpxdimen]{{notebooks_MudCreek_Landslide_optical_15_1}.png}

\end{sphinxuseclass}
\end{sphinxuseclass}
\end{nbsphinxfancyoutput}

\end{sphinxuseclass}
\begin{sphinxuseclass}{nboutput}
\begin{sphinxuseclass}{nblast}
\hrule height -\fboxrule\relax
\vspace{\nbsphinxcodecellspacing}

\makeatletter\setbox\nbsphinxpromptbox\box\voidb@x\makeatother

\begin{nbsphinxfancyoutput}

\begin{sphinxuseclass}{output_area}
\begin{sphinxuseclass}{}
\noindent\sphinxincludegraphics[width=690\sphinxpxdimen,height=697\sphinxpxdimen]{{notebooks_MudCreek_Landslide_optical_15_2}.png}

\end{sphinxuseclass}
\end{sphinxuseclass}
\end{nbsphinxfancyoutput}

\end{sphinxuseclass}
\end{sphinxuseclass}

\subsection{Interactive Time Series Analysis in a Dash Web App}
\label{\detokenize{notebooks/MudCreek_Landslide_optical:Interactive-Time-Series-Analysis-in-a-Dash-Web-App}}
\sphinxAtStartPar
This plot is part of an interactive web application. Users have the ability to select groups of data points using either the box select or lasso tool, which are superimposed on a satellite image base map. The selected data will then be reflected in the time series chart. Additionally, users can customize the axis labels to suit their dataset; for example, if the dataset measures displacement or velocity in meters, the y\sphinxhyphen{}axis label can be adjusted accordingly from the default mm. Furthermore,
users can download their analysis and the generated figures directly from the web application.


\subsection{Screenshot of the dash appp interface}
\label{\detokenize{notebooks/MudCreek_Landslide_optical:Screenshot-of-the-dash-appp-interface}}
\begin{sphinxuseclass}{nbinput}
\begin{sphinxuseclass}{nblast}
{
\begin{sphinxVerbatim}[commandchars=\\\{\}]
\llap{\color{nbsphinxin}[ ]:\,\hspace{\fboxrule}\hspace{\fboxsep}}\PYG{c+ch}{\PYGZsh{}![Step 1](./data/mudcreek/optical/screenshot.png)}
\end{sphinxVerbatim}
}

\end{sphinxuseclass}
\end{sphinxuseclass}

\subsection{sample exported analysis from Time\sphinxhyphen{}series}
\label{\detokenize{notebooks/MudCreek_Landslide_optical:sample-exported-analysis-from-Time-series}}
\noindent\sphinxincludegraphics[width=1308\sphinxpxdimen,height=450\sphinxpxdimen]{{2dprofile2}.png}

\noindent\sphinxincludegraphics[width=1000\sphinxpxdimen,height=800\sphinxpxdimen]{{2dvelmap1}.png}

\begin{sphinxuseclass}{nbinput}
{
\begin{sphinxVerbatim}[commandchars=\\\{\}]
\llap{\color{nbsphinxin}[1]:\,\hspace{\fboxrule}\hspace{\fboxsep}}\PYG{k+kn}{import} \PYG{n+nn}{akhdefo\PYGZus{}functions}

\PYG{n}{akhdefo\PYGZus{}functions}\PYG{o}{.}\PYG{n}{akhdefo\PYGZus{}dashApp}\PYG{p}{(}
\PYG{n}{Path\PYGZus{}to\PYGZus{}Shapefile}\PYG{o}{=}\PYG{l+s+s1}{\PYGZsq{}}\PYG{l+s+s1}{./data/mudcreek/optical/geo/temp\PYGZus{}shapefile\PYGZus{}dir/updated\PYGZus{}shapefiles/20161002\PYGZus{}20170516\PYGZus{}20170516\PYGZus{}2DVEL.shp}\PYG{l+s+s1}{\PYGZsq{}}\PYG{p}{,}
                                  \PYG{n}{port}\PYG{o}{=}\PYG{l+m+mi}{8047}\PYG{p}{,} \PYG{n}{BaseMap}\PYG{o}{=}\PYG{k+kc}{True}\PYG{p}{,} \PYG{n}{basemap\PYGZus{}type}\PYG{o}{=}\PYG{l+s+s1}{\PYGZsq{}}\PYG{l+s+s1}{image}\PYG{l+s+s1}{\PYGZsq{}}\PYG{p}{)}
\end{sphinxVerbatim}
}

\end{sphinxuseclass}
\begin{sphinxuseclass}{nboutput}
\begin{sphinxuseclass}{nblast}
{

\kern-\sphinxverbatimsmallskipamount\kern-\baselineskip
\kern+\FrameHeightAdjust\kern-\fboxrule
\vspace{\nbsphinxcodecellspacing}

\sphinxsetup{VerbatimColor={named}{white}}
\begin{sphinxuseclass}{output_area}
\begin{sphinxuseclass}{}


\begin{sphinxVerbatim}[commandchars=\\\{\}]
<IPython.lib.display.IFrame at 0x261a3e4bd90>
\end{sphinxVerbatim}



\end{sphinxuseclass}
\end{sphinxuseclass}
}

\end{sphinxuseclass}
\end{sphinxuseclass}
\begin{sphinxuseclass}{nbinput}
\begin{sphinxuseclass}{nblast}
{
\begin{sphinxVerbatim}[commandchars=\\\{\}]
\llap{\color{nbsphinxin}[ ]:\,\hspace{\fboxrule}\hspace{\fboxsep}}
\PYG{n}{akhdefo\PYGZus{}dashApp}\PYG{p}{(}
\PYG{n}{Path\PYGZus{}to\PYGZus{}Shapefile}\PYG{o}{=}\PYG{l+s+s1}{\PYGZsq{}}\PYG{l+s+s1}{./data/mudcreek/optical/geo/temp\PYGZus{}shapefile\PYGZus{}dir/updated\PYGZus{}shapefiles/20180627\PYGZus{}20220926\PYGZus{}20220926\PYGZus{}N.shp}\PYG{l+s+s1}{\PYGZsq{}}\PYG{p}{,}
                                  \PYG{n}{port}\PYG{o}{=}\PYG{l+m+mi}{8056}\PYG{p}{,} \PYG{n}{BaseMap}\PYG{o}{=}\PYG{k+kc}{True}\PYG{p}{,} \PYG{n}{basemap\PYGZus{}type}\PYG{o}{=}\PYG{l+s+s1}{\PYGZsq{}}\PYG{l+s+s1}{image}\PYG{l+s+s1}{\PYGZsq{}}\PYG{p}{)}
\end{sphinxVerbatim}
}

\end{sphinxuseclass}
\end{sphinxuseclass}
\begin{sphinxuseclass}{nbinput}
\begin{sphinxuseclass}{nblast}
{
\begin{sphinxVerbatim}[commandchars=\\\{\}]
\llap{\color{nbsphinxin}[ ]:\,\hspace{\fboxrule}\hspace{\fboxsep}}
\PYG{n}{akhdefo\PYGZus{}dashApp}\PYG{p}{(}
\PYG{n}{Path\PYGZus{}to\PYGZus{}Shapefile}\PYG{o}{=}\PYG{l+s+s1}{\PYGZsq{}}\PYG{l+s+s1}{./data/mudcreek/optical/geo/temp\PYGZus{}shapefile\PYGZus{}dir/updated\PYGZus{}shapefiles/20180627\PYGZus{}20220926\PYGZus{}20220926\PYGZus{}E.shp}\PYG{l+s+s1}{\PYGZsq{}}\PYG{p}{,}
                                  \PYG{n}{port}\PYG{o}{=}\PYG{l+m+mi}{8055}\PYG{p}{,} \PYG{n}{BaseMap}\PYG{o}{=}\PYG{k+kc}{True}\PYG{p}{,} \PYG{n}{basemap\PYGZus{}type}\PYG{o}{=}\PYG{l+s+s1}{\PYGZsq{}}\PYG{l+s+s1}{image}\PYG{l+s+s1}{\PYGZsq{}}\PYG{p}{)}
\end{sphinxVerbatim}
}

\end{sphinxuseclass}
\end{sphinxuseclass}

\subsection{Kriging Interpolation using Akhdefo workflow}
\label{\detokenize{notebooks/MudCreek_Landslide_optical:Kriging-Interpolation-using-Akhdefo-workflow}}
\begin{sphinxuseclass}{nbinput}
\begin{sphinxuseclass}{nblast}
{
\begin{sphinxVerbatim}[commandchars=\\\{\}]
\llap{\color{nbsphinxin}[ ]:\,\hspace{\fboxrule}\hspace{\fboxsep}}\PYG{k+kn}{import} \PYG{n+nn}{akhdefo\PYGZus{}functions}
\PYG{k+kn}{from} \PYG{n+nn}{akhdefo\PYGZus{}functions} \PYG{k+kn}{import} \PYG{n}{Auto\PYGZus{}Variogram}

\PYG{n}{Auto\PYGZus{}Variogram}\PYG{p}{(}\PYG{n}{data}\PYG{o}{=}\PYG{l+s+s1}{\PYGZsq{}}\PYG{l+s+s1}{./data/mudcreek/optical/geo/temp\PYGZus{}shapefile\PYGZus{}dir/updated\PYGZus{}shapefiles/20161002\PYGZus{}20170516\PYGZus{}20170516\PYGZus{}N.shp}\PYG{l+s+s1}{\PYGZsq{}}\PYG{p}{,} \PYG{n}{column\PYGZus{}attribute}\PYG{o}{=}\PYG{l+s+s1}{\PYGZsq{}}\PYG{l+s+s1}{VEL}\PYG{l+s+s1}{\PYGZsq{}}
               \PYG{p}{,} \PYG{n}{pixel\PYGZus{}size}\PYG{o}{=}\PYG{l+m+mi}{10}\PYG{p}{,} \PYG{n}{num\PYGZus{}chunks}\PYG{o}{=}\PYG{l+m+mi}{10}\PYG{p}{,} \PYG{n}{out\PYGZus{}fileName}\PYG{o}{=}\PYG{l+s+s1}{\PYGZsq{}}\PYG{l+s+s1}{N}\PYG{l+s+s1}{\PYGZsq{}}\PYG{p}{,} \PYG{n}{geo\PYGZus{}folder}\PYG{o}{=}\PYG{l+s+s1}{\PYGZsq{}}\PYG{l+s+s1}{./data/mudcreek/optical/interpolated\PYGZus{}VEL\PYGZus{}products}\PYG{l+s+s1}{\PYGZsq{}}\PYG{p}{,} \PYG{n}{plot\PYGZus{}folder}\PYG{o}{=}\PYG{l+s+s1}{\PYGZsq{}}\PYG{l+s+s1}{./data/mudcreek/optical/interpolated\PYGZus{}VEL\PYGZus{}products}\PYG{l+s+s1}{\PYGZsq{}}
               \PYG{p}{,} \PYG{n}{smoothing\PYGZus{}kernel}\PYG{o}{=}\PYG{l+m+mi}{2}\PYG{p}{,} \PYG{n}{latlon}\PYG{o}{=}\PYG{k+kc}{False}\PYG{p}{,} \PYG{n}{aoi\PYGZus{}shapefile}\PYG{o}{=}\PYG{l+s+s1}{\PYGZsq{}}\PYG{l+s+s1}{./data/mudcreek/AOI.shp}\PYG{l+s+s1}{\PYGZsq{}}\PYG{p}{,} \PYG{n}{krig\PYGZus{}method}\PYG{o}{=}\PYG{l+s+s1}{\PYGZsq{}}\PYG{l+s+s1}{universal}\PYG{l+s+s1}{\PYGZsq{}}\PYG{p}{)}


\PYG{n}{Auto\PYGZus{}Variogram}\PYG{p}{(}\PYG{n}{data}\PYG{o}{=}\PYG{l+s+s1}{\PYGZsq{}}\PYG{l+s+s1}{./data/mudcreek/optical/geo/temp\PYGZus{}shapefile\PYGZus{}dir/updated\PYGZus{}shapefiles/20161002\PYGZus{}20170516\PYGZus{}20170516\PYGZus{}E.shp}\PYG{l+s+s1}{\PYGZsq{}}\PYG{p}{,} \PYG{n}{column\PYGZus{}attribute}\PYG{o}{=}\PYG{l+s+s1}{\PYGZsq{}}\PYG{l+s+s1}{VEL}\PYG{l+s+s1}{\PYGZsq{}}
               \PYG{p}{,} \PYG{n}{pixel\PYGZus{}size}\PYG{o}{=}\PYG{l+m+mi}{10}\PYG{p}{,} \PYG{n}{num\PYGZus{}chunks}\PYG{o}{=}\PYG{l+m+mi}{10}\PYG{p}{,} \PYG{n}{out\PYGZus{}fileName}\PYG{o}{=}\PYG{l+s+s1}{\PYGZsq{}}\PYG{l+s+s1}{E}\PYG{l+s+s1}{\PYGZsq{}}\PYG{p}{,} \PYG{n}{geo\PYGZus{}folder}\PYG{o}{=}\PYG{l+s+s1}{\PYGZsq{}}\PYG{l+s+s1}{./data/mudcreek/optical/interpolated\PYGZus{}VEL\PYGZus{}products}\PYG{l+s+s1}{\PYGZsq{}}\PYG{p}{,} \PYG{n}{plot\PYGZus{}folder}\PYG{o}{=}\PYG{l+s+s1}{\PYGZsq{}}\PYG{l+s+s1}{./data/mudcreek/optical/interpolated\PYGZus{}VEL\PYGZus{}products}\PYG{l+s+s1}{\PYGZsq{}}
               \PYG{p}{,} \PYG{n}{smoothing\PYGZus{}kernel}\PYG{o}{=}\PYG{l+m+mi}{2}\PYG{p}{,} \PYG{n}{latlon}\PYG{o}{=}\PYG{k+kc}{False}\PYG{p}{,} \PYG{n}{aoi\PYGZus{}shapefile}\PYG{o}{=}\PYG{l+s+s1}{\PYGZsq{}}\PYG{l+s+s1}{./data/mudcreek/AOI.shp}\PYG{l+s+s1}{\PYGZsq{}} \PYG{p}{,} \PYG{n}{krig\PYGZus{}method}\PYG{o}{=}\PYG{l+s+s1}{\PYGZsq{}}\PYG{l+s+s1}{universal}\PYG{l+s+s1}{\PYGZsq{}}\PYG{p}{)}



\PYG{n}{Auto\PYGZus{}Variogram}\PYG{p}{(}\PYG{n}{data}\PYG{o}{=}\PYG{l+s+s1}{\PYGZsq{}}\PYG{l+s+s1}{./data/mudcreek/optical/geo/temp\PYGZus{}shapefile\PYGZus{}dir/updated\PYGZus{}shapefiles/20161002\PYGZus{}20170516\PYGZus{}20170516\PYGZus{}2DVEL.shp}\PYG{l+s+s1}{\PYGZsq{}}\PYG{p}{,} \PYG{n}{column\PYGZus{}attribute}\PYG{o}{=}\PYG{l+s+s1}{\PYGZsq{}}\PYG{l+s+s1}{VEL}\PYG{l+s+s1}{\PYGZsq{}}
               \PYG{p}{,} \PYG{n}{pixel\PYGZus{}size}\PYG{o}{=}\PYG{l+m+mi}{10}\PYG{p}{,} \PYG{n}{num\PYGZus{}chunks}\PYG{o}{=}\PYG{l+m+mi}{10}\PYG{p}{,} \PYG{n}{out\PYGZus{}fileName}\PYG{o}{=}\PYG{l+s+s1}{\PYGZsq{}}\PYG{l+s+s1}{2DVEL}\PYG{l+s+s1}{\PYGZsq{}}\PYG{p}{,} \PYG{n}{geo\PYGZus{}folder}\PYG{o}{=}\PYG{l+s+s1}{\PYGZsq{}}\PYG{l+s+s1}{./data/mudcreek/optical/interpolated\PYGZus{}VEL\PYGZus{}products}\PYG{l+s+s1}{\PYGZsq{}}\PYG{p}{,} \PYG{n}{plot\PYGZus{}folder}\PYG{o}{=}\PYG{l+s+s1}{\PYGZsq{}}\PYG{l+s+s1}{./data/mudcreek/optical/interpolated\PYGZus{}VEL\PYGZus{}products}\PYG{l+s+s1}{\PYGZsq{}}
               \PYG{p}{,} \PYG{n}{smoothing\PYGZus{}kernel}\PYG{o}{=}\PYG{l+m+mi}{2}\PYG{p}{,} \PYG{n}{latlon}\PYG{o}{=}\PYG{k+kc}{False}\PYG{p}{,} \PYG{n}{aoi\PYGZus{}shapefile}\PYG{o}{=}\PYG{l+s+s1}{\PYGZsq{}}\PYG{l+s+s1}{./data/mudcreek/AOI.shp}\PYG{l+s+s1}{\PYGZsq{}} \PYG{p}{,} \PYG{n}{krig\PYGZus{}method}\PYG{o}{=}\PYG{l+s+s1}{\PYGZsq{}}\PYG{l+s+s1}{universal}\PYG{l+s+s1}{\PYGZsq{}}\PYG{p}{)}




\PYG{n}{Auto\PYGZus{}Variogram}\PYG{p}{(}\PYG{n}{data}\PYG{o}{=}\PYG{l+s+s1}{\PYGZsq{}}\PYG{l+s+s1}{./data/mudcreek/optical/geo/temp\PYGZus{}shapefile\PYGZus{}dir/updated\PYGZus{}shapefiles/20161002\PYGZus{}20170516\PYGZus{}20170516\PYGZus{}2DVEL.shp}\PYG{l+s+s1}{\PYGZsq{}}\PYG{p}{,} \PYG{n}{column\PYGZus{}attribute}\PYG{o}{=}\PYG{l+s+s1}{\PYGZsq{}}\PYG{l+s+s1}{aspect}\PYG{l+s+s1}{\PYGZsq{}}
               \PYG{p}{,} \PYG{n}{pixel\PYGZus{}size}\PYG{o}{=}\PYG{l+m+mi}{10}\PYG{p}{,} \PYG{n}{num\PYGZus{}chunks}\PYG{o}{=}\PYG{l+m+mi}{10}\PYG{p}{,} \PYG{n}{out\PYGZus{}fileName}\PYG{o}{=}\PYG{l+s+s1}{\PYGZsq{}}\PYG{l+s+s1}{aspect}\PYG{l+s+s1}{\PYGZsq{}}\PYG{p}{,} \PYG{n}{geo\PYGZus{}folder}\PYG{o}{=}\PYG{l+s+s1}{\PYGZsq{}}\PYG{l+s+s1}{./data/mudcreek/optical/interpolated\PYGZus{}VEL\PYGZus{}products}\PYG{l+s+s1}{\PYGZsq{}}\PYG{p}{,} \PYG{n}{plot\PYGZus{}folder}\PYG{o}{=}\PYG{l+s+s1}{\PYGZsq{}}\PYG{l+s+s1}{./data/mudcreek/optical/interpolated\PYGZus{}VEL\PYGZus{}products}\PYG{l+s+s1}{\PYGZsq{}}
               \PYG{p}{,} \PYG{n}{smoothing\PYGZus{}kernel}\PYG{o}{=}\PYG{l+m+mi}{2}\PYG{p}{,} \PYG{n}{latlon}\PYG{o}{=}\PYG{k+kc}{False}\PYG{p}{,} \PYG{n}{aoi\PYGZus{}shapefile}\PYG{o}{=}\PYG{l+s+s1}{\PYGZsq{}}\PYG{l+s+s1}{./data/mudcreek/AOI.shp}\PYG{l+s+s1}{\PYGZsq{}} \PYG{p}{,} \PYG{n}{krig\PYGZus{}method}\PYG{o}{=}\PYG{l+s+s1}{\PYGZsq{}}\PYG{l+s+s1}{universal}\PYG{l+s+s1}{\PYGZsq{}}\PYG{p}{)}

\end{sphinxVerbatim}
}

\end{sphinxuseclass}
\end{sphinxuseclass}

\subsection{Calculating Approximate Volume of bedrock from the displacement and digitial elevation model}
\label{\detokenize{notebooks/MudCreek_Landslide_optical:Calculating-Approximate-Volume-of-bedrock-from-the-displacement-and-digitial-elevation-model}}
\sphinxAtStartPar
In May 20th 2017, a catstrophic landslide occured in MudCreek, California USA. The Total removal of bedrock and materials measured from lidar and photogrammetry topography change detecion estimate about 5 million cubic yards equall to approximately 3 million cubic meters. In this example, we estimated landslide failure volume based on the time\sphinxhyphen{}series linear\sphinxhyphen{}annual velocity calulated approximately one year before landslide occur. \#\#\# Our method gave potential landslide volume two million and
nine\sphinxhyphen{}handred seventy six thousand cubic meter which is very close to the actual landslide volume measured from lidar dem change detection. \#\#\# \#\#\#\# Reference for actual volume collapse estimation \#\#\#\#
\begin{itemize}
\item {} 
\sphinxAtStartPar
All the datasets are avilable on github.

\item {} 
\sphinxAtStartPar
For the current example, displacemnts dervied from 3 meter optical imagery and the pre collapse dem used 3 meter dem (originally 10m NED but resampled to match the source displacement measurments). it worthwhile to mention that the dem resolution has influnece in the accuracy of the volume estimation.

\end{itemize}

\begin{sphinxuseclass}{nbinput}
{
\begin{sphinxVerbatim}[commandchars=\\\{\}]
\llap{\color{nbsphinxin}[15]:\,\hspace{\fboxrule}\hspace{\fboxsep}}\PYG{k+kn}{import} \PYG{n+nn}{akhdefo\PYGZus{}functions}
\PYG{n}{akhdefo\PYGZus{}functions}\PYG{o}{.}\PYG{n}{displacement\PYGZus{}to\PYGZus{}volume}\PYG{p}{(}\PYG{n}{dem\PYGZus{}path}\PYG{o}{=}\PYG{l+s+s1}{\PYGZsq{}}\PYG{l+s+s1}{./data/mudcreek/dem.tif}\PYG{l+s+s1}{\PYGZsq{}}\PYG{p}{,} \PYG{n}{aspect\PYGZus{}path}\PYG{o}{=}\PYG{l+s+s1}{\PYGZsq{}}\PYG{l+s+s1}{./data/mudcreek/optical/interpolated\PYGZus{}VEL\PYGZus{}products/aspect.tif}\PYG{l+s+s1}{\PYGZsq{}}\PYG{p}{,}
                                         \PYG{n}{displacement\PYGZus{}path}\PYG{o}{=}\PYG{l+s+s1}{\PYGZsq{}}\PYG{l+s+s1}{./data/mudcreek/optical/interpolated\PYGZus{}VEL\PYGZus{}products/2DVEL.tif}\PYG{l+s+s1}{\PYGZsq{}}\PYG{p}{,}
                                         \PYG{n}{slope\PYGZus{}output\PYGZus{}path}\PYG{o}{=}\PYG{l+s+s1}{\PYGZsq{}}\PYG{l+s+s1}{./data/mudcreek/optical/interpolated\PYGZus{}VEL\PYGZus{}products/slope.tif}\PYG{l+s+s1}{\PYGZsq{}}\PYG{p}{,}
                                         \PYG{n}{height\PYGZus{}output\PYGZus{}path}\PYG{o}{=}\PYG{l+s+s1}{\PYGZsq{}}\PYG{l+s+s1}{./data/mudcreek/optical/interpolated\PYGZus{}VEL\PYGZus{}products/height.tif}\PYG{l+s+s1}{\PYGZsq{}}\PYG{p}{,} \PYG{n}{volume\PYGZus{}output\PYGZus{}path}\PYG{o}{=}\PYG{l+s+s1}{\PYGZsq{}}\PYG{l+s+s1}{./data/mudcreek/optical/interpolated\PYGZus{}VEL\PYGZus{}products/volume.tif}\PYG{l+s+s1}{\PYGZsq{}}\PYG{p}{)}
\end{sphinxVerbatim}
}

\end{sphinxuseclass}
\begin{sphinxuseclass}{nboutput}
\begin{sphinxuseclass}{nblast}
{

\kern-\sphinxverbatimsmallskipamount\kern-\baselineskip
\kern+\FrameHeightAdjust\kern-\fboxrule
\vspace{\nbsphinxcodecellspacing}

\sphinxsetup{VerbatimColor={named}{white}}
\begin{sphinxuseclass}{output_area}
\begin{sphinxuseclass}{}


\begin{sphinxVerbatim}[commandchars=\\\{\}]
Total Volume: -2976346.0 cubic meter
\end{sphinxVerbatim}



\end{sphinxuseclass}
\end{sphinxuseclass}
}

\end{sphinxuseclass}
\end{sphinxuseclass}

\subsection{Caluclating Plune and Trend and 3D Velocity of deformation}
\label{\detokenize{notebooks/MudCreek_Landslide_optical:Caluclating-Plune-and-Trend-and-3D-Velocity-of-deformation}}
\begin{sphinxuseclass}{nbinput}
{
\begin{sphinxVerbatim}[commandchars=\\\{\}]
\llap{\color{nbsphinxin}[2]:\,\hspace{\fboxrule}\hspace{\fboxsep}}\PYG{k+kn}{import} \PYG{n+nn}{akhdefo\PYGZus{}functions}

\PYG{n}{akhdefo\PYGZus{}functions}\PYG{o}{.}\PYG{n}{Akhdefo\PYGZus{}inversion}\PYG{p}{(}\PYG{n}{horizontal\PYGZus{}InSAR}\PYG{o}{=}\PYG{l+s+s1}{\PYGZsq{}}\PYG{l+s+s1}{./data/mudcreek/optical/interpolated\PYGZus{}VEL\PYGZus{}products/E.tif}\PYG{l+s+s1}{\PYGZsq{}}\PYG{p}{,}
                                    \PYG{n}{Vertical\PYGZus{}InSAR}\PYG{o}{=}\PYG{l+s+s1}{\PYGZsq{}}\PYG{l+s+s1}{./data/mudcreek/optical/interpolated\PYGZus{}VEL\PYGZus{}products/height.tif}\PYG{l+s+s1}{\PYGZsq{}}\PYG{p}{,} \PYG{n}{EW\PYGZus{}Akhdefo}\PYG{o}{=}\PYG{l+s+s1}{\PYGZsq{}}\PYG{l+s+s1}{./data/mudcreek/optical/interpolated\PYGZus{}VEL\PYGZus{}products/E.tif}\PYG{l+s+s1}{\PYGZsq{}}\PYG{p}{,}
                                    \PYG{n}{NS\PYGZus{}Akhdefo}\PYG{o}{=}\PYG{l+s+s1}{\PYGZsq{}}\PYG{l+s+s1}{./data/mudcreek/optical/interpolated\PYGZus{}VEL\PYGZus{}products/N.tif}\PYG{l+s+s1}{\PYGZsq{}}\PYG{p}{,} \PYG{n}{dem\PYGZus{}path}\PYG{o}{=}\PYG{l+s+s1}{\PYGZsq{}}\PYG{l+s+s1}{./data/mudcreek/dem.tif}\PYG{l+s+s1}{\PYGZsq{}}\PYG{p}{,} \PYG{n}{output\PYGZus{}folder}\PYG{o}{=}\PYG{l+s+s1}{\PYGZsq{}}\PYG{l+s+s1}{./data/mudcreek/optical/interpolated\PYGZus{}VEL\PYGZus{}products/inversion}\PYG{l+s+s1}{\PYGZsq{}}
                                    \PYG{p}{)}
\end{sphinxVerbatim}
}

\end{sphinxuseclass}
\begin{sphinxuseclass}{nboutput}
{

\kern-\sphinxverbatimsmallskipamount\kern-\baselineskip
\kern+\FrameHeightAdjust\kern-\fboxrule
\vspace{\nbsphinxcodecellspacing}

\sphinxsetup{VerbatimColor={named}{white}}
\begin{sphinxuseclass}{output_area}
\begin{sphinxuseclass}{}


\begin{sphinxVerbatim}[commandchars=\\\{\}]
Trend in degree raw data:  -89.98021 0.66595054
\end{sphinxVerbatim}



\end{sphinxuseclass}
\end{sphinxuseclass}
}

\end{sphinxuseclass}
\begin{sphinxuseclass}{nboutput}
{

\kern-\sphinxverbatimsmallskipamount\kern-\baselineskip
\kern+\FrameHeightAdjust\kern-\fboxrule
\vspace{\nbsphinxcodecellspacing}

\sphinxsetup{VerbatimColor={named}{nbsphinx-stderr}}
\begin{sphinxuseclass}{output_area}
\begin{sphinxuseclass}{stderr}


\begin{sphinxVerbatim}[commandchars=\\\{\}]
c:\textbackslash{}Users\textbackslash{}mahmud\textbackslash{}anaconda3\textbackslash{}envs\textbackslash{}akhdefov2\textbackslash{}lib\textbackslash{}site-packages\textbackslash{}earthpy\textbackslash{}spatial.py:620: RuntimeWarning:

invalid value encountered in sqrt

\end{sphinxVerbatim}



\end{sphinxuseclass}
\end{sphinxuseclass}
}

\end{sphinxuseclass}
\begin{sphinxuseclass}{nboutput}
\hrule height -\fboxrule\relax
\vspace{\nbsphinxcodecellspacing}

\makeatletter\setbox\nbsphinxpromptbox\box\voidb@x\makeatother

\begin{nbsphinxfancyoutput}

\begin{sphinxuseclass}{output_area}
\begin{sphinxuseclass}{}
\noindent\sphinxincludegraphics[width=953\sphinxpxdimen,height=668\sphinxpxdimen]{{notebooks_MudCreek_Landslide_optical_30_2}.png}

\end{sphinxuseclass}
\end{sphinxuseclass}
\end{nbsphinxfancyoutput}

\end{sphinxuseclass}
\begin{sphinxuseclass}{nboutput}
{

\kern-\sphinxverbatimsmallskipamount\kern-\baselineskip
\kern+\FrameHeightAdjust\kern-\fboxrule
\vspace{\nbsphinxcodecellspacing}

\sphinxsetup{VerbatimColor={named}{nbsphinx-stderr}}
\begin{sphinxuseclass}{output_area}
\begin{sphinxuseclass}{stderr}


\begin{sphinxVerbatim}[commandchars=\\\{\}]
c:\textbackslash{}Users\textbackslash{}mahmud\textbackslash{}anaconda3\textbackslash{}envs\textbackslash{}akhdefov2\textbackslash{}lib\textbackslash{}site-packages\textbackslash{}earthpy\textbackslash{}spatial.py:620: RuntimeWarning:

invalid value encountered in sqrt

\end{sphinxVerbatim}



\end{sphinxuseclass}
\end{sphinxuseclass}
}

\end{sphinxuseclass}
\begin{sphinxuseclass}{nboutput}
\hrule height -\fboxrule\relax
\vspace{\nbsphinxcodecellspacing}

\makeatletter\setbox\nbsphinxpromptbox\box\voidb@x\makeatother

\begin{nbsphinxfancyoutput}

\begin{sphinxuseclass}{output_area}
\begin{sphinxuseclass}{}
\noindent\sphinxincludegraphics[width=961\sphinxpxdimen,height=668\sphinxpxdimen]{{notebooks_MudCreek_Landslide_optical_30_4}.png}

\end{sphinxuseclass}
\end{sphinxuseclass}
\end{nbsphinxfancyoutput}

\end{sphinxuseclass}
\begin{sphinxuseclass}{nboutput}
{

\kern-\sphinxverbatimsmallskipamount\kern-\baselineskip
\kern+\FrameHeightAdjust\kern-\fboxrule
\vspace{\nbsphinxcodecellspacing}

\sphinxsetup{VerbatimColor={named}{nbsphinx-stderr}}
\begin{sphinxuseclass}{output_area}
\begin{sphinxuseclass}{stderr}


\begin{sphinxVerbatim}[commandchars=\\\{\}]
c:\textbackslash{}Users\textbackslash{}mahmud\textbackslash{}anaconda3\textbackslash{}envs\textbackslash{}akhdefov2\textbackslash{}lib\textbackslash{}site-packages\textbackslash{}earthpy\textbackslash{}spatial.py:620: RuntimeWarning:

invalid value encountered in sqrt

\end{sphinxVerbatim}



\end{sphinxuseclass}
\end{sphinxuseclass}
}

\end{sphinxuseclass}
\begin{sphinxuseclass}{nboutput}
\begin{sphinxuseclass}{nblast}
\hrule height -\fboxrule\relax
\vspace{\nbsphinxcodecellspacing}

\makeatletter\setbox\nbsphinxpromptbox\box\voidb@x\makeatother

\begin{nbsphinxfancyoutput}

\begin{sphinxuseclass}{output_area}
\begin{sphinxuseclass}{}
\noindent\sphinxincludegraphics[width=958\sphinxpxdimen,height=668\sphinxpxdimen]{{notebooks_MudCreek_Landslide_optical_30_6}.png}

\end{sphinxuseclass}
\end{sphinxuseclass}
\end{nbsphinxfancyoutput}

\end{sphinxuseclass}
\end{sphinxuseclass}

\subsection{Visualizing Products}
\label{\detokenize{notebooks/MudCreek_Landslide_optical:Visualizing-Products}}\begin{itemize}
\item {} \subsubsection*{Post failure Mudcreek orthophoto}

\sphinxAtStartPar
\sphinxincludegraphics[width=1532\sphinxpxdimen,height=1057\sphinxpxdimen]{{MudCreek27May2017}.jpg}

\end{itemize}

\noindent\sphinxincludegraphics[width=953\sphinxpxdimen,height=752\sphinxpxdimen]{{3DVEL_WithVectorMotion}.png}

\begin{sphinxuseclass}{nbinput}
{
\begin{sphinxVerbatim}[commandchars=\\\{\}]
\llap{\color{nbsphinxin}[2]:\,\hspace{\fboxrule}\hspace{\fboxsep}}\PYG{k+kn}{import} \PYG{n+nn}{akhdefo\PYGZus{}functions}
\PYG{k+kn}{import} \PYG{n+nn}{cmocean}\PYG{n+nn}{.}\PYG{n+nn}{cm} \PYG{k}{as} \PYG{n+nn}{cmo}

\PYG{n}{akhdefo\PYGZus{}functions}\PYG{o}{.}\PYG{n}{akhdefo\PYGZus{}viewer}\PYG{p}{(}\PYG{n}{path\PYGZus{}to\PYGZus{}dem\PYGZus{}file}\PYG{o}{=}\PYG{l+s+s1}{\PYGZsq{}}\PYG{l+s+s1}{./data/mudcreek/basemap.tif}\PYG{l+s+s1}{\PYGZsq{}}\PYG{p}{,} \PYG{n}{raster\PYGZus{}file}\PYG{o}{=}\PYG{l+s+s1}{\PYGZsq{}}\PYG{l+s+s1}{./data/mudcreek/optical/interpolated\PYGZus{}VEL\PYGZus{}products/2DVEL.tif}\PYG{l+s+s1}{\PYGZsq{}}\PYG{p}{,} \PYG{n}{output\PYGZus{}folder}\PYG{o}{=}\PYG{l+s+s1}{\PYGZsq{}}\PYG{l+s+s1}{./data/mudcreek/optical/Figs\PYGZus{}analysis/}\PYG{l+s+s1}{\PYGZsq{}}\PYG{p}{,} \PYG{n}{title}\PYG{o}{=}\PYG{l+s+s1}{\PYGZsq{}}\PYG{l+s+s1}{2D Velocity with Vector Motion}\PYG{l+s+s1}{\PYGZsq{}}\PYG{p}{,}
                   \PYG{n}{pixel\PYGZus{}resolution\PYGZus{}meters}\PYG{o}{=}\PYG{k+kc}{None}\PYG{p}{,} \PYG{n}{output\PYGZus{}file\PYGZus{}name}\PYG{o}{=}\PYG{l+s+s2}{\PYGZdq{}}\PYG{l+s+s2}{2DVEL\PYGZus{}WithVectorMotion.png}\PYG{l+s+s2}{\PYGZdq{}}\PYG{p}{,}
                   \PYG{n}{alpha}\PYG{o}{=}\PYG{l+m+mf}{0.4}\PYG{p}{,} \PYG{n}{unit\PYGZus{}conversion}\PYG{o}{=}\PYG{k+kc}{None}\PYG{p}{,} \PYG{n}{no\PYGZus{}data\PYGZus{}mask}\PYG{o}{=}\PYG{k+kc}{True}\PYG{p}{,}
                   \PYG{n}{colormap}\PYG{o}{=}\PYG{l+s+s1}{\PYGZsq{}}\PYG{l+s+s1}{hot\PYGZus{}r}\PYG{l+s+s1}{\PYGZsq{}}\PYG{p}{,} \PYG{n}{min\PYGZus{}value}\PYG{o}{=}\PYG{k+kc}{None}\PYG{p}{,} \PYG{n}{max\PYGZus{}value}\PYG{o}{=}\PYG{k+kc}{None}\PYG{p}{,}
                   \PYG{n}{normalize}\PYG{o}{=}\PYG{k+kc}{True}\PYG{p}{,} \PYG{n}{colorbar\PYGZus{}label}\PYG{o}{=}\PYG{l+s+s1}{\PYGZsq{}}\PYG{l+s+s1}{meter/year}\PYG{l+s+s1}{\PYGZsq{}}\PYG{p}{,} \PYG{n}{show\PYGZus{}figure}\PYG{o}{=}\PYG{k+kc}{True}\PYG{p}{,} \PYG{n}{aspect\PYGZus{}raster}\PYG{o}{=}\PYG{l+s+s1}{\PYGZsq{}}\PYG{l+s+s1}{./data/mudcreek/optical/interpolated\PYGZus{}VEL\PYGZus{}products/aspect.tif}\PYG{l+s+s1}{\PYGZsq{}}\PYG{p}{,} \PYG{n}{cmap\PYGZus{}aspect}\PYG{o}{=}\PYG{l+s+s1}{\PYGZsq{}}\PYG{l+s+s1}{hsv}\PYG{l+s+s1}{\PYGZsq{}} \PYG{p}{,} \PYG{n}{step}\PYG{o}{=}\PYG{l+m+mi}{5}\PYG{p}{)}
\end{sphinxVerbatim}
}

\end{sphinxuseclass}
\begin{sphinxuseclass}{nboutput}
\begin{sphinxuseclass}{nblast}
\hrule height -\fboxrule\relax
\vspace{\nbsphinxcodecellspacing}

\makeatletter\setbox\nbsphinxpromptbox\box\voidb@x\makeatother

\begin{nbsphinxfancyoutput}

\begin{sphinxuseclass}{output_area}
\begin{sphinxuseclass}{}
\noindent\sphinxincludegraphics[width=953\sphinxpxdimen,height=752\sphinxpxdimen]{{notebooks_MudCreek_Landslide_optical_33_0}.png}

\end{sphinxuseclass}
\end{sphinxuseclass}
\end{nbsphinxfancyoutput}

\end{sphinxuseclass}
\end{sphinxuseclass}
\begin{sphinxuseclass}{nbinput}
{
\begin{sphinxVerbatim}[commandchars=\\\{\}]
\llap{\color{nbsphinxin}[6]:\,\hspace{\fboxrule}\hspace{\fboxsep}}\PYG{k+kn}{import} \PYG{n+nn}{akhdefo\PYGZus{}functions}
\PYG{k+kn}{import} \PYG{n+nn}{cmocean}\PYG{n+nn}{.}\PYG{n+nn}{cm} \PYG{k}{as} \PYG{n+nn}{cmo}

\PYG{n}{akhdefo\PYGZus{}functions}\PYG{o}{.}\PYG{n}{akhdefo\PYGZus{}viewer}\PYG{p}{(}\PYG{n}{path\PYGZus{}to\PYGZus{}dem\PYGZus{}file}\PYG{o}{=}\PYG{l+s+s1}{\PYGZsq{}}\PYG{l+s+s1}{./data/mudcreek/basemap.tif}\PYG{l+s+s1}{\PYGZsq{}}\PYG{p}{,} \PYG{n}{raster\PYGZus{}file}\PYG{o}{=}\PYG{l+s+s1}{\PYGZsq{}}\PYG{l+s+s1}{./data/mudcreek/optical/interpolated\PYGZus{}VEL\PYGZus{}products/inversion/D3D.tif}\PYG{l+s+s1}{\PYGZsq{}}\PYG{p}{,} \PYG{n}{output\PYGZus{}folder}\PYG{o}{=}\PYG{l+s+s1}{\PYGZsq{}}\PYG{l+s+s1}{./data/mudcreek/optical/Figs\PYGZus{}analysis/}\PYG{l+s+s1}{\PYGZsq{}}\PYG{p}{,} \PYG{n}{title}\PYG{o}{=}\PYG{l+s+s1}{\PYGZsq{}}\PYG{l+s+s1}{3D Velocity with Vector Motion}\PYG{l+s+s1}{\PYGZsq{}}\PYG{p}{,}
                   \PYG{n}{pixel\PYGZus{}resolution\PYGZus{}meters}\PYG{o}{=}\PYG{k+kc}{None}\PYG{p}{,} \PYG{n}{output\PYGZus{}file\PYGZus{}name}\PYG{o}{=}\PYG{l+s+s2}{\PYGZdq{}}\PYG{l+s+s2}{3DVEL\PYGZus{}WithVectorMotion.png}\PYG{l+s+s2}{\PYGZdq{}}\PYG{p}{,}
                   \PYG{n}{alpha}\PYG{o}{=}\PYG{l+m+mf}{0.4}\PYG{p}{,} \PYG{n}{unit\PYGZus{}conversion}\PYG{o}{=}\PYG{k+kc}{None}\PYG{p}{,} \PYG{n}{no\PYGZus{}data\PYGZus{}mask}\PYG{o}{=}\PYG{k+kc}{True}\PYG{p}{,}
                   \PYG{n}{colormap}\PYG{o}{=}\PYG{l+s+s1}{\PYGZsq{}}\PYG{l+s+s1}{hot\PYGZus{}r}\PYG{l+s+s1}{\PYGZsq{}}\PYG{p}{,} \PYG{n}{min\PYGZus{}value}\PYG{o}{=}\PYG{k+kc}{None}\PYG{p}{,} \PYG{n}{max\PYGZus{}value}\PYG{o}{=}\PYG{k+kc}{None}\PYG{p}{,}
                   \PYG{n}{normalize}\PYG{o}{=}\PYG{k+kc}{True}\PYG{p}{,} \PYG{n}{colorbar\PYGZus{}label}\PYG{o}{=}\PYG{l+s+s1}{\PYGZsq{}}\PYG{l+s+s1}{meter/year}\PYG{l+s+s1}{\PYGZsq{}}\PYG{p}{,} \PYG{n}{show\PYGZus{}figure}\PYG{o}{=}\PYG{k+kc}{True}\PYG{p}{,} \PYG{n}{aspect\PYGZus{}raster}\PYG{o}{=}\PYG{l+s+s1}{\PYGZsq{}}\PYG{l+s+s1}{./data/mudcreek/optical/interpolated\PYGZus{}VEL\PYGZus{}products/aspect.tif}\PYG{l+s+s1}{\PYGZsq{}}\PYG{p}{,} \PYG{n}{cmap\PYGZus{}aspect}\PYG{o}{=}\PYG{l+s+s1}{\PYGZsq{}}\PYG{l+s+s1}{hsv}\PYG{l+s+s1}{\PYGZsq{}} \PYG{p}{,} \PYG{n}{step}\PYG{o}{=}\PYG{l+m+mi}{5}\PYG{p}{)}
\end{sphinxVerbatim}
}

\end{sphinxuseclass}
\begin{sphinxuseclass}{nboutput}
\begin{sphinxuseclass}{nblast}
\hrule height -\fboxrule\relax
\vspace{\nbsphinxcodecellspacing}

\makeatletter\setbox\nbsphinxpromptbox\box\voidb@x\makeatother

\begin{nbsphinxfancyoutput}

\begin{sphinxuseclass}{output_area}
\begin{sphinxuseclass}{}
\noindent\sphinxincludegraphics[width=953\sphinxpxdimen,height=752\sphinxpxdimen]{{notebooks_MudCreek_Landslide_optical_34_0}.png}

\end{sphinxuseclass}
\end{sphinxuseclass}
\end{nbsphinxfancyoutput}

\end{sphinxuseclass}
\end{sphinxuseclass}
\begin{sphinxuseclass}{nbinput}
{
\begin{sphinxVerbatim}[commandchars=\\\{\}]
\llap{\color{nbsphinxin}[3]:\,\hspace{\fboxrule}\hspace{\fboxsep}}\PYG{k+kn}{import} \PYG{n+nn}{akhdefo\PYGZus{}functions}
\PYG{k+kn}{import} \PYG{n+nn}{cmocean}\PYG{n+nn}{.}\PYG{n+nn}{cm} \PYG{k}{as} \PYG{n+nn}{cmo}

\PYG{n}{akhdefo\PYGZus{}functions}\PYG{o}{.}\PYG{n}{akhdefo\PYGZus{}viewer}\PYG{p}{(}\PYG{n}{path\PYGZus{}to\PYGZus{}dem\PYGZus{}file}\PYG{o}{=}\PYG{l+s+s1}{\PYGZsq{}}\PYG{l+s+s1}{./data/mudcreek/basemap.tif}\PYG{l+s+s1}{\PYGZsq{}}\PYG{p}{,} \PYG{n}{raster\PYGZus{}file}\PYG{o}{=}\PYG{l+s+s1}{\PYGZsq{}}\PYG{l+s+s1}{./data/mudcreek/optical/interpolated\PYGZus{}VEL\PYGZus{}products/height.tif}\PYG{l+s+s1}{\PYGZsq{}}\PYG{p}{,}
                                 \PYG{n}{output\PYGZus{}folder}\PYG{o}{=}\PYG{l+s+s1}{\PYGZsq{}}\PYG{l+s+s1}{./data/mudcreek/optical/Figs\PYGZus{}analysis/}\PYG{l+s+s1}{\PYGZsq{}}\PYG{p}{,} \PYG{n}{title}\PYG{o}{=}\PYG{l+s+s1}{\PYGZsq{}}\PYG{l+s+s1}{height\PYGZus{}change with Vector Motion}\PYG{l+s+s1}{\PYGZsq{}}\PYG{p}{,}
                   \PYG{n}{pixel\PYGZus{}resolution\PYGZus{}meters}\PYG{o}{=}\PYG{k+kc}{None}\PYG{p}{,} \PYG{n}{output\PYGZus{}file\PYGZus{}name}\PYG{o}{=}\PYG{l+s+s2}{\PYGZdq{}}\PYG{l+s+s2}{height\PYGZus{}change\PYGZus{}WithVectorMotion.png}\PYG{l+s+s2}{\PYGZdq{}}\PYG{p}{,}
                   \PYG{n}{alpha}\PYG{o}{=}\PYG{l+m+mf}{0.4}\PYG{p}{,} \PYG{n}{unit\PYGZus{}conversion}\PYG{o}{=}\PYG{k+kc}{None}\PYG{p}{,} \PYG{n}{no\PYGZus{}data\PYGZus{}mask}\PYG{o}{=}\PYG{k+kc}{True}\PYG{p}{,}
                   \PYG{n}{colormap}\PYG{o}{=}\PYG{l+s+s1}{\PYGZsq{}}\PYG{l+s+s1}{bwr}\PYG{l+s+s1}{\PYGZsq{}}\PYG{p}{,} \PYG{n}{min\PYGZus{}value}\PYG{o}{=}\PYG{k+kc}{None}\PYG{p}{,} \PYG{n}{max\PYGZus{}value}\PYG{o}{=}\PYG{k+kc}{None}\PYG{p}{,}
                   \PYG{n}{normalize}\PYG{o}{=}\PYG{k+kc}{True}\PYG{p}{,} \PYG{n}{colorbar\PYGZus{}label}\PYG{o}{=}\PYG{l+s+s1}{\PYGZsq{}}\PYG{l+s+s1}{meter}\PYG{l+s+s1}{\PYGZsq{}}\PYG{p}{,} \PYG{n}{show\PYGZus{}figure}\PYG{o}{=}\PYG{k+kc}{True}\PYG{p}{,} \PYG{n}{aspect\PYGZus{}raster}\PYG{o}{=}\PYG{l+s+s1}{\PYGZsq{}}\PYG{l+s+s1}{./data/mudcreek/optical/interpolated\PYGZus{}VEL\PYGZus{}products/aspect.tif}\PYG{l+s+s1}{\PYGZsq{}}\PYG{p}{,} \PYG{n}{cmap\PYGZus{}aspect}\PYG{o}{=}\PYG{l+s+s1}{\PYGZsq{}}\PYG{l+s+s1}{hsv}\PYG{l+s+s1}{\PYGZsq{}} \PYG{p}{,} \PYG{n}{step}\PYG{o}{=}\PYG{l+m+mi}{5}\PYG{p}{)}
\end{sphinxVerbatim}
}

\end{sphinxuseclass}
\begin{sphinxuseclass}{nboutput}
\begin{sphinxuseclass}{nblast}
\hrule height -\fboxrule\relax
\vspace{\nbsphinxcodecellspacing}

\makeatletter\setbox\nbsphinxpromptbox\box\voidb@x\makeatother

\begin{nbsphinxfancyoutput}

\begin{sphinxuseclass}{output_area}
\begin{sphinxuseclass}{}
\noindent\sphinxincludegraphics[width=965\sphinxpxdimen,height=758\sphinxpxdimen]{{notebooks_MudCreek_Landslide_optical_35_0}.png}

\end{sphinxuseclass}
\end{sphinxuseclass}
\end{nbsphinxfancyoutput}

\end{sphinxuseclass}
\end{sphinxuseclass}
\begin{sphinxuseclass}{nbinput}
{
\begin{sphinxVerbatim}[commandchars=\\\{\}]
\llap{\color{nbsphinxin}[4]:\,\hspace{\fboxrule}\hspace{\fboxsep}}\PYG{k+kn}{import} \PYG{n+nn}{akhdefo\PYGZus{}functions}
\PYG{k+kn}{import} \PYG{n+nn}{cmocean}\PYG{n+nn}{.}\PYG{n+nn}{cm} \PYG{k}{as} \PYG{n+nn}{cmo}

\PYG{n}{akhdefo\PYGZus{}functions}\PYG{o}{.}\PYG{n}{akhdefo\PYGZus{}viewer}\PYG{p}{(}\PYG{n}{path\PYGZus{}to\PYGZus{}dem\PYGZus{}file}\PYG{o}{=}\PYG{l+s+s1}{\PYGZsq{}}\PYG{l+s+s1}{./data/mudcreek/basemap.tif}\PYG{l+s+s1}{\PYGZsq{}}\PYG{p}{,} \PYG{n}{raster\PYGZus{}file}\PYG{o}{=}\PYG{l+s+s1}{\PYGZsq{}}\PYG{l+s+s1}{./data/mudcreek/optical/interpolated\PYGZus{}VEL\PYGZus{}products/volume.tif}\PYG{l+s+s1}{\PYGZsq{}}\PYG{p}{,}
                                 \PYG{n}{output\PYGZus{}folder}\PYG{o}{=}\PYG{l+s+s1}{\PYGZsq{}}\PYG{l+s+s1}{./data/mudcreek/optical/Figs\PYGZus{}analysis/}\PYG{l+s+s1}{\PYGZsq{}}\PYG{p}{,} \PYG{n}{title}\PYG{o}{=}\PYG{l+s+s1}{\PYGZsq{}}\PYG{l+s+s1}{Volume\PYGZus{}change with Vector Motion}\PYG{l+s+s1}{\PYGZsq{}}\PYG{p}{,}
                   \PYG{n}{pixel\PYGZus{}resolution\PYGZus{}meters}\PYG{o}{=}\PYG{k+kc}{None}\PYG{p}{,} \PYG{n}{output\PYGZus{}file\PYGZus{}name}\PYG{o}{=}\PYG{l+s+s2}{\PYGZdq{}}\PYG{l+s+s2}{Volume\PYGZus{}change\PYGZus{}WithVectorMotion.png}\PYG{l+s+s2}{\PYGZdq{}}\PYG{p}{,}
                   \PYG{n}{alpha}\PYG{o}{=}\PYG{l+m+mf}{0.4}\PYG{p}{,} \PYG{n}{unit\PYGZus{}conversion}\PYG{o}{=}\PYG{k+kc}{None}\PYG{p}{,} \PYG{n}{no\PYGZus{}data\PYGZus{}mask}\PYG{o}{=}\PYG{k+kc}{True}\PYG{p}{,}
                   \PYG{n}{colormap}\PYG{o}{=}\PYG{l+s+s1}{\PYGZsq{}}\PYG{l+s+s1}{bwr}\PYG{l+s+s1}{\PYGZsq{}}\PYG{p}{,} \PYG{n}{min\PYGZus{}value}\PYG{o}{=}\PYG{k+kc}{None}\PYG{p}{,} \PYG{n}{max\PYGZus{}value}\PYG{o}{=}\PYG{k+kc}{None}\PYG{p}{,}
                   \PYG{n}{normalize}\PYG{o}{=}\PYG{k+kc}{True}\PYG{p}{,} \PYG{n}{colorbar\PYGZus{}label}\PYG{o}{=}\PYG{l+s+s1}{\PYGZsq{}}\PYG{l+s+s1}{cubic meter}\PYG{l+s+s1}{\PYGZsq{}}\PYG{p}{,} \PYG{n}{show\PYGZus{}figure}\PYG{o}{=}\PYG{k+kc}{True}\PYG{p}{,} \PYG{n}{aspect\PYGZus{}raster}\PYG{o}{=}\PYG{l+s+s1}{\PYGZsq{}}\PYG{l+s+s1}{./data/mudcreek/optical/interpolated\PYGZus{}VEL\PYGZus{}products/aspect.tif}\PYG{l+s+s1}{\PYGZsq{}}\PYG{p}{,} \PYG{n}{cmap\PYGZus{}aspect}\PYG{o}{=}\PYG{l+s+s1}{\PYGZsq{}}\PYG{l+s+s1}{hsv}\PYG{l+s+s1}{\PYGZsq{}} \PYG{p}{,} \PYG{n}{step}\PYG{o}{=}\PYG{l+m+mi}{10}\PYG{p}{)}
\end{sphinxVerbatim}
}

\end{sphinxuseclass}
\begin{sphinxuseclass}{nboutput}
\begin{sphinxuseclass}{nblast}
\hrule height -\fboxrule\relax
\vspace{\nbsphinxcodecellspacing}

\makeatletter\setbox\nbsphinxpromptbox\box\voidb@x\makeatother

\begin{nbsphinxfancyoutput}

\begin{sphinxuseclass}{output_area}
\begin{sphinxuseclass}{}
\noindent\sphinxincludegraphics[width=970\sphinxpxdimen,height=758\sphinxpxdimen]{{notebooks_MudCreek_Landslide_optical_36_0}.png}

\end{sphinxuseclass}
\end{sphinxuseclass}
\end{nbsphinxfancyoutput}

\end{sphinxuseclass}
\end{sphinxuseclass}
\sphinxstepscope


\section{Mount Meager Radar Imagery Example}
\label{\detokenize{notebooks/Meager_radar:Mount-Meager-Radar-Imagery-Example}}\label{\detokenize{notebooks/Meager_radar::doc}}
\begin{sphinxuseclass}{nbinput}
\begin{sphinxuseclass}{nblast}
{
\begin{sphinxVerbatim}[commandchars=\\\{\}]
\llap{\color{nbsphinxin}[ ]:\,\hspace{\fboxrule}\hspace{\fboxsep}}
\end{sphinxVerbatim}
}

\end{sphinxuseclass}
\end{sphinxuseclass}
\begin{sphinxuseclass}{nbinput}
\begin{sphinxuseclass}{nblast}
{
\begin{sphinxVerbatim}[commandchars=\\\{\}]
\llap{\color{nbsphinxin}[ ]:\,\hspace{\fboxrule}\hspace{\fboxsep}}
\end{sphinxVerbatim}
}

\end{sphinxuseclass}
\end{sphinxuseclass}
\begin{sphinxuseclass}{nbinput}
\begin{sphinxuseclass}{nblast}
{
\begin{sphinxVerbatim}[commandchars=\\\{\}]
\llap{\color{nbsphinxin}[ ]:\,\hspace{\fboxrule}\hspace{\fboxsep}}
\end{sphinxVerbatim}
}

\end{sphinxuseclass}
\end{sphinxuseclass}
\begin{sphinxuseclass}{nbinput}
\begin{sphinxuseclass}{nblast}
{
\begin{sphinxVerbatim}[commandchars=\\\{\}]
\llap{\color{nbsphinxin}[ ]:\,\hspace{\fboxrule}\hspace{\fboxsep}}
\end{sphinxVerbatim}
}

\end{sphinxuseclass}
\end{sphinxuseclass}
\begin{sphinxuseclass}{nbinput}
\begin{sphinxuseclass}{nblast}
{
\begin{sphinxVerbatim}[commandchars=\\\{\}]
\llap{\color{nbsphinxin}[ ]:\,\hspace{\fboxrule}\hspace{\fboxsep}}
\end{sphinxVerbatim}
}

\end{sphinxuseclass}
\end{sphinxuseclass}
\begin{sphinxuseclass}{nbinput}
\begin{sphinxuseclass}{nblast}
{
\begin{sphinxVerbatim}[commandchars=\\\{\}]
\llap{\color{nbsphinxin}[ ]:\,\hspace{\fboxrule}\hspace{\fboxsep}}
\end{sphinxVerbatim}
}

\end{sphinxuseclass}
\end{sphinxuseclass}
\begin{sphinxuseclass}{nbinput}
\begin{sphinxuseclass}{nblast}
{
\begin{sphinxVerbatim}[commandchars=\\\{\}]
\llap{\color{nbsphinxin}[ ]:\,\hspace{\fboxrule}\hspace{\fboxsep}}
\end{sphinxVerbatim}
}

\end{sphinxuseclass}
\end{sphinxuseclass}

\chapter{About!}
\label{\detokenize{index:about}}
\sphinxstepscope


\section{Affiliation}
\label{\detokenize{aboutMe:affiliation}}\label{\detokenize{aboutMe::doc}}

\begin{quote}


\end{quote}

\sphinxstepscope


\section{PHYSICAL VOLCANOLOGY RESEARCH GROUP}
\label{\detokenize{aboutUS:physical-volcanology-research-group}}\label{\detokenize{aboutUS::doc}}

\begin{quote}


\end{quote}

\sphinxstepscope


\section{Resume}
\label{\detokenize{CV_Mahmud_2023:resume}}\label{\detokenize{CV_Mahmud_2023::doc}}

\subsection{Mahmud Mustafa Muhammad}
\label{\detokenize{CV_Mahmud_2023:mahmud-mustafa-muhammad}}
\sphinxAtStartPar
Contact Information:
\sphinxhyphen{} Address: Department of Earth Sciences, Simon Fraser University, 8888 University Drive, Burnaby, British Columbia, Canada
\sphinxhyphen{} Phone: 778\sphinxhyphen{}957\sphinxhyphen{}3928
\sphinxhyphen{} Email: \sphinxhref{mailto:mahmudm@sfu.ca}{mahmudm@sfu.ca} , \sphinxhref{mailto:mahmud.muhamm1@gmail.com}{mahmud.muhamm1@gmail.com}

\sphinxAtStartPar
—


\subsection{Professional Preparation}
\label{\detokenize{CV_Mahmud_2023:professional-preparation}}\begin{itemize}
\item {} 
\sphinxAtStartPar
2019\sphinxhyphen{}Present: PhD Candidate, Department of Earth Science, Simon Fraser University, BC, Canada

\item {} 
\sphinxAtStartPar
2014\sphinxhyphen{}2016: M.Sc. in Geoscience, University of Nevada Las Vegas, USA

\item {} 
\sphinxAtStartPar
2013\sphinxhyphen{}2014: Diploma in ESL, University of St. Thomas/ELS, Houston, TX, USA

\item {} 
\sphinxAtStartPar
2007\sphinxhyphen{}2010: B.Sc. in Geology, University of Salahaddin\sphinxhyphen{}Erbil, Iraq

\end{itemize}

\sphinxAtStartPar
—


\subsection{Professional Appointments}
\label{\detokenize{CV_Mahmud_2023:professional-appointments}}\begin{itemize}
\item {} 
\sphinxAtStartPar
2018\sphinxhyphen{}Present: Research Assistant, Earth Sciences, Simon Fraser University, Canada

\item {} 
\sphinxAtStartPar
2016\sphinxhyphen{}2018: Assistant Lecturer, Geology Department, University of Salahaddin\sphinxhyphen{}Erbil, Iraq

\item {} 
\sphinxAtStartPar
2014\sphinxhyphen{}2016: Research Assistant, Geoscience Department, University of Nevada Las Vegas, USA

\item {} 
\sphinxAtStartPar
2013\sphinxhyphen{}2016: Exchange Scholar, Institute of International Education, Houston, TX, United States

\item {} 
\sphinxAtStartPar
2011\sphinxhyphen{}2013: Teaching Assistant, Geology Department, University of Salahaddin\sphinxhyphen{}Erbil, Iraq

\end{itemize}

\sphinxAtStartPar
—


\subsection{Professional Affiliations}
\label{\detokenize{CV_Mahmud_2023:professional-affiliations}}\begin{itemize}
\item {} 
\sphinxAtStartPar
Member, Association of Mineral Exploration, BC, Canada (since 2019)

\item {} 
\sphinxAtStartPar
Member, Geological Society of America (GSA) (since 2014)

\item {} 
\sphinxAtStartPar
Member, American Association of Petroleum Geologists (AAPG) (since April 2012)

\item {} 
\sphinxAtStartPar
Secretary, UNLV AAPG Student Chapter (August 2015\sphinxhyphen{}March 2016)

\item {} 
\sphinxAtStartPar
Member, Kurdistan Geologists Syndicate (since November 2010)

\end{itemize}

\sphinxAtStartPar
—


\subsection{Awards and Scholarships}
\label{\detokenize{CV_Mahmud_2023:awards-and-scholarships}}\begin{itemize}
\item {} 
\sphinxAtStartPar
Graduate Fellowship Grant, Simon Fraser University (7000 CAD, 2022)

\item {} 
\sphinxAtStartPar
Richard B. \& Cynthia Waitt Research Award in Field\sphinxhyphen{}based Research, GSA (3000 USD, 2022)

\item {} 
\sphinxAtStartPar
Internship, Accelerate Mitacs Program (TREALTAMIRA and Canadian Space Agency), 95000 CAD (2019)

\item {} 
\sphinxAtStartPar
Edwards \& Olswang Geology Scholarship, UNLV\sphinxhyphen{}Geoscience Department (300 USD, 2016)

\item {} 
\sphinxAtStartPar
Nevada Petroleum \& Geothermal Society (NPGS) Award (1200 USD, 2015)

\item {} 
\sphinxAtStartPar
U.S. Geological Survey EdMap Student Award (6000 USD, 2015)

\item {} 
\sphinxAtStartPar
Iraqi Scholars and Leaders Program, Exxon Mobil (120,000 USD, 2013)

\end{itemize}

\sphinxAtStartPar
—


\subsection{Key Projects}
\label{\detokenize{CV_Mahmud_2023:key-projects}}\begin{enumerate}
\sphinxsetlistlabels{\arabic}{enumi}{enumii}{}{.}%
\item {} 
\sphinxAtStartPar
AkhDefo Python\sphinxhyphen{}Based Software: Developed for processing ground\sphinxhyphen{}based and satellite optical and Radar images to monitor landslide stability. {[}PyPI Link{]}(\sphinxurl{https://pypi.org/project/akhdefo-functions/})

\item {} 
\sphinxAtStartPar
Bedrock Mapping for Mount Meager Geothermal Research Initiative: Contributed to the Garibaldi Volcanic Belt Geothermal Energy Project. {[}Technical Report{]}(\sphinxurl{https://publications.gc.ca/site/eng/9.913602/publication.html})

\item {} 
\sphinxAtStartPar
Structural Geology Applications in Geothermal Systems Exploration, Mt. Meager, BC: Research funded by Simon Fraser University, Natural Resource Canada, and Geoscience BC. {[}Journal Publication{]}(\sphinxurl{https://doi.org/10.1139/cjes-2023-0077})

\item {} 
\sphinxAtStartPar
Mount Currie 2020 Geology Fieldwork Report: Investigated key geological parameters. Funded by Simon Fraser University and Squamish Lillooet Regional District.

\item {} 
\sphinxAtStartPar
Expanded Bedrock Map of Mount Meager Volcanic Complex: Collaborative project with Geological Survey of Canada. {[}Publication Link{]}(\sphinxurl{https://publications.gc.ca/site/eng/9.913602/publication.html})

\end{enumerate}

\sphinxAtStartPar
—


\subsection{Journal Publications}
\label{\detokenize{CV_Mahmud_2023:journal-publications}}\begin{itemize}
\item {} 
\sphinxAtStartPar
Muhammad, M., Williams\sphinxhyphen{}Jones, G., Barendregt, R.W. Structural Geology of the Mount Meager Volcanic Complex, BC, Canada. Canadian Journal of Earth Sciences. {[}Link{]}(\sphinxurl{https://doi.org/10.1139/cjes-2023-0077})

\item {} 
\sphinxAtStartPar
Muhammad, M. et al. (2022). Applications of Image\sphinxhyphen{}Based Computer Vision for Remote Surveillance of Slope Instability. Front. Earth Sci. {[}DOI{]}(10.3389/feart.2022.909078)

\item {} 
\sphinxAtStartPar
M.M. Muhammad, A.H. Awdal: Automatic mapping of lineaments using shaded relief images derived from digital elevation model (DEM) in Erbil\sphinxhyphen{}Kurdistan, northeast Iraq. Advances in Natural and Applied Sciences 01/2012; 6(2\sphinxhyphen{}2):138\sphinxhyphen{}146

\end{itemize}

\sphinxAtStartPar
—


\subsection{Technical Reports}
\label{\detokenize{CV_Mahmud_2023:technical-reports}}\begin{itemize}
\item {} 
\sphinxAtStartPar
Muhammad, M., et al. (2020). Applications of Structural Geology to the Exploration of Geothermal Systems, Mt. Meager, BC.

\item {} 
\sphinxAtStartPar
Harris, M., Muhammad, M., et al. (2020). Bedrock Mapping for Mount Meager Geothermal Research Initiative.

\item {} 
\sphinxAtStartPar
Muhammad, M., Williams\sphinxhyphen{}Jones, G., Stead, D., and Donati, D., 2020, Mount Currie 2020 Geology Fieldwork Report: Key Geological Parameters Controlling Main Mount Currie Scarp Structure: Technical report submitted to Squamish\sphinxhyphen{}Lillooet Regional District.

\item {} 
\sphinxAtStartPar
Price, T., Evans, M., Muhammad, M., Hinson, M., McIntyre, A., Peck, A., Hahn, P., and Taylor, W.J., 2017, Preliminary geologic map of the Lower Pahranagat Lake NW quadrangle, Lincoln County, Nevada: Nevada Bureau of Mines and Geology Open\sphinxhyphen{}File Report 17\sphinxhyphen{}3, scale 1:24,000.

\end{itemize}

\sphinxAtStartPar
—


\subsection{Conference Proceedings}
\label{\detokenize{CV_Mahmud_2023:conference-proceedings}}\begin{itemize}
\item {} 
\sphinxAtStartPar
Akhdefo Software: A tool for land deformation Monitoring using daily satellite optical imagery,6th wolrd landslide Forum, held in Florence(Italy) from 14 to 17 Novemeber 2023

\item {} 
\sphinxAtStartPar
Taylor, Wanda, Abdelhaleem, S., Ely, R., Evans, M., Muhammad, M., Peck, A., Price, T.2, Reed,
Nathan and Reid, Andrew J., 2022, The Boundary Zone Between the Northern and Central Basin \&
Range Sub provinces: A Developing Rift Segment Boundary, Geological Society of America
Abstracts with Programs, Joint 118th Annual Cordilleran/72nd Annual Rocky Mountain Section Meeting, \sphinxurl{https://doi.org/10.1130/abs/2022CD-374351}

\item {} 
\sphinxAtStartPar
Mahmud Muhammad, Martin Harris, Glyn Williams Jones, and James K. Russell: Structural and Tectonic activity of Mount Meager Complex: Implications for quality of Geothermal Exploration, AME\sphinxhyphen{}Roundup 2020 poster sessions, Vancouver, January 2020.

\item {} 
\sphinxAtStartPar
Mahmud Muhammad: Groundwater Contamination in the Kurdistan region of Iraq: Geological controls on hydrocarbon leakage. International Conference on Applied Science, Energy and Environment (ICASEE\sphinxhyphen{} 2018), Ishik University\sphinxhyphen{}Erbil, Erbil, Kurdistan Region of Iraq; 04/2018

\item {} 
\sphinxAtStartPar
Mahmud M. Muhammad, Wanda J. Taylor: Kinematics of Apparent Reverse Faults Within Extensional Tectonics: Case Study from The Pahranagat Shear Zone (PSZ), Basin and Range,
Nevada, USA. Joint 70th Annual Rocky Mountain GSA Section / 114th Annual Cordilleran GSA Section Meeting \sphinxhyphen{} 2018; 01/2018, DOI:10.1130/abs/2018RM\sphinxhyphen{}313605

\item {} 
\sphinxAtStartPar
Mahmud Muhammad, Wanda J. Taylor: Structural Evolution of The Maynard Lake Fault (MLF) Within the Left\sphinxhyphen{}Lateral Pahranagat Shear Zone (PSZ), Nevada, USA. 112th Annual GSA Cordilleran Section Meeting; 04/2016, DOI:10.1130/abs/2016CD\sphinxhyphen{}274230.

\end{itemize}

\sphinxAtStartPar
—


\subsection{Published Geology Maps}
\label{\detokenize{CV_Mahmud_2023:published-geology-maps}}\begin{itemize}
\item {} 
\sphinxAtStartPar
Mahmud Muhammad, Glyn Williams\sphinxhyphen{}Jones, and René W. Barendregt. Structural geology of the Mount Meager Volcanic Complex, BC, Canada: implications for geothermal energy and geohazards. Canadian Journal of Earth Sciences. \sphinxhref{https://cdnsciencepub.com/doi/full/10.1139/cjes-2023-0077\#fig12}{Detailed geology map of MMVC, GVB, British Columbia, Canada at a scale 1:30 000}

\item {} 
\sphinxAtStartPar
\sphinxhref{https://doi.org/10.4095/329886}{Harris, M., et al. (2022). Mount Meager Volcanic Complex, Garibaldi Volcanic Belt, British Columbia. Geological Survey of Canada, Link}

\item {} 
\sphinxAtStartPar
\sphinxhref{https://pubs.nbmg.unr.edu/Prelim-Lower-Pahranagat-Lake-NW-p/of2017-03.htm}{Price, T., Evans, M., Muhammad, M., Hinson, M., McIntyre, A., Peck, A., Hahn, P., and Taylor, W.J., 2017, Preliminary geologic map of the Lower Pahranagat Lake NW quadrangle, Lincoln County, Nevada: Nevada Bureau of Mines and Geology Open\sphinxhyphen{}File Report 17\sphinxhyphen{}3, scale 1:24,000.}

\end{itemize}

\sphinxAtStartPar
—


\subsection{Software Skills}
\label{\detokenize{CV_Mahmud_2023:software-skills}}\begin{itemize}
\item {} 
\sphinxAtStartPar
ESRI\sphinxhyphen{}ArcGIS (ArcMap, ArcScene, ArcGlobe)

\item {} 
\sphinxAtStartPar
PCI\sphinxhyphen{}Geomatica and ENVI

\item {} 
\sphinxAtStartPar
LeapFrog (Geothermal and Geology)

\item {} 
\sphinxAtStartPar
Python (GIS, Computer Vision, Machine Learning, Image Processing)

\item {} 
\sphinxAtStartPar
Synthetic Aperture Radar Image Processing (ISCE2, SARSCAPE, GMTSAR)

\end{itemize}

\sphinxAtStartPar
—


\subsection{Software Developed}
\label{\detokenize{CV_Mahmud_2023:software-developed}}\begin{itemize}
\item {} 
\sphinxAtStartPar
AkhDefo Open\sphinxhyphen{}Source Code: Python\sphinxhyphen{}based software for land deformation monitoring using high temporal resolution ground and satellite optical imagery. {[}PyPI Link{]}(\sphinxurl{https://pypi.org/project/akhdefo-functions/})

\end{itemize}


\chapter{Indices and tables}
\label{\detokenize{index:indices-and-tables}}\begin{itemize}
\item {} 
\sphinxAtStartPar
\DUrole{xref,std,std-ref}{genindex}

\item {} 
\sphinxAtStartPar
\DUrole{xref,std,std-ref}{modindex}

\item {} 
\sphinxAtStartPar
\DUrole{xref,std,std-ref}{search}

\end{itemize}


\renewcommand{\indexname}{Python Module Index}
\begin{sphinxtheindex}
\let\bigletter\sphinxstyleindexlettergroup
\bigletter{a}
\item\relax\sphinxstyleindexentry{akhdefo\_functions}\sphinxstyleindexpageref{akhdefo_functions:\detokenize{module-akhdefo_functions}}
\item\relax\sphinxstyleindexentry{akhdefo\_functions.Akhdefo\_Coreg}\sphinxstyleindexpageref{akhdefo_functions:\detokenize{module-akhdefo_functions.Akhdefo_Coreg}}
\item\relax\sphinxstyleindexentry{akhdefo\_functions.Akhdefo\_GOI}\sphinxstyleindexpageref{akhdefo_functions:\detokenize{module-akhdefo_functions.Akhdefo_GOI}}
\item\relax\sphinxstyleindexentry{akhdefo\_functions.Akhdefo\_Tools}\sphinxstyleindexpageref{akhdefo_functions:\detokenize{module-akhdefo_functions.Akhdefo_Tools}}
\item\relax\sphinxstyleindexentry{akhdefo\_functions.Akhdefo\_TS}\sphinxstyleindexpageref{akhdefo_functions:\detokenize{module-akhdefo_functions.Akhdefo_TS}}
\item\relax\sphinxstyleindexentry{akhdefo\_functions.Akhdefo\_utils}\sphinxstyleindexpageref{akhdefo_functions:\detokenize{module-akhdefo_functions.Akhdefo_utils}}
\item\relax\sphinxstyleindexentry{akhdefo\_functions.AkhdefoPlot}\sphinxstyleindexpageref{akhdefo_functions:\detokenize{module-akhdefo_functions.AkhdefoPlot}}
\item\relax\sphinxstyleindexentry{akhdefo\_functions.Filter\_PreProc}\sphinxstyleindexpageref{akhdefo_functions:\detokenize{module-akhdefo_functions.Filter_PreProc}}
\item\relax\sphinxstyleindexentry{akhdefo\_functions.Mosaic\_Crop}\sphinxstyleindexpageref{akhdefo_functions:\detokenize{module-akhdefo_functions.Mosaic_Crop}}
\item\relax\sphinxstyleindexentry{akhdefo\_functions.OpticalFlow}\sphinxstyleindexpageref{akhdefo_functions:\detokenize{module-akhdefo_functions.OpticalFlow}}
\item\relax\sphinxstyleindexentry{akhdefo\_functions.Stacked\_Velocity}\sphinxstyleindexpageref{akhdefo_functions:\detokenize{module-akhdefo_functions.Stacked_Velocity}}
\item\relax\sphinxstyleindexentry{akhdefo\_functions.Unzip\_CopyFiles}\sphinxstyleindexpageref{akhdefo_functions:\detokenize{module-akhdefo_functions.Unzip_CopyFiles}}
\item\relax\sphinxstyleindexentry{akhdefo\_functions.video\_streamer}\sphinxstyleindexpageref{akhdefo_functions:\detokenize{module-akhdefo_functions.video_streamer}}
\end{sphinxtheindex}

\renewcommand{\indexname}{Index}
\printindex
\end{document}